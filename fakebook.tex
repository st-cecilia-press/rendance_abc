\documentclass[11pt]{book}
\usepackage{fancyhdr}
\usepackage{multicol}
\usepackage{makeidx}
\usepackage[hidelinks]{hyperref}
\usepackage{refcount}
\usepackage{abc}
\usepackage{tocloft}
\usepackage[utf8]{inputenc}
\usepackage[top=48pt,headheight=18pt,headsep=12pt,bottom=48pt,inner=48pt,outer=48pt]{geometry}
\makeindex

\begin{document}
%\setuplayouts
\frontmatter

\pagestyle{fancy}

\fancyhf{}

\fancyhead[LE,RO]{\thepage}
\fancyhead[RE,LO]{DRAFT \today -- {\bf DO NOT DISTRIBUTE}}
%\pagebreak
\renewcommand{\headrulewidth}{0pt}
\renewcommand{\footrulewidth}{0pt}
\setlength{\parindent}{0pt}
\setlength{\parskip}{11pt plus 2pt minus 2pt}

%\begin{figure}
%
%\currentpage
%
%\drawparametersfalse
%
%\drawpage
%
%\caption{Page layout for this document} \label{fig:ptrs}
%
%\end{figure}
%\pagebreak
%

\section*{Note from the Editor}

The goal of this book is to include as many tunes as possible for extant
pre-1651 dances. It also includes music for dances choreographed by SCA
memebers in a variety of styles as well as some tunes for later English Country
dances and modern folk dances that are danced in the SCA.

We have provided this fakebook both as a printable PDF and as ABC files,
obtainable from \url{https://github.com/orgs/st-cecilia-press/rendance\_abc}.
ABC is a simple format for music readable both by computers and by humans that
is commonly used for folk music and dance tunes. There is also a lot of
software to display, play, search, and transpose ABC files. We particularly
recommend EasyABC on Mac, Linux and Windows:
\url{http://www.nilsliberg.se/ksp/easyabc/}. There are many other software
packages including for Android and iOS devices. A full list can be found at
\url{http://abcnotation.com/software}.

We have aimed to match the key of the tune as well as the marked chords to a
tune's most recent appearance in the Pennsic Pile. If it has not recently
appeared in the Pennsic Pile, we have typically kept the original key in the
source material and added an attempt at a reasonable harmonization.

Our belief is that all these tunes are freely usable within the SCA. Most are
many hundreds of years old. However, please make your own determination before
using any of this material in a non-SCA/non-educational setting. Sources of
harmonizations are noted in the ABC files mentioned above.

Some of these tunes have many more chords marked than most chord players would
reasonably want to play. These extra chord markings may still be useful for
players improvising a bass line or for adding passing notes when playing a
chordal accompaniment.

Thanks to Emma Badowski for organizing this material in a previous edition of
this fakebook and especially for harmonizing many of the tunes from Playford
1651 that have not previously appeared in the Pennsic Pile.

Please note that this book is {\em not} an official publication of the Society
for Creative Anachronism.

Aaron Elkiss

\clearpage
\begin{multicols}{2}

\renewcommand\cftchapafterpnum{\vskip\baselineskip}
\setlength{\cftsubsecindent}{0pt}
\setlength{\cftsubsecnumwidth}{0pt}
\tableofcontents
\end{multicols}

\clearpage
\mainmatter

\chapter{Basse Danse}

Basse danse (or bassadanza in Italian) was popular across Europe in the 15th
and early 16th centuries. One of the most important sources for basse danse is
Ms 9085 in the Bibliotheque Royale, Brussels (c. 1445) This manuscript gives
only a slow-moving tenor, or cantus firmus, as the melody for most of the
dances. Musicians normally would have improvised multipart polyphony above the
tenor line. One simple way to improve a melody from the tenor line is to play
it in the style of the basse dance section from ``Rostiboli Gioioso''.  Most of
the basse danses are notated here in 6/4 time, and an appropriate tempo would
be approximately dotted half note = 40-45.

\clearpage

\index{Alenchon}
\addcontentsline{toc}{subsection}{Alenchon}
\begin{abc}[name=latex_basse_dance1]
X:1
I:linebreak $
T:Alenchon
M:6/4
L:1/8
K:C clef=treble-8 octave=1 
A,12| C12| D12| A,12| F,12| E,12| D,12| D,12| F,12| 
A,12| D12| C12| A,12| F,12| D,12| E,12| D,12| D,12| F,12| G,12|
A,12| A,12| D12| A,12| F,12| D,12| F,12| E,12| D,12| D,12| D,12|]


\end{abc}
\index{Allemande, la}
\addcontentsline{toc}{subsection}{La Allemande}
\begin{abc}[name=latex_basse_dance2]
X:2
I:linebreak $
T:La Allemande
C:Paul Butler
M:6/4
L:1/8
K:Gdor
D12 | G12| G12 | A12 | D12 | D12 | C12 | A12 | 
F12 | D12 | G12 | G12 | F12 | C12 | D12 | G12 |
 G12 | A12 | D12 | D12 | D12 | G12 |]


\end{abc}
\index{Aliot nouvella}
\addcontentsline{toc}{subsection}{Aliot nouvella}
\begin{abc}[name=latex_basse_dance3]
X:3
I:linebreak $
T:Aliot nouvella
M:6/4
L:1/8
K:Ddor clef=treble-8 octave=1 
D,12 | D,12 | F,12 | A,12 | B,12 |
A,12 | C12 | D12 | C12 |
A,12 | A,12 | G,12 | F,12 |
F,12 | B,12 | A,12 | E,12 |
D,12 | F,12 | E,12 | D,12 |
D,12 | F,12 | E,12 | F,12 |
G,12 | A,12 | _B,12 | A,12 |
G,12 | F,12 | F,12 | F,12 |
E,12 | D,12 | D,12 | D,12 |]


\end{abc}
\index{Amours}
\addcontentsline{toc}{subsection}{Amours}
\begin{abc}[name=latex_basse_dance4]
X:4
I:linebreak $
T:Amours
M:3/2
L:1/8
K:C clef=treble-8 octave=1 
F,12| E,12| F,12| F,12|
E,12| D,12| E,12| ^F,12|
G,12| =F,12| E,12| F,12|
F,12| E,12| D,12| E,12|
^F,12| G,12| G,12|]


\end{abc}
\index{Avignon}
\addcontentsline{toc}{subsection}{Avignon}
\begin{abc}[name=latex_basse_dance5]
X:5
I:linebreak $
T:Avignon
M:3/2
L:1/8
K:C clef=treble-8 octave=1 
D,12| C,12| D,12|
E,12| F,12| E,12| D,12|
D,12| G,12| F,12| E,12|
D,12| A,12| F,12| E,12|
E,12| G,12| G,12| A,12|
A,12| C12| B,12| A,12|
A,12| G,12| F,12| E,12|
E,12| G,12| A,12| E,12|
D,12| F,12| E,12| D,12|
D,12| A,,12| C,12| D,12|
E,12| F,12| E,12| D,12|
D,12| D,12|]


\end{abc}
\index{Barbesieux}
\addcontentsline{toc}{subsection}{Barbesieux}
\begin{abc}[name=latex_basse_dance6]
X:6
I:linebreak $
T:Barbesieux
M:6/4
L:1/8
K:Dm clef=treble-8 octave=1
B,12 | D12 | C12 | B,12 | A,12 |
G,12 | F,12 | A,12 | G,12 |
A,12 | A,12 | D12 | C12 |
=B,12 | A,12 | G,12 | F,12 |
E,12 | D,12 | F,12 | G,12 |
F,12 | E,12 | D,12 | D,12 |
D12 | E12 | D12 | C12 |
=B,12 | A,12 | A,12 | B,12 |
A,12 | F,12 | E,12 | D,12 |
D,12 | D,12 |]


\end{abc}
\index{Barcelonne}
\addcontentsline{toc}{subsection}{Barcelonne}
\begin{abc}[name=latex_basse_dance7]
X:7
I:linebreak $
T:Barcelonne
M:6/4
L:1/8
K:C clef=treble-8 octave=1 
G,12| B,12| D12| C12|
B,12| A,12| D,12| E,12|
D,12| D,12| A,12| B,12|
D12| C12| B,12| A,12|
G,12| F,12| E,12| D,12|
D,12| F,12| E,12| D,12|
F,12| A,12| C12| A,12|
F,12| D,12| F,12| E,12|
D,12| D,12| D,12|]


\end{abc}
\index{Basine, la}
\addcontentsline{toc}{subsection}{La basine}
\begin{abc}[name=latex_basse_dance8]
X:8
I:linebreak $
T:La basine
M:6/4
L:1/8
K:C clef=treble-8 octave=1 
F,12| D,12| E,12| D,12|
F,12| E,12| F,12| A,12|
B,12| A,12| C12| B,12|
A,12| A,12| A,12| B,12|
A,12| G,12| F,12| F,12|
E,12| D,12| D,12| D,12|]


\end{abc}
\index{Bayonne}
\addcontentsline{toc}{subsection}{Bayonne}
\begin{abc}[name=latex_basse_dance9]
X:9
I:linebreak $
T:Bayonne
M:6/2
L:1/8
K:C clef=treble-8 octave=1 
A,12| B,12| D12| C12| B,12| A,12| D12| C12| A,12| F,12| E,12| D,12|
w:B:~R b ss d r ss d d d ss r r w: T:~R b ss d d d ss r r r b ss 
E,12| D,12| D,12| G,12| B,12| A,12| G,12| A,12| D12| C12| B,12| A,12|
w:r b ss d r b ss d r ss d d
w:d ss r r r b ss d d d ss r 
A,12| B,12| A,12| G,12| F,12| D,12| E,12| D,12| D,12| D,12|]
w:d r r r b ss d r b cw: r r b ss d r r r b c


\end{abc}
\index{Beaulte de Castile}
\addcontentsline{toc}{subsection}{Beaulte de Castile}
\begin{abc}[name=latex_basse_dance10]
X:10
I:linebreak $
T:Beaulte de Castile
M:6/8
L:1/8
K:C clef=treble-8 octave=1 
C6| C6| G,6| C6|
C6| C6| G,6| C6|
C6| G,6| C4-CC| D2C B,2A,|
G,4-G,G,| D2C B,A,/2G,/2A,/2^F,/2| G,6| =F,6|
F,6| F,4-F,G,| D2C B,A,/2G,/2A,/2^F,/2| G,6|
=F,6| F,6| F,4-F,G,| D2C B,A,/2G,/2A,/2^F,/2| G,6|]


\end{abc}
\index{Beaulte}
\addcontentsline{toc}{subsection}{Beaulte}
\begin{abc}[name=latex_basse_dance11]
X:11
I:linebreak $
T:Beaulte
M:6/4
L:1/8
K:C clef=treble-8 octave=1 
D12| D12| C12| C12|
A,12| G,12| F,12| F,12|
G,12| A,12| D,12| F,12|
E,12| D,12| F,12| G,12|
A,12| D12| C12| B,12|
A,12| A,12| A,12| D12|
A,12| D,12| F,12| E,12|
D,12| E,12| F,12| E,12|
D,12| G,12| F,12| D,12|
E,12| D,12| D,12| D,12|]


\end{abc}
\index{Belle, la}
\addcontentsline{toc}{subsection}{La Belle}
\begin{abc}[name=latex_basse_dance12]
X:12
I:linebreak $
T:La Belle
M:6/4
L:1/8
K:C clef=treble-8 octave=1 
D12| 
A,12| D,12| E,12| D,12| 
D,12| A,12| C12| D12| 
C12| B,12| A,12| A,12| 
B,12| B,12| A,12| F,12| 
E,12| D,12| F,12| E,12| 
D,12| D,12| A,12| B,12| 
A,12| G,12| F,12| E,12| 
D,12| D,12| D,12| D,12|]


\end{abc}
\index{Casuelle la Nouvelle}
\addcontentsline{toc}{subsection}{Casuelle la Nouvelle}
\begin{abc}[name=latex_basse_dance13]
X:13
I:linebreak $
T:Casuelle la Nouvelle
T:La Spagna
N:adapted from Heinrich Isaac setting
K:F major clef=G-8
M:6/4
L:1/2
(d3 | d3) | A3 | G3 | B3 | A3 | (G3 | G3) | B3 | c3 | (d3 | d3) |
f3 | e3 | d3 | f3 | g3 | g3 | c3 | d3 | c3 | c3 | g3 | g3 | 
f3 | f3 | g3 | c3 | B3 || _e3 | (d3 | d3) | G3 | A3 | c3 | d3 | f3 | 
_e3 | d3 | c3 | B3 | A3 | G3 | d3 | (G3  | HG3 ) |]


\end{abc}
\index{Cupido}
\addcontentsline{toc}{subsection}{Cupido}
\begin{abc}[name=latex_basse_dance14]
X:14
T:Cupido
T:Canzon di Pifari
C:Cornazano, c. 1465
M:6/4
L:1/8
K:Gmaj clef=G-8
B12 | B12 | d12 | e12 | A12 | B12 | A12 | A12 | e12  | d12 | c12 | d12 | e12 | e12 | A12 | d12 |
c12 | c12 | B12 | B12 | G12 | B12 | d12 | e12 | A12 | B12 | A12 | A12 | B12 | c12 | A12 | G12 |
e12 | d12 | c12 | c12 | d12 | e12 | c12 | d12 | A12 | B12 | A12 | B12 | A12 |]


\end{abc}
\index{Danse de Cleves}
\addcontentsline{toc}{subsection}{Danse de Cleves}
\begin{abc}[name=latex_basse_dance15]
X:15
I:linebreak $
T:Danse de Cleves
M:6/4
L:1/8
K:D minor
P:A
G2 | d2cBc/B/A/G/ A2G2GA/B/ | c2AFG2 F2F2F2 | B2B2c2 d2d2c2 | B2cBA2 G2G2G2 || 
P:B
d2cBGB A2G2GA/B/ | c2ABG2 F2F2F2 | 
B2B2c2 d2G2_e2 | dc2BA2 G2G2G2 |: 
P:C
g2g2g2 a2d2c/d/e/f/ | gfede2 d2d2G2 :| 
P:D
d2cBc/B/A/G/ A2G2GA/B/ | c2AFG2 F2F2F2 | 
B2B2c2 d2d2c2 | B2cBA2 G2G2G2 || 
P:E
g2g2g2 a2d2ef | g2fde2 d2d2d2 | g2g2g2 a2d2ef | g2fge2 d2d2G2 || 
P:F
d2cB2G A2G2AB | c2BAG2 F2F2F2 | B2B2c2 d2d2c2 | B2GBA2 G2G2G2 || 
P:G
d2cB2c A2G2AB | c2BAG2 F2F2F2 | 
B2B2c2 d2d2c2 | B2GBA2 G6 |] 


\end{abc}
\index{Doulce amour, la}
\addcontentsline{toc}{subsection}{La doulce amour}
\begin{abc}[name=latex_basse_dance16]
X:16
I:linebreak $
T:La doulce amour
M:6/4
L:1/8
K:C clef=treble-8 octave=1 
A,12| A,12| F,12| E,12|
A,12| A,12| D,12| E,12|
D,12| D,12| A,12| A,12|
D,12| G,12| E,12| F,12|
E,12| E,12| A,12| A,12|
C12| B,12| A,12| D,12|
A,12| A,12| A,12| G,12|
F,12| A,12| B,12| A,12|
A,12| C12| B,12| A,12|
G,12| E,12| D,12| D,12|
D,12|]


\end{abc}
\index{Doulz espoir, le}
\addcontentsline{toc}{subsection}{Le doulz espoir}
\begin{abc}[name=latex_basse_dance17]
X:17
I:linebreak $
T:Le doulz espoir
M:6/4
L:1/8
K:Ddor clef=treble-8 octave=1 
F,12| D,12| E,12| F,12|
A,12| G,12| F,12| F,12|
C12| A,12| C12| B,12|
A,12| G,12| F,12| F,12|
A,12| G,12| D,12| C,12|
C,12| D,12| D,12| C,12|
F,12| A,12| C12| B,12|
A,12| G,12| F,12| C12|
D12| C12| F,12| C12|
B,12| A,12| C12| B,12|
A,12| G,12| F,12| E,12|
D,12| D,12| A,12| C12|
B,12| C12| B,12| A,12|
G,12| G,12| C12| B,12|
A,12| A,12| A,12| G,12|
F,12| F,12| F,12|]


\end{abc}
\index{Basse danse du roy, la}
\addcontentsline{toc}{subsection}{La basse danse du roy}
\begin{abc}[name=latex_basse_dance18]
X:18
I:linebreak $
T:La basse danse du roy
M:6/4
L:1/8
K:C clef=treble-8 octave=1 
F,12| F,12| E,12| A,12|
A,12| E,12| D,12| C,12|
C,12| E,12| E,12| F,12|
A,12| G,12| F,12| E,12|
E,12| E,12| F,12| G,12|
F,12| E,12| D,12| C,12|
C,12| C,12| D,12| C,12|
C,12| G,12| F,12| A,12|
G,12| G,12| E,12| F,12|
E,12| D,12| C,12| C,12|
F,12| G,12| A,12| A,12|
E,12| D,12| C,12| C,12|
C,12|]


\end{abc}
\index{Engoulesme}
\addcontentsline{toc}{subsection}{Engoulesme}
\begin{abc}[name=latex_basse_dance19]
X:19
I:linebreak $
T:Engoulesme
M:6/4
L:1/8
K:C clef=treble-8 octave=1 
F,12| A,12| D12| C12|
A,12| G,12| F,12| E,12|
D,12| D,12| A,12| D12|
A,12| G,12| A,12| G,12|
D12| C12| B,12| A,12| G,12| 
D,12| E,12| D,12| D,12| 
D12| A,12| F,12| D,12| 
F,12| G,12| F,12| E,12| 
D,12| D,12| D,12|]


\end{abc}
\index{Joyeulx espoyr, le}
\addcontentsline{toc}{subsection}{Le Joyeulx espoyr}
\begin{abc}[name=latex_basse_dance20]
X:20
I:linebreak $
T:Le Joyeulx espoyr
M:6/4
L:1/8
K:C clef=treble-8 octave=1 
G,12| E,12| F,12| C,12|
G,12| B,12| A,12| G,12|
B,12| C12| G,12| F,12|
C,12| F,12| E,12| D,12|
C,12| G,12| C,12| D,12|
C,12| C,12| C,12|]


\end{abc}
\index{Filles a marier}
\addcontentsline{toc}{subsection}{Filles a marier}
\begin{abc}[name=latex_basse_dance21]
X:21
I:linebreak $
T:Filles a marier
M:6/4
L:1/8
K:C clef=treble-8 octave=1 
A,12| A,12| C12| G,12|
C12| D12| C12| C12|
C12| D12| E12| D12|
C12| B,12| A,12| A,12|
E12| B,12| D12| G,12|
C12| D12| C12| C12|
C12| D12| E12| D12|
C12| B,12| A,12| A,12|
A,12|]


\end{abc}
\index{Florentine}
\addcontentsline{toc}{subsection}{Florentine}
\begin{abc}[name=latex_basse_dance22]
X:22
I:linebreak $
T:Florentine
M:3/2
L:1/8
K:C clef=treble-8 octave=1 
x12| A,12| F,12| G,12|
D,12| A,12| C12| B,12|
A,12| ^C12| D12| A,12|
G,12| D,12| G,12| F,12|
E,12| D,12| A,12| D,12|
E,12| D,12| D,12| A,12|
F,12| G,12| D,12| A,12|
=C12| B,12| A,12| ^C12|
D12| A,12| G,12| D,12|
G,12| F,12| E,12| D,12|
A,12| D,12| E,12| D,12|
D,12| D,12|]


\end{abc}
\index{Franchoise nouvelle}
\addcontentsline{toc}{subsection}{Franchoise nouvelle}
\begin{abc}[name=latex_basse_dance23]
X:23
I:linebreak $
T:Franchoise nouvelle
M:6/2
L:1/8
K:C clef=treble-8 octave=1 
C4C4C4 B,4G,4A,2B,2| 
C3 B,A,2 G,2A,4 G,4G,4G,4| 
C4C4C4 B,3 A,G,4A,2B,2| 
C4G,2 B,2A,4 G,4G,4C2B,2|
A,4G,2 F,3 E,D,2 E,4C,4C,4| 
F,4E,2 C,2D,4 E,4E,4C2B,2| 
A,4G,2 F,3 E,D,2 E,4C,4C,4| 
F,3 E,D,2 C,2D,4 C,4C,4C,4|
C,4C,4D,4 E,4E,4F,4| 
E,4D,4D,4 C,4C,4C,4| 
C,4C,4D,4 E,4E,4F,4| 
E,2 (3G,4-G,F,4- F,D,4-D, C,4C,4C,4|
C4C4C4 B,4G,4A,2B,2| 
C3 B,A,2 G,2A,4 G,4G,4G,4| 
C4C4C4 B,3 A,G,4A,2B,2| 
C4G,2 B,2A,4 G,4G,4C2B,2|
A,4G,2 F,3 E,D,2 E,4C,4C,4| 
F,4E,2 C,2D,4 E,4E,4C2B,2| 
A,4G,2 F,3 E,D,2 E,4C,4C,4| 
F,3 E,D,2 C,2D,4 C,4C,4C,4|
C,4C,4D,4 E,4E,4F,4| 
E,4D,4D,4 C,4C,4C,4| 
C,4C,4D,4 E,4E,4F,4| 
E,2 (3G,4-G,F,4- F,D,4-D, C,4C,4C,4|
C,16 |]


\end{abc}
\index{Grant Rouen, le}
\addcontentsline{toc}{subsection}{Le grant Rouen}
\begin{abc}[name=latex_basse_dance24]
X:24
I:linebreak $
T:Le grant Rouen
M:6/4
L:1/8
K:C clef=treble-8 octave=1 
A,12| G,12| C12| C,12|
E,12| D,12| C,12| C,12|
G,12| G,12| A,12| C12|
B,12| A,12| G,12| G,12|
G,12| B,12| C12| C12|
A,12| F,12| E,12| E,12|
E,12| D,12| C,12| D,12|
C,12| B,12| A,12| C12|
D12| D12| G,12| A,12|
G,12| F,12| E,12| E,12|
A,12| B,12| C12| F,12|
C,12| D,12| C,12| C,12|
C,12|]


\end{abc}
\index{Grant Thorin, le}
\addcontentsline{toc}{subsection}{Le grant Thorin}
\begin{abc}[name=latex_basse_dance25]
X:25
I:linebreak $
T:Le grant Thorin
M:6/4
L:1/8
K:C clef=treble-8 octave=1
A,12| A,12| F,12| G,12| A,12| 
C12| B,12| B,12| A,12| A,12| 
B,12| F,12| E,12| D,12| E,12| E,12|
D,12| D,12| B,12| B,12| C12| B,12| 
A,12| A,12| B,12| A,12| D12| 
C12| B,12| A,12| G,12| F,12|
F,12| A,12| C12| B,12| A,12| 
C12| B,12| A,12| A,12| F,12| 
A,12| G,12| F,12| G,12| F,12| 
F,12| F,12|]


\end{abc}
\index{Hault et le bas, le}
\addcontentsline{toc}{subsection}{Le hault et le bas}
\begin{abc}[name=latex_basse_dance26]
X:26
I:linebreak $
T:Le hault et le bas
M:6/4
L:1/8
K:C clef=treble-8 octave=1 
F,12| D,12| E,12| D,12| F,12| 
G,12| A,12| A,12| A,12| C12| 
D12| A,12| D,12| E,12| D,12| 
D,12| F,12| D,12| E,12| D,12|
F,12| G,12| A,12| A,12| A,12| 
C12| D12| A,12| D,12| E,12| 
D,12| D,12| D,12|]


\end{abc}
\index{Haulte Borgongne (Lydian), la}
\addcontentsline{toc}{subsection}{La haulte Borgongne (Lydian)}
\begin{abc}[name=latex_basse_dance27]
X:27
I:linebreak $
T:La haulte Borgongne (Lydian)
M:6/4
L:1/8
K:C clef=treble-8 octave=1
B,12| A,12| E,12| B,,12| C,12| 
D,12| _E,12| F,12| F,12| A,12| 
G,12| G,12| F,12| F,12| G,12| 
A,12| F,12| D,12| _E,12| _E,12|
D,12| C,12| B,,12| C,12| D,12| 
F,12| D,12| C,12| B,,12| B,,12| 
B,12| C12| F,12| G,12| F,12| 
F,12| C12| =B,12| A,12| G,12|
F,12| =E,12| D,12| D,12| F,12| 
G,12| F,12| C,12| D,12| C,12| 
_B,,12| B,,12| B,,12|]


\end{abc}
\index{Haulte Borgongne (Dorian), la}
\addcontentsline{toc}{subsection}{La haulte Borgongne (Dorian)}
\begin{abc}[name=latex_basse_dance28]
X:28
I:linebreak $
T:La haulte Borgongne (Dorian)
M:6/4
L:1/8
K:C clef=treble-8 octave=1 
D12| C12| G,12| D,12| E,12| 
F,12| G,12| A,12| A,12| C12| 
B,12| B,12| A,12| A,12| B,12| 
C12| A,12| F,12| G,12| G,12| F,12| 
E,12| D,12| E,12| F,12| A,12| 
F,12| E,12| D,12| D,12| D12| 
E12| A,12| B,12| A,12| A,12|
E12| D12| C12| B,12| A,12| 
G,12| F,12| F,12| A,12| B,12| 
A,12| E,12| F,12| E,12| D,12| D,12|
D,12|]


\end{abc}
\index{Je languis}
\addcontentsline{toc}{subsection}{Je languis}
\begin{abc}[name=latex_basse_dance29]
X:29
I:linebreak $
T:Je languis
M:6/4
L:1/8
K:C clef=treble-8 octave=1 
C,12| E,12| G,12| A,12| G,12| 
G,12| G,12| A,12| C12| B,12| 
A,12| G,12| F,12| E,12| E,12| 
C,12| C,12| E,12| G,12| G,12|
A,12| G,12| G,12| C,12| E,12| 
G,12| A,12| G,12| F,12| E,12| 
D,12| C,12| C,12| C12| C12| 
B,12| D12| B,12| A,12| G,12|
G,12| A,12| G,12| G,12| G,12|]


\end{abc}
\index{Je sui povere de leesse}
\addcontentsline{toc}{subsection}{Je sui povere de leesse}
\begin{abc}[name=latex_basse_dance30]
X:30
I:linebreak $
T:Je sui povere de leesse
M:6/4
L:1/8
K:C clef=treble-8 octave=1 
C,12| C,12| E,12| E,12| G,12| 
F,12| E,12| D,12| C,12| C,12| 
G,12| F,12| C,12| D,12| C,12| 
C,12| C,12| E,12| G,12| A,12|
G,12| C12| B,12| C12| G,12| 
G,12| B,12| B,12| E,12| E,12| 
G,12| F,12| E,12| E,12| G,12| 
A,12| G,12| F,12| C,12| D,12|
C,12| C,12| C,12|]


\end{abc}
\index{Joieusement}
\addcontentsline{toc}{subsection}{Joieusement}
\begin{abc}[name=latex_basse_dance31]
X:31
I:linebreak $
T:Joieusement
M:6/4
L:1/8
K:C clef=treble-8 octave=1 
A,12| F,12| E,12| E,12| F,12| 
G,12| C,12| E,12| D,12| C,12| 
C,12| E,12| F,12| G,12| C12| 
B,12| A,12| A,12| G,12| C12|
G,12| C,12| E,12| D,12| C,12| 
C,12| D,12| E,12| D,12| C,12| 
G,12| G,12| F,12| E,12| E,12| 
C,12| D,12| C,12| C,12| C,12|]


\end{abc}
\index{Joieux de Brucelles, le}
\addcontentsline{toc}{subsection}{Le joieux de Brucelles}
\begin{abc}[name=latex_basse_dance32]
X:32
I:linebreak $
T:Le joieux de Brucelles
M:6/4
L:1/8
K:C clef=treble-8 octave=1 
D,12| F,12| A,12| G,12| C12| 
B,12| A,12| G,12| G,12| G,12| 
C12| A,12| D,12| E,12| D,12| 
D,12| G,12| F,12| E,12| E,12|
A,12| D12| G,12| D,12| F,12| 
E,12| D,12| D,12| F,12| A,12| 
D,12| E,12| D,12| D,6 x6| D,12|]


\end{abc}
\index{Languir en mille destresse}
\addcontentsline{toc}{subsection}{Languir en mille destresse}
\begin{abc}[name=latex_basse_dance33]
X:33
I:linebreak $
T:Languir en mille destresse
M:6/4
L:1/8
K:C clef=treble-8 octave=1 
C,12| E,12| G,12| A,12| G,12| 
E,12| C,12| C,12| G,12| G,12| 
A,12| C12| B,12| A,12| G,12| 
F,12| E,12| E,12| C,12| C,12|
E,12| E,12| G,12| A,12| G,12| 
G,12| C,12| E,12| G,12| A,12| 
G,12| F,12| E,12| D,12| C,12| 
C,12| C,12|


\end{abc}
\index{Lauro}
\addcontentsline{toc}{subsection}{Lauro}
\begin{abc}[name=latex_basse_dance34]
X:34
I:linebreak $
T:Lauro
T:Re di Spagna
C:Vatican, Cap. 283
N:trans. Al Cofrin
N:for dance by Ebreo (15th C)
K:F major clef=G-8
M:6/4
L:1/4
d3 d3 | A3 G3 | d2 c B A2 | G3 G3 | 
B3 c3 | d3 d3 | f3 _e3 | d3 ^f3 | g3 g3 ||
c3 d3 | c3 c3 | g3 g3 | f3 f3 | g3 c3 | B3 _e3 | d3 d3 | 
G3 A3 | c3 d3 | f3 e3 | d3 c3 | B3 A3 | G3 A3 | G6 |]


\end{abc}
\index{Lyron}
\addcontentsline{toc}{subsection}{Lyron}
\begin{abc}[name=latex_basse_dance35]
X:35
I:linebreak $
T:Lyron
M:6/4
L:1/8
K:C clef=treble-8 octave=1 
G,12| F,12| E,12| D,12| F,12| 
E,12| D,12| D,12| G,12| G,12| 
C12| D12| E12| D12| C12| 
B,12| B,12| A,12| F,12| E,12|
D,12| D,12| E,12| D,12| F,12| 
G,12| C12| B,12| A,12| B,12| 
C12| A,12| G,12| G,12| G,12|]


\end{abc}
\index{Maistresse}
\addcontentsline{toc}{subsection}{Maistresse}
\begin{abc}[name=latex_basse_dance36]
X:36
I:linebreak $
T:Maistresse
M:6/4
L:1/8
K:C clef=treble-8 octave=1 
D12| D12| C12| B,12| A,12| 
D,12| G,12| A,12| G,12| G,12| 
F,12| G,12| A,12| C12| G,12| 
A,12| G,12| G,12| D12| 
E12| D12| G,12| F,12| E,12|
D,12| D,12| A,12| B,12| A,12| 
G,12| C12| B,12| A,12| A,12| 
D,12| F,12| G,12| D12| 
B,12| A,12| G,12| G,12| G,12|]


\end{abc}
\index{M'amour}
\addcontentsline{toc}{subsection}{M'amour}
\begin{abc}[name=latex_basse_dance37]
X:37
I:linebreak $
T:M'amour
M:6/4
L:1/8
K:C clef=treble-8 octave=1 
F,12| D,12| E,12| D,12| A,12| 
D12| C12| B,12| A,12| B,12| A,12| 
A,12| B,12| C12| D12| D12|
C12| B,12| A,12| A,12| A,12| 
B,12| A,12| G,12| F,12| E,12| 
D,12| E,12| D,12| D,12| D,12|]


\end{abc}
\index{Marchon la dureau}
\addcontentsline{toc}{subsection}{Marchon la dureau}
\begin{abc}[name=latex_basse_dance38]
X:38
I:linebreak $
T:Marchon la dureau
M:6/4
L:1/8
K:C clef=treble-8 octave=1 
F,12| E,12| F,12| G,12|
A,12| A,12| D,12| F,12|
G,12| E,12| D,12| F,12|
E,12| F,12| G,12| A,12|
A,12| D,12| F,12| E,12|
D,12| A,12| B,12| C12|
B,12| A,12| A,12| B,12|
C12| B,12| A,12| A,12|
G,12| A,12| B,12| C12|
A,12| C12| B,12| A,12|
F,12| F,12| F,12|]


\end{abc}
\index{Margaritte, la}
\addcontentsline{toc}{subsection}{La Margaritte}
\begin{abc}[name=latex_basse_dance39]
X:39
I:linebreak $
T:La Margaritte
M:6/4
L:1/8
K:C clef=treble-8 octave=1 
C12| D12| B,12| B,12| G,12| F,12| 
E,12| E,12| G,12| A,12| C,12| E,12|
D,12| C,12| C,12| E,12| F,12| G,12| 
C12| G,12| G,12| C12| G,12| C,12|
E,12| D,12| C,12| C,12| E,12| D,12| 
E,12| D,12| C,12| G,12| F,12| D,12|
C,12| C,12| C,12|]


\end{abc}
\index{Mois de may, le}
\addcontentsline{toc}{subsection}{Le mois de may}
\begin{abc}[name=latex_basse_dance40]
X:40
I:linebreak $
T:Le mois de may
M:6/4
L:1/8
K:C clef=treble-8 octave=1 
C12| C12| D12| B,12| A,12| A,12| 
C12| E12| D12| C12| D12| C12|
A,12| B,12| G,12| C12| D12| 
E12| D12| C12| A,12| C12| 
B,12| G,12| C12| A,12| D12| 
C12| B,12| B,12| C12| B,12|
A,12| A,12| A,12|]


\end{abc}
\index{Mon Cousin, je me recommende}
\addcontentsline{toc}{subsection}{Mon Cousin, je me recommende}
\begin{abc}[name=latex_basse_dance41]
X:41
I:linebreak $
T:Mon Cousin, je me recommende
M:3/2
L:1/8
K:C clef=treble-8 octave=1 
G,12| F,12| D,12| E,12| F,12| 
E,12| F,12| G,12| F,12| E,12| 
D,12| E,12|]


\end{abc}
\index{Mon leal desire}
\addcontentsline{toc}{subsection}{Mon leal desire}
\begin{abc}[name=latex_basse_dance42]
X:42
I:linebreak $
T:Mon leal desire
M:6/4
L:1/8
K:Ddor clef=treble-8 octave=1 
F,12 | G,12 | A,12 | A,12 | D12 |
C12 | D12 | A,12 | C12 |
_B,12 | A,12 | A,12 | F,12 |
A,12 | G,12 | D,12 | E,12 |
E,12 | D,12 | D,12 | A,12 |
A,12 | B,12 | B,12 | D12 |
E12 | D12 | D12 | A,12 |
B,12 | A,12 | A,12 | E,12 |
F,12 | E,12 | D,12 | A,12 |
C12 | D12 | D12 | A,12 |
F,12 | D,12 | F,12 | E,12 |
E,12 | A,12 | B,12 | A,12 |
A,12 | F,12 | E,12 | D,12 |
E,12 | D,12 | D,12 | D,12 |]


\end{abc}
\index{Ma meiulx ammee}
\addcontentsline{toc}{subsection}{Ma meiulx ammee}
\begin{abc}[name=latex_basse_dance43]
X:43
I:linebreak $
T:Ma meiulx ammee
M:6/4
L:1/8
K:C clef=treble-8 octave=1 
G,12| D,12| F,12| A,12| G,12| 
G,12| C12| B,12| A,12| A,12| 
D,12| G,12| A,12| G,12| C12| 
D12| G,12| A,12| B,12| C12| D12| 
D12| B,12| A,12| G,12| A,12| 
B,12| A,12| G,12| G,12| G,12|]


\end{abc}
\index{Navaroise, la}
\addcontentsline{toc}{subsection}{La navaroise}
\begin{abc}[name=latex_basse_dance44]
X:44
I:linebreak $
T:La navaroise
M:6/4
L:1/8
K:C clef=treble-8 octave=1 
D12| D12| A,12| F,12| D,12| E,12| 
D,12| D,12| F,12| A,12| D12| D12|
C12| B,12| A,12| G,12| F,12| 
E,12| D,12| D,12| A,12| A,12| D12| 
D12| A,12| F,12| G,12| F,12|
D,12| E,12| D,12| D,12| D,12|]


\end{abc}
\index{Non pareille, la}
\addcontentsline{toc}{subsection}{La non pareille}
\begin{abc}[name=latex_basse_dance45]
X:45
I:linebreak $
T:La non pareille
M:6/4
L:1/8
K:C clef=treble-8 octave=1 
C12| C12| D12| D12| A,12| D12| 
C12| B,12| A,12| A,12| C12| B,12|
A,12| D,12| E,12| F,12| G,12| A,12| 
C12| A,12| B,12| C12| C12| F,12|
C,12| E,12| E,12| D,12| C,12| 
C,12| C,12|]


\end{abc}
\index{Orleans}
\addcontentsline{toc}{subsection}{Orleans}
\begin{abc}[name=latex_basse_dance46]
X:46
I:linebreak $
T:Orleans
M:3/4
L:1/8
K:Gmix clef=treble-8 octave=1 
G,12 | A,12 | G,12 | D,12 | A,12 | D12 | C12 | B,12 | A,12 | A,12 | 
w:B,T:~R b ss d d d d d ss r  
B,12 | C12 | A,12 | F,12 | D,12 | E,12 | D,12 | D,12 | F,12 | A,12 | 
w:r r b ss d d d r r r 
G,12 | B,12 | D12 | A,12 | F,12 | D,12 | E,12 | D,12 | A,12 | 
w:b ss d ss r r r b ss 
D12 | C12 | B,12 | C12 | A,12 | G,12 | G,12 | G,12 |]
w:d d d r r r b c


\end{abc}
\index{Passe rose}
\addcontentsline{toc}{subsection}{Passe rose}
\begin{abc}[name=latex_basse_dance47]
X:47
I:linebreak $
T:Passe rose
M:6/4
L:1/8
K:C clef=treble-8 octave=1 
A,12| G,12| B,12| D12| C12| 
B,12| D12| D12| B,12| A,12| 
A,12| A,12| B,12| C12| B,12| 
A,12| G,12| F,12| E,12| E,12|
F,12| G,12| A,12| G,12| F,12| 
E,12| D,12| D,12| D,12|]


\end{abc}
\index{Petit rouen, le}
\addcontentsline{toc}{subsection}{Le petit rouen}
\begin{abc}[name=latex_basse_dance48]
X:48
I:linebreak $
T:Le petit rouen
M:6/4
L:1/8
K:C clef=treble-8 octave=1 
C,12| E,12| A,12| G,12| E,12| 
D,12| C,12| C,12| G,12| A,12| 
G,12| C12| B,12| A,12| G,12| 
G,12| G,12| A,12| B,12| A,12|
G,12| F,12| E,12| E,12| C,12| 
C,12| G,12| G,12| F,12| E,12| 
D,12| D,12| G,12| E,12| C,12| 
F,12| E,12| D,12| C,12| C,12| C,12|]


\end{abc}
\index{Portingaloise, la}
\addcontentsline{toc}{subsection}{La portingaloise}
\begin{abc}[name=latex_basse_dance49]
X:49
I:linebreak $
T:La portingaloise
M:6/4
L:1/8
K:Ddor clef=treble-8 octave=1 
G,12 | A,12 | _B,12 | D12 | C12 |
B,12 | A,12 | A,12 | A,12 |
B,12 | A,12 | G,12 | F,12 |
D,12 | E,12 | D,12 | D,12 |
F,12 | E,12 | F,12 | G,12 |
D12 | C12 | B,12 | A,12 |
G,12 | D,12 | E,12 | D,12 |
D,12 | D,12 |]


\end{abc}
\index{Potevine, la}
\addcontentsline{toc}{subsection}{La potevine}
\begin{abc}[name=latex_basse_dance50]
X:50
I:linebreak $
T:La potevine
M:6/4
L:1/8
K:C clef=treble-8 octave=1 
D,12| E,12| D,12| F,12| G,12| 
A,12| C12| B,12| A,12| A,12| 
A,12| B,12| A,12| G,12| A,12| 
G,12| G,12| A,12| G,12| F,12|
E,12| E,12| F,12| G,12| A,12| 
A,12| A,12| B,12| C12| B,12| 
A,12| G,12| G,12| A,12| A,12| 
G,12| F,12| F,12| G,12| F,12|
A,12| A,12| G,12| G,12| G,12|]


\end{abc}
\index{Rochelle, la}
\addcontentsline{toc}{subsection}{La Rochelle}
\begin{abc}[name=latex_basse_dance51]
X:51
I:linebreak $
T:La Rochelle
M:6/2
L:1/8
K:C clef=treble-8 octave=1 
F,12| E,12| D,12| E,12| D,12| D,12|  D12| C12| D12| E12| D12| C12|
w:B:~R b ss d d d ss r r r b ss 
w:T:~R b ss d d d ss r r r b ss  
B,12| A,12| A,12| A,12| B,12| G,12| F,12| E,12| D,12| D,12| F,12| A,12|
w:d r r r b ss d d d ss r r w: d r r r b ss d d d r r r 
D12| C12| A,12| F,12| D,12| E,12| D,12| D,12| D,12|]
w:r b ss d r r r b c
w:b ss d ss r r r b c


\end{abc}
\index{Petit roysin, le}
\addcontentsline{toc}{subsection}{Le petit roysin}
\begin{abc}[name=latex_basse_dance52]
X:52
I:linebreak $
T:Le petit roysin
M:6/4
L:1/8
K:C clef=treble-8 octave=1 
D,12| F,12| A,12| G,12| D,12| 
F,12| E,12| D,12| D,12| F,12| G,12| 
A,12| A,12| G,12| F,12| E,12|
E,12| A,12| G,12| D,12| E,12| 
F,12| E,12| D,12| D,12| E,12| 
F,12| E,12| D,12| G,12| A,12| 
A,12| G,12| F,12| E,12| D,12|
C,12| D,12| F,12| E,12| D,12| D,12| D,12|]


\end{abc}
\index{Basse Danse du roy d'Espaingne, le}
\addcontentsline{toc}{subsection}{Le Basse Danse du roy d'Espaingne}
\begin{abc}[name=latex_basse_dance53]
X:53
I:linebreak $
T:Le Basse Danse du roy d'Espaingne
M:6/4
L:1/8
K:C clef=treble-8 octave=1 
F,12| F,12| E,12| G,12| A,12| A,12| 
D,12| E,12| D,12| D,12| A,12| 
A,12| C12| B,12| A,12| A,12| G,12| 
F,12| E,12| E,12| F,12| E,12| D,12| 
D,12| A,12| A,12| B,12| B,12| C12| 
C12| D12| D12| A,12| A,12| A,12| 
G,12| F,12| D,12| F,12| E,12| D,12| 
E,12| D,12| D,12| D,12|]


\end{abc}
\index{Sans fair de vous departe}
\addcontentsline{toc}{subsection}{Sans fair de vous departe}
\begin{abc}[name=latex_basse_dance54]
X:54
I:linebreak $
T:Sans fair de vous departe
M:6/4
L:1/8
K:C clef=treble-8 octave=1 
C,12| E,12| F,12| D,12| E,12| 
D,12| C,12| C,12| G,12| E,12| G,12| 
F,12| E,12| D,12| C,12| C,12|
G,12| A,12| B,12| C12| B,12| 
A,12| G,12| G,12| G,12| C12| 
G,12| A,12| E,12| F,12| E,12| 
E,12| A,12| B,12| C12| D12|
E12| D12| C12| G,12| G,12| 
A,12| G,12| F,12| E,12| D,12| 
D,12| G,12| E,12| C,12| F,12| 
E,12| D,12| C,12| C,12| C,12|]


\end{abc}
\index{Ma soverayne}
\addcontentsline{toc}{subsection}{Ma soverayne}
\begin{abc}[name=latex_basse_dance55]
X:55
I:linebreak $
T:Ma soverayne
M:6/4
L:1/8
K:C clef=treble-8 octave=1 
E,12| E,12| G,12| F,12| E,12| G,12| 
A,12| B,12| B,12| C12| B,12| A,12|
G,12| ^F,12| G,12| C,12| G,12| A,12| 
G,12| =F,12| E,12| D,12| E,12| D,12|
C,12| D,12| C,12| D,12| F,12| E,12| 
F,12| G,12| F,12| E,12| D,12| C,12|
D,12| C,12| C,12| C,12|]


\end{abc}
\index{Tantaine, la}
\addcontentsline{toc}{subsection}{La tantaine}
\begin{abc}[name=latex_basse_dance56]
X:56
I:linebreak $
T:La tantaine
M:6/4
L:1/8
K:C clef=treble-8 octave=1 
A,12 | G,12 | F,12 | E,12 | D,12 | E,12 | D,12 | D,12 | F,12 |
w:B,T:~R b ss d d d d d ss
A,12 | _B,12 | A,12 | A,12 | A,12 | B,12 | C12 | B,12 |
w:r r r b ss d d d 
A,12 | F,12 | E,12 | D,12 | D,12 | F,12 | G,12 | F,12 |
w:r r r b ss d ss r r 
D12 | C12 | B,12 | A,12 | G,12 | F,12 | E,12 | D,12 |
w:r r b ss d d d r 
E,12 | D,12 | D,12 | D,12 |]
w:r r b c


\end{abc}
\index{Theme A}
\addcontentsline{toc}{subsection}{Theme A}
\begin{abc}[name=latex_basse_dance57]
X:57
I:linebreak $
T:Theme A
C:Faugues, Missa la basse dance
M:6/8
L:1/8
K:C clef=treble-8 octave=1 
A,6| C6| A,4G,2| F,3 x2F,| G,6| A,4G,2| 
F,4E,2| D,3 x2D,| A,4G,2| F,4A,2| C4B,2| A,3 x2A,|
A,3 G,3| F,3 G,3| F,4E,2| D,2F, E,D,2| C,3 x2C,| 
G,4-G,G,| F,3 A,3| D,2F,2E,2| D,6|]


\end{abc}
\index{Theme B}
\addcontentsline{toc}{subsection}{Theme B}
\begin{abc}[name=latex_basse_dance58]
X:58
I:linebreak $
T:Theme B
C:Faugues, Missa la basse dance
M:6/8
L:1/8
K:C clef=treble-8 octave=1 
D,6| F,4D,2| C,4G,2| F,2D,2E,2| D,6| 
x6| A,4F,2| G,6| A,2F, A,G,2| F,3 x2F,| 
D4C2| D6| D,4E,2| D,3 x2D,| F,3 G,3| 
A,4-A,A,| C4B,2| A,3 x2A,| D3 C3| A,3 F,3|
G,6| A,6| D,2F,2E,2| 
D,6|]


\end{abc}
\index{Torin}
\addcontentsline{toc}{subsection}{Torin}
\begin{abc}[name=latex_basse_dance59]
X:59
I:linebreak $
T:Torin
M:6/4
L:1/8
K:C clef=treble-8 octave=1 
C,12| E,12| G,12| A,12| C,12| 
D,12| C,12| C,12| B,12| C12| G,12| 
A,12| B,12| A,12| G,12| G,12| A,12| 
B,12| C12| C12| G,12| F,12| E,12| 
E,12| F,12| D,12| E,12| C,12|
D,12| D,12| C,12| B,12| C12| 
D12| C12| B,12| A,12| G,12| 
G,12| C,12| G,12| D,12| C,12| 
D,12| C,12| C,12| C,12|]


\end{abc}
\index{Triste plaiser}
\addcontentsline{toc}{subsection}{Triste plaiser}
\begin{abc}[name=latex_basse_dance60]
X:60
I:linebreak $
T:Triste plaiser
M:6/4
L:1/8
K:C clef=treble-8 octave=1 
G,12| F,12| D,12| F,12| A,12| 
G,12| F,12| F,12| G,12| G,12| 
B,12| B,12| A,12| A,12| C12| D12|
C12| C12| B,12| A,12| G,12| B,12| 
G,12| F,12| D,12| F,12| G,12| G,12|
G,12| F,12| G,12| B,12| B,12| 
C12| D12| C12| C12| G,12| 
B,12| A,12| G,12| G,12| G,12|]


\end{abc}
\index{Ulises}
\addcontentsline{toc}{subsection}{Ulises}
\begin{abc}[name=latex_basse_dance61]
X:61
I:linebreak $
T:Ulises
M:6/4
L:1/8
K:C clef=treble-8 octave=1 
C12| D12| C12| E12|
D12| C12| A,12| F,12|
D,12| E,12| D,12| D,12|
F,12| G,12| A,12| A,12|
A,12| B,12| D12| C12|
A,12| F,12| D,12| E,12|
D,12| G,12| F,12| E,12|
D,12| F,12| E,12| D,12|
D,12| D,12|]


\end{abc}
\index{Une fois avant que morir}
\addcontentsline{toc}{subsection}{Une fois avant que morir}
\begin{abc}[name=latex_basse_dance62]
X:62
I:linebreak $
T:Une fois avant que morir
M:6/4
L:1/8
K:Ddor clef=treble-8 octave=1 
D,12 | A,12 | C12 | D12 | A,12 |
G,12 | F,12 | E,12 | D,12 |
D,12 | D12 | D12 | E12 |
D12 | C12 | B,12 | A,12 |
D,12 | A,12 | B,12 | F,12 |
E,12 | G,12 | A,12 | F,12 |
E,12 | D,12 | F,12 | G,12 |
A,12 | _B,12 | A,12 | G,12 |
F,12 | F,12 | D12 | C12 |
D12 | A,12 | G,12 | F,12 |
E,12 | D,12 | D,12 | D,12 |]


\end{abc}
\index{Venise}
\addcontentsline{toc}{subsection}{Venise}
\begin{abc}[name=latex_basse_dance63]
X:63
I:linebreak $
T:Venise
M:6/4
L:1/8
K:Ddor clef=treble-8 octave=1 
F,12 | F,12 | D,12 | E,12 | D,12 |
C12 | A,12 | F,12 | E,12 |
D,12 | D,12 | ^C12 | ^C12 |
D12 | D12 | A,12 | B,12 |
A,12 | A,12 | D12 | D12 |
C12 | B,12 | A,12 | G,12 |
F,12 | E,12 | D,12 | D,12 |
A,12 | D12 | C12 | A,12 |
F,12 | G,12 | F,12 | F,12 |
A,12 | A,12 | _B,12 | A,12 |
^C12 | D12 | A,12 | F,12 |
E,12 | D,12 | D,12 | A,12 |
C12 | A,12 | F,12 | E,12 |
E,12 | D,12 | E,12 | D,12 |
D,12 | D,12 |]


\end{abc}

\index{Verdelete}
\addcontentsline{toc}{subsection}{Verdelete}
\begin{abc}[name=latex_basse_dance64]
X:64
I:linebreak $
T:Verdelete
M:6/4
L:1/8
K:C clef=treble-8 octave=1 
D12| D12| C12| C12|
A,12| G,12| F,12| F,12|
G,12| A,12| D,12| F,12|
E,12| D,12| F,12| G,12|
A,12| D12| C12| B,12|
A,12| A,12| A,12| D12|
A,12| D,12| F,12| E,12|
D,12| D,12| E,12| F,12|
E,12| D,12| A,12| G,12|
F,12| E,12| D,12| E,12|
D,12| D,12| D,12|]
\end{abc}


\chapter{15th Century Italian Dances}

The primary sources for 15th Century Italian dance are manuscripts from the
mid- to late 15th century containing dances by (among others) the dancing
masters Domenico da Piacenza (c. 1400-1470) and his student Guglielmo Ebreo (c.
1420-1848) (also known as Giovanni Ambrosio after his conversion from Judaism
to Catholocism).

15th century Italian dance is somewhat unusual in that dances often change
between ``tempi'', which are marked in each dance. The various tempi are
transcribed as:

\begin{itemize}

\item Bassadanza: 6/4
\item Quadernaria: 4/4
\item Saltarello: 6/8 or occassionally 3/4
\item Piva: 2/4 or 6/8

\end{itemize}

As a rough guide for tempo, keeping a constant tempo of approximately quarter
note = 120 (or dotted quarter = 120 for 6/8 piva sections) regardless of the
various tempi should work for many of the dances.

(See {\em Joy and Jealousy} by Vivian Stephens and Monica Cellio for additional
information; it is available online at
\url{http://sca.uwaterloo.ca/~praetzel/Joy-J-book/}). 


\clearpage
\index{Amoroso}
\addcontentsline{toc}{subsection}{Amoroso}
\begin{abc}[name=latex_15italian1]
X:1
I:linebreak $
T:Amoroso
C:Giovanni Ambrosio (Guglielmo Ebreo da Pesaro), c. 1475 (PnA)
N:Transcribed by Monica Cellio (She'erah bat Shlomo). Matches Pile 2018; Joy & Jealousy.
P:Play two times through
M:C
L:1/8
K:D dorian clef=G-8
P:A (3x)
"^Drone:D/A;Piva"DEFE D2EF | GAGF E2D2 | DEFE D2AG | FDE2 "^       (3)" D4 ::
P:B
A2A2 G2c2 | A2A2 G2AB |
c2ed cAB2 |
M:2/4
A4 ::
M:C
P:C
ddcB/A/ A2AB | c3B/A/ G2AB | c2ed cAB2 |
M:2/4
A4 ::
M:C
P:D
A2A2 G2c2 | A2A2 G2G2 | A2A2 GEF2 | E2E2 E2F2 | FEFG AG/F/E2 | E2E2 E/E/E/E/F3/G/ |
A2A2 A2AG | F3D F2E2 |
M:2/4
D4 :|


\end{abc}
\index{Anello}
\addcontentsline{toc}{subsection}{Anello}
\begin{abc}[name=latex_15italian2]
X:2
I:linebreak $
T:Anello
C:Domenico da Piacenza, c. 1425 (PnD)
N:Transcribed by Monica Cellio (She'erah bat Shlomo). Matches Pile 2018; Joy & Jealousy.
P:Play two times through
M:C
L:1/8
M:C
K:F major clef=G-8
P:A (3x)
"^Drone:F/C;Quadernaria"c2de f2f2 | ecde c2c2 "^(3)":| c2de f2f2 | cBAG F2F2 |
P:B
A/B/cz2 A/B/cz2 | A/B/cz2 def2 |
f2cB AGF2 ::
P:C
A2AG F2FB | AFGA F2F2 ::
M:2/4
P:D
f2 f2 | e2 e2 | df ed | c2 c2 :|
M:C
P:E
A/B/cz2 A/B/cz2 | A/B/cz2 f/e/cde | c2c2 FAG2 | F8 |]


\end{abc}
\index{Belfiore}
\addcontentsline{toc}{subsection}{Belfiore}
\begin{abc}[name=latex_15italian3]
X:3
T:Belfiore
C:Domenico da Piacenza, c. 1425 (PnD)
P:Play three times through
M:C
L:1/8
K:G major clef=G-8
P:A (3x)
"^Drone:G/D;Quadernaria"G3A B2c2 | B2dc B2AG | G3A B2c2 | B2cB A2"^      (3)"G2 ::\
P:B (3x)
d2d3/e/ d3/c/B2 "^(3)"::
M:3/2
P:C
d2 B2 d2 B2 d2 B2 |:\
M:C
P:D (3x)
d2d3/e/ d3/c/"^       (3)"B2 ::\
M:2/4
P:E (3x)
G2A>B | cz z2 | G2 A>B | cz zd | c3B | AG z2  "^(3)":|
P:F
M:C
d2 B>c d4 ||\
P:G
M:2/4
G2A>B | cz z2 | G2 A>B | cz zd | cB A2 | cB A2 | (G4 | G4 ) |]


\end{abc}
\index{Belreguardo}
\addcontentsline{toc}{subsection}{Belreguardo}
\begin{abc}[name=latex_15italian4]
X:4
T:Belreguardo
C:Domenico da Piacenza, c. 1425 (PnD)
N:Transcribed by Monica Cellio (She'erah bat Shlomo).
N:Matches Pennsic Pile 46; transposed down a 4th from Joy & Jealousy.
M:6/8
L:1/8
K:Bb lyd
P:A (2x or 3x)
"^Saltarello"
|:B2B AG2 | F2F GA2 | B2f dc2 | B3 B3 :| \
B2B AG2 | F2F GA2 | d2c BA2 | G3 G2G/2A/2 :|
M:6/4
P:B
K:clef=G-8
"^Bassadanza"B6 B6 | B6 B6 | c6 c6 | c6 c6 | d4 Bc d4 Bc | d4 Bc d2 d4 |\
c6 c6 | c4 AB c4 AB | c4 AB c2 c4 | B6 B6 |
B6 B6 |:\
P:C
d3efg f2e4 | d3efg f2e4 | d6 d6 | d6 d6 :|\
P:D
c6 c6 | c4 e2 d2 c4 | B6 B6 | B6 B6 | B6 B6 |]


\end{abc}
\index{Chirintana}
\addcontentsline{toc}{subsection}{Chirintana}
\begin{abc}[name=latex_15italian5]
X:5
T:Chirintana
C:Al Cofrin
P:AABB; repeat CCDD until done
M:C
L:1/8
K:E dorian
P:A
"^Drone:E/B;Quadernaria "B2AG F2E2 | F2FG DEF2 |\
M:2/4
G2 E2 |\
M:C
B2AG F2E2 | F2FG DEF2 |\
M:2/4
E2 E2 ::
M:C
P:B
e2dc B2A=c | B2B2 =cBA2 |\
M:2/4
B2 B2 |\
M:C
e2dc B2A=c | B2B2 AGFG |\
M:2/4
E2 E2 :|
M:6/8
"^Rhythm Interlude - Pivas"E2E EEE | E2E EEE | E2E EEE | E2z z3 |:
K:G major
P:C
"^Pivas ad nauseum"SE3 EFG | A3 ABc | BAG AGF | E3 E2D |\
E3 EFG | A3 ABc | BAG AGF | E3 E3 ::
P:D
G2G EG2 | B2B GB2 | B2A GAB | A3 AGF |\
G2G EG2 | B2B GB2 | BAG AGF | E3 "^D.S."E3 :|


\end{abc}
\index{Chirintana}
\addcontentsline{toc}{subsection}{Chirintana}
\begin{abc}[name=latex_15italian6]
X:6
T:Chirintana
T:T'Andernaken / Laet Ons Mit Hartzen
C:Emma Badowski, based on anonymous 15th C. Dutch melodies
P:AABB; repeat C until done
M:C|
L:1/4
K:D dorian
P:A
"^Drone:D/A;Quadernaria"A_B A/G/F/A/ | dc/_B/ A/F/G | AA cd | AF/G/ AG/F/ | E/D/E DD ::
P:B
FF cc/_B/ | AG FF/G/ | AA FG | AD/A/ GF |  [1 E/D/E DD :|]  [2 E/D/E DD |
M:6/8
P:C
A/ |:"^Piva"FE/ DC/ | FE/ DA/ | AB/ cA/ | cB/ AA/ | FE/ DC/ | FE/ DF/ | FF/ GF/ | E3/ DE/ |
G/F/E/ ED/ | E3/- EF/ | FF/ GF/ | E3/ DE/ | G/F/E/ ED/ | E3/- EA/ | FE/ DC/ |
FE/ DA/ | AB/ cA/ | cB/ AA/ | FE/ DC/ | FE/ DF/ | FF/ GF/ |  [1 E3/ DA/ :|]  [2 E3/ D3/ :|


\end{abc}
\index{Colonesse}
\addcontentsline{toc}{subsection}{Colonesse}
\begin{abc}[name=latex_15italian7]
X:7
T:Colonesse
P:Play two times through
C:Guglielmo Ebreo da Pesaro, 1463 (PnG)
N:Transcribed by Monica Cellio (She'erah bat Shlomo). Matches Pennsic Pile 46; Joy & Jealousy.
M:6/8
L:1/8
K:F major clef=G-8
P:A
"^Drone:F/C;Saltarello" FFF GGG | AAA BBB | AAA GGG |  [1-3 FFF FFF :|]  [4 FFc AGG | FFF FFF ::
M:6/4
P:B (3x)
"^Bassadanza" f6 f3 e/f/ g/f/e/d/ |  c2 c2 c2 c6 | d2 d2 d2 d2 d2 z2 | d2 d2 f2 e2 d2 d2 | c2 c2 c2 c2 c2 c2" (3)"::
M:2/4
P:C
"^Piva" cd ed | fd BB | GA B2 :|\
M:C
P:D
"^Quadernaria" f3 g f4 | dcde f3e | d>Bcc B4 |]


\end{abc}
\index{Figlia di Guielmina}
\addcontentsline{toc}{subsection}{Figlia di Guielmina}
\begin{abc}[name=latex_15italian8]
X:8
T:Figlia di Guielmina
C:Domenico da Piacenza, c. 1425 (PnD)
N:Transcribed by Al Cofrin (Albrect / Avatar of Catsprey). Matches Pennsic Pile 46
P:Intro:A; AABCDE x 2
M:C
K:Gmaj
P:A
"^Drone:D/A;Quadernaria"G2BB ccAA | GGdd dddc | A2G2 d2e2 | c2dG BBcc | AAG2 G4 :|
P:B
M:6/4
"^Bassadanza"A4 D2=F2 E4 | D12 || \
P:C
D6D6 | E4D2C6 | G6G6 |  A6D6 | A4 A2A4A2 | G4A2 c4c2 | d4 d2d4d2 | c4B2A6 ||
P:D
M:C
"^Quadernaria"eece eedc | eece eedc | dddd cBAd | cBAA GFEG | AAGF E2z2 |
P:E
M:6/8
"^Piva"EEE E2E | A2A F2D | EEE D2D | E3 A2B | c2A B2c | A2A B2B  | EEE E2E | A2A F2D | EEE D3 | -D3-D2z |]


\end{abc}
\index{Gelosia}
\addcontentsline{toc}{subsection}{Gelosia}
\begin{abc}[name=latex_15italian9]
X:9
I:linebreak $
T:Gelosia
C:Domenico da Piacenza, c. 1425 (PnD)
N:Transcribed by Monica Cellio (She'erah bat Shlomo). Matches Pile 2018; Joy & Jealousy.
P:Play three times through
L:1/8
M:C
K:G dorian clef=G-8
P:A (3x)
"^Drone:G/D;Quadernaria"BABc d2d2 | fe/f/dc BA/B/GA "^(3)":|
P:B
BABc d2d2 | cB/c/BA G2G2 |:
P:C
gfed gfed | gfef d2d2 |
BABc d2d2 :|
P:D
cBcA G2F2 |:
P:E
BABc d2d2 | f2fe d2d2 ::
P:F (3x)
g2fe d2"^      (3)"d2 ::
M:2/4
P:G
cB/c/ AG | BA/B/ AG | F2 F2 :|


\end{abc}
\index{Gratiosa}
\addcontentsline{toc}{subsection}{Gratiosa}
\begin{abc}[name=latex_15italian10]
X:10
I:linebreak $
T:Gratiosa
C:Guglielmo Ebreo da Pesaro, 1463 (PnG)
N:Transcribed by Monica Cellio (She'erah bat Shlomo). Matches Joy & Jealousy.
M:C
L:1/8
K:C major clef=G-8
P:A
"^Drone:G/D;Quadernaria"G2Bc d4 | e2Bc d4 | G2Bc d3c | B3/G/A2 G4 ::
P:B
e2dc B4 | e2dc B4 |
G2Bc d3c | B3/G/A2 G4 ::
M:6/4
P:C
"^Bassadanza" G12 | A12 | B12 | A12 | G12 | G12 ::
M:2/4
P:D
"^Piva"GABA | cBGG | FE G2 :|
M:C
d3e d4 | B3/A/Bc d3c | B3/G/ AA G4 |]


\end{abc}
\index{Ingrata}
\addcontentsline{toc}{subsection}{Ingrata}
\begin{abc}[name=latex_15italian11]
X:11
T:Ingrata
C:Domenico da Piacenza, c. 1425 (PnD)
N:Transcribed by Monica Cellio (She'erah bat Shlomo). Matches Joy & Jealousy.
M:6/8
L:1/8
K:F major clef=G-8
P:A
"^Drone:F/C;Saltarello"F2G AB2 | c2c de2 | f2e dc2 | B2B AG2 | F3 F3 :| \
M:C
P:B
"^Quadernaria"f2fg f2f2 | ed/c/ de c2c2 |
M:2/4
AA BB || \
M:6/8
P:C
"^Saltarello"c2A GF2 | E2A AB2 | c2A GF2 | E2A AB2 || \
M:6/4
P:D
"^Bassadanza"c6 c6 | f6 f6 | c6 B6 | A6 A6 |
B4A4G4 | F6 F6 |:\
P:E
A6 B6 | c6 c4A2 | B4A4G4 | F6 F6 :| \
P:F
f6 f6 | f6 f6 |:
M:6/8
P:G (3x)
"^Piva"A2G A2B | c2d c2B | A3 A3 "^(3)":| \
P:H
A2A GF2 | E3 E3 | A2G A2B | \
c3 c3 | B2A B2G | F3 F3 |]


\end{abc}
\index{Jupiter (Giove)}
\addcontentsline{toc}{subsection}{Jupiter (Giove)}
\begin{abc}[name=latex_15italian12]
X:12
I:linebreak $
T:Jupiter (Giove)
C:Domenico da Piacenza, c. 1425 (PnD)
N:Transcribed by Monica Cellio (She'erah bat Shlomo). Matches Joy & Jealousy.
M:C
L:1/8
K:Cmaj clef=G-8
P:A
"^Drone:C/G;Quadernaria" cccc GGGG | AABB cccz | cccc GGGG |
M:6/4
"^Bassadanza"A4A2 B4B2 | c6 c6 ::
P:B
_B6 B6 | c6 c6 | d6 d6 | d4f2 _e2d4 |
c6 c6 ::
M:6/8
P:C (3x)
"^Piva"c2B c2d | B2c e2f | d3 "^        (3)"d3 ::
P:D
"^Saltarello"c3 c3 | _B3 _e3 | d2e dc2 |
B3 e2d |
M:6/4
"^Bassadanza"c4B2 G2_A4 | G6 G6 :|
P:D
c12 | e12 |]


\end{abc}
\index{Legiadra}
\addcontentsline{toc}{subsection}{Legiadra}
\begin{abc}[name=latex_15italian13]
X:13
I:linebreak $
T:Legiadra
C:Guglielmo Ebreo da Pesaro, 1463 (PnG)
N:Transcribed by Monica Cellio (She'erah bat Shlomo). Matches Joy & Jealousy.
M:6/8
L:1/8
K:F major clef=G-8
P:A
"^Drone:F/C;Saltarello"ccc ddd | eee ggg | ddd eee | ddd ccc | ddd eee | fff eee |
ddd ccc | g/e/dd ccc |
ccc ccc |:
M:6/4
P:B
"^Bassadanza"f2f2f2 f2f2de | f2f2f2 f3efd | c2c2c2 c2c2c2 | d2d2d2 d2d2z2 | d2d2fe d2d2c2 | c2c2c2 c2c2AB |
c2c2c2 c2c2c2 ::
M:C
P:C
"^Quadernaria"g3a g2g2 | a2gf e2e2 | ff/f/f2 ede2 ||
M:6/4
"^Bassadanza"e2e2e2 e2e2e2 | d2d2d2 d2d2d2 | c2c2c2 c2c2AB | cccccc ::
M:6/8
P:D
"^Piva"cde dfd | BBA GB2 :|
M:C
P:E
"^Quadernaria"c3d c3B | AFGG FF F2 |]


\end{abc}
\index{Leoncello}
\addcontentsline{toc}{subsection}{Leoncello}
\begin{abc}[name=latex_15italian14]
X:14
T:Leoncello
P:Ax5 BB CC D E F
C:Domenico da Piacenza, c. 1425 (PnD)
N:Transcribed by Monica Cellio (She'erah bat Shlomo). Matches Pennsic Pile 46; Joy & Jealousy.
M:C
L:1/8
K:Gdor clef=G-8
P:A (5x)
"^Drone:F/C;Quadernaria"F2FG F2Fc | cBAG F2"^      (5)"F2 ::\
P:B
A2AG F2F2 | G2G2 A2A2 | FFGG A2A2 ::\
P:C
c3/2B/2 A3/2B/2 cBAG | FFGG A2A2 :|
P:D
M:6/4
"^Bassadanza"A6 A6 | G6 G6 | F6 F6 | A6 A6 | G6 G6 | F6 F6 ||\
P:E
F6 F6 | F6 F6 |
c6 c6 | c6 c6 | d6 d6 | c6 c6 | c6 c6 |\
M:C
P:F
"^Quadernaria"ABc2 z4 | ABc2 z4 |]


\end{abc}
\index{Marchesana}
\addcontentsline{toc}{subsection}{Marchesana}
\begin{abc}[name=latex_15italian15]
X:15
T:Marchesana
N:Transcribed by Monica Cellio (She'erah bat Shlomo). Matches Pennsic Pile 46; Joy & Jealousy.
C:Domenico da Piacenza, c. 1425 (PnD)
M:C
L:1/8
K:F major clef=G-8
P:A (3x)
"^Drone:F/C;Quadernaria"c2cd c2cd | e2d2 c2c2 "^(3)":| c2cd c2cB | AFGG F2F2 |:\
P:B
c2cd c2cc | c2c2 F2z2 :|
M:6/4
P:C
"^Bassadanza"  c6 c6 | d6 d6 | d6 d6 | c6 c6 | B6 B6 | A6 A6 | G6 G6 | G6 G6 | F6 F6 |
 c6 c6 | A6 A6 | G6 G6 | F6 F6 |:\
M:C
P:D
"^Quadernaria"c2z2 c2z2 | c3/B/A3/G/ F2F2 :| F2A2 G2F2 | F2z6 |]


\end{abc}
\index{Mercantia}
\addcontentsline{toc}{subsection}{Mercantia}
\begin{abc}[name=latex_15italian16]
X:16
T:Mercantia
C:Domenico da Piacenza, c. 1425 (PnD)
N:Transcribed by Monica Cellio (She'erah bat Shlomo). Matches Pennsic Pile 46; Joy & Jealousy.
K:F major
K:clef=G-8
M:6/8
L:1/8
P:A (3x)
"^Drone:F/C;Saltarello"G2G AB2 | c2c cBA | G2G AB2 | c2c c"^      (3)"z2 ::\
M:C
P:B
"^Quadernaria"A4 A4 | A2G2 A2B2 | c4 c4 :|
M:6/4
P:C
"^Bassadanza"B6 B6 | c6 c6 | d6 d6 | c6 c6  |:\
P:D
B6 B6 | F6 F6 | G6 G6 | F6 F6 :| \
M:3/4
P:E
c2c2Az || \
M:6/4
P:F
B6 F6 | c4B2 A2G4 ||
M:C
P:G
"^Quadernaria"F2FG A2B2 | BAG2 F2FF | \
M:3/4
P:H
c2c2Az | \
M:6/4
P:I
"^Bassadanza"B6 B6 | F6 F6 | G6 G6 | F6 F6 |]


\end{abc}
\index{Petit Riens}
\addcontentsline{toc}{subsection}{Petit Riens}
\begin{abc}[name=latex_15italian17]
X:17
I:linebreak $
T:Petit Riens
N:Transcribed by Monica Cellio (She'erah bat Shlomo). Matches Pile 2018; Joy & Jealousy.
C:Giovanni Ambrosio (Guglielmo Ebreo da Pesaro), c. 1475 (PnA)
P:Play three times through
M:6/8
L:1/8
K:G mix
P:A
"^Drone:G/D;Piva"GAB GAB | c2B A2G | d2d d2d | e2d/c/ B3 | GAB GAB | c2B A2G |
dd/d/d dcB | A3 G3 :|
P:B
c2c c2c | c2c B3 | d2d d2d | e2e d3 |
c2c c2c | c2c B3 | d2d d2d | e2e d3 |
c2c c2c | c2c B3 | d2d d2d | e2e d3 ||
g2g f2a | g2^f/e/ d3 |
g2g f2a | g2^f/e/ d3 |
g2g f2a | g2^f/e/ d3 ||
d2d d2c | BA2 G3 |
d2d d2c | BA2 G3 |
d2d d2c | BA2 G3 ||
c2c c2c | c2c B3 | d2d d2d | e2e d3 |
g2g f2a | g2^f/e/ d3 |
d2d d2c | BA2 G3 |]


\end{abc}
\index{Petite Rose}
\addcontentsline{toc}{subsection}{Petite Rose}
\begin{abc}[name=latex_15italian18]
X:18
T:Petite Rose
T:Spingardo
C:Joan Ambrosio Dalza, adapted by Monique Rio
M:6/8
L:1/8
K:Gmaj
P:A
e | "^Piva""G5"d2c B2A | G2D G2e | d2c B2A | G2d c2B | "F5"A2c B2A | "G5"G2A B2c | "F5"A2c B2A | "G5"G2G G2 ::
P:B
"G5"d | d3 z2d | d3 z2d | d2c B2A |  [1 G3 z2 :|]  [2 G2 |:\
P:C
"G5"GF2G |"D5"A2GE2F | "G5"G2 ::
P:D
"G5"GF2G | "D5"A2GA2c | "G5"B2 ::\
P:E
"D5"GF2G | "C5"E2GF2E | "D5"D2 :|\
P:F
z2 GF | "C5"E3"D5"F3 |"E5"G3d2c | "C5"BAG "D5"GAF | "G5"G3-G2 |]


\end{abc}
\index{Pizocara}
\addcontentsline{toc}{subsection}{Pizocara}
\begin{abc}[name=latex_15italian19]
X:19
I:linebreak $
N:Transcribed by Monica Cellio (She'erah bat Shlomo). Matches Pennsic Pile 46; Joy & Jealousy.
C:Domenico da Piacenza, c. 1425 (PnD)
T:Pizocara
L:1/8
M:6/8
K:F major clef=G-8
P:A (3x)
"^Piva"F2F c2c | c2c d2d | f2f e2e | d2d c2"^      (3)"z ::
P:B (4x)
B2B B2B | B2c d2z "^(4)":|
M:6/4
P:C
"^Bassadanza" f12 | d12  |:
P:D (3x)
_e6 e6 | f6 f6 | e2d4  B2d2c2 |
B6 B6 "^(3)":|
B12 | B12 ||
M:6/8
P:E
"^Saltarello"e2e2e2 | f2f2e2 | d2B dc2 | B2B2B2 | e2e2e2 | f2f2e2 | d2B dc2 | B2d2c2 |
B2_e dc2 | B3 B3 |:
P:F (3x)
"^Piva"G2F G2A | B3 B3 | d2c d2e | f3 f2e | d2B c2d | B3 B3 "^(3)":|


\end{abc}
\index{Prexonera}
\addcontentsline{toc}{subsection}{Prexonera}
\begin{abc}[name=latex_15italian20]
X:20
I:linebreak $
T:Prexonera
C:Domenico da Piacenza, c. 1425 (PnD)
N:Transcribed by Monica Cellio (She'erah bat Shlomo). Matches Joy & Jealousy.
K:Bb major
K:clef=G-8
M:6/4
L:1/8
P:A
"^Drone:F/C;Bassadanza"B6 B6 | c6 c6 |
M:3/4
d4e2 |
M:6/4
d2c4 B6 | f6 f6 ::
P:B
g4g2 f2e4 | d6 d4Bc | d6 z6 | g4g2 f2e4 |
d6 d4Bc | d6 z6 | B4F2 G2A4 | B6 B6 :|
M:C
P:C
"^Quadernaria"BBcc d2d2 | ffee d2d2 | BBcc d2d2 | ffee d2d2 | d2z2 d2z2 | B2c2 d2d2 |
d2z2 d2z2 |
M:2/4
B2 c2 ||
M:6/8
P:D
"^Saltarello"d2g fe2 | d2g fe2 | d2g fe2 ||
M:6/4
P:E
"^Bassadanza"B6 B6 | c6 c6 |
M:3/4
d4e2 |
M:6/4
d2c4 B6 | f6 f6 |]


\end{abc}
\index{Rostiboli Gioioso}
\addcontentsline{toc}{subsection}{Rostiboli Gioioso}
\begin{abc}[name=latex_15italian21]
X:21
T:Rostiboli Gioioso
C:Guglielmo Ebreo da Pesaro, 1463 (PnG)
N:Transcribed & arranged by Al Cofrin (Albrect (Avatar) of Catsprey). Matches Joy & Jealousy.
N:Matches Pile 2018
P:Play two times through
M:C
M:6/4
L:1/8
K:Fmaj clef=G-8
P:A
"^Bassadanza""F"c2c2c2 c2c2AB | c2c2c2 c2c2c2 | "C"G2G2G2 G2G2GA | "Gm"B2B2B2 "F"A2A2A2 | "C"G2G2G2 G2G2AB |
"F"c2c2c2 c2c2AB | c2c2c2 c2c2c2 | "Dm"F2F2F2 "Bb"F2F2F2 | "F"A2A2B2 "Csus4"A2G2G2 | "F"F2F2F2 F2F2F2 ::
P:B
"C"G2G2A2 G2G2F2 | G2G2A2 G2G2G2 | "Bb"B2B2B2 "F"A2A2A2 | "C"G2G2G2 G2G2F2 |
"C"G2G2A2 G2G2F2 | G2G2A2 G2G2G2 | "F"c2c2B2 "Csus4"A2G2G2 | "F"F2F2F2 F2F2F2 ::
P:C
M:6/8
"^Saltarello""F"ccc ccc | "Gm"BBB BBB | "F"ccc AAA | "C"GGG GGG | "F"ccc ccc | "Gm"BBB BBB | "F"FFF "Csus4"AGG | "F"F2F F3 ::
P:D
M:12/8
"^Piva"F |"C"G2 AG2 FG2 AG2G |"F"c2 c BA2"C"G3G2F | "C"G2 AG2 FG2 AG2G | "F"c2 B A"Csus4"G2"F"F3F2 :|


\end{abc}
\index{Sobria}
\addcontentsline{toc}{subsection}{Sobria}
\begin{abc}[name=latex_15italian22]
X:22
I:linebreak $
T:Sobria
C:Domenico da Piacenza, c. 1425 (PnD)
N:Transcribed by Monica Cellio (She'erah bat Shlomo). Matches Joy & Jealousy.
K:clef=G-8
K:G minor
M:3/4
L:1/8
P:A (3x)
"^Drone:G/D;Bassadanza"d2d2d2 | c4c2 | d2dcB2 | AA/B/ c/d/cB2 | A4A2 "^(3)":|
M:2/4
P:B
"^Piva"FEFG | AA A2 | G2 FE | D2 C2 ||
P:C
d2 d3/e/ | d2 d3/d/ |
cA BB | A2 A3/d/ | cA BB | A4 |:
M:6/4
P:D (3x)
"^Bassadanza"F6 F6 | G4z2 A4z2 | AABc dc B6 | A6 "^         (3)"A6 ::
M:C
P:E
"^Quadernaria"d2d3/e/ d2d3/d/ | cA BB A2A3/d/ | cA BB A4 |
M:6/4
"^Bassadanza"G12 | A12 | c8z2Bc ||
M:C
"^Quadernaria"d2cA B2A2 |
M:2/4
A2 A2 :|
M:3/4
P:F
"^Saltarello"D2D2E2 | F2F2F2 | A2BAG2 | F2F2F2 | D2D2E2 | F2F2F2 |
A2BAG2 | F2F2DC | G2FDE2 | D4D2 |:
M:2/4
P:G (3x)
"^Piva"F2 FE/D/ | C2 C z/C/ | D3/F/ E/D3/ | C2 C2 "^(3)":|


\end{abc}
\index{Spero}
\addcontentsline{toc}{subsection}{Spero}
\begin{abc}[name=latex_15italian23]
X:23
T:Spero
P:Play two times through
C:Guglielmo Ebreo da Pesaro, 1463 (PnG)
N:Transcribed by Monica Cellio (She'erah bat Shlomo). Matches Pennsic Pile 46; Joy & Jealousy.
M:6/8
L:1/8
K:F major clef=G-8
P:A
"^Drone:F/C;Piva"F2F A2B | c2c c2f | e2d c2B | A2A Az2 ::\
P:B
B3 B2A/G/ | F2F z2c | \
B2A G3 | F2F z3 :|
M:C
P:C
"^Quadernaria"cccc FFFz | GGGG FFFz || \
M:6/8
P:D
"^Saltarello"f3 f3 | c3 c3 | F3 F3 | c3 B3/A/G | F3 c3/A/G | F3 F3 ||
M:6/4
P:E
"^Bassadanza" F6 F6 | G6 G6 | A6 A6 | B6 B6 | B6 B6 | A6 A6 | G6 G6 | F6 F6 ||
M:6/8
P:F
"^Piva"F2G A2A | B2c A2A | G2G F2z | F2G A2A | B2c A2A | G2G F2A/B/ |
c2c z3 | dc2 cB2 | A2A G2G | F2F z2c | A2A G2G | F2F F3 |]


\end{abc}
\index{Tesara}
\addcontentsline{toc}{subsection}{Tesara}
\begin{abc}[name=latex_15italian24]
X:24
T:Tesara
N:Transcribed by Monica Cellio (She'erah bat Shlomo). Matches Pennsic Pile 46; Joy & Jealousy.
C:Domenico da Piacenza, c. 1425 (PnD)
M:6/8
L:1/8
K:G mix clef=G-8
P:A (3x)
"^Drone:G/D;Saltarello"d3 c3/B/A | G3 G3/F/E | D3 "^         (3)"D3 ::\
P:B
"^Piva"D2D G2D | G2A B2A | B2c d2d | B2G F2F |
E2E D2D | A3 G3/F/E | D3 z2c | d3 z2c | d3 z2d | e2e c2B | d3 d3 ::
P:C
G2G G2E | D3 D3/E/F | G3 G3 "^(4)":| d3 c3/B/A | G3 F3/D/E | D3 D2D | E2D E2F | G3 G3 |:
G2G G2E | D3 D3/E/F | G3 G3 "^(4)":| d3 c3/B/A | G3 F3/D/E | D3 D2D | E2D E2F | G3 G3 |:
P:D
"^Saltarello"G2G2A2 | B2B2d2 | B3 c3/2B/2A | G3 F3/2D/2E | D2E2D2 | D3/2E/2F G3 ::
P:E (4x)
"^Piva"d2c BA2 | G2G G2G | D3 E3 | D3 D3 "^(4)":| \
P:F
F2D G2D | G2A B2A | B2A B2c | d3 z2d | e2e c2c |:
P:G (4x)
"^Saltarello"e3 A3 | A3 G2d | e3 A3 | A3 G2d "^(4)":| \
P:H
d3 z2c | d3 z2d | e3 e3/2c/2c | d3 d3 | G3 F3/2D/2E | D6 |]


\end{abc}
\index{Verçepe}
\addcontentsline{toc}{subsection}{Verçepe}
\begin{abc}[name=latex_15italian25]
X:25
I:linebreak $
T:Verçepe
C:Domenico da Piacenza, c. 1425 (PnD)
N:Transcribed by Monica Cellio (She'erah bat Shlomo). Matches Pennsic Pile 46; Joy & Jealousy.
M:6/8
L:1/8
K:D dorian clef=G-8
P:A (3x)
"^Drone:D/A;Saltarello"d2d B2B | c2c A2A | d2d cB2 | A3 "^        (3)"A3 ::
M:6/4
P:B
"^Bassadanza"D6 D6 | F6 F6 | A6 A4A2 |
M:3/4
G2F4 |
M:6/4
E6 E6 | F4G2 F2E4 | D6 D6 ::
M:C
P:C
"^Quadernaria"d2dd d3d | AABc d2z2 :|
M:2/4
f2 e2 |
M:6/8
P:D
"^Saltarello"d2d d2d | c2d cB2 | A3 A3 | G3 D3 | F2E D3 |:
M:6/4
P:E
"^Bassadanza"d6 d6 | c4d2 c2B4 | A6 A6 |
M:6/8
"^Saltarello"D2D F2F | A2A G2G | D3 F3 | E3 D3 ::
M:C
P:F
"^Quadernaria"dd zd d3d | DDFF D4 :|


\end{abc}
\index{Vita di Cholino}
\addcontentsline{toc}{subsection}{Vita di Cholino}
\begin{abc}[name=latex_15italian26]
X:26
I:linebreak $
T:Vita di Cholino
N:Matches Pile 2018; Joy & Jealousy
C:modified by V. Stephens from "La Vida de Culin"
P:One dance:5 times through. Play:two dances.
M:C
L:1/8
K:C major
E2 |:"C"G2G2 G2G2 | "F"A4 z2A2 | "F"A2A3/B/ c2c3/B/ | "C"G4 z2G2 | "C"c2c2 c2B2 | "F"A4 A2A2 |
"C"G2G2 "Dm"F2F2 | "C"E4 z2C2 | "C"E4 "G"D4 | "C"C4 z2C2 | "C"E4 "G"D4 | "C"C4 z2C2 |
 [1-4 "C"E4 "Dm"F4 | "G"G6E2 :|]  [5 "C"E4 "G"D4 | "C"C8 |]


\end{abc}


\index{Voltate in ça Rosina}
\addcontentsline{toc}{subsection}{Voltate in ça Rosina}
\begin{abc}[name=latex_15italian27]
X:27
I:linebreak $
T:Voltate in ça Rosina
C:Giovanni Ambrosio (Guglielmo Ebreo da Pesaro), c. 1475 (PnA)
N:Transcribed by Monica Cellio (She'erah bat Shlomo). Matches Pile 2018; Joy & Jealousy.
P:Play two times through
M:C
L:1/8
K:A minor
P:A (3x)
"^Drone:A/E;Quadernaria"c2cd e2e2 | eded c2c2 | c2cd e2e2 | eded c2c2 | dddc B2B2 | c2B2 A4 |
dddc B2B2 | c2B2 "^        (3)"A4 ::
M:2/4
P:B (2x or 4x)
"^Piva"c2 cd | e2 e2 | ed ed | c2 c2 | dd dc | B2 B2 |
c2 B2 | A4 :|
\end{abc}


\chapter{Dances from the Gresley Manuscript}

The Gresley manuscript dates to the late 15th or early 16th century and was
re-discovered in Derbyshire, England. It contains choreography for 26 dances
and music for 13, with 8 of those having both music and the dance steps. We
have re-used other music from the manuscript for some of the dances missing
music and have included newly-composed music by Master Martin Bildner for the remainder, provided
under a Creative Commons Attribution-NonCommercial-ShareAlike license (see 
\url{https://creativecommons.org/licenses/by-nc-sa/3.0/}).
Reconstructions very, so always check the music with the dance master!

The dances are primarily transcribed in a lively 6/8 time; a tempo of dotted
quarter = 115-120 should work well.

\clearpage
\input{gresley.tex}

\chapter{Dances from the Inns of Court}

The dances in this section are from the Inns of Court: professional
associations for English barristers dating to the 15th century. There are
several known manuscripts dating from the mid-16th to mid-17th century
informally describing these dances, eight of which are believed to have been
performed in a fixed order at the beginning of revels at the Inns of Court. We
have preserved that order (for Quadran Pavane through Black Alman) to
facilitate dancing the entire suite, also known as ``The Old Measures''.

Tempos vary wildly, so check with the dancing master for their preference.
Reconstructions vary as well, so check for the desired roadmaps especially for
the more unusual ones such as Turkelone and Tinternell. We have included
suggested numbers of repeats when playing all 8 Old Measures as a suite, but
you may want to play more times through the dance if playing just one of the
dances.

\clearpage
\input{inns.tex}

\chapter{16th Century Italian Dances}

The major sources for 16th century Italian dances are the published books of
Fabritio Caroso (c. 1526-1605) and Cesare Negri (c. 1535-1605).

Many of the dances included in this collection are {\em cascarda}, a bouncy,
triple time kind of dance unique to Caroso. We have used a 3/4 time signature
for these but the dances should really be felt in 1, with a tempo of
approximately dotted half = 110-120.

The other dances (mostly {\em balletti}) in common time such as Bizzarria and
Lo Spagnoletto should work well with a tempo of half note = 100-110. Some of
these dances shift to 3/4 time partway through; let dotted half note in the 3/4
section = half note in the common time section. 

A few exceptions: Passo e Mezzo is written with doubled note values in cut
time, so use a tempo of whole note = 100-110. There are also a few dances we
have transcribed in 3/4 that are not cascarda such as Contrapasso and
Villanella. For Contrapasso, use a tempo of dotted half = 50-55. For
Villanella, always check with the dance master: it is sometimes danced (at the
same speed) to the music played slowly for 3 repeats and sometimes to the music
played twice as fast for 6 repeats. 

\clearpage
\index{Allegrezza d'Amore}
\addcontentsline{toc}{subsection}{Allegrezza d'Amore}
\begin{abc}[name=latex_16italian1]
X:1
T:Allegrezza d'Amore
C:Fabritio Caroso, il Ballarino, 1581
I:linebreak $
N:Transcribed & arranged by Monique Rio. Matches Pennsic Pile 46
M:3/4
L:1/8
K:C major
g2f2 |:
P:A
"C"e3de2 | c2d2e2 | "Bb"f4f2 | f2g2f2 | "C"e3de2 | "Am"c3dc2 | 
"G"d4d2 |  [1 "G"d2g2f2 :|]  [2 "G"d4"Am"c2 :| 
P:B
"G"d4d2 | d4"Am"c2 | "G"d4d2 | 
"G"d4"Am"c2 | "G"d4d2 | d2d2"C"c2 | "G"d3cB2 | "D"A3GA2 | "G"B4B2 | 
B2g2f2 | "C"e3dc2 | "G"d2c2d2 | "C"e4e2 | e2_B2c2 | "Bb"d4d2 | 
d4"F"c2 | "Bb"d4d2 | d2g2f2 | "C"e3dc2 | "G"d3cd2 | "C"e4e2 | 
e4 |] 


\end{abc}
\index{Alta Regina}
\addcontentsline{toc}{subsection}{Alta Regina}
\begin{abc}[name=latex_16italian2]
X:2
T:Alta Regina
C:Fabritio Caroso, il Ballarino, 1581
N:Transcribed & arranged by Aaron Elkiss. Matches Pennsic Pile 46
P:For Alta Regina:AB x 6 For Squilina:A x 21
M:3/4
L:1/8
K:F major
P:A
"C"G2F2G2 | G4"F"A2 | "Eb"B4B2 | B2G2A2 | \
B2A2B2 | "F"A2G2F2 | "C"G4G2 | G4G2 | \
G2F2G2 | G4"F"A2 | "Bb"B4B2 | B2c2B2 | 
"F"A2G2F2 | "C"G2F2G2 | "F"A4A2 | "F"F2G2A2 || \
P:B
"Bb"B6 | B2A2G2 | "F"A6 | A2G2F2 | \
"C"G4"Dm"F2 | "Bb"F4"C"G2 | "F"A4A2 | A6 :| 


\end{abc}
\index{Bassa Toscana}
\addcontentsline{toc}{subsection}{Bassa Toscana}
\begin{abc}[name=latex_16italian3]
X:3
I:linebreak $
C:Fabritio Caroso, il Ballarino, 1581
T:Bassa Toscana
N:Transcribed by Aaron Elkiss
M:C
L:1/8
M:C
K:F major
P:A (5x)
"Gm"G2G2 GBAG | "D"^F2F2 F2GA | "Gm"BGAB "F"cFGA | "Bb"B2B2 B4 | "F"AGAB cBAG | "F"AGFE FGAB | 
"F"AFGA "Eb"GBAG | "D"^F2F2 F4 | "Gm"G2GA GFEG | "C"GDEF ECDE | "F"FBAG F_EDC | "Bb"D4 D4 | 
"Bb"B2B2 B2"F"A2 | "Gm"GABG cBAG | "D"^F2G2 G2F2 | "G"G4 G4 "^(5)":| 
M:6/8
P:B
"Gm"G2G G2A/G/ | "D"^F2F F2G/A/ | "Bb"B2B "F"A2G/A/ | "Bb"B3 B3 | "F"AGA B2A | "F"A3 A2A | 
"Gm"B2A G2A/G/ | "D"^F2F F3 |:"Gm"G2G G3/E/F | "C"G3 G2D/E/ | "F"F2B/A/ G2F/_E/ | "Bb"D3 D3 | 
"Bb"B2B B2"F"A | "Gm"G2A/B/ B2A/G/ | "D"^F2G G2"D"F | "G"G3 G3 :| 


\end{abc}
\index{Bella Gioiosa}
\addcontentsline{toc}{subsection}{Bella Gioiosa}
\begin{abc}[name=latex_16italian4]
X:4
I:linebreak $
C:Fabritio Caroso, il Ballarino, 1581
T:Bella Gioiosa
N:Transcribed & arranged by Al Cofrin. Matches Pile 2018
P:AA BBB x 7 (or sometimes AA BBB AA x 6)
M:3/4
L:1/8
K:G major
P:A
d2 |:"G"d2c2d2 | G2A2B2 | "C"c2d2e2 | c2d2c2 | "G"B2c2B2 | G2A2G2 | 
"D"A6 | A4d2 | "G"d2c2A2 | G2A2B2 | "C"c2d2e2 | c2d2c2 | 
"G"B2A2G2 | "D"F2G2A2 | "G"G6 | G4 ::
dc | "G"B2A2G2 | "D"F2G2F2 | "G"G4G2 | G4 "^3":| 


\end{abc}
\index{Bianco Fiore, il}
\addcontentsline{toc}{subsection}{Il Bianco Fiore}
\begin{abc}[name=latex_16italian5]
X:5
T:Il Bianco Fiore
C:Cesare Negri, le Grazie d'Amore, 1602
N:Transcribed & arranged by Emma Badowski
M:6/4
L:1/8
K:F major
P:A
"F"F2E2F2 G2A2B2 | c6 A2B2c2 | "Bb"d4"F"c4"Gm"B4 | "F"A6 G2A4 | "Dm"F2E2F2 G2A2B2 | "F"c6 B2A2G2 | \
"Dm"F4"Csus4"F4"C"E4 | "F"F12 ::
P:B
"Dm"A4A4G2F2 | "C"E6 D2C4 | "F"c2B2A4G2F2 | "C"E6 D2C4 | \
"F"F2G2A2 G2F2E2 | "Bb"D6 C2D2E2 | "Dm"F4"Csus4"F4"C"E4 | "F"F12 ::
M:C
P:C
"C"c6B2 | "F"A4 "Dm"F4 | "Gm"G2A2 B2G2 | "F"A4 F4 | c6B2 | A4 G2F2 | \
"C"E4 F4 | "F"F8 :| 


\end{abc}
\index{Bizzarria d'Amore}
\addcontentsline{toc}{subsection}{Bizzarria d'Amore}
\begin{abc}[name=latex_16italian6]
X:6
T:Bizzarria d'Amore
C:Cesare Negri, le Grazie d'Amore, 1602
N:Transcribed & arranged by Monique Rio. Matches Pile 2018.
P:AA BB CC x 6
M:C
L:1/8
K:F major
P:A
c2 | "F"c2A2 B2c2 | "Bb"d3c B2d2 | "F"c2B2 "C"A2G2 | A4 z2A2 |\
"C"G2E2 "F"F2D2 | "C"C4 "F"c2BA | "Csus4"G2F2 "C"F2E2 | "F"F4 z2 ::
P:B
GA | "Gm"B4 "F"A4 | "C"G4 "Bb"d2cB | "F"A2"G"G2 "Dsus4"G2"D"^F2 | "G"G4 z2 ::\
P:C
GF | "C"E2C4GF | E2C4AB | "Am"c2"Bb"BA "C"G2G2 | "F"F4 z2 :|


\end{abc}
\index{Caccia d'Amore, la}
\addcontentsline{toc}{subsection}{La Caccia d'Amore}
\begin{abc}[name=latex_16italian7]
X:7
I:linebreak $
T:La Caccia d'Amore
N:Edited by Aaron Elkiss. Matches Pile 2018
C:Giovanni Giacomo Gastoldi, Balletti a cinque voci, 1591
M:C
L:1/8
K:D minor
"F"f2 | "Gm"d2"F"f2 "Bb"f2"C"e2 | "F"f4 f2f2 | "Gm"d2"F"f2 "Bb"f2"C"e2 | "F"f4 z2ff | "C"e2e2 "Dm"d2d2 | "A"^c2c2 z2"F"ff | 
"C"e2"Dm"d2 "Asus4"d2"A"^c2 | "D"d4 z2 ::"Dm"d2 | "C"e2"G"d2 "Am"e2"D"^f2 | "G"g4 g2"F"c2 | "Bb"d2"F"c2 "Gm"d2"C"e2 | "F"f4 z2ff | 
"C"e2ee "Dm"d2dd | "A"^c2c2 z2"F"ff | "C"e2"D"d2 "Gm"d2"A"^c2 | "D"d6 :| 


\end{abc}
\index{Candida Luna}
\addcontentsline{toc}{subsection}{Candida Luna}
\begin{abc}[name=latex_16italian8]
X:8
I:linebreak $
T:Candida Luna
C:Fabritio Caroso, il Ballarino, 1581
N:Transcribed & arranged by Aaron Elkiss. Matches Pile 2018
P:AA BB CC x 3
M:3/4
L:1/8
K:C major
P:A
"C"c2 | "G"B3AB2 | "Em"G3AB2 | "F"A3GA2 | A3B"C"c2 | "G"B4"F"A2 | A4"G"B2 | 
"C"c4c2 | c4 ::
P:B
"C"c2 | "G"B3AB2 | "C"G3AB2 | "F"A3GA2 | "Dm"F3GA2 | 
"C"G3FG2 | E3FG2 | "Dm"F3EF2 | "Bb"D3EF2 | "C"E3DE2 | E4"G"D2 | 
"Am"E3DE2 | E3DC2 | "G"D4"F"C2 | C4"G"D2 | "C"E4E2 | E4 ::
P:C
DE | 
"Bb"F3EF2 | D3EF2 | "Am"E3DE2 | E3DC2 | "G"D4"F"C2 | C4"G"D2 | 
"C"E4E2 | E4 :| 


\end{abc}
\index{Castellana, la}
\addcontentsline{toc}{subsection}{La Castellana}
\begin{abc}[name=latex_16italian9]
X:9
T:La Castellana
C:Fabritio Caroso, il Ballarino, 1581
N:Transcribed & arranged by Aaron Elkiss. Matches Pennsic Pile 46
P:AABBCC x 3
M:3/4
L:1/8
K:D minor
P:A
D2E2 |:"Dm"F3EF2 | "C"G3FG2 | "F"A4A2 | A3B"Dm"c2 | "Gm"B4"F"A2 | "C"G3FG2 | \
"F"A4A2 |  [1 "F"A2D2E2 :|]  [2 "F"A2E2F2 ::
P:B
"C"G4G2 | G2E2F2 | G4G2 | \
"C"G2A2G2 | "Dm"A4"C"G2 | "Dm"F4"G"D2 | "A"E4E2 | E4F2 ::
P:C
"C"G3FG2 | \
E3FG2 | "Dm"F3ED2 | "Em"E3^CD2 | "A"E4"Dm"D2 | "G"D4"A"^C2 | "D"D4D2 | \
 [1 "D"D4F2 :|]  [2 "D"D2 :| 


\end{abc}
\index{Chiara Stella}
\addcontentsline{toc}{subsection}{Chiara Stella}
\begin{abc}[name=latex_16italian10]
X:10
T:Chiara Stella
C:Fabritio Caroso, il Ballarino, 1581
N:Transcribed & arranged by Dennis Sherman. Matches Pile 2018
P:ABB x 4
M:3/4
L:1/8
K:D minor
P:A
"A"^C2D2E2 | E4E2 | E4E2 | "Dm"F4G2 | "F"A6 | "C"G2F2G2 | \
"F"A6 | A4G2 | "Dm"F3ED2 | "A"^C2D2E2 
| "Dm"D4D2 | D4D2 | \
"A"^C2D2E2 | E4E2 | E4E2 | "Dm"F4G2 | "F"A6 | "C"G2F2G2 | \
"F"A6 | A4G2 | "Dm"F3ED2 
| "A"^C2D2E2 | "Dm"D4F2 | F3ED2 | \
"A"E4^C2 | "G"=B,3^CD2 | "A"E4"Dm"F2 | F3ED2 | "A"E4^C2 | "G"=B,3^CD2 |\
"A"E4E2 | E6 |:
P:B
"F"A4A2 | A4"C"G2 | "F"A4A2 | A4"C"G2 | \
"Dm"F3ED2 | "A"^C2D2E2 | "Dm"D4D2 | D6 :| 


\end{abc}
\index{Chiaranzana}
\addcontentsline{toc}{subsection}{Chiaranzana}
\begin{abc}[name=latex_16italian11]
X:11
I:linebreak $
T:Chiaranzana
C:Fabritio Caroso, il Ballarino, 1581
N:Transcribed & arranged by Emma Badowski. Matches Pennsic Pile 46
M:6/4
L:1/8
K:A minor
"F"A4A2 B2c2d2 | "C"e2d2c2 g2f2e2 | "Dm"f2e2d2 f2e2d2 | "A"^c2B2A2 G2F2E2 | "F"F2G2A2 B2c2d2 | "C"e2g2f2 e2d2c2 | 
"G"B2f2e2 d2"A"d2^c2 | "D"d2e2f2 e2d2c2 | "G"B2GABc d2"A"d2^c2 | "D"d4d2 e2d2c2 | "G"B2GABc d2"A"d2^c2 | "D"d8d4 ::
M:3/4
"F"f4f2 | f3ed2 | "C"e3dc2 | g2f2e2 | "Dm"f3ed2 | f3ed2 | 
"A"^c3BA2 | G2F2E2 | "F"F3GA2 | "G"B3cd2 | "C"e3dc2 | e2d2c2 | 
"G"B3ed2- | "A"d2d2^c2 | "D"d3e^f2 | e2d2c2 | "G"B3ed2- | "A"d2d2^c2 | 
"D"d6 | e2d2c2 | "G"B3ed2- | "A"d2d2^c2 | "D"d4d2 | d6 :| 


\end{abc}
\index{Contentezza d'Amore}
\addcontentsline{toc}{subsection}{Contentezza d'Amore}
\begin{abc}[name=latex_16italian12]
X:12
T:Contentezza d'Amore
C:Cesare Negri, le Grazie d'Amore, 1602
N:Transcribed & arranged by Robert Smith. Matches Pennsic Pile 46
P:Ax5 B Cx3
M:C
L:1/8
K:F major
AB |:\
P:A
"F"c2c2 c2"Gm"B2 | "F"A2A2 A2GA | "Bb"B2B2 B2"F"A2 | "Gm"G2G2 G2AB | "F"c2c2 c2"Gm"B2 | "F"A3G A2GF |
"G"G2G2 G2"D"D2 | "G"G2G2 G2z2 | "F"A2AB A2"C"G2 | "F"A2A2 A2GA | "Gm"B2B2 B2"F"A2 | "G"G2G2 G2z2 |
"F"ABAG FEDC | "G"D4 G4 | "C"G2G2 G2"Dm"F2 | "C"G2GG cBAG | "F"ABAG FEDC | "G"G4 C2G2 |
"C"G2G2 G2"Dm"F2 | "C"G2G2 cBAG | "F"AF"C"GA "Bb"BAGF | "C"G2"F"F2 "Bb"F2"C"FE | "F"A2A2 A2"C"G2 |  [1-4 "F"A3A A2AB :|] \
 [5 "F"A3A A4 || 
M:3/4
P:B
"F"c4"Gm"B2 | "F"A2G2A2 | "Gm"B4"D"A2 | "Gm"G4G2 | G3AB2 | B3AG2 | \
G4"D"^F2 | "G"G4G2 | 
"F"c4"Gm"B2 | "F"A3GA2 | "Gm"B4"D"A2 | "Gm"G4G2 | \
G4F2 | "C"E3DC2 | "Gsus4"D6 | "C"E4E2 |:
P:C
"C"G4"Dm"F2 | "C"E3DC2 | \
"G"D2C2"G"D2 | "C"E4E2 | "F"c4"Gm"B2 | "F"A3GF2 | "C"G2F2G2 |  [1-2 "F"A4A2 :|] \
 [3 "F"A6 |] 


\end{abc}
\index{Conto Dell'Orco, il}
\addcontentsline{toc}{subsection}{Il Conto Dell'Orco}
\begin{abc}[name=latex_16italian13]
X:13
T:Il Conto Dell'Orco
C:Fabritio Caroso, il Ballarino, 1581
N:Transcribed & arranged by Dave Lankford. Matches Pennsic Pile 46
P:(AABB)x2 Cx2 or 3
M:C
L:1/8
K:C major
P:A
 EF | "C"G2"F"A2 "C"G2EF | "C"G2"F"A2 "C"G2EF | "C"GEFG "F"A3/G/FE | "G"DCCD "C"E2 ::
P:B
cd | "C"e2"F"f2 "C"e2cd | "C"e2"F"f2 "C"e2cd | \
"C"ecde "F"f3/e/dc | "G"BAAB "^Repeat AABB!""C"c2 ::
P:C
EF | "C"G2"F"A2 "C"G2EF | "C"G2"F"A2 "C"G2EF | "C"GEFG "F"A3/G/FE | "G"DCCD "C"E2 :| 


\end{abc}
\index{Contrapasso}
\addcontentsline{toc}{subsection}{Contrapasso}
\begin{abc}[name=latex_16italian14]
X:14
T:Contrapasso
C:Fabritio Caroso, Nobiltà di Dame, 1600
N:Transcribed & arranged by Monique Rio. Matches Pile 2018.
%%text for Contrapasso in Due & in Ruota:AAA BBB AA BBB
%%text for Contra Passo (Chigi):AA BBB AA BBB
%%text for Contrapasso Nuovo:AAA BBB AAA BBB
M:3/4
L:1/8
K:F major
P:A
DE | "F"F2F2"Csus4"G2 | "F"A2AGF_E | "Bb"D2F2"C"G2 | "F"A2A2DE | FGAF"C"GE | "F"FBAGF_E | "Bb"DE"Dm"F2"C"GE | "F"F2F2 ::
P:B
AB |"F"c2cFGA | "Bb"B2B2B2 | B2B3G | "F"A2A2AB | c2cFGA | "Bb"B2B2B2 | B2B3G | "F"A2A2z2 |
FGAFGA | "Bb"B2B2"F"A2 | "Eb"G2G2"Dm"F2 | "C"G2G2GB | "F"ABAGFE | "Bb"D2"C"EG"Dm"FA | "Bb"GF"Csus4"G2"C"E2 | "F"F2F2 :|

\end{abc}
\index{Dimostra, lo}
\addcontentsline{toc}{subsection}{Lo Dimostra}
\begin{abc}[name=latex_il_papa2]
X:2
I:linebreak $
T:Lo Dimostra
C:Nathan Kronenfeld
P:AAA BB C
M:6/4
L:1/8
K:C major
P:A
 |: "G"G3AB2 B3cd2 | "C"e3dc2 c3de2 | "D"d4A2 A3Bc2 | "D"d4A2 A3BA2 | "Em"G4E2 E6 | "Am"G3AB2 B3cB2 | 
"G"c3BA2 A3Bc2 | "C"B3AG2 G3AG2 | "C"G3FE2 "Dm"D4C2 | "G"C3DE2 "C"D3EF2 | "F"G3AB2 c4G2 | "F"A3GF2 "D"F3GA2 | 
"G"A3Bc2 d4A2 | "G"G3AB2 "F"d3cB2 | "C"B3AG2 A3BA2 | "G"G3FE2 "C"E3FG2 | "C"G3AB2 "^ (3)"c6 :: 
P:B
"F"G3FE2 E3FG2 | 
"G"A3GF2 F3GA2 | "G"B3AG2 "C"G3AB2 |  [1 "G"G3AB2 "C"c4A2 :|]  [2 "Am"G3AB2 c6 :| 
P:C
"G"c3Bc2 A3Bc2 | "F"B3AB2 G3AB2 | 
"G"A3GA2 "C"F3GA2 | G3AB2 c6 |] 


\end{abc}
\index{Fedelta}
\addcontentsline{toc}{subsection}{Fedelta}
\begin{abc}[name=latex_16italian15]
X:15
T:Fedelta
C:Fabritio Caroso, il Ballarino, 1581
N:Transcribed & arranged by Aaron Elkiss. Matches Pile 2018
P:AAB x 3
M:3/4
L:1/8
K:D major
P:A
"D"F2 |:F4F2 | F4F2 | "G"G4G2 | G4G2 | "D"F4F2 | "A"E4D2 | \
E4E2 | "A"E4"D"F2 
| "D"F4D2 | E4F2 | "G"G4G2 | "D"F4F2 | \
"A"E4"D"D2 | "A"C4E2 | "D"D4D2 | D4F2 :| 
P:B
"G"G4"D"F2 | "A"E4E2 | \
"D"F4F2 | F4F2 | "G"G4"D"F2 | "Asus4"E4"A"E2 | "D"D4D2 | D4 |] 


\end{abc}
\index{Fiamma d'Amore}
\addcontentsline{toc}{subsection}{Fiamma d'Amore}
\begin{abc}[name=latex_16italian16]
X:16
T:Fiamma d'Amore
C:Fabritio Caroso, il Ballarino, 1581
N:Transcribed & arranged by Al Cofrin. Matches Pennsic Pile 46
P:AA B x 4
M:3/4
L:1/8
K:D minor
P:A
 |:F2 | "Dm"F3EF2 | "C"G3FG2 | "F"A4A2 | A4"C"G2 | "Bb"F3ED2 | "A"E3DE2 | "D"D4D2 | D4 :|
P:B
F2 |"Dm"F3EF2 | "C"G3FG2 | "F"A4A2 | A4c2 | "Gm"B4A2 | "C"G3FG2 | "F"A4A2 | A4F2 |
"Dm"F3EF2 | "C"G3FG2 | "F"A4A2 | A4G2 | "Bb"F3ED2 | "A"E3DE2 | "D"D4D2 | D6 |
M:2/4
"Bb"F2 FG | "F"A4 | "Bb"F2 FG | "F"A4 |\
M:3/4
"F"A4"C"G2 | "Bb"F3ED2 | "A"E3DE2 | "D"D4D2 | D4 |]


\end{abc}
\index{Ballo del Fiore}
\index{Torche, Bransle de la}
\index{Bransle!Torche, de la}
\index{Fiore, Ballo del}
\addcontentsline{toc}{subsection}{Ballo del Fiore}
\begin{abc}[name=latex_16italian17]
X:17
T:Ballo del Fiore
T:Bransle de la Torche
C:Michael Praetorius, Terpsichore, 1612
P:Intro:A; one dance = (AB)x4
N:Edited by Aaron Elkiss. Matches Pile 2018
M:C
L:1/8
K:D dorian
P:A
"Dm"d3e f2f2 | "C"e4 e2e2 | "Dm"d3d d2d2 | "A"^c4 A4 | "Dm"d3e f2f2 | "C"e3f g2"Am"a2 | \
"Dm"f2ed "A"^cde2 | "D"d4 d4 || 
P:B
"F"a3g fgaf | "Em"g2ef geag | "Dm"f3e defg | "A"a4 a4 | \
"F"a3g fgaf | "C"g3f efge | "Dm"f2gf "A"ede2 | "D"d4 d4 :| 


\end{abc}
\index{Florido Giglio}
\addcontentsline{toc}{subsection}{Florido Giglio}
\begin{abc}[name=latex_16italian18]
X:18
T:Florido Giglio
C:Fabritio Caroso, il Ballarino, 1581
N:Transcribed & arranged by Aaron Elkiss. Matches Pennsic Pile 46
P:AABBCDD AABBCCDDx2 AABBCDD
M:3/4
L:1/8
K:G mixolydian
P:A
d2c2 | "G"B3AB2 | G2A2B2    | "F"c4c2  | c2d2c2 | "G"B2A2G2 | "D"A2G2A2 | "G"B6 | B2 ::
P:B
F2G2 | "F"A4A2  | A2A2B2    | c4c2     | c2d2c2 | "G"B2A2G2 | "D"A2G2A2 | "G"B6 | B2 ::
P:C (No repeat for 1st & 4th verses)
C2D2 | "C"E4G2  | "D"A2G2A2 | "G"B3AG2 | F2E2D2 | "C"E3^FG2 | "D"A2G2A2 | "G"B6 | B2 ::
P:D
F2G2 | "D"A6    | A2^F2G2   | "D"A6    | A3dc2  | "G"B2A2G2 | "D"A2G2A2 | "G"B6 | B2 :|


\end{abc}
\index{Fulgente Stella}
\addcontentsline{toc}{subsection}{Fulgente Stella}
\begin{abc}[name=latex_16italian19]
X:19
I:linebreak $
T:Fulgente Stella
C:Fabritio Caroso, il Ballarino, 1581
N:Transcribed & arranged by Aaron Elkiss. Matches Pennsic Pile 46
P:AABB x4
M:3/4
L:1/8
K:F major
P:A
 | "G5"G4A2 | "Gm"B3AG2 | "D"A4A2 | "A"A6 | "G5"G2A2B2 | "Gm"B3AG2 | 
"D"A4A2 | A6 ::
M:2/4
P:B
"Bb"B4 | B2 "F"c2 | "Bb"d2 d2 | d2 d2 | d2 "F"c2 | "Gm"B3/A/ G2 | 
"F"A4 | A2 "C"G2 | "F"A2 "Bb"B2 | "F"c2 A2 | "Gm"B2 A2 | "C"G2 "F"F2 | 
"C"E^F G2 | "Dsus4"A2 "D"^F2 | "G"G4 | G4 :| 


\end{abc}
\index{Furioso all'Italiana}
\addcontentsline{toc}{subsection}{Furioso all'Italiana}
\begin{abc}[name=latex_16italian20]
X:20
I:linebreak $
T:Furioso all'Italiana
C:Fabritio Caroso, Nobiltà di Dame, 1600
N:Transcribed & arranged by Al Cofrin. Matches Pennsic Pile 46
P:Ax10 Bx3 C Bx2 C B
M:C
L:1/8
K:G major
BA |:
P:A
"G"B2B2 c2A2 | B2B2 B2dc | BGAB GFEG | "D"A2A2 A2A2 | A2A2 G2F2 | "C"E2ED EFG2 | 
"Dsus4"A2G2 G2GF |  [1-9 "G"G2G2 G2BA :|] 
M:3/4
 [10 "G"G4G2 | "G"G4B2 ::
P:B
"G"B3AB2 | "Am"c3Bc2 | "G"B4B2 | B4d2 | 
"Am"c4B2 | "D"A4G2 | A4A2 | A4A2 | A3GA2 | "Em"G4F2 | 
"C"E4E2 | E3FG2 | "D"A4G2 | "C"G4"D"F2 | "G"G4G2 |  [1-2 G4B2 :|] 
 [3 "G"G6 :| 
M:3/2
P:C
"G"B4B4A2B2 | "C"c4c4c4 | c4c8 | "G"B4B4B4 | "Em"G4"F"A4"G"B4 | "C"c4c2B2A2G2 | 
F2E2"Dsus4"G4F4 | "G"G4G4z2"^To Bx2 C B"B2 |] 


\end{abc}
\index{Giunto m'ha Amore}
\addcontentsline{toc}{subsection}{Giunto m'ha Amore}
\begin{abc}[name=latex_16italian21]
X:21
T:Giunto m'ha Amore
N:Transcribed & arranged by Dave Lankford. Matches Pennsic Pile 46.
C:Fabritio Caroso, il Ballarino, 1581
P:AABBx5
M:3/4
L:1/8
K:D minor
P:A
"Dm"F2 |:F3EF2 | "C"G3FG2 | "F"A6 | A4G2 | "Dm"F3ED2 | "A"^C3DE2 | "Dm"D4D2 | D4F2 ::\
P:B
"Dm"F3EF2 | "C"G3FG2 | 
"F"A6 | A4c2 | c3Bc2 | "Gm"B3cB2 | "F"A6 | A4G2 |\
"Dm"F3ED2 | "Am"E3DE2 | "Dm"D4D2 |  [1 "Dm"D4F2 :|]  [2 "Dm"D4 :|


\end{abc}
\index{Gloria d'Amore}
\addcontentsline{toc}{subsection}{Gloria d'Amore}
\begin{abc}[name=latex_16italian22]
X:22
T:Gloria d'Amore
N:Transcribed & arranged by Dave Lankford. Matches Pennsic Pile 46.
C:Fabritio Caroso, il Ballarino, 1581
T:Cascarda
P:Play five times
M:3/4
L:1/8
K:D minor
"Gm"G4G2 | G3AG2 | "D"^F4F2 | ^F4GA | "Bb"B4B2 | "F"A4GA | "Bb"B4B2 | B6 | "F"A4GA | 
A3Bc2 | "Bb"B4B2 | B4A2 | "Gm"G3AG2 | "D"^F2G2F2 | "G"G4"Gm"B2 | "Gm"B3AG2 | "D"^F4D2 | 
D4E2 | ^F4F2 | ^F6 | "F"A3GA2 | A3Bc2 | "Bb"B4B2 | B4A2 | "Gm"G3AG2 | "D"^F2G2F2 | "G"G6 | G6 :|


\end{abc}
\index{Gracca Amorosa}
\addcontentsline{toc}{subsection}{Gracca Amorosa}
\begin{abc}[name=latex_16italian23]
X:23
T:Gracca Amorosa
C:Fabritio Caroso, il Ballarino, 1581
P:Play five times
M:6/8
L:1/8
K:F major
   "F"c3   c2B    | A3     A2c     | c2B       A2A       |     \
   "C"G3   G2C    | G3     G2F     | G3        G2F       |
   "Gm"G2A B2G    | "F"A3  A3      | c3        c2B       |     \
   A3      A2c    | c2B    A2A     | "C"G3     G2F       |
|:"Bb"F2F "C"G2G | "F"A2A "Eb"G2B | "F"A3/G/F "C"G3/F/G | [1  "F"A3 A2F :|]  [2  "F"A3 A3 :|]
\end{abc}

\index{Lucretia}
\addcontentsline{toc}{subsection}{Lucretia}
\begin{abc}[name=latex_il_papa_3]
X:3
T:Lucretia
P:AA B CC x5
M:6/4
L:1/8
C:Nathan Kronenfeld
K:C major
P:A
 |: "G"d4c2 B4AG | "D"A2B2A2 "G"G6 | "G"G4d2 d4cB | "C"c4e2 e4dc | "G"B4d2 d4cB | "D"A4G2 A6 :| 
P:B
"G"G6 "D"^F4A2 | "G"B2c2B2 d6 | "G"G4d2 d4cB | "C"c4e2 e4dc | "G"B4d2 d4cB | "D"A4G2 A6 |: 
P:C
"G"G6 "D"^F4A2 | "G"B2c2B2 d6 | "G"d4c2 B4AG | "D"A2B2A2 "G"G6 :| 


\end{abc}
\index{Maraviglia d'Amore}
\addcontentsline{toc}{subsection}{Maraviglia d'Amore}
\begin{abc}[name=latex_16italian24]
X:24
I:linebreak $
C:Fabritio Caroso, il Ballarino, 1581
T:Maraviglia d'Amore
N:Transcribed & arranged by Aaron Elkiss. Matches Pennsic Pile 46
P:ABBCC x 4
M:3/4
L:1/8
K:G major
P:A
"G"B2 | "D"A2B2"Em"G2 | "D"A4"G"B2 | "C"c4c2 | "C"c4"G"B2 | "F"A2B2A2 | "Em"G2F2"C"G2 | 
"D"A4A2 | A4"G"B2 | "D"A2B2"C"G2 | "F"A4"G"B2 | "C"c4c2 | c2d2"G"B2 | 
"D"A2B2"Em"G2 | "C"E2G2"D"F2 | "G"G4G2 | G2B2c2 |:
P:B
"G"d4d2 | d2e2"Am"c2 | 
"G"B4B2 | B2c2d2 | "Am"c4"G"B2 | "D"A2B2c2 | "G"B4B2 |  [1 "G"B2B2c2 :|] 
 [2 "G"B4B2 ::
P:C
"D"A4"C"G2 | "D"A4"G"B2 | "C"c4c2 | c2d2c2 | "G"B2A2G2 | 
"D"A2G2A2 | "G"B4B2 |  [1 "G"B4B2 :|]  [2 "G"B4 |] 


\end{abc}
\index{Ombrosa Valle}
\addcontentsline{toc}{subsection}{Ombrosa Valle}
\begin{abc}[name=latex_16italian25]
X:25
T:Ombrosa Valle
C:Fabritio Caroso, il Ballarino, 1581
N:Transcribed & arranged by Aaron Elkiss. Matches Pennsic Pile 46
P:AB x 7
M:C
L:1/8
K:C major
P:A
"C"g4 g2"Dm"f2 | "C"e4 cBcd | e2"G"d2 "Am"c2dc | "G"d4 d2d2 | d3c dcBA | B2G2 d2"F"c2 | 
"Bb"d2"C"e2 "Dm"f2"G"d2 | "C"e4 e4 | "C"g4 g2"Dm"f2 | "C"e4 cBcd | "C"e2"G"d2 "Am"c2dc | "G"d4 d2"F"c2 || 
P:B
"Bb"f2f2 "G"d2d2 | "C"e2e2 "Bb"d2"Dm"f2 | "C"e2e2 "G"d2d2 | "C"e4 e2"F"c2 | "F"c2c2 "G"d2df | "C"e2e2 "Bb"d2"Dm"f2 | \ 
"C"e2dc "G"B2AB |  [1-6 "C"c4 c2c2 :|]  [7 "C"c4 c4 :| 


\end{abc}
\index{Passo e Mezzo}
\addcontentsline{toc}{subsection}{Passo e Mezzo}
\begin{abc}[name=latex_16italian26]
X:26
T:Passo e Mezzo
C:Fabritio Caroso, il Ballarino, 1581
N:Transcribed & arranged by Dave Lankford. Matches Pennsic Pile 46.
%%text for Passo e Mezzo:11 times through
%%text for Dolce Amoroso Fuoco:5 times through
%%text for Ardente Sola:7 times through
M:C|
L:1/4
K:G dorian
"Gm"d2 d2 | dc BA | Bf ed | cB AG | "F"A2 A2 | AG AB | cd cB | AF GA |
"Gm"B2 d2 | dc BA | Bd cB | "D"AG ^FE | ^F2 F2 | ^F2 d2 | dc de | fe dc |
"Gm"d2 d2 | dc BA | Bf ed | cB AG | "F"A2 A2 | AG AB | "F"c2 "Gm"B2 | "F"AF "Dm"GA |
"Gm"Bd cB | "D"AG ^FE | ^F2 "G"G2 | "Am"A2 "D"A2 | "G"G2 G2 | G2 "D"^F2 | "G"G4- | G4 |]


\end{abc}
\index{Rose e Viole}
\addcontentsline{toc}{subsection}{Rose e Viole}
\begin{abc}[name=latex_16italian27]
X:27
I:linebreak $
T:Rose e Viole
N:Transcribed & arranged by Paul Butler. Matches Pennsic Pile 46
C:attrib. Antonio Casteliono, 1536
P:AABB
M:3/4
L:1/8
K:C major
P:A
de | "F"f4f2 | f4e2 | "Dm"d4d2 | d2"C"c2c2 | "G"B4B2 | "Am"c2A2c2 | 
"Em"B6- | B4de | "F"f4f2 | f4e2 | "Dm"d4d2 | d2"C"c2c2 | 
"G"B4c2 | "Am"BAc2B2 | "C"c4c2 | c4de | "F"f4f2 | f4e2 | 
"Dm"d4d2 | d2"C"c2c2 | "G"B4B2 | "Am"c2e2c2 | "Em"B6- | B4de | 
"F"f4f2 | f4e2 | "Dm"d4d2 | d2"C"c2c2 | "G"B4c2 | "Am"BAc2B2 | 
"C"c4c2 | c4 ::
P:B
de | "F"f4f2 | f4e2 | "Dm"d4d2 | d2"C"c2c2 | 
"G"B4B2 | "Am"c2A2c2 | "Em"B6- | B4de | "F"f4f2 | f4e2 | 
"Dm"d4d2 | d2"C"c2c2 | "G"B4c2 | "Am"BAc2B2 | "C"c4c2 | c6 | 
"F"AGABcA | "Dm"d6 | "G"GABcBA | "Em"e6 | "Am"A2c2d2 | "G"d4B2 | 
"Am"A4B2 | "C"e3fe2 | "F"c6 | "Dm"d3ed2 | "G"B2c2d2 | "Em"e6 | 
"Am"ABcABc | "Dm"d4c2 | "C"c3cB2 | "C"c4 :| 


\end{abc}
\index{Se pensando al partire}
\addcontentsline{toc}{subsection}{Se pensando al partire}
\begin{abc}[name=latex_16italian28]
X:28
T:Se pensando al partire
C:Fabritio Caroso, il Ballarino, 1581
T:Balletto
N:Transcribed & arranged by Emma Badowski
%P:AA BBB
M:C
L:1/8
K:F major
P:A
F2 |:"Bb"F2F2 "C"G2G2 | "F"A2A2 A2"C"G2 | "Bb"F2"Gm"ED "Asus4"E4 | "Dm"D6D2 | \
D2"Am"E2 "Bb"F2"C"G2 | "F"A2A2 A2"C"G2 | "Dm"F2ED "Asus4"E2E2 | "D"^F2F2 F4 ::
P:B
"F"A4 "C"G3A | "Gm"B2B2 "F"A2c2 | "Gm"B2"F"A2 "C"G4 | "F"A4 A4 | \
A4 "C"G3A | "Gm"B2B2 "F"A2c2 | "Gm"B2"F"A2 "C"G4 | "F"A8 |
A2A2 "Dm"A2"C"G2 | "Dm"A4 "Bb"F4 | "Bb"F3E "Gm"G2"Dm"F2 | "Asus4"E4 "Dm"D4 | \
D2A2 A2"C"G2 | "Dm"A4 "Bb"F4 | F3A "Gm"G2"Dm"F2 | "Asus4"E4 "D"^F4 |
"Bb"F2F2 "C"G2G2 | "F"A2A2 A2"C"G2 | "Bb"F2"Gm"ED "Asus4"E4 | "Dm"D6 D2 | \
D2"Am"E2 "Bb"F2"C"G2 | "F"A2A2 A2"C"G2 | "Dm"F2ED "Asus4"E2E2 | "D"^F2F2 F4 "^(3)":|


\end{abc}
\index{Spagnoletta}
\addcontentsline{toc}{subsection}{Spagnoletta}
\begin{abc}[name=latex_16italian29]
X:29
I:linebreak $
T:Spagnoletta
C:Fabritio Caroso, il Ballarino, 1581
N:Transcribed & arranged by David Yardley. Matches Pennsic Pile 46
P:5 times through (6 for Spagnoletta Nuova)
M:3/4
L:1/8
K:G dorian
GA | "Gm"d4d2 | "F"c3Bc2 | "Bb"d4d2 | c2d2e2 | f3ed2 | "F"c3Bc2 | 
"Bb"d4d2 | "D"^F2G2A2 | "Gm"d4d2 | "F"c3Bc2 | "Bb"d4d2 | c2d2e2 | 
f3ed2 | "F"c3Bc2 | "Bb"d4d2 | c2d2e2 | f3ed2 | "Gm"d3cB2 | 
"F"c4c2 | c4c2 | "Gm"B3AG2 | "D"A3GA2 | "Gm"d4d2 | "C"c2d2e2 | 
"Bb"f3ed2 | "Gm"d3cB2 | "F"c4c2 | c3Bc2 | "Gm"B3AG2 | "D"d4d2 | 
"Gm"z2zAB2 | B3AG2 | "D"A4"C"G2 | "D"A3GA2 | "Gm"z2zdd2 | d3cB2 | 
"D"A4"C"G2 | "D"A3GA2 | "Gm"z2zAB2 | B3AG2 | "D"A4"C"G2 | "D"A3GA2 | 
"Gm"z2zdd2 | d3cB2 | "D"A4"C"G2 | "D"A3GA2 | "G"d4 :| 


\end{abc}
\index{Spagnoletto, lo}
\addcontentsline{toc}{subsection}{Lo Spagnoletto}
\begin{abc}[name=latex_16italian30]
X:30
T:Lo Spagnoletto
C:Cesare Negri, le Grazie d'Amore, 1602
N:Transcribed & arranged by Dave Lankford. Matches Pennsic Pile 46
P:AABBCC x 7
M:C
L:1/8
K:F major
P:A
 B2 | "Gm"B2AB "F"c2Bc | "Bb"d2d2 B2d2 | "F"c2B2 c2c2 | "Bb"B4 B2 ::
P:B
d2 | "F"c2B2 A2"C"G2 | "D"^F4 D2dc | "Gm"B2AG "D"^F2A2 | "G"G6 ::
P:C
FG | "F"A4 F2AB | c4 A2dc | "Gm"B2AG "D"^F2A2 | "G"G6 :| 


\end{abc}
% \index{Torneo Amoroso}
% \addcontentsline{toc}{subsection}{Torneo Amoroso}
% \begin{abc}[name=latex_16italian31]
% X:31
% T:Torneo Amoroso
% C:Cesare Negri, Le Grazie d'Amore, 1602
% P:(AABBCC)x2 DDEEFFGGHHJJ
% M:C|
% L:1/4
% K:F major
% P:A
% "G"G2 "Am"cc | "G"=B2 =B2 | "F"cc Ac | "Gm"B2 B2 | "F"A/G/A/B/ c"Bb"B | "F"A/B/c/A/ B/A/G/F/ | "C"E"F"F "Csus4"FE | "F"F4 ::
% P:B
% "C"G3/A/ G"Dm"F | "C"E2 "Bb"D2 | "Bb"DD "C"E2 | "F"F4 | "C"GF GA | "C"GF E3/C/ | "Dm"D/E/F "Csus4"FE | "F"F4 ::
% P:C
% "F"AA/B/ cA | "Bb"B3/c/ dc/B/ | "F"AA/B/ cA | "Bb"B3/c/ dc/B/ | "F"A3/B/ cA | "Bb"B3/A/ G"F"F | "C"E"F"F "^Repeat AABBCC!""Csus4"FE | "F"F4 ::
% M:6/8
% P:D
% "^Saltarello""F"FF/ "C"EE/ | "F"F3//G//A/ "Eb"GG/ | "Bb"BA/ "Cm"G"Dm"F/ | D/"C"E "F"F3/ ::\
% P:E
% "Bb"BA/ "C"GF/ | "F"F3//G//A/ "Eb"GF/ | "Bb"BA/ "C"G"Dm"F/ | F/"C"E "F"F3/ ::
% P:F
% "F"F3//G//A/ "Bb"B3//c//B/ | "F"F3//G//A/ "Bb"B3/ | "Bb"d3//c//B/ "F"A3//G//F/ | "C"E3//F//G/ "F"F3/ ::\
% M:6/4
% P:G
% "^Galliarda""Bb"FDF "C"E3/D/C | "F"cAc "Gm"B3/c/B | "Eb"G/A/B/c/B "F"A3/C/D/E/ | "F"F"Csus4"FE "F"F3 ::
% M:6/8
% P:H
% "^Saltarello""Eb"GA/ "C"GF/ | "Dm"F3//G//A/ "C"GF/ | "Bb"BA/ "Eb"G"Bb"F/ | F/"Csus4"F/E/ "F"F3/ ::\
% P:J
% "F"F3//G//A/ "Bb"BB/ | "F"F3//G//A/ "Bb"BB/ | "Bb"d3//c//B/ "F"A3//G//F/ | "Bb"D/"C"G "F"F3/ :| "^Reverance"F3 |] 
% \end{abc}

\index{Villanella}
\addcontentsline{toc}{subsection}{Villanella}
\begin{abc}[name=latex_16italian32]
X:32
I:linebreak $
T:Villanella
C:Fabritio Caroso, Il Ballarino, 1581
N:Transcribed & arranged by Kathy Van Stone. Matches Pennsic Pile 46
P:AABB x 6 (fast) or AABB x 3 (slow)
M:3/4
L:1/8
K:G major
P:A
"G"d4c2 | B3AB2 | "Am"A3Bc2 | "G"B4B2 | "D"A6 | "C"G6 | "D"A6 | "G"B4B2 ::
P:B
"G"B4B2 | "F"A4A2 | A4"C"G2 | "F"A4A2 | A3Bc2 | "G"B3AG2 | "D"A3GA2 | "G"B4B2 :| 
\end{abc}

\index{Vita, la}
\addcontentsline{toc}{subsection}{La Vita}
\begin{abc}[name=latex_il_papa1]
X:1
I:linebreak $
T:La Vita
C:Nathan Kronenfeld
P:5x
M:6/4
L:1/8
K:C major
P:A
"Am"E6 A3Bc2 | "Dm"d2A2d2 "C"c3BAG | "Am"c4B2 "Dm"d3cBA | "E"A4^G2 "Am2."A6 || 
P:B
"Am"B3GB2 "G"A2G2E2 | "Am"B3GB2 "Am"A2G2E2 | 
EFGFED E6 |] 
\end{abc}


\chapter{Dances from Arbeau's {\em Orchésographie}}

Published in 1589 in Langres, France, Orchésographie includes music and
instructions for many different kinds of dances. Numerically speaking, the bulk
of the dances in Arbeau are {\em bransles}. Most of the bransles are in duple
time and should be played at about half note = 115.  The triple time bransles
are Bransle Gay and Bransle de Poictou; for these, a tempo of dotted half =
60-65 should work.

Arbeau also includes instructions for the pavane and galliard, music for which
also appears in the Improvised Dances section.

\clearpage
\input{arbeau.tex}

\chapter{Improvised Dances}

Improvised dances such as the pavane and galliard were very popular in the 16th
century all over Europe. Music and instructions for these dances appear in
numerous sources. Reductions are provided from such sources as Tylman Susato's
{\em Danserye} of 1551 and Praetorius' {\em Terpsichore} of 1612. We have also
included the tunes used in the SCA for some early Italian improvised dances,
the Piva and the Saltarello.

The Canarie is transcribed in 6/4. For the Canarie, use a tempo of
approximately dotted half = 70.

Galliards can be transcribed in either 3/2 or 6/4. We have chosen to use 3/2
for clarity for some of the more rhythmically complex settings while halving
the original note values and using 6/4 for the more straightforward ones. The
tempo for galliards (for the 6/4 settings) can be anywhere from dotted half =
45 - 60, depending on the whims of the dancing master. For transcriptions in
3/2 use dotted whole = 45 - 60 instead (two measures of a 3/2 galliard equating
to one measure of 6/4 galliard). The Volta is really just a variation on the
galliard and can be played as such.

The pavanes are transcribed in cut time, and again, the tempo can range from
half note = 45 to 60.

Preferences vary, so always check with the dancing master for desired tempo.
Additionally, modern choreographies have been created for some of these tunes,
so be sure to confirm the roadmap with the dancing master if these are being
danced.

\clearpage
\input{improvised.tex}

\chapter{The English Dancing Master, 1651}

This section includes all 105 dances in the first edition of John Playford's
{\em The English Dancing Master} of 1651.  The dances are generally transcribed
in either cut time or in 6/4. For cut time use a tempo of approximately half
note = 115 or for 6/4, dotted half = 115.  Some dances such as Chestnut are
often danced slower, so be sure to check with the dancing master just in case.

\clearpage
\input{playford.tex}

\chapter{Other English Country Dances}

This section includes a variety of other English Country dances including
reconstructed music for dances from pre-Playford manuscripts as well as
selected dances from later editions of Playford or later English Country dance
authors that have been danced in the SCA.

\clearpage
\index{Beginning of the World, the}
\index{St. Ledgers}
\addcontentsline{toc}{subsection}{The Beginning of the World}
\begin{abc}[name=latex_playford_later1]
X:1
T:The Beginning of the World
T:or, St. Ledgers
C:adapted from Sellinger's Round by William Byrd and John Playford by Henry of Maldon
P:AA B CC x 3
L:1/8
K:G
P:A
G |: G>AB GAB | c2 c d>e=f | ged c2B | [1 (C3 C) g :| [2 (C3 C) d ] |\
P:B
e3 e>dc | d3 d2 d | 
Bd>c BAG | A3 A2 B \
P:C
|:c>dc B2 G | c>de d2B | A2 G F>EF | [1 G2 G G>AB :| [2 (G3 G2) |]


\end{abc}
\index{Black Nag}
\addcontentsline{toc}{subsection}{Black Nag}
\begin{abc}[name=latex_playford_later2]
X:1
C:John Playford, the Dancing Master, 1670
T:Black Nag
N:arr. Jay Ter Louw. Matches Pile 2018
P:AA BB x 3
M:6/4
L:1/8
K:A minor
E2 |: \
P:A
"Am"A3BA2 "Em"B3AB2 | "Am"c3Bc2 "G"B2c2d2 | "Am"e3dc2 "G"B3AB2 | "Am"A6- A4E2 :: \
P:B
"Em"B2G2E2 B2G2E2 | B2G2E2 B2G2E2 | 
"Am"e2c2A2 e2c2A2 | e2c2A2 e2c2A2 | "Em"B2G2E2 B2G2E2 | B2G2E2 B2c2d2 | \
"Am"e3dc2 "E"B3cB2 |  [1 "Am"A6- A4A2 :|]  [2 "Am"A12 |] 


\end{abc}
\index{Epping Forest}
\addcontentsline{toc}{subsection}{Epping Forest}
\begin{abc}[name=latex_playford_later3]
X:2
I:linebreak $
C:John Playford, the Dancing Master, 1670
T:Epping Forest
K:D minor
M:6/4
L:1/8
P:A
 |: "Gm"d4d2 "F"c3BA2 | "Gm"B3AG2 "D"^F4e2 | "Bb"f3ed2 "F"c4B2 | "Dm"A6- A4d2 | "Bb"f3ed2 "F"c4A2 | "Gm"B3AG2 "D"^F4D2 | 
"C"E3FG2 "D"G4^F2 |  [1 "Gm"G12 :|]  [2 "Gm"G6 G4 :: 
P:B
Bc | "Bb"d6 "F"c6 | "Bb"B6- B4de | "Dm"f6 "A"e6 | 
"Dm"d6- d4 :: 
P:C
d2 | "Gm"d3ed2 "F"c3BA2 | "Gm"B3AG2 "D"^F4D2 | "C"E3FG2 "D"G4^F2 |  [1 "Gm"G6- G4 :|]  [2 "Gm"G12 |] 


\end{abc}
\index{Female Sailor}
\addcontentsline{toc}{subsection}{Female Sailor}
\begin{abc}[name=latex_playford_later4]
X:3
T:Female Sailor
T:Marche pour les Matelots
C:Marain Marais, Alcyone, 1706
M:6/8
L:1/8
K:E minor
"Em"E2B "B"B2A | "Em"G3 "D"A3 | "Em"B2A G2F | G2F E2^D | E2B "B"B2A | "Em"G3 "D"A3 | \
"Em"B2A G2F | E6 :|
"Em"g2f e2^d | e3 "B"B3 | "Em"g2f e2^d | e3 "E"B2d | \
"Am"c2A "Em"B2F | "Em"G2E G2A | B3 "A"^c2^d | "Em"e6 |
 "Em"g2f e2^d | e3 "B"B3 | \
"Em"g2f e2^d | e3 "E"B2d | "Am"c2A "Em"B2F | "Em"G2E G2A | B3 "B"F2E | "Em"E6 |] 


\end{abc}
\index{Hole in the Wall}
\addcontentsline{toc}{subsection}{Hole in the Wall}
\begin{abc}[name=latex_playford_later5]
X:4
I:linebreak $
T:Hole in the Wall
C:John Playford, the Dancing Master, 1695
K:G major
M:3/2
L:1/4
"G"B3/c/ B/c/d"D"Ad | "Em"G3/A/ G/A/B"Bm"FB | "C"E3/F/ E/F/G"G"DB | "Dsus4"G3F"G"G2 | "G"B3/c/ B/c/d"D"Ad | \
"Em"G3/A/ G/A/B"Bm"FB | 
"C"E3/F/ E/F/G"G"DB | "C Dsus4"G3F"G"G2 ||
"Em"g3/f/ e/f/g"D"fe | "B"^d3/e/ d/e/fBf | "Em"g3/f/ e/f/g"D"fe | "Am    B"e3^d"Em"e2 | 
"C"E3/F/ E/F/G"D"F/G/A | "G"G3/A/ G/A/B"Am"A/B/c | "G"B3/c/ B/c/d"D"Dd | "G D"B3A/B/"G"G2 |] 


\end{abc}
\index{Jamaica}
\addcontentsline{toc}{subsection}{Jamaica}
\begin{abc}[name=latex_playford_later6]
X:5
I:linebreak $
T:Jamaica
C:John Playford, the Dancing Master, 1670
M:C|
L:1/4
K:F major
"F"FA AB/c/ | "Bb"d"F"c "Bb"d2 | "F"cA AG/F/ | "Csus4"G2 "F"F2 | "F"FA AB/c/ | "Bb"d"F"c "Bb"d2 | 
"F"cA AG/F/ | "Csus4"G2 "F"F2 |: 
"F"ff "C"ee | "Bb"dd "F"cA | "Bb"ff e/f/g | "Bb"d2 "F"c2 | 
"F"ff "C"ed/c/ | "Bb"dd "F"cA | "Bb"B/c/d "F"cB/A/ | "Csus4"G2 "F"F2 :| 


\end{abc}
\index{Jumbling of Harry, the}
\addcontentsline{toc}{subsection}{The Jumbling of Harry}
\begin{abc}[name=latex_playford_later7]
X:12
T:The Jumbling of Harry
C:Jumbled by Henry of Maldon
C:for the Lovelace Ms. dance "The Jumbling of Joan"
P:AA BB x 3
L:1/8
M:C|
K:G
P:A
d3 c B2A2 | | BABc d2cd | e2d2 cdcB | A6 B2 | c2 B2 AGAB | [1 G8 :| [2 G6 G2 
|:F4 D2 F2 | A4 G2 A2 | B3 A G2 A2 | | F2 D2 E2 F2 | G3 A Bc d2 | e3 d cdcB | A2 G2 G2 F2 | [1 G6 G2 :| [2 G8 |]


\end{abc}
\index{Kelsterne Gardens}
\addcontentsline{toc}{subsection}{Kelsterne Gardens}
\begin{abc}[name=latex_playford_later8]
X:6
T:Kelsterne Gardens
C:John Playford, the Dancing Master, the Third Volume 1726
M:C|
B:Barnes 2nd ed.
Z:2007 John Chambers <jc@trillian.mit.edu>
F:http://trillian.mit.edu/~jc/music/abc/England/tune/KelsterneGardens_Dm.abc	 2007-09-26 22:34:42 UT
L:1/8
K:D minor
"Dm"D2d2 dcBA | "Gm"B2G2 E2G2 | "C"C2c2 cBAG | "F"BAGF "C"AGFE | \
"Dm"D2d2 dcBA | "Gm"B2G2 E2G2 | "Dm"A2F2 "A7"A,2^C2 | "Dm"D8 :: 
"Dm"d2a2 a2ga | "Gm"bagf "C"e2c'2 | "Bb"d2b4d2 | "A"^c2a4ga | \
"Gm"bagf "A7"ed^ce | "A7"A2^c2 "Dm"d4 :| 


\end{abc}
\index{Lightly Love}
\addcontentsline{toc}{subsection}{Lightly Love}
\begin{abc}[name=latex_playford_later9]
X:7
I:linebreak $
T:Lightly Love
T:Light of Love, or Earl of Bedford
P:One round: AA BBB BBB; end with AA BB
C:16th. C English
K:C major
M:6/4
L:1/8
P:A
 |: "C"G3Gc2 B2G2E2 | "F"F3GE2 "G"D4D2 | "C"G3Gc2 B2G2E2 | "F"F3E"G"D2 "C"C4C2 :: 
P:B
"F"F3GF2 "C"E3DC2 | "F"F3GA2 "G"D4D2 | 
"C"G3Gc2 B2G2E2 | "F"F3E"G"D2 "C"C4C2 :| 


\end{abc}
\index{Oranges and Lemons}
\addcontentsline{toc}{subsection}{Oranges and Lemons}
\begin{abc}[name=latex_playford_later10]
X:8
T:Oranges and Lemons
B:3LF / Barnes
C:John Playford, the Dancing Master, 1670
M:C|
L:1/8
K:G
%
%
"D" d2 e2 afed | "G" e3 d "(A)" B3 A | "D" d3 f afed | [1 "A" e3 f "D" d4 :| [2 "A" e3 f "D" d2 ga |]
"G" b6 fg | "D" a6 fe | d3 f afed | "G" e3 d "(A)" B3 A | "D" d3 f afed | "A" e3 f "D" d2 ga |
"G" b6 fg | "D" a6 fe | d3 f afed | "G" e3 d "(A)" B3 A | "D" d3 f afed | [1 "A" e3 f "D" d2 ga :| [2 "A" e3 f "D" d4 |]


\end{abc}
\index{Prince William}
\addcontentsline{toc}{subsection}{Prince William}
\begin{abc}[name=latex_playford_later11]
X:9
T:Prince William
M:C|
C:Walsh c. 1731
B:3LF/Barnes
L:1/4
K:G
D | "G" G2 B A/G/| "D" A2 D "C" uc| "G" B2 "D" A2| "G" G/F/G/A/ G "D" A | "G" B G D B | "D" A3 "Em" G| "D" F d "A" A ^c | "D" d3 :|
|:c/d/|"G" d2 "Am" e d | "C" c>B "D" A d | "Am" c B "D" A "G" G | "D" F/G/A/F/ "G" D2 | "G" G F/G/ "D" A G/A/ | "G" B A/B/ "C" c d | "G" B A/G/ "D" D F | "G" G3 :|


\end{abc}
\index{Sellinger's Round}
\addcontentsline{toc}{subsection}{Sellinger's Round}
\begin{abc}[name=latex_playford_later12]
X:10
T:Sellinger's Round
N:arr. Robert Smith; matches Pile 2018
C:William Byrd, My Ladye Nevells Booke, 1591
P:AA BB x 4
M:6/4
L:1/8
K:C major
P:A
 |: "G"G6 "C"G3AB2 | "C"c6 c3de2 | "Dm"d4c2 "G"B3AB2 |  [1 "C"c12 :|]  [2 "C"c6- c4d2 :: \
P:B
"C"e6 e3dc2 | "G"d6- d4d2 | 
"G"B3cd2 d3cB2 | "D"A6 d4"G"B2 | "C"c3dc2 "G"B4G2 | "F"A3Bc2 "G"B4G2 | \
"F"A4"C"G2 "D"^F3EF2 |  [1 "G"G6- G4d2 :|]  [2 "G"G12 :| 


\end{abc}

\index{Trenchmore}
\index{Tomorrow the fox will come to towne}
\addcontentsline{toc}{subsection}{Trenchmore}
\begin{abc}[name=latex_playford_later13]
X:11
I:linebreak $
T:Trenchmore
T:Tomorrow the fox will come to towne
C:Thomas Ravenscroft, Deuteromelia, 1609
K:G major
M:6/4
L:1/8
"G"G2 | "D"F2F2"G"G2 "D"A4"G"B2 | "D"A4"G"G2 "D"A6 | "G"G6 B6 | "G"d4"C"c2 "G"B4G2 | "D"F2F2"G"G2 "D"A4"G"B2 | "F"c4"G"B2 "D"A4"G"Bc | 
"D"d4"Em"B2 "Am D"A4A2 | "G"G6- G4G2 | "D"F4"G"G2 "D"A4"G"B2 | "D"A4F2 A4"G"D2 | "G"G2G2G2 B4B2 | "G"d4"C"c2 "G"B4G2 | 
"D"F4"G"G2 "D"A4"G"B2 | "F"c4"G"B2 "D"A6 | "G"B6 d6 | "G"d4B2 d4G2 | "D"F4"G"G2 "D"A4"G"B2 | "F"c4"G"B2 "D"A4"G"Bc | 
"D"d4"Em"B2 "Am D"A4A2 | "G"G6- G4 |] 
\end{abc}


\chapter{Modern Folk Dances and SCA Choreography}

This section includes folk dances from later traditions that are popular within
the SCA as well as music for SCA choreographies in a variety of styles.

\clearpage
\input{other.tex}

\clearpage

\printindex

\end{document}
