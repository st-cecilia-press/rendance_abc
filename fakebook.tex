\documentclass[11pt]{book}
\usepackage{fancyhdr}
\usepackage{multicol}
\usepackage{makeidx}
\usepackage[hidelinks]{hyperref}
\usepackage{refcount}
\usepackage{abc}
\usepackage{tocloft}
\usepackage[utf8]{inputenc}
\usepackage[top=48pt,headheight=18pt,headsep=12pt,bottom=48pt,inner=48pt,outer=48pt]{geometry}
\makeindex

\begin{document}
%\setuplayouts
\frontmatter

\pagestyle{fancy}

\fancyhf{}

\fancyhead[LE,RO]{\thepage}
\fancyhead[RE,LO]{DRAFT \today -- {\bf DO NOT DISTRIBUTE}}
%\pagebreak
\renewcommand{\headrulewidth}{0pt}
\renewcommand{\footrulewidth}{0pt}
\setlength{\parindent}{0pt}
\setlength{\parskip}{11pt plus 2pt minus 2pt}

%\begin{figure}
%
%\currentpage
%
%\drawparametersfalse
%
%\drawpage
%
%\caption{Page layout for this document} \label{fig:ptrs}
%
%\end{figure}
%\pagebreak
%

\section*{Note from the Editor}

The goal of this book is to include as many tunes as possible for extant
pre-1651 dances. It also includes music for dances choreographed by SCA
memebers in a variety of styles as well as some tunes for later English Country
dances and modern folk dances that are danced in the SCA.

We have provided this fakebook both as a printable PDF and as ABC files,
obtainable from \url{https://github.com/orgs/st-cecilia-press/rendance\_abc}.
ABC is a simple format for music readable both by computers and by humans that
is commonly used for folk music and dance tunes. There is also a lot of
software to display, play, search, and transpose ABC files. We particularly
recommend EasyABC on Mac, Linux and Windows:
\url{http://www.nilsliberg.se/ksp/easyabc/}. There are many other software
packages including for Android and iOS devices. A full list can be found at
\url{http://abcnotation.com/software}.

We have aimed to match the key of the tune as well as the marked chords to a
tune's most recent appearance in the Pennsic Pile. If it has not recently
appeared in the Pennsic Pile, we have typically kept the original key in the
source material and added an attempt at a reasonable harmonization.

Our belief is that all these tunes are freely usable within the SCA. Most are
many hundreds of years old. However, please make your own determination before
using any of this material in a non-SCA/non-educational setting. Sources of
harmonizations are noted in the ABC files mentioned above.

Some of these tunes have many more chords marked than most chord players would
reasonably want to play. These extra chord markings may still be useful for
players improvising a bass line or for adding passing notes when playing a
chordal accompaniment.

Thanks to Emma Badowski for organizing this material in a previous edition of
this fakebook and especially for harmonizing many of the tunes from Playford
1651 that have not previously appeared in the Pennsic Pile.

Please note that this book is {\em not} an official publication of the Society
for Creative Anachronism.

Aaron Elkiss

\clearpage
\begin{multicols}{2}

\renewcommand\cftchapafterpnum{\vskip\baselineskip}
\setlength{\cftsubsecindent}{0pt}
\setlength{\cftsubsecnumwidth}{0pt}
\tableofcontents
\end{multicols}

\clearpage
\mainmatter

\chapter{Basse Danse}

Basse danse (or bassadanza in Italian) was popular across Europe in the 15th
and early 16th centuries. One of the most important sources for basse danse is
Ms 9085 in the Bibliotheque Royale, Brussels (c. 1445) This manuscript gives
only a slow-moving tenor, or cantus firmus, as the melody for most of the
dances. Musicians normally would have improvised multipart polyphony above the
tenor line. One simple way to improve a melody from the tenor line is to play
it in the style of the basse dance section from ``Rostiboli Gioioso''.  Most of
the basse danses are notated here in 6/4 time, and an appropriate tempo would
be approximately dotted half note = 40-45.

\clearpage

\index{Alenchon}
\addcontentsline{toc}{subsection}{Alenchon}
\begin{abc}[name=latex_basse_dance1]
X:1
I:linebreak $
T:Alenchon
M:6/4
L:1/8
K:C clef=treble-8 octave=1 
A,12| C12| D12| A,12| F,12| E,12| D,12| D,12| F,12| 
A,12| D12| C12| A,12| F,12| D,12| E,12| D,12| D,12| F,12| G,12|
A,12| A,12| D12| A,12| F,12| D,12| F,12| E,12| D,12| D,12| D,12|]


\end{abc}
\index{Allemande, la}
\addcontentsline{toc}{subsection}{La Allemande}
\begin{abc}[name=latex_basse_dance2]
X:2
I:linebreak $
T:La Allemande
C:Paul Butler
M:6/4
L:1/8
K:Gdor
D12 | G12| G12 | A12 | D12 | D12 | C12 | A12 | 
F12 | D12 | G12 | G12 | F12 | C12 | D12 | G12 |
 G12 | A12 | D12 | D12 | D12 | G12 |]


\end{abc}
\index{Aliot nouvella}
\addcontentsline{toc}{subsection}{Aliot nouvella}
\begin{abc}[name=latex_basse_dance3]
X:3
I:linebreak $
T:Aliot nouvella
M:6/4
L:1/8
K:Ddor clef=treble-8 octave=1 
D,12 | D,12 | F,12 | A,12 | B,12 |
A,12 | C12 | D12 | C12 |
A,12 | A,12 | G,12 | F,12 |
F,12 | B,12 | A,12 | E,12 |
D,12 | F,12 | E,12 | D,12 |
D,12 | F,12 | E,12 | F,12 |
G,12 | A,12 | _B,12 | A,12 |
G,12 | F,12 | F,12 | F,12 |
E,12 | D,12 | D,12 | D,12 |]


\end{abc}
\index{Amours}
\addcontentsline{toc}{subsection}{Amours}
\begin{abc}[name=latex_basse_dance4]
X:4
I:linebreak $
T:Amours
M:3/2
L:1/8
K:C clef=treble-8 octave=1 
F,12| E,12| F,12| F,12|
E,12| D,12| E,12| ^F,12|
G,12| =F,12| E,12| F,12|
F,12| E,12| D,12| E,12|
^F,12| G,12| G,12|]


\end{abc}
\index{Avignon}
\addcontentsline{toc}{subsection}{Avignon}
\begin{abc}[name=latex_basse_dance5]
X:5
I:linebreak $
T:Avignon
M:3/2
L:1/8
K:C clef=treble-8 octave=1 
D,12| C,12| D,12|
E,12| F,12| E,12| D,12|
D,12| G,12| F,12| E,12|
D,12| A,12| F,12| E,12|
E,12| G,12| G,12| A,12|
A,12| C12| B,12| A,12|
A,12| G,12| F,12| E,12|
E,12| G,12| A,12| E,12|
D,12| F,12| E,12| D,12|
D,12| A,,12| C,12| D,12|
E,12| F,12| E,12| D,12|
D,12| D,12|]


\end{abc}
\index{Barbesieux}
\addcontentsline{toc}{subsection}{Barbesieux}
\begin{abc}[name=latex_basse_dance6]
X:6
I:linebreak $
T:Barbesieux
M:6/4
L:1/8
K:Dm clef=treble-8 octave=1
B,12 | D12 | C12 | B,12 | A,12 |
G,12 | F,12 | A,12 | G,12 |
A,12 | A,12 | D12 | C12 |
=B,12 | A,12 | G,12 | F,12 |
E,12 | D,12 | F,12 | G,12 |
F,12 | E,12 | D,12 | D,12 |
D12 | E12 | D12 | C12 |
=B,12 | A,12 | A,12 | B,12 |
A,12 | F,12 | E,12 | D,12 |
D,12 | D,12 |]


\end{abc}
\index{Barcelonne}
\addcontentsline{toc}{subsection}{Barcelonne}
\begin{abc}[name=latex_basse_dance7]
X:7
I:linebreak $
T:Barcelonne
M:6/4
L:1/8
K:C clef=treble-8 octave=1 
G,12| B,12| D12| C12|
B,12| A,12| D,12| E,12|
D,12| D,12| A,12| B,12|
D12| C12| B,12| A,12|
G,12| F,12| E,12| D,12|
D,12| F,12| E,12| D,12|
F,12| A,12| C12| A,12|
F,12| D,12| F,12| E,12|
D,12| D,12| D,12|]


\end{abc}
\index{Basine, la}
\addcontentsline{toc}{subsection}{La basine}
\begin{abc}[name=latex_basse_dance8]
X:8
I:linebreak $
T:La basine
M:6/4
L:1/8
K:C clef=treble-8 octave=1 
F,12| D,12| E,12| D,12|
F,12| E,12| F,12| A,12|
B,12| A,12| C12| B,12|
A,12| A,12| A,12| B,12|
A,12| G,12| F,12| F,12|
E,12| D,12| D,12| D,12|]


\end{abc}
\index{Bayonne}
\addcontentsline{toc}{subsection}{Bayonne}
\begin{abc}[name=latex_basse_dance9]
X:9
I:linebreak $
T:Bayonne
M:6/2
L:1/8
K:C clef=treble-8 octave=1 
A,12| B,12| D12| C12| B,12| A,12| D12| C12| A,12| F,12| E,12| D,12|
w:B:~R b ss d r ss d d d ss r r w: T:~R b ss d d d ss r r r b ss 
E,12| D,12| D,12| G,12| B,12| A,12| G,12| A,12| D12| C12| B,12| A,12|
w:r b ss d r b ss d r ss d d
w:d ss r r r b ss d d d ss r 
A,12| B,12| A,12| G,12| F,12| D,12| E,12| D,12| D,12| D,12|]
w:d r r r b ss d r b cw: r r b ss d r r r b c


\end{abc}
\index{Beaulte de Castile}
\addcontentsline{toc}{subsection}{Beaulte de Castile}
\begin{abc}[name=latex_basse_dance10]
X:10
I:linebreak $
T:Beaulte de Castile
M:6/8
L:1/8
K:C clef=treble-8 octave=1 
C6| C6| G,6| C6|
C6| C6| G,6| C6|
C6| G,6| C4-CC| D2C B,2A,|
G,4-G,G,| D2C B,A,/2G,/2A,/2^F,/2| G,6| =F,6|
F,6| F,4-F,G,| D2C B,A,/2G,/2A,/2^F,/2| G,6|
=F,6| F,6| F,4-F,G,| D2C B,A,/2G,/2A,/2^F,/2| G,6|]


\end{abc}
\index{Beaulte}
\addcontentsline{toc}{subsection}{Beaulte}
\begin{abc}[name=latex_basse_dance11]
X:11
I:linebreak $
T:Beaulte
M:6/4
L:1/8
K:C clef=treble-8 octave=1 
D12| D12| C12| C12|
A,12| G,12| F,12| F,12|
G,12| A,12| D,12| F,12|
E,12| D,12| F,12| G,12|
A,12| D12| C12| B,12|
A,12| A,12| A,12| D12|
A,12| D,12| F,12| E,12|
D,12| E,12| F,12| E,12|
D,12| G,12| F,12| D,12|
E,12| D,12| D,12| D,12|]


\end{abc}
\index{Belle, la}
\addcontentsline{toc}{subsection}{La Belle}
\begin{abc}[name=latex_basse_dance12]
X:12
I:linebreak $
T:La Belle
M:6/4
L:1/8
K:C clef=treble-8 octave=1 
D12| 
A,12| D,12| E,12| D,12| 
D,12| A,12| C12| D12| 
C12| B,12| A,12| A,12| 
B,12| B,12| A,12| F,12| 
E,12| D,12| F,12| E,12| 
D,12| D,12| A,12| B,12| 
A,12| G,12| F,12| E,12| 
D,12| D,12| D,12| D,12|]


\end{abc}
\index{Casuelle la Nouvelle}
\addcontentsline{toc}{subsection}{Casuelle la Nouvelle}
\begin{abc}[name=latex_basse_dance13]
X:13
I:linebreak $
T:Casuelle la Nouvelle
T:La Spagna
N:adapted from Heinrich Isaac setting
K:F major clef=G-8
M:6/4
L:1/2
(d3 | d3) | A3 | G3 | B3 | A3 | (G3 | G3) | B3 | c3 | (d3 | d3) |
f3 | e3 | d3 | f3 | g3 | g3 | c3 | d3 | c3 | c3 | g3 | g3 | 
f3 | f3 | g3 | c3 | B3 || _e3 | (d3 | d3) | G3 | A3 | c3 | d3 | f3 | 
_e3 | d3 | c3 | B3 | A3 | G3 | d3 | (G3  | HG3 ) |]


\end{abc}
\index{Cupido}
\addcontentsline{toc}{subsection}{Cupido}
\begin{abc}[name=latex_basse_dance14]
X:14
T:Cupido
T:Canzon di Pifari
C:Cornazano, c. 1465
M:6/4
L:1/8
K:Gmaj clef=G-8
B12 | B12 | d12 | e12 | A12 | B12 | A12 | A12 | e12  | d12 | c12 | d12 | e12 | e12 | A12 | d12 |
c12 | c12 | B12 | B12 | G12 | B12 | d12 | e12 | A12 | B12 | A12 | A12 | B12 | c12 | A12 | G12 |
e12 | d12 | c12 | c12 | d12 | e12 | c12 | d12 | A12 | B12 | A12 | B12 | A12 |]


\end{abc}
\index{Danse de Cleves}
\addcontentsline{toc}{subsection}{Danse de Cleves}
\begin{abc}[name=latex_basse_dance15]
X:15
I:linebreak $
T:Danse de Cleves
M:6/4
L:1/8
K:D minor
P:A
G2 | d2cBc/B/A/G/ A2G2GA/B/ | c2AFG2 F2F2F2 | B2B2c2 d2d2c2 | B2cBA2 G2G2G2 || 
P:B
d2cBGB A2G2GA/B/ | c2ABG2 F2F2F2 | 
B2B2c2 d2G2_e2 | dc2BA2 G2G2G2 |: 
P:C
g2g2g2 a2d2c/d/e/f/ | gfede2 d2d2G2 :| 
P:D
d2cBc/B/A/G/ A2G2GA/B/ | c2AFG2 F2F2F2 | 
B2B2c2 d2d2c2 | B2cBA2 G2G2G2 || 
P:E
g2g2g2 a2d2ef | g2fde2 d2d2d2 | g2g2g2 a2d2ef | g2fge2 d2d2G2 || 
P:F
d2cB2G A2G2AB | c2BAG2 F2F2F2 | B2B2c2 d2d2c2 | B2GBA2 G2G2G2 || 
P:G
d2cB2c A2G2AB | c2BAG2 F2F2F2 | 
B2B2c2 d2d2c2 | B2GBA2 G6 |] 


\end{abc}
\index{Doulce amour, la}
\addcontentsline{toc}{subsection}{La doulce amour}
\begin{abc}[name=latex_basse_dance16]
X:16
I:linebreak $
T:La doulce amour
M:6/4
L:1/8
K:C clef=treble-8 octave=1 
A,12| A,12| F,12| E,12|
A,12| A,12| D,12| E,12|
D,12| D,12| A,12| A,12|
D,12| G,12| E,12| F,12|
E,12| E,12| A,12| A,12|
C12| B,12| A,12| D,12|
A,12| A,12| A,12| G,12|
F,12| A,12| B,12| A,12|
A,12| C12| B,12| A,12|
G,12| E,12| D,12| D,12|
D,12|]


\end{abc}
\index{Doulz espoir, le}
\addcontentsline{toc}{subsection}{Le doulz espoir}
\begin{abc}[name=latex_basse_dance17]
X:17
I:linebreak $
T:Le doulz espoir
M:6/4
L:1/8
K:Ddor clef=treble-8 octave=1 
F,12| D,12| E,12| F,12|
A,12| G,12| F,12| F,12|
C12| A,12| C12| B,12|
A,12| G,12| F,12| F,12|
A,12| G,12| D,12| C,12|
C,12| D,12| D,12| C,12|
F,12| A,12| C12| B,12|
A,12| G,12| F,12| C12|
D12| C12| F,12| C12|
B,12| A,12| C12| B,12|
A,12| G,12| F,12| E,12|
D,12| D,12| A,12| C12|
B,12| C12| B,12| A,12|
G,12| G,12| C12| B,12|
A,12| A,12| A,12| G,12|
F,12| F,12| F,12|]


\end{abc}
\index{Basse danse du roy, la}
\addcontentsline{toc}{subsection}{La basse danse du roy}
\begin{abc}[name=latex_basse_dance18]
X:18
I:linebreak $
T:La basse danse du roy
M:6/4
L:1/8
K:C clef=treble-8 octave=1 
F,12| F,12| E,12| A,12|
A,12| E,12| D,12| C,12|
C,12| E,12| E,12| F,12|
A,12| G,12| F,12| E,12|
E,12| E,12| F,12| G,12|
F,12| E,12| D,12| C,12|
C,12| C,12| D,12| C,12|
C,12| G,12| F,12| A,12|
G,12| G,12| E,12| F,12|
E,12| D,12| C,12| C,12|
F,12| G,12| A,12| A,12|
E,12| D,12| C,12| C,12|
C,12|]


\end{abc}
\index{Engoulesme}
\addcontentsline{toc}{subsection}{Engoulesme}
\begin{abc}[name=latex_basse_dance19]
X:19
I:linebreak $
T:Engoulesme
M:6/4
L:1/8
K:C clef=treble-8 octave=1 
F,12| A,12| D12| C12|
A,12| G,12| F,12| E,12|
D,12| D,12| A,12| D12|
A,12| G,12| A,12| G,12|
D12| C12| B,12| A,12| G,12| 
D,12| E,12| D,12| D,12| 
D12| A,12| F,12| D,12| 
F,12| G,12| F,12| E,12| 
D,12| D,12| D,12|]


\end{abc}
\index{Joyeulx espoyr, le}
\addcontentsline{toc}{subsection}{Le Joyeulx espoyr}
\begin{abc}[name=latex_basse_dance20]
X:20
I:linebreak $
T:Le Joyeulx espoyr
M:6/4
L:1/8
K:C clef=treble-8 octave=1 
G,12| E,12| F,12| C,12|
G,12| B,12| A,12| G,12|
B,12| C12| G,12| F,12|
C,12| F,12| E,12| D,12|
C,12| G,12| C,12| D,12|
C,12| C,12| C,12|]


\end{abc}
\index{Filles a marier}
\addcontentsline{toc}{subsection}{Filles a marier}
\begin{abc}[name=latex_basse_dance21]
X:21
I:linebreak $
T:Filles a marier
M:6/4
L:1/8
K:C clef=treble-8 octave=1 
A,12| A,12| C12| G,12|
C12| D12| C12| C12|
C12| D12| E12| D12|
C12| B,12| A,12| A,12|
E12| B,12| D12| G,12|
C12| D12| C12| C12|
C12| D12| E12| D12|
C12| B,12| A,12| A,12|
A,12|]


\end{abc}
\index{Florentine}
\addcontentsline{toc}{subsection}{Florentine}
\begin{abc}[name=latex_basse_dance22]
X:22
I:linebreak $
T:Florentine
M:3/2
L:1/8
K:C clef=treble-8 octave=1 
x12| A,12| F,12| G,12|
D,12| A,12| C12| B,12|
A,12| ^C12| D12| A,12|
G,12| D,12| G,12| F,12|
E,12| D,12| A,12| D,12|
E,12| D,12| D,12| A,12|
F,12| G,12| D,12| A,12|
=C12| B,12| A,12| ^C12|
D12| A,12| G,12| D,12|
G,12| F,12| E,12| D,12|
A,12| D,12| E,12| D,12|
D,12| D,12|]


\end{abc}
\index{Franchoise nouvelle}
\addcontentsline{toc}{subsection}{Franchoise nouvelle}
\begin{abc}[name=latex_basse_dance23]
X:23
I:linebreak $
T:Franchoise nouvelle
M:6/2
L:1/8
K:C clef=treble-8 octave=1 
C4C4C4 B,4G,4A,2B,2| 
C3 B,A,2 G,2A,4 G,4G,4G,4| 
C4C4C4 B,3 A,G,4A,2B,2| 
C4G,2 B,2A,4 G,4G,4C2B,2|
A,4G,2 F,3 E,D,2 E,4C,4C,4| 
F,4E,2 C,2D,4 E,4E,4C2B,2| 
A,4G,2 F,3 E,D,2 E,4C,4C,4| 
F,3 E,D,2 C,2D,4 C,4C,4C,4|
C,4C,4D,4 E,4E,4F,4| 
E,4D,4D,4 C,4C,4C,4| 
C,4C,4D,4 E,4E,4F,4| 
E,2 (3G,4-G,F,4- F,D,4-D, C,4C,4C,4|
C4C4C4 B,4G,4A,2B,2| 
C3 B,A,2 G,2A,4 G,4G,4G,4| 
C4C4C4 B,3 A,G,4A,2B,2| 
C4G,2 B,2A,4 G,4G,4C2B,2|
A,4G,2 F,3 E,D,2 E,4C,4C,4| 
F,4E,2 C,2D,4 E,4E,4C2B,2| 
A,4G,2 F,3 E,D,2 E,4C,4C,4| 
F,3 E,D,2 C,2D,4 C,4C,4C,4|
C,4C,4D,4 E,4E,4F,4| 
E,4D,4D,4 C,4C,4C,4| 
C,4C,4D,4 E,4E,4F,4| 
E,2 (3G,4-G,F,4- F,D,4-D, C,4C,4C,4|
C,16 |]


\end{abc}
\index{Grant Rouen, le}
\addcontentsline{toc}{subsection}{Le grant Rouen}
\begin{abc}[name=latex_basse_dance24]
X:24
I:linebreak $
T:Le grant Rouen
M:6/4
L:1/8
K:C clef=treble-8 octave=1 
A,12| G,12| C12| C,12|
E,12| D,12| C,12| C,12|
G,12| G,12| A,12| C12|
B,12| A,12| G,12| G,12|
G,12| B,12| C12| C12|
A,12| F,12| E,12| E,12|
E,12| D,12| C,12| D,12|
C,12| B,12| A,12| C12|
D12| D12| G,12| A,12|
G,12| F,12| E,12| E,12|
A,12| B,12| C12| F,12|
C,12| D,12| C,12| C,12|
C,12|]


\end{abc}
\index{Grant Thorin, le}
\addcontentsline{toc}{subsection}{Le grant Thorin}
\begin{abc}[name=latex_basse_dance25]
X:25
I:linebreak $
T:Le grant Thorin
M:6/4
L:1/8
K:C clef=treble-8 octave=1
A,12| A,12| F,12| G,12| A,12| 
C12| B,12| B,12| A,12| A,12| 
B,12| F,12| E,12| D,12| E,12| E,12|
D,12| D,12| B,12| B,12| C12| B,12| 
A,12| A,12| B,12| A,12| D12| 
C12| B,12| A,12| G,12| F,12|
F,12| A,12| C12| B,12| A,12| 
C12| B,12| A,12| A,12| F,12| 
A,12| G,12| F,12| G,12| F,12| 
F,12| F,12|]


\end{abc}
\index{Hault et le bas, le}
\addcontentsline{toc}{subsection}{Le hault et le bas}
\begin{abc}[name=latex_basse_dance26]
X:26
I:linebreak $
T:Le hault et le bas
M:6/4
L:1/8
K:C clef=treble-8 octave=1 
F,12| D,12| E,12| D,12| F,12| 
G,12| A,12| A,12| A,12| C12| 
D12| A,12| D,12| E,12| D,12| 
D,12| F,12| D,12| E,12| D,12|
F,12| G,12| A,12| A,12| A,12| 
C12| D12| A,12| D,12| E,12| 
D,12| D,12| D,12|]


\end{abc}
\index{Haulte Borgongne (Lydian), la}
\addcontentsline{toc}{subsection}{La haulte Borgongne (Lydian)}
\begin{abc}[name=latex_basse_dance27]
X:27
I:linebreak $
T:La haulte Borgongne (Lydian)
M:6/4
L:1/8
K:C clef=treble-8 octave=1
B,12| A,12| E,12| B,,12| C,12| 
D,12| _E,12| F,12| F,12| A,12| 
G,12| G,12| F,12| F,12| G,12| 
A,12| F,12| D,12| _E,12| _E,12|
D,12| C,12| B,,12| C,12| D,12| 
F,12| D,12| C,12| B,,12| B,,12| 
B,12| C12| F,12| G,12| F,12| 
F,12| C12| =B,12| A,12| G,12|
F,12| =E,12| D,12| D,12| F,12| 
G,12| F,12| C,12| D,12| C,12| 
_B,,12| B,,12| B,,12|]


\end{abc}
\index{Haulte Borgongne (Dorian), la}
\addcontentsline{toc}{subsection}{La haulte Borgongne (Dorian)}
\begin{abc}[name=latex_basse_dance28]
X:28
I:linebreak $
T:La haulte Borgongne (Dorian)
M:6/4
L:1/8
K:C clef=treble-8 octave=1 
D12| C12| G,12| D,12| E,12| 
F,12| G,12| A,12| A,12| C12| 
B,12| B,12| A,12| A,12| B,12| 
C12| A,12| F,12| G,12| G,12| F,12| 
E,12| D,12| E,12| F,12| A,12| 
F,12| E,12| D,12| D,12| D12| 
E12| A,12| B,12| A,12| A,12|
E12| D12| C12| B,12| A,12| 
G,12| F,12| F,12| A,12| B,12| 
A,12| E,12| F,12| E,12| D,12| D,12|
D,12|]


\end{abc}
\index{Je languis}
\addcontentsline{toc}{subsection}{Je languis}
\begin{abc}[name=latex_basse_dance29]
X:29
I:linebreak $
T:Je languis
M:6/4
L:1/8
K:C clef=treble-8 octave=1 
C,12| E,12| G,12| A,12| G,12| 
G,12| G,12| A,12| C12| B,12| 
A,12| G,12| F,12| E,12| E,12| 
C,12| C,12| E,12| G,12| G,12|
A,12| G,12| G,12| C,12| E,12| 
G,12| A,12| G,12| F,12| E,12| 
D,12| C,12| C,12| C12| C12| 
B,12| D12| B,12| A,12| G,12|
G,12| A,12| G,12| G,12| G,12|]


\end{abc}
\index{Je sui povere de leesse}
\addcontentsline{toc}{subsection}{Je sui povere de leesse}
\begin{abc}[name=latex_basse_dance30]
X:30
I:linebreak $
T:Je sui povere de leesse
M:6/4
L:1/8
K:C clef=treble-8 octave=1 
C,12| C,12| E,12| E,12| G,12| 
F,12| E,12| D,12| C,12| C,12| 
G,12| F,12| C,12| D,12| C,12| 
C,12| C,12| E,12| G,12| A,12|
G,12| C12| B,12| C12| G,12| 
G,12| B,12| B,12| E,12| E,12| 
G,12| F,12| E,12| E,12| G,12| 
A,12| G,12| F,12| C,12| D,12|
C,12| C,12| C,12|]


\end{abc}
\index{Joieusement}
\addcontentsline{toc}{subsection}{Joieusement}
\begin{abc}[name=latex_basse_dance31]
X:31
I:linebreak $
T:Joieusement
M:6/4
L:1/8
K:C clef=treble-8 octave=1 
A,12| F,12| E,12| E,12| F,12| 
G,12| C,12| E,12| D,12| C,12| 
C,12| E,12| F,12| G,12| C12| 
B,12| A,12| A,12| G,12| C12|
G,12| C,12| E,12| D,12| C,12| 
C,12| D,12| E,12| D,12| C,12| 
G,12| G,12| F,12| E,12| E,12| 
C,12| D,12| C,12| C,12| C,12|]


\end{abc}
\index{Joieux de Brucelles, le}
\addcontentsline{toc}{subsection}{Le joieux de Brucelles}
\begin{abc}[name=latex_basse_dance32]
X:32
I:linebreak $
T:Le joieux de Brucelles
M:6/4
L:1/8
K:C clef=treble-8 octave=1 
D,12| F,12| A,12| G,12| C12| 
B,12| A,12| G,12| G,12| G,12| 
C12| A,12| D,12| E,12| D,12| 
D,12| G,12| F,12| E,12| E,12|
A,12| D12| G,12| D,12| F,12| 
E,12| D,12| D,12| F,12| A,12| 
D,12| E,12| D,12| D,6 x6| D,12|]


\end{abc}
\index{Languir en mille destresse}
\addcontentsline{toc}{subsection}{Languir en mille destresse}
\begin{abc}[name=latex_basse_dance33]
X:33
I:linebreak $
T:Languir en mille destresse
M:6/4
L:1/8
K:C clef=treble-8 octave=1 
C,12| E,12| G,12| A,12| G,12| 
E,12| C,12| C,12| G,12| G,12| 
A,12| C12| B,12| A,12| G,12| 
F,12| E,12| E,12| C,12| C,12|
E,12| E,12| G,12| A,12| G,12| 
G,12| C,12| E,12| G,12| A,12| 
G,12| F,12| E,12| D,12| C,12| 
C,12| C,12|


\end{abc}
\index{Lauro}
\addcontentsline{toc}{subsection}{Lauro}
\begin{abc}[name=latex_basse_dance34]
X:34
I:linebreak $
T:Lauro
T:Re di Spagna
C:Vatican, Cap. 283
N:trans. Al Cofrin
N:for dance by Ebreo (15th C)
K:F major clef=G-8
M:6/4
L:1/4
d3 d3 | A3 G3 | d2 c B A2 | G3 G3 | 
B3 c3 | d3 d3 | f3 _e3 | d3 ^f3 | g3 g3 ||
c3 d3 | c3 c3 | g3 g3 | f3 f3 | g3 c3 | B3 _e3 | d3 d3 | 
G3 A3 | c3 d3 | f3 e3 | d3 c3 | B3 A3 | G3 A3 | G6 |]


\end{abc}
\index{Lyron}
\addcontentsline{toc}{subsection}{Lyron}
\begin{abc}[name=latex_basse_dance35]
X:35
I:linebreak $
T:Lyron
M:6/4
L:1/8
K:C clef=treble-8 octave=1 
G,12| F,12| E,12| D,12| F,12| 
E,12| D,12| D,12| G,12| G,12| 
C12| D12| E12| D12| C12| 
B,12| B,12| A,12| F,12| E,12|
D,12| D,12| E,12| D,12| F,12| 
G,12| C12| B,12| A,12| B,12| 
C12| A,12| G,12| G,12| G,12|]


\end{abc}
\index{Maistresse}
\addcontentsline{toc}{subsection}{Maistresse}
\begin{abc}[name=latex_basse_dance36]
X:36
I:linebreak $
T:Maistresse
M:6/4
L:1/8
K:C clef=treble-8 octave=1 
D12| D12| C12| B,12| A,12| 
D,12| G,12| A,12| G,12| G,12| 
F,12| G,12| A,12| C12| G,12| 
A,12| G,12| G,12| D12| 
E12| D12| G,12| F,12| E,12|
D,12| D,12| A,12| B,12| A,12| 
G,12| C12| B,12| A,12| A,12| 
D,12| F,12| G,12| D12| 
B,12| A,12| G,12| G,12| G,12|]


\end{abc}
\index{M'amour}
\addcontentsline{toc}{subsection}{M'amour}
\begin{abc}[name=latex_basse_dance37]
X:37
I:linebreak $
T:M'amour
M:6/4
L:1/8
K:C clef=treble-8 octave=1 
F,12| D,12| E,12| D,12| A,12| 
D12| C12| B,12| A,12| B,12| A,12| 
A,12| B,12| C12| D12| D12|
C12| B,12| A,12| A,12| A,12| 
B,12| A,12| G,12| F,12| E,12| 
D,12| E,12| D,12| D,12| D,12|]


\end{abc}
\index{Marchon la dureau}
\addcontentsline{toc}{subsection}{Marchon la dureau}
\begin{abc}[name=latex_basse_dance38]
X:38
I:linebreak $
T:Marchon la dureau
M:6/4
L:1/8
K:C clef=treble-8 octave=1 
F,12| E,12| F,12| G,12|
A,12| A,12| D,12| F,12|
G,12| E,12| D,12| F,12|
E,12| F,12| G,12| A,12|
A,12| D,12| F,12| E,12|
D,12| A,12| B,12| C12|
B,12| A,12| A,12| B,12|
C12| B,12| A,12| A,12|
G,12| A,12| B,12| C12|
A,12| C12| B,12| A,12|
F,12| F,12| F,12|]


\end{abc}
\index{Margaritte, la}
\addcontentsline{toc}{subsection}{La Margaritte}
\begin{abc}[name=latex_basse_dance39]
X:39
I:linebreak $
T:La Margaritte
M:6/4
L:1/8
K:C clef=treble-8 octave=1 
C12| D12| B,12| B,12| G,12| F,12| 
E,12| E,12| G,12| A,12| C,12| E,12|
D,12| C,12| C,12| E,12| F,12| G,12| 
C12| G,12| G,12| C12| G,12| C,12|
E,12| D,12| C,12| C,12| E,12| D,12| 
E,12| D,12| C,12| G,12| F,12| D,12|
C,12| C,12| C,12|]


\end{abc}
\index{Mois de may, le}
\addcontentsline{toc}{subsection}{Le mois de may}
\begin{abc}[name=latex_basse_dance40]
X:40
I:linebreak $
T:Le mois de may
M:6/4
L:1/8
K:C clef=treble-8 octave=1 
C12| C12| D12| B,12| A,12| A,12| 
C12| E12| D12| C12| D12| C12|
A,12| B,12| G,12| C12| D12| 
E12| D12| C12| A,12| C12| 
B,12| G,12| C12| A,12| D12| 
C12| B,12| B,12| C12| B,12|
A,12| A,12| A,12|]


\end{abc}
\index{Mon Cousin, je me recommende}
\addcontentsline{toc}{subsection}{Mon Cousin, je me recommende}
\begin{abc}[name=latex_basse_dance41]
X:41
I:linebreak $
T:Mon Cousin, je me recommende
M:3/2
L:1/8
K:C clef=treble-8 octave=1 
G,12| F,12| D,12| E,12| F,12| 
E,12| F,12| G,12| F,12| E,12| 
D,12| E,12|]


\end{abc}
\index{Mon leal desire}
\addcontentsline{toc}{subsection}{Mon leal desire}
\begin{abc}[name=latex_basse_dance42]
X:42
I:linebreak $
T:Mon leal desire
M:6/4
L:1/8
K:Ddor clef=treble-8 octave=1 
F,12 | G,12 | A,12 | A,12 | D12 |
C12 | D12 | A,12 | C12 |
_B,12 | A,12 | A,12 | F,12 |
A,12 | G,12 | D,12 | E,12 |
E,12 | D,12 | D,12 | A,12 |
A,12 | B,12 | B,12 | D12 |
E12 | D12 | D12 | A,12 |
B,12 | A,12 | A,12 | E,12 |
F,12 | E,12 | D,12 | A,12 |
C12 | D12 | D12 | A,12 |
F,12 | D,12 | F,12 | E,12 |
E,12 | A,12 | B,12 | A,12 |
A,12 | F,12 | E,12 | D,12 |
E,12 | D,12 | D,12 | D,12 |]


\end{abc}
\index{Ma meiulx ammee}
\addcontentsline{toc}{subsection}{Ma meiulx ammee}
\begin{abc}[name=latex_basse_dance43]
X:43
I:linebreak $
T:Ma meiulx ammee
M:6/4
L:1/8
K:C clef=treble-8 octave=1 
G,12| D,12| F,12| A,12| G,12| 
G,12| C12| B,12| A,12| A,12| 
D,12| G,12| A,12| G,12| C12| 
D12| G,12| A,12| B,12| C12| D12| 
D12| B,12| A,12| G,12| A,12| 
B,12| A,12| G,12| G,12| G,12|]


\end{abc}
\index{Navaroise, la}
\addcontentsline{toc}{subsection}{La navaroise}
\begin{abc}[name=latex_basse_dance44]
X:44
I:linebreak $
T:La navaroise
M:6/4
L:1/8
K:C clef=treble-8 octave=1 
D12| D12| A,12| F,12| D,12| E,12| 
D,12| D,12| F,12| A,12| D12| D12|
C12| B,12| A,12| G,12| F,12| 
E,12| D,12| D,12| A,12| A,12| D12| 
D12| A,12| F,12| G,12| F,12|
D,12| E,12| D,12| D,12| D,12|]


\end{abc}
\index{Non pareille, la}
\addcontentsline{toc}{subsection}{La non pareille}
\begin{abc}[name=latex_basse_dance45]
X:45
I:linebreak $
T:La non pareille
M:6/4
L:1/8
K:C clef=treble-8 octave=1 
C12| C12| D12| D12| A,12| D12| 
C12| B,12| A,12| A,12| C12| B,12|
A,12| D,12| E,12| F,12| G,12| A,12| 
C12| A,12| B,12| C12| C12| F,12|
C,12| E,12| E,12| D,12| C,12| 
C,12| C,12|]


\end{abc}
\index{Orleans}
\addcontentsline{toc}{subsection}{Orleans}
\begin{abc}[name=latex_basse_dance46]
X:46
I:linebreak $
T:Orleans
M:3/4
L:1/8
K:Gmix clef=treble-8 octave=1 
G,12 | A,12 | G,12 | D,12 | A,12 | D12 | C12 | B,12 | A,12 | A,12 | 
w:B,T:~R b ss d d d d d ss r  
B,12 | C12 | A,12 | F,12 | D,12 | E,12 | D,12 | D,12 | F,12 | A,12 | 
w:r r b ss d d d r r r 
G,12 | B,12 | D12 | A,12 | F,12 | D,12 | E,12 | D,12 | A,12 | 
w:b ss d ss r r r b ss 
D12 | C12 | B,12 | C12 | A,12 | G,12 | G,12 | G,12 |]
w:d d d r r r b c


\end{abc}
\index{Passe rose}
\addcontentsline{toc}{subsection}{Passe rose}
\begin{abc}[name=latex_basse_dance47]
X:47
I:linebreak $
T:Passe rose
M:6/4
L:1/8
K:C clef=treble-8 octave=1 
A,12| G,12| B,12| D12| C12| 
B,12| D12| D12| B,12| A,12| 
A,12| A,12| B,12| C12| B,12| 
A,12| G,12| F,12| E,12| E,12|
F,12| G,12| A,12| G,12| F,12| 
E,12| D,12| D,12| D,12|]


\end{abc}
\index{Petit rouen, le}
\addcontentsline{toc}{subsection}{Le petit rouen}
\begin{abc}[name=latex_basse_dance48]
X:48
I:linebreak $
T:Le petit rouen
M:6/4
L:1/8
K:C clef=treble-8 octave=1 
C,12| E,12| A,12| G,12| E,12| 
D,12| C,12| C,12| G,12| A,12| 
G,12| C12| B,12| A,12| G,12| 
G,12| G,12| A,12| B,12| A,12|
G,12| F,12| E,12| E,12| C,12| 
C,12| G,12| G,12| F,12| E,12| 
D,12| D,12| G,12| E,12| C,12| 
F,12| E,12| D,12| C,12| C,12| C,12|]


\end{abc}
\index{Portingaloise, la}
\addcontentsline{toc}{subsection}{La portingaloise}
\begin{abc}[name=latex_basse_dance49]
X:49
I:linebreak $
T:La portingaloise
M:6/4
L:1/8
K:Ddor clef=treble-8 octave=1 
G,12 | A,12 | _B,12 | D12 | C12 |
B,12 | A,12 | A,12 | A,12 |
B,12 | A,12 | G,12 | F,12 |
D,12 | E,12 | D,12 | D,12 |
F,12 | E,12 | F,12 | G,12 |
D12 | C12 | B,12 | A,12 |
G,12 | D,12 | E,12 | D,12 |
D,12 | D,12 |]


\end{abc}
\index{Potevine, la}
\addcontentsline{toc}{subsection}{La potevine}
\begin{abc}[name=latex_basse_dance50]
X:50
I:linebreak $
T:La potevine
M:6/4
L:1/8
K:C clef=treble-8 octave=1 
D,12| E,12| D,12| F,12| G,12| 
A,12| C12| B,12| A,12| A,12| 
A,12| B,12| A,12| G,12| A,12| 
G,12| G,12| A,12| G,12| F,12|
E,12| E,12| F,12| G,12| A,12| 
A,12| A,12| B,12| C12| B,12| 
A,12| G,12| G,12| A,12| A,12| 
G,12| F,12| F,12| G,12| F,12|
A,12| A,12| G,12| G,12| G,12|]


\end{abc}
\index{Rochelle, la}
\addcontentsline{toc}{subsection}{La Rochelle}
\begin{abc}[name=latex_basse_dance51]
X:51
I:linebreak $
T:La Rochelle
M:6/2
L:1/8
K:C clef=treble-8 octave=1 
F,12| E,12| D,12| E,12| D,12| D,12|  D12| C12| D12| E12| D12| C12|
w:B:~R b ss d d d ss r r r b ss 
w:T:~R b ss d d d ss r r r b ss  
B,12| A,12| A,12| A,12| B,12| G,12| F,12| E,12| D,12| D,12| F,12| A,12|
w:d r r r b ss d d d ss r r w: d r r r b ss d d d r r r 
D12| C12| A,12| F,12| D,12| E,12| D,12| D,12| D,12|]
w:r b ss d r r r b c
w:b ss d ss r r r b c


\end{abc}
\index{Petit roysin, le}
\addcontentsline{toc}{subsection}{Le petit roysin}
\begin{abc}[name=latex_basse_dance52]
X:52
I:linebreak $
T:Le petit roysin
M:6/4
L:1/8
K:C clef=treble-8 octave=1 
D,12| F,12| A,12| G,12| D,12| 
F,12| E,12| D,12| D,12| F,12| G,12| 
A,12| A,12| G,12| F,12| E,12|
E,12| A,12| G,12| D,12| E,12| 
F,12| E,12| D,12| D,12| E,12| 
F,12| E,12| D,12| G,12| A,12| 
A,12| G,12| F,12| E,12| D,12|
C,12| D,12| F,12| E,12| D,12| D,12| D,12|]


\end{abc}
\index{Basse Danse du roy d'Espaingne, le}
\addcontentsline{toc}{subsection}{Le Basse Danse du roy d'Espaingne}
\begin{abc}[name=latex_basse_dance53]
X:53
I:linebreak $
T:Le Basse Danse du roy d'Espaingne
M:6/4
L:1/8
K:C clef=treble-8 octave=1 
F,12| F,12| E,12| G,12| A,12| A,12| 
D,12| E,12| D,12| D,12| A,12| 
A,12| C12| B,12| A,12| A,12| G,12| 
F,12| E,12| E,12| F,12| E,12| D,12| 
D,12| A,12| A,12| B,12| B,12| C12| 
C12| D12| D12| A,12| A,12| A,12| 
G,12| F,12| D,12| F,12| E,12| D,12| 
E,12| D,12| D,12| D,12|]


\end{abc}
\index{Sans fair de vous departe}
\addcontentsline{toc}{subsection}{Sans fair de vous departe}
\begin{abc}[name=latex_basse_dance54]
X:54
I:linebreak $
T:Sans fair de vous departe
M:6/4
L:1/8
K:C clef=treble-8 octave=1 
C,12| E,12| F,12| D,12| E,12| 
D,12| C,12| C,12| G,12| E,12| G,12| 
F,12| E,12| D,12| C,12| C,12|
G,12| A,12| B,12| C12| B,12| 
A,12| G,12| G,12| G,12| C12| 
G,12| A,12| E,12| F,12| E,12| 
E,12| A,12| B,12| C12| D12|
E12| D12| C12| G,12| G,12| 
A,12| G,12| F,12| E,12| D,12| 
D,12| G,12| E,12| C,12| F,12| 
E,12| D,12| C,12| C,12| C,12|]


\end{abc}
\index{Ma soverayne}
\addcontentsline{toc}{subsection}{Ma soverayne}
\begin{abc}[name=latex_basse_dance55]
X:55
I:linebreak $
T:Ma soverayne
M:6/4
L:1/8
K:C clef=treble-8 octave=1 
E,12| E,12| G,12| F,12| E,12| G,12| 
A,12| B,12| B,12| C12| B,12| A,12|
G,12| ^F,12| G,12| C,12| G,12| A,12| 
G,12| =F,12| E,12| D,12| E,12| D,12|
C,12| D,12| C,12| D,12| F,12| E,12| 
F,12| G,12| F,12| E,12| D,12| C,12|
D,12| C,12| C,12| C,12|]


\end{abc}
\index{Tantaine, la}
\addcontentsline{toc}{subsection}{La tantaine}
\begin{abc}[name=latex_basse_dance56]
X:56
I:linebreak $
T:La tantaine
M:6/4
L:1/8
K:C clef=treble-8 octave=1 
A,12 | G,12 | F,12 | E,12 | D,12 | E,12 | D,12 | D,12 | F,12 |
w:B,T:~R b ss d d d d d ss
A,12 | _B,12 | A,12 | A,12 | A,12 | B,12 | C12 | B,12 |
w:r r r b ss d d d 
A,12 | F,12 | E,12 | D,12 | D,12 | F,12 | G,12 | F,12 |
w:r r r b ss d ss r r 
D12 | C12 | B,12 | A,12 | G,12 | F,12 | E,12 | D,12 |
w:r r b ss d d d r 
E,12 | D,12 | D,12 | D,12 |]
w:r r b c


\end{abc}
\index{Theme A}
\addcontentsline{toc}{subsection}{Theme A}
\begin{abc}[name=latex_basse_dance57]
X:57
I:linebreak $
T:Theme A
C:Faugues, Missa la basse dance
M:6/8
L:1/8
K:C clef=treble-8 octave=1 
A,6| C6| A,4G,2| F,3 x2F,| G,6| A,4G,2| 
F,4E,2| D,3 x2D,| A,4G,2| F,4A,2| C4B,2| A,3 x2A,|
A,3 G,3| F,3 G,3| F,4E,2| D,2F, E,D,2| C,3 x2C,| 
G,4-G,G,| F,3 A,3| D,2F,2E,2| D,6|]


\end{abc}
\index{Theme B}
\addcontentsline{toc}{subsection}{Theme B}
\begin{abc}[name=latex_basse_dance58]
X:58
I:linebreak $
T:Theme B
C:Faugues, Missa la basse dance
M:6/8
L:1/8
K:C clef=treble-8 octave=1 
D,6| F,4D,2| C,4G,2| F,2D,2E,2| D,6| 
x6| A,4F,2| G,6| A,2F, A,G,2| F,3 x2F,| 
D4C2| D6| D,4E,2| D,3 x2D,| F,3 G,3| 
A,4-A,A,| C4B,2| A,3 x2A,| D3 C3| A,3 F,3|
G,6| A,6| D,2F,2E,2| 
D,6|]


\end{abc}
\index{Torin}
\addcontentsline{toc}{subsection}{Torin}
\begin{abc}[name=latex_basse_dance59]
X:59
I:linebreak $
T:Torin
M:6/4
L:1/8
K:C clef=treble-8 octave=1 
C,12| E,12| G,12| A,12| C,12| 
D,12| C,12| C,12| B,12| C12| G,12| 
A,12| B,12| A,12| G,12| G,12| A,12| 
B,12| C12| C12| G,12| F,12| E,12| 
E,12| F,12| D,12| E,12| C,12|
D,12| D,12| C,12| B,12| C12| 
D12| C12| B,12| A,12| G,12| 
G,12| C,12| G,12| D,12| C,12| 
D,12| C,12| C,12| C,12|]


\end{abc}
\index{Triste plaiser}
\addcontentsline{toc}{subsection}{Triste plaiser}
\begin{abc}[name=latex_basse_dance60]
X:60
I:linebreak $
T:Triste plaiser
M:6/4
L:1/8
K:C clef=treble-8 octave=1 
G,12| F,12| D,12| F,12| A,12| 
G,12| F,12| F,12| G,12| G,12| 
B,12| B,12| A,12| A,12| C12| D12|
C12| C12| B,12| A,12| G,12| B,12| 
G,12| F,12| D,12| F,12| G,12| G,12|
G,12| F,12| G,12| B,12| B,12| 
C12| D12| C12| C12| G,12| 
B,12| A,12| G,12| G,12| G,12|]


\end{abc}
\index{Ulises}
\addcontentsline{toc}{subsection}{Ulises}
\begin{abc}[name=latex_basse_dance61]
X:61
I:linebreak $
T:Ulises
M:6/4
L:1/8
K:C clef=treble-8 octave=1 
C12| D12| C12| E12|
D12| C12| A,12| F,12|
D,12| E,12| D,12| D,12|
F,12| G,12| A,12| A,12|
A,12| B,12| D12| C12|
A,12| F,12| D,12| E,12|
D,12| G,12| F,12| E,12|
D,12| F,12| E,12| D,12|
D,12| D,12|]


\end{abc}
\index{Une fois avant que morir}
\addcontentsline{toc}{subsection}{Une fois avant que morir}
\begin{abc}[name=latex_basse_dance62]
X:62
I:linebreak $
T:Une fois avant que morir
M:6/4
L:1/8
K:Ddor clef=treble-8 octave=1 
D,12 | A,12 | C12 | D12 | A,12 |
G,12 | F,12 | E,12 | D,12 |
D,12 | D12 | D12 | E12 |
D12 | C12 | B,12 | A,12 |
D,12 | A,12 | B,12 | F,12 |
E,12 | G,12 | A,12 | F,12 |
E,12 | D,12 | F,12 | G,12 |
A,12 | _B,12 | A,12 | G,12 |
F,12 | F,12 | D12 | C12 |
D12 | A,12 | G,12 | F,12 |
E,12 | D,12 | D,12 | D,12 |]


\end{abc}
\index{Venise}
\addcontentsline{toc}{subsection}{Venise}
\begin{abc}[name=latex_basse_dance63]
X:63
I:linebreak $
T:Venise
M:6/4
L:1/8
K:Ddor clef=treble-8 octave=1 
F,12 | F,12 | D,12 | E,12 | D,12 |
C12 | A,12 | F,12 | E,12 |
D,12 | D,12 | ^C12 | ^C12 |
D12 | D12 | A,12 | B,12 |
A,12 | A,12 | D12 | D12 |
C12 | B,12 | A,12 | G,12 |
F,12 | E,12 | D,12 | D,12 |
A,12 | D12 | C12 | A,12 |
F,12 | G,12 | F,12 | F,12 |
A,12 | A,12 | _B,12 | A,12 |
^C12 | D12 | A,12 | F,12 |
E,12 | D,12 | D,12 | A,12 |
C12 | A,12 | F,12 | E,12 |
E,12 | D,12 | E,12 | D,12 |
D,12 | D,12 |]


\end{abc}

\index{Verdelete}
\addcontentsline{toc}{subsection}{Verdelete}
\begin{abc}[name=latex_basse_dance64]
X:64
I:linebreak $
T:Verdelete
M:6/4
L:1/8
K:C clef=treble-8 octave=1 
D12| D12| C12| C12|
A,12| G,12| F,12| F,12|
G,12| A,12| D,12| F,12|
E,12| D,12| F,12| G,12|
A,12| D12| C12| B,12|
A,12| A,12| A,12| D12|
A,12| D,12| F,12| E,12|
D,12| D,12| E,12| F,12|
E,12| D,12| A,12| G,12|
F,12| E,12| D,12| E,12|
D,12| D,12| D,12|]
\end{abc}


\chapter{15th Century Italian Dances}

The primary sources for 15th Century Italian dance are manuscripts from the
mid- to late 15th century containing dances by (among others) the dancing
masters Domenico da Piacenza (c. 1400-1470) and his student Guglielmo Ebreo (c.
1420-1848) (also known as Giovanni Ambrosio after his conversion from Judaism
to Catholocism).

15th century Italian dance is somewhat unusual in that dances often change
between ``tempi'', which are marked in each dance. The various tempi are
transcribed as:

\begin{itemize}

\item Bassadanza: 6/4
\item Quadernaria: 4/4
\item Saltarello: 6/8 or occassionally 3/4
\item Piva: 2/4 or 6/8

\end{itemize}

As a rough guide for tempo, keeping a constant tempo of approximately quarter
note = 120 (or dotted quarter = 120 for 6/8 piva sections) regardless of the
various tempi should work for many of the dances.

(See {\em Joy and Jealousy} by Vivian Stephens and Monica Cellio for additional
information; it is available online at
\url{http://sca.uwaterloo.ca/~praetzel/Joy-J-book/}). 


\clearpage
\index{Amoroso}
\addcontentsline{toc}{subsection}{Amoroso}
\begin{abc}[name=latex_15italian1]
X:1
I:linebreak $
T:Amoroso
C:Giovanni Ambrosio (Guglielmo Ebreo da Pesaro), c. 1475 (PnA)
N:Transcribed by Monica Cellio (She'erah bat Shlomo). Matches Pile 2018; Joy & Jealousy.
P:Play two times through
M:C
L:1/8
K:D dorian clef=G-8
P:A (3x)
"^Drone:D/A;Piva"DEFE D2EF | GAGF E2D2 | DEFE D2AG | FDE2 "^       (3)" D4 ::
P:B
A2A2 G2c2 | A2A2 G2AB |
c2ed cAB2 |
M:2/4
A4 ::
M:C
P:C
ddcB/A/ A2AB | c3B/A/ G2AB | c2ed cAB2 |
M:2/4
A4 ::
M:C
P:D
A2A2 G2c2 | A2A2 G2G2 | A2A2 GEF2 | E2E2 E2F2 | FEFG AG/F/E2 | E2E2 E/E/E/E/F3/G/ |
A2A2 A2AG | F3D F2E2 |
M:2/4
D4 :|


\end{abc}
\index{Anello}
\addcontentsline{toc}{subsection}{Anello}
\begin{abc}[name=latex_15italian2]
X:2
I:linebreak $
T:Anello
C:Domenico da Piacenza, c. 1425 (PnD)
N:Transcribed by Monica Cellio (She'erah bat Shlomo). Matches Pile 2018; Joy & Jealousy.
P:Play two times through
M:C
L:1/8
M:C
K:F major clef=G-8
P:A (3x)
"^Drone:F/C;Quadernaria"c2de f2f2 | ecde c2c2 "^(3)":| c2de f2f2 | cBAG F2F2 |
P:B
A/B/cz2 A/B/cz2 | A/B/cz2 def2 |
f2cB AGF2 ::
P:C
A2AG F2FB | AFGA F2F2 ::
M:2/4
P:D
f2 f2 | e2 e2 | df ed | c2 c2 :|
M:C
P:E
A/B/cz2 A/B/cz2 | A/B/cz2 f/e/cde | c2c2 FAG2 | F8 |]


\end{abc}
\index{Belfiore}
\addcontentsline{toc}{subsection}{Belfiore}
\begin{abc}[name=latex_15italian3]
X:3
T:Belfiore
C:Domenico da Piacenza, c. 1425 (PnD)
P:Play three times through
M:C
L:1/8
K:G major clef=G-8
P:A (3x)
"^Drone:G/D;Quadernaria"G3A B2c2 | B2dc B2AG | G3A B2c2 | B2cB A2"^      (3)"G2 ::\
P:B (3x)
d2d3/e/ d3/c/B2 "^(3)"::
M:3/2
P:C
d2 B2 d2 B2 d2 B2 |:\
M:C
P:D (3x)
d2d3/e/ d3/c/"^       (3)"B2 ::\
M:2/4
P:E (3x)
G2A>B | cz z2 | G2 A>B | cz zd | c3B | AG z2  "^(3)":|
P:F
M:C
d2 B>c d4 ||\
P:G
M:2/4
G2A>B | cz z2 | G2 A>B | cz zd | cB A2 | cB A2 | (G4 | G4 ) |]


\end{abc}
\index{Belreguardo}
\addcontentsline{toc}{subsection}{Belreguardo}
\begin{abc}[name=latex_15italian4]
X:4
T:Belreguardo
C:Domenico da Piacenza, c. 1425 (PnD)
N:Transcribed by Monica Cellio (She'erah bat Shlomo).
N:Matches Pennsic Pile 46; transposed down a 4th from Joy & Jealousy.
M:6/8
L:1/8
K:Bb lyd
P:A (2x or 3x)
"^Saltarello"
|:B2B AG2 | F2F GA2 | B2f dc2 | B3 B3 :| \
B2B AG2 | F2F GA2 | d2c BA2 | G3 G2G/2A/2 :|
M:6/4
P:B
K:clef=G-8
"^Bassadanza"B6 B6 | B6 B6 | c6 c6 | c6 c6 | d4 Bc d4 Bc | d4 Bc d2 d4 |\
c6 c6 | c4 AB c4 AB | c4 AB c2 c4 | B6 B6 |
B6 B6 |:\
P:C
d3efg f2e4 | d3efg f2e4 | d6 d6 | d6 d6 :|\
P:D
c6 c6 | c4 e2 d2 c4 | B6 B6 | B6 B6 | B6 B6 |]


\end{abc}
\index{Chirintana}
\addcontentsline{toc}{subsection}{Chirintana}
\begin{abc}[name=latex_15italian5]
X:5
T:Chirintana
C:Al Cofrin
P:AABB; repeat CCDD until done
M:C
L:1/8
K:E dorian
P:A
"^Drone:E/B;Quadernaria "B2AG F2E2 | F2FG DEF2 |\
M:2/4
G2 E2 |\
M:C
B2AG F2E2 | F2FG DEF2 |\
M:2/4
E2 E2 ::
M:C
P:B
e2dc B2A=c | B2B2 =cBA2 |\
M:2/4
B2 B2 |\
M:C
e2dc B2A=c | B2B2 AGFG |\
M:2/4
E2 E2 :|
M:6/8
"^Rhythm Interlude - Pivas"E2E EEE | E2E EEE | E2E EEE | E2z z3 |:
K:G major
P:C
"^Pivas ad nauseum"SE3 EFG | A3 ABc | BAG AGF | E3 E2D |\
E3 EFG | A3 ABc | BAG AGF | E3 E3 ::
P:D
G2G EG2 | B2B GB2 | B2A GAB | A3 AGF |\
G2G EG2 | B2B GB2 | BAG AGF | E3 "^D.S."E3 :|


\end{abc}
\index{Chirintana}
\addcontentsline{toc}{subsection}{Chirintana}
\begin{abc}[name=latex_15italian6]
X:6
T:Chirintana
T:T'Andernaken / Laet Ons Mit Hartzen
C:Emma Badowski, based on anonymous 15th C. Dutch melodies
P:AABB; repeat C until done
M:C|
L:1/4
K:D dorian
P:A
"^Drone:D/A;Quadernaria"A_B A/G/F/A/ | dc/_B/ A/F/G | AA cd | AF/G/ AG/F/ | E/D/E DD ::
P:B
FF cc/_B/ | AG FF/G/ | AA FG | AD/A/ GF |  [1 E/D/E DD :|]  [2 E/D/E DD |
M:6/8
P:C
A/ |:"^Piva"FE/ DC/ | FE/ DA/ | AB/ cA/ | cB/ AA/ | FE/ DC/ | FE/ DF/ | FF/ GF/ | E3/ DE/ |
G/F/E/ ED/ | E3/- EF/ | FF/ GF/ | E3/ DE/ | G/F/E/ ED/ | E3/- EA/ | FE/ DC/ |
FE/ DA/ | AB/ cA/ | cB/ AA/ | FE/ DC/ | FE/ DF/ | FF/ GF/ |  [1 E3/ DA/ :|]  [2 E3/ D3/ :|


\end{abc}
\index{Colonesse}
\addcontentsline{toc}{subsection}{Colonesse}
\begin{abc}[name=latex_15italian7]
X:7
T:Colonesse
P:Play two times through
C:Guglielmo Ebreo da Pesaro, 1463 (PnG)
N:Transcribed by Monica Cellio (She'erah bat Shlomo). Matches Pennsic Pile 46; Joy & Jealousy.
M:6/8
L:1/8
K:F major clef=G-8
P:A
"^Drone:F/C;Saltarello" FFF GGG | AAA BBB | AAA GGG |  [1-3 FFF FFF :|]  [4 FFc AGG | FFF FFF ::
M:6/4
P:B (3x)
"^Bassadanza" f6 f3 e/f/ g/f/e/d/ |  c2 c2 c2 c6 | d2 d2 d2 d2 d2 z2 | d2 d2 f2 e2 d2 d2 | c2 c2 c2 c2 c2 c2" (3)"::
M:2/4
P:C
"^Piva" cd ed | fd BB | GA B2 :|\
M:C
P:D
"^Quadernaria" f3 g f4 | dcde f3e | d>Bcc B4 |]


\end{abc}
\index{Figlia di Guielmina}
\addcontentsline{toc}{subsection}{Figlia di Guielmina}
\begin{abc}[name=latex_15italian8]
X:8
T:Figlia di Guielmina
C:Domenico da Piacenza, c. 1425 (PnD)
N:Transcribed by Al Cofrin (Albrect / Avatar of Catsprey). Matches Pennsic Pile 46
P:Intro:A; AABCDE x 2
M:C
K:Gmaj
P:A
"^Drone:D/A;Quadernaria"G2BB ccAA | GGdd dddc | A2G2 d2e2 | c2dG BBcc | AAG2 G4 :|
P:B
M:6/4
"^Bassadanza"A4 D2=F2 E4 | D12 || \
P:C
D6D6 | E4D2C6 | G6G6 |  A6D6 | A4 A2A4A2 | G4A2 c4c2 | d4 d2d4d2 | c4B2A6 ||
P:D
M:C
"^Quadernaria"eece eedc | eece eedc | dddd cBAd | cBAA GFEG | AAGF E2z2 |
P:E
M:6/8
"^Piva"EEE E2E | A2A F2D | EEE D2D | E3 A2B | c2A B2c | A2A B2B  | EEE E2E | A2A F2D | EEE D3 | -D3-D2z |]


\end{abc}
\index{Gelosia}
\addcontentsline{toc}{subsection}{Gelosia}
\begin{abc}[name=latex_15italian9]
X:9
I:linebreak $
T:Gelosia
C:Domenico da Piacenza, c. 1425 (PnD)
N:Transcribed by Monica Cellio (She'erah bat Shlomo). Matches Pile 2018; Joy & Jealousy.
P:Play three times through
L:1/8
M:C
K:G dorian clef=G-8
P:A (3x)
"^Drone:G/D;Quadernaria"BABc d2d2 | fe/f/dc BA/B/GA "^(3)":|
P:B
BABc d2d2 | cB/c/BA G2G2 |:
P:C
gfed gfed | gfef d2d2 |
BABc d2d2 :|
P:D
cBcA G2F2 |:
P:E
BABc d2d2 | f2fe d2d2 ::
P:F (3x)
g2fe d2"^      (3)"d2 ::
M:2/4
P:G
cB/c/ AG | BA/B/ AG | F2 F2 :|


\end{abc}
\index{Gratiosa}
\addcontentsline{toc}{subsection}{Gratiosa}
\begin{abc}[name=latex_15italian10]
X:10
I:linebreak $
T:Gratiosa
C:Guglielmo Ebreo da Pesaro, 1463 (PnG)
N:Transcribed by Monica Cellio (She'erah bat Shlomo). Matches Joy & Jealousy.
M:C
L:1/8
K:C major clef=G-8
P:A
"^Drone:G/D;Quadernaria"G2Bc d4 | e2Bc d4 | G2Bc d3c | B3/G/A2 G4 ::
P:B
e2dc B4 | e2dc B4 |
G2Bc d3c | B3/G/A2 G4 ::
M:6/4
P:C
"^Bassadanza" G12 | A12 | B12 | A12 | G12 | G12 ::
M:2/4
P:D
"^Piva"GABA | cBGG | FE G2 :|
M:C
d3e d4 | B3/A/Bc d3c | B3/G/ AA G4 |]


\end{abc}
\index{Ingrata}
\addcontentsline{toc}{subsection}{Ingrata}
\begin{abc}[name=latex_15italian11]
X:11
T:Ingrata
C:Domenico da Piacenza, c. 1425 (PnD)
N:Transcribed by Monica Cellio (She'erah bat Shlomo). Matches Joy & Jealousy.
M:6/8
L:1/8
K:F major clef=G-8
P:A
"^Drone:F/C;Saltarello"F2G AB2 | c2c de2 | f2e dc2 | B2B AG2 | F3 F3 :| \
M:C
P:B
"^Quadernaria"f2fg f2f2 | ed/c/ de c2c2 |
M:2/4
AA BB || \
M:6/8
P:C
"^Saltarello"c2A GF2 | E2A AB2 | c2A GF2 | E2A AB2 || \
M:6/4
P:D
"^Bassadanza"c6 c6 | f6 f6 | c6 B6 | A6 A6 |
B4A4G4 | F6 F6 |:\
P:E
A6 B6 | c6 c4A2 | B4A4G4 | F6 F6 :| \
P:F
f6 f6 | f6 f6 |:
M:6/8
P:G (3x)
"^Piva"A2G A2B | c2d c2B | A3 A3 "^(3)":| \
P:H
A2A GF2 | E3 E3 | A2G A2B | \
c3 c3 | B2A B2G | F3 F3 |]


\end{abc}
\index{Jupiter (Giove)}
\addcontentsline{toc}{subsection}{Jupiter (Giove)}
\begin{abc}[name=latex_15italian12]
X:12
I:linebreak $
T:Jupiter (Giove)
C:Domenico da Piacenza, c. 1425 (PnD)
N:Transcribed by Monica Cellio (She'erah bat Shlomo). Matches Joy & Jealousy.
M:C
L:1/8
K:Cmaj clef=G-8
P:A
"^Drone:C/G;Quadernaria" cccc GGGG | AABB cccz | cccc GGGG |
M:6/4
"^Bassadanza"A4A2 B4B2 | c6 c6 ::
P:B
_B6 B6 | c6 c6 | d6 d6 | d4f2 _e2d4 |
c6 c6 ::
M:6/8
P:C (3x)
"^Piva"c2B c2d | B2c e2f | d3 "^        (3)"d3 ::
P:D
"^Saltarello"c3 c3 | _B3 _e3 | d2e dc2 |
B3 e2d |
M:6/4
"^Bassadanza"c4B2 G2_A4 | G6 G6 :|
P:D
c12 | e12 |]


\end{abc}
\index{Legiadra}
\addcontentsline{toc}{subsection}{Legiadra}
\begin{abc}[name=latex_15italian13]
X:13
I:linebreak $
T:Legiadra
C:Guglielmo Ebreo da Pesaro, 1463 (PnG)
N:Transcribed by Monica Cellio (She'erah bat Shlomo). Matches Joy & Jealousy.
M:6/8
L:1/8
K:F major clef=G-8
P:A
"^Drone:F/C;Saltarello"ccc ddd | eee ggg | ddd eee | ddd ccc | ddd eee | fff eee |
ddd ccc | g/e/dd ccc |
ccc ccc |:
M:6/4
P:B
"^Bassadanza"f2f2f2 f2f2de | f2f2f2 f3efd | c2c2c2 c2c2c2 | d2d2d2 d2d2z2 | d2d2fe d2d2c2 | c2c2c2 c2c2AB |
c2c2c2 c2c2c2 ::
M:C
P:C
"^Quadernaria"g3a g2g2 | a2gf e2e2 | ff/f/f2 ede2 ||
M:6/4
"^Bassadanza"e2e2e2 e2e2e2 | d2d2d2 d2d2d2 | c2c2c2 c2c2AB | cccccc ::
M:6/8
P:D
"^Piva"cde dfd | BBA GB2 :|
M:C
P:E
"^Quadernaria"c3d c3B | AFGG FF F2 |]


\end{abc}
\index{Leoncello}
\addcontentsline{toc}{subsection}{Leoncello}
\begin{abc}[name=latex_15italian14]
X:14
T:Leoncello
P:Ax5 BB CC D E F
C:Domenico da Piacenza, c. 1425 (PnD)
N:Transcribed by Monica Cellio (She'erah bat Shlomo). Matches Pennsic Pile 46; Joy & Jealousy.
M:C
L:1/8
K:Gdor clef=G-8
P:A (5x)
"^Drone:F/C;Quadernaria"F2FG F2Fc | cBAG F2"^      (5)"F2 ::\
P:B
A2AG F2F2 | G2G2 A2A2 | FFGG A2A2 ::\
P:C
c3/2B/2 A3/2B/2 cBAG | FFGG A2A2 :|
P:D
M:6/4
"^Bassadanza"A6 A6 | G6 G6 | F6 F6 | A6 A6 | G6 G6 | F6 F6 ||\
P:E
F6 F6 | F6 F6 |
c6 c6 | c6 c6 | d6 d6 | c6 c6 | c6 c6 |\
M:C
P:F
"^Quadernaria"ABc2 z4 | ABc2 z4 |]


\end{abc}
\index{Marchesana}
\addcontentsline{toc}{subsection}{Marchesana}
\begin{abc}[name=latex_15italian15]
X:15
T:Marchesana
N:Transcribed by Monica Cellio (She'erah bat Shlomo). Matches Pennsic Pile 46; Joy & Jealousy.
C:Domenico da Piacenza, c. 1425 (PnD)
M:C
L:1/8
K:F major clef=G-8
P:A (3x)
"^Drone:F/C;Quadernaria"c2cd c2cd | e2d2 c2c2 "^(3)":| c2cd c2cB | AFGG F2F2 |:\
P:B
c2cd c2cc | c2c2 F2z2 :|
M:6/4
P:C
"^Bassadanza"  c6 c6 | d6 d6 | d6 d6 | c6 c6 | B6 B6 | A6 A6 | G6 G6 | G6 G6 | F6 F6 |
 c6 c6 | A6 A6 | G6 G6 | F6 F6 |:\
M:C
P:D
"^Quadernaria"c2z2 c2z2 | c3/B/A3/G/ F2F2 :| F2A2 G2F2 | F2z6 |]


\end{abc}
\index{Mercantia}
\addcontentsline{toc}{subsection}{Mercantia}
\begin{abc}[name=latex_15italian16]
X:16
T:Mercantia
C:Domenico da Piacenza, c. 1425 (PnD)
N:Transcribed by Monica Cellio (She'erah bat Shlomo). Matches Pennsic Pile 46; Joy & Jealousy.
K:F major
K:clef=G-8
M:6/8
L:1/8
P:A (3x)
"^Drone:F/C;Saltarello"G2G AB2 | c2c cBA | G2G AB2 | c2c c"^      (3)"z2 ::\
M:C
P:B
"^Quadernaria"A4 A4 | A2G2 A2B2 | c4 c4 :|
M:6/4
P:C
"^Bassadanza"B6 B6 | c6 c6 | d6 d6 | c6 c6  |:\
P:D
B6 B6 | F6 F6 | G6 G6 | F6 F6 :| \
M:3/4
P:E
c2c2Az || \
M:6/4
P:F
B6 F6 | c4B2 A2G4 ||
M:C
P:G
"^Quadernaria"F2FG A2B2 | BAG2 F2FF | \
M:3/4
P:H
c2c2Az | \
M:6/4
P:I
"^Bassadanza"B6 B6 | F6 F6 | G6 G6 | F6 F6 |]


\end{abc}
\index{Petit Riens}
\addcontentsline{toc}{subsection}{Petit Riens}
\begin{abc}[name=latex_15italian17]
X:17
I:linebreak $
T:Petit Riens
N:Transcribed by Monica Cellio (She'erah bat Shlomo). Matches Pile 2018; Joy & Jealousy.
C:Giovanni Ambrosio (Guglielmo Ebreo da Pesaro), c. 1475 (PnA)
P:Play three times through
M:6/8
L:1/8
K:G mix
P:A
"^Drone:G/D;Piva"GAB GAB | c2B A2G | d2d d2d | e2d/c/ B3 | GAB GAB | c2B A2G |
dd/d/d dcB | A3 G3 :|
P:B
c2c c2c | c2c B3 | d2d d2d | e2e d3 |
c2c c2c | c2c B3 | d2d d2d | e2e d3 |
c2c c2c | c2c B3 | d2d d2d | e2e d3 ||
g2g f2a | g2^f/e/ d3 |
g2g f2a | g2^f/e/ d3 |
g2g f2a | g2^f/e/ d3 ||
d2d d2c | BA2 G3 |
d2d d2c | BA2 G3 |
d2d d2c | BA2 G3 ||
c2c c2c | c2c B3 | d2d d2d | e2e d3 |
g2g f2a | g2^f/e/ d3 |
d2d d2c | BA2 G3 |]


\end{abc}
\index{Petite Rose}
\addcontentsline{toc}{subsection}{Petite Rose}
\begin{abc}[name=latex_15italian18]
X:18
T:Petite Rose
T:Spingardo
C:Joan Ambrosio Dalza, adapted by Monique Rio
M:6/8
L:1/8
K:Gmaj
P:A
e | "^Piva""G5"d2c B2A | G2D G2e | d2c B2A | G2d c2B | "F5"A2c B2A | "G5"G2A B2c | "F5"A2c B2A | "G5"G2G G2 ::
P:B
"G5"d | d3 z2d | d3 z2d | d2c B2A |  [1 G3 z2 :|]  [2 G2 |:\
P:C
"G5"GF2G |"D5"A2GE2F | "G5"G2 ::
P:D
"G5"GF2G | "D5"A2GA2c | "G5"B2 ::\
P:E
"D5"GF2G | "C5"E2GF2E | "D5"D2 :|\
P:F
z2 GF | "C5"E3"D5"F3 |"E5"G3d2c | "C5"BAG "D5"GAF | "G5"G3-G2 |]


\end{abc}
\index{Pizocara}
\addcontentsline{toc}{subsection}{Pizocara}
\begin{abc}[name=latex_15italian19]
X:19
I:linebreak $
N:Transcribed by Monica Cellio (She'erah bat Shlomo). Matches Pennsic Pile 46; Joy & Jealousy.
C:Domenico da Piacenza, c. 1425 (PnD)
T:Pizocara
L:1/8
M:6/8
K:F major clef=G-8
P:A (3x)
"^Piva"F2F c2c | c2c d2d | f2f e2e | d2d c2"^      (3)"z ::
P:B (4x)
B2B B2B | B2c d2z "^(4)":|
M:6/4
P:C
"^Bassadanza" f12 | d12  |:
P:D (3x)
_e6 e6 | f6 f6 | e2d4  B2d2c2 |
B6 B6 "^(3)":|
B12 | B12 ||
M:6/8
P:E
"^Saltarello"e2e2e2 | f2f2e2 | d2B dc2 | B2B2B2 | e2e2e2 | f2f2e2 | d2B dc2 | B2d2c2 |
B2_e dc2 | B3 B3 |:
P:F (3x)
"^Piva"G2F G2A | B3 B3 | d2c d2e | f3 f2e | d2B c2d | B3 B3 "^(3)":|


\end{abc}
\index{Prexonera}
\addcontentsline{toc}{subsection}{Prexonera}
\begin{abc}[name=latex_15italian20]
X:20
I:linebreak $
T:Prexonera
C:Domenico da Piacenza, c. 1425 (PnD)
N:Transcribed by Monica Cellio (She'erah bat Shlomo). Matches Joy & Jealousy.
K:Bb major
K:clef=G-8
M:6/4
L:1/8
P:A
"^Drone:F/C;Bassadanza"B6 B6 | c6 c6 |
M:3/4
d4e2 |
M:6/4
d2c4 B6 | f6 f6 ::
P:B
g4g2 f2e4 | d6 d4Bc | d6 z6 | g4g2 f2e4 |
d6 d4Bc | d6 z6 | B4F2 G2A4 | B6 B6 :|
M:C
P:C
"^Quadernaria"BBcc d2d2 | ffee d2d2 | BBcc d2d2 | ffee d2d2 | d2z2 d2z2 | B2c2 d2d2 |
d2z2 d2z2 |
M:2/4
B2 c2 ||
M:6/8
P:D
"^Saltarello"d2g fe2 | d2g fe2 | d2g fe2 ||
M:6/4
P:E
"^Bassadanza"B6 B6 | c6 c6 |
M:3/4
d4e2 |
M:6/4
d2c4 B6 | f6 f6 |]


\end{abc}
\index{Rostiboli Gioioso}
\addcontentsline{toc}{subsection}{Rostiboli Gioioso}
\begin{abc}[name=latex_15italian21]
X:21
T:Rostiboli Gioioso
C:Guglielmo Ebreo da Pesaro, 1463 (PnG)
N:Transcribed & arranged by Al Cofrin (Albrect (Avatar) of Catsprey). Matches Joy & Jealousy.
N:Matches Pile 2018
P:Play two times through
M:C
M:6/4
L:1/8
K:Fmaj clef=G-8
P:A
"^Bassadanza""F"c2c2c2 c2c2AB | c2c2c2 c2c2c2 | "C"G2G2G2 G2G2GA | "Gm"B2B2B2 "F"A2A2A2 | "C"G2G2G2 G2G2AB |
"F"c2c2c2 c2c2AB | c2c2c2 c2c2c2 | "Dm"F2F2F2 "Bb"F2F2F2 | "F"A2A2B2 "Csus4"A2G2G2 | "F"F2F2F2 F2F2F2 ::
P:B
"C"G2G2A2 G2G2F2 | G2G2A2 G2G2G2 | "Bb"B2B2B2 "F"A2A2A2 | "C"G2G2G2 G2G2F2 |
"C"G2G2A2 G2G2F2 | G2G2A2 G2G2G2 | "F"c2c2B2 "Csus4"A2G2G2 | "F"F2F2F2 F2F2F2 ::
P:C
M:6/8
"^Saltarello""F"ccc ccc | "Gm"BBB BBB | "F"ccc AAA | "C"GGG GGG | "F"ccc ccc | "Gm"BBB BBB | "F"FFF "Csus4"AGG | "F"F2F F3 ::
P:D
M:12/8
"^Piva"F |"C"G2 AG2 FG2 AG2G |"F"c2 c BA2"C"G3G2F | "C"G2 AG2 FG2 AG2G | "F"c2 B A"Csus4"G2"F"F3F2 :|


\end{abc}
\index{Sobria}
\addcontentsline{toc}{subsection}{Sobria}
\begin{abc}[name=latex_15italian22]
X:22
I:linebreak $
T:Sobria
C:Domenico da Piacenza, c. 1425 (PnD)
N:Transcribed by Monica Cellio (She'erah bat Shlomo). Matches Joy & Jealousy.
K:clef=G-8
K:G minor
M:3/4
L:1/8
P:A (3x)
"^Drone:G/D;Bassadanza"d2d2d2 | c4c2 | d2dcB2 | AA/B/ c/d/cB2 | A4A2 "^(3)":|
M:2/4
P:B
"^Piva"FEFG | AA A2 | G2 FE | D2 C2 ||
P:C
d2 d3/e/ | d2 d3/d/ |
cA BB | A2 A3/d/ | cA BB | A4 |:
M:6/4
P:D (3x)
"^Bassadanza"F6 F6 | G4z2 A4z2 | AABc dc B6 | A6 "^         (3)"A6 ::
M:C
P:E
"^Quadernaria"d2d3/e/ d2d3/d/ | cA BB A2A3/d/ | cA BB A4 |
M:6/4
"^Bassadanza"G12 | A12 | c8z2Bc ||
M:C
"^Quadernaria"d2cA B2A2 |
M:2/4
A2 A2 :|
M:3/4
P:F
"^Saltarello"D2D2E2 | F2F2F2 | A2BAG2 | F2F2F2 | D2D2E2 | F2F2F2 |
A2BAG2 | F2F2DC | G2FDE2 | D4D2 |:
M:2/4
P:G (3x)
"^Piva"F2 FE/D/ | C2 C z/C/ | D3/F/ E/D3/ | C2 C2 "^(3)":|


\end{abc}
\index{Spero}
\addcontentsline{toc}{subsection}{Spero}
\begin{abc}[name=latex_15italian23]
X:23
T:Spero
P:Play two times through
C:Guglielmo Ebreo da Pesaro, 1463 (PnG)
N:Transcribed by Monica Cellio (She'erah bat Shlomo). Matches Pennsic Pile 46; Joy & Jealousy.
M:6/8
L:1/8
K:F major clef=G-8
P:A
"^Drone:F/C;Piva"F2F A2B | c2c c2f | e2d c2B | A2A Az2 ::\
P:B
B3 B2A/G/ | F2F z2c | \
B2A G3 | F2F z3 :|
M:C
P:C
"^Quadernaria"cccc FFFz | GGGG FFFz || \
M:6/8
P:D
"^Saltarello"f3 f3 | c3 c3 | F3 F3 | c3 B3/A/G | F3 c3/A/G | F3 F3 ||
M:6/4
P:E
"^Bassadanza" F6 F6 | G6 G6 | A6 A6 | B6 B6 | B6 B6 | A6 A6 | G6 G6 | F6 F6 ||
M:6/8
P:F
"^Piva"F2G A2A | B2c A2A | G2G F2z | F2G A2A | B2c A2A | G2G F2A/B/ |
c2c z3 | dc2 cB2 | A2A G2G | F2F z2c | A2A G2G | F2F F3 |]


\end{abc}
\index{Tesara}
\addcontentsline{toc}{subsection}{Tesara}
\begin{abc}[name=latex_15italian24]
X:24
T:Tesara
N:Transcribed by Monica Cellio (She'erah bat Shlomo). Matches Pennsic Pile 46; Joy & Jealousy.
C:Domenico da Piacenza, c. 1425 (PnD)
M:6/8
L:1/8
K:G mix clef=G-8
P:A (3x)
"^Drone:G/D;Saltarello"d3 c3/B/A | G3 G3/F/E | D3 "^         (3)"D3 ::\
P:B
"^Piva"D2D G2D | G2A B2A | B2c d2d | B2G F2F |
E2E D2D | A3 G3/F/E | D3 z2c | d3 z2c | d3 z2d | e2e c2B | d3 d3 ::
P:C
G2G G2E | D3 D3/E/F | G3 G3 "^(4)":| d3 c3/B/A | G3 F3/D/E | D3 D2D | E2D E2F | G3 G3 |:
G2G G2E | D3 D3/E/F | G3 G3 "^(4)":| d3 c3/B/A | G3 F3/D/E | D3 D2D | E2D E2F | G3 G3 |:
P:D
"^Saltarello"G2G2A2 | B2B2d2 | B3 c3/2B/2A | G3 F3/2D/2E | D2E2D2 | D3/2E/2F G3 ::
P:E (4x)
"^Piva"d2c BA2 | G2G G2G | D3 E3 | D3 D3 "^(4)":| \
P:F
F2D G2D | G2A B2A | B2A B2c | d3 z2d | e2e c2c |:
P:G (4x)
"^Saltarello"e3 A3 | A3 G2d | e3 A3 | A3 G2d "^(4)":| \
P:H
d3 z2c | d3 z2d | e3 e3/2c/2c | d3 d3 | G3 F3/2D/2E | D6 |]


\end{abc}
\index{Verçepe}
\addcontentsline{toc}{subsection}{Verçepe}
\begin{abc}[name=latex_15italian25]
X:25
I:linebreak $
T:Verçepe
C:Domenico da Piacenza, c. 1425 (PnD)
N:Transcribed by Monica Cellio (She'erah bat Shlomo). Matches Pennsic Pile 46; Joy & Jealousy.
M:6/8
L:1/8
K:D dorian clef=G-8
P:A (3x)
"^Drone:D/A;Saltarello"d2d B2B | c2c A2A | d2d cB2 | A3 "^        (3)"A3 ::
M:6/4
P:B
"^Bassadanza"D6 D6 | F6 F6 | A6 A4A2 |
M:3/4
G2F4 |
M:6/4
E6 E6 | F4G2 F2E4 | D6 D6 ::
M:C
P:C
"^Quadernaria"d2dd d3d | AABc d2z2 :|
M:2/4
f2 e2 |
M:6/8
P:D
"^Saltarello"d2d d2d | c2d cB2 | A3 A3 | G3 D3 | F2E D3 |:
M:6/4
P:E
"^Bassadanza"d6 d6 | c4d2 c2B4 | A6 A6 |
M:6/8
"^Saltarello"D2D F2F | A2A G2G | D3 F3 | E3 D3 ::
M:C
P:F
"^Quadernaria"dd zd d3d | DDFF D4 :|


\end{abc}
\index{Vita di Cholino}
\addcontentsline{toc}{subsection}{Vita di Cholino}
\begin{abc}[name=latex_15italian26]
X:26
I:linebreak $
T:Vita di Cholino
N:Matches Pile 2018; Joy & Jealousy
C:modified by V. Stephens from "La Vida de Culin"
P:One dance:5 times through. Play:two dances.
M:C
L:1/8
K:C major
E2 |:"C"G2G2 G2G2 | "F"A4 z2A2 | "F"A2A3/B/ c2c3/B/ | "C"G4 z2G2 | "C"c2c2 c2B2 | "F"A4 A2A2 |
"C"G2G2 "Dm"F2F2 | "C"E4 z2C2 | "C"E4 "G"D4 | "C"C4 z2C2 | "C"E4 "G"D4 | "C"C4 z2C2 |
 [1-4 "C"E4 "Dm"F4 | "G"G6E2 :|]  [5 "C"E4 "G"D4 | "C"C8 |]


\end{abc}


\index{Voltate in ça Rosina}
\addcontentsline{toc}{subsection}{Voltate in ça Rosina}
\begin{abc}[name=latex_15italian27]
X:27
I:linebreak $
T:Voltate in ça Rosina
C:Giovanni Ambrosio (Guglielmo Ebreo da Pesaro), c. 1475 (PnA)
N:Transcribed by Monica Cellio (She'erah bat Shlomo). Matches Pile 2018; Joy & Jealousy.
P:Play two times through
M:C
L:1/8
K:A minor
P:A (3x)
"^Drone:A/E;Quadernaria"c2cd e2e2 | eded c2c2 | c2cd e2e2 | eded c2c2 | dddc B2B2 | c2B2 A4 |
dddc B2B2 | c2B2 "^        (3)"A4 ::
M:2/4
P:B (2x or 4x)
"^Piva"c2 cd | e2 e2 | ed ed | c2 c2 | dd dc | B2 B2 |
c2 B2 | A4 :|
\end{abc}


\chapter{Dances from the Gresley Manuscript}

The Gresley manuscript dates to the late 15th or early 16th century and was
re-discovered in Derbyshire, England. It contains choreography for 26 dances
and music for 13, with 8 of those having both music and the dance steps. We
have re-used other music from the manuscript for some of the dances missing
music and have included newly-composed music by Master Martin Bildner for the remainder, provided
under a Creative Commons Attribution-NonCommercial-ShareAlike license (see 
\url{https://creativecommons.org/licenses/by-nc-sa/3.0/}).
Reconstructions very, so always check the music with the dance master!

The dances are primarily transcribed in a lively 6/8 time; a tempo of dotted
quarter = 115-120 should work well.

\clearpage
\index{Aras}
\addcontentsline{toc}{subsection}{Aras}
\begin{abc}[name=latex_gresley1]
X:1
I:linebreak $
T:Aras
T:for two
C:Richard Schweitzer
P:ABBCCDEF
M:C
M:6/8
L:1/8
K:G major
P:A
"^Drone: G/D"G2G B2A | G2E D3 | E2E D2D | E2E D3 | E2E D2D | G2A B2G | 
E2E D2D | G2A B3 |: 
P:B
d2d B2B | A2B c2B | B3 A3 :: 
P:C
B3 B2G | 
A2c B2A :| 
P:D
G2G BAG | A2c B2A | G2G BAG | A2B c3 || 
P:E
d2d B2B | 
A2B cBA | d2d B2B | A2B c3 || 
M:9/8
P:F
d3 e3 d3 | 
M:6/8
dB2 cA2 | G6 |] 


\end{abc}
\index{Armyn}
\addcontentsline{toc}{subsection}{Armyn}
\begin{abc}[name=latex_gresley2]
X:2
I:linebreak $
T:Armyn
T:for three
C:Richard Schweitzer
P:AA BBB CCC DDD E FFF G
M:C
M:6/8
L:1/8
K:Dmix
P:A
"^Drone: D/A"D2D F2G | A2A A3 | B2d ^c2B | A6 :: 
P:B (3x)
B3 A3 | G3 F2"^        (3)"A :: 
P:C (3x)
A2A A2A | A2B A2G | F3 "^        (3)"D3 :: 
P:D
B3 A3 | G3 F2B | A2A A2B | 
A6 :| 
P:E
B3 A3 | G3 F3 | FGA FE2 | D6 |: 
P:F (3x)
E3 F3 | 
G3 "^         (3)"E3 :| D2
P:G
D F2G | A2A A2F | FGA FE2 | D6 |] 


\end{abc}
\index{Bugill}
\addcontentsline{toc}{subsection}{Bugill}
\begin{abc}[name=latex_gresley3]
X:3
I:linebreak $
T:Bugill
T:for three
C:Richard Schweitzer
P:AAA BBB CDEE
M:C
M:6/8
L:1/8
K:Gmaj
P:A (3x)
"^Drone G/D"d3 B3 | A2B c2B | G3 B3 | A2B c3 | d3 B3 | A2B c3 | 
d2c GA2 | G6 "^(3)":: 
P:B (3x)
A2A A2A | A2B c3 "^(3)":| B2
P:C
A G2B | d6 | 
P:D
A2G A2G | A2B c3 |: 
P:E
d2c BA2 | G6 :| 


\end{abc}
\index{Damesyn}
\addcontentsline{toc}{subsection}{Damesyn}
\begin{abc}[name=latex_gresley4]
X:4
I:linebreak $
T:Damesyn
T:for three
C:Gresley Manuscript, c. 1500
C:Music for This enderis day
P:AAA B CC D
M:C
M:6/4
L:1/8
K:Dmix
P:A
"^Drone: D/A"A2ABA3GFED2 | G3AB2 BGF2E2 "^(3)":| A2
P:B
ABA2 AGFED2 | A3BA2 AFE2D2 |: 
M:6/8
P:C
E2F G2E | F2G A3 | F2G A2F | A2A F3 | B3 A3 | G3 F3 :| 
P:D
FGA FE2 | D6 |] 


\end{abc}
\index{Eglamour}
\addcontentsline{toc}{subsection}{Eglamour}
\begin{abc}[name=latex_gresley5]
X:5
I:linebreak $
T:Eglamour
T:for three
C:Gresley Manuscript, c. 1500
P:AAA BBB CCC DDD
M:C
M:6/8
L:1/8
K:Dmix
P:A (3x)
"^Drone: D/A"A2A G3 | F2F D2D | A2G FE2 | "^         (3)"D6 :: 
P:B (3x)
A2G FE2 | "^         (3)"D6 :: 
P:C (3x)
A3 G3 | F3 "^        (3)"D3 :: 
P:D (3x)
A2G FE2 | "^          (3)"D6 :| 


\end{abc}
\index{Egle}
\addcontentsline{toc}{subsection}{Egle}
\begin{abc}[name=latex_gresley6]
X:6
T:Egle
T:for three
C:Richard Schweitzer
P:AAA B C DD E
M:C
M:6/8
L:1/8
K:Gmaj
P:A (3x)
"^Drone: G/D"G3 F3 | E2F D3 | G2G A2G | B2A G3 | B2A GF2 | E6 | \
G3 F3 | E2F G3 | 
A2A A2G | F2G A3 | B2A G2F | G6 "^(3)":| \
P:B
B2A G2F | G6 || \
P:C
A2A A2G | F2G A3 | 
G2G AGA | B6 | \
A2A A2G | F2G A3 | B2A G2F | G6 |: \
P:D
G3 F3 | E3 G3 | 
A2A A2G | A6 :| \
P:E
G3 F3 | G3 A3 | B2A GF2 | G6 |] 


\end{abc}
\index{Esperans}
\addcontentsline{toc}{subsection}{Esperans}
\begin{abc}[name=latex_gresley7]
X:7
T:Esperans
T:for three
C:Gresley Manuscript, c. 1500
P:AAA BBB CDEFG
M:6/8
L:1/8
K:D dorian
P:A (3x)
"^Drone: D/A"F2F FG2 | ED2 C2C | F2F FG2 | ED2 C2C | F2E F2G | A2A A3 | \
M:9/8
D3 G2F GA2 | \
M:6/8
"^           (3)"G6 :: 
P:B (3x)
c2c A2A | c2c d3 | c2B AA2 | G6 "^(3)":| \
P:C
F2F G2E | D6 | \
M:9/8
P:D
A3 G2E F3 | \
M:6/8
G6 | 
M:9/8
P:E
A3 A3 A3 | \
M:6/8
G6 || \
P:F
A3 F3 | G3 A3 | \
M:9/8
_B3 G3 F3 | \
M:6/8
P:G
G3 A3 | GF2 DE2 | D6 |] 


\end{abc}
\index{Greene Gynger}
\addcontentsline{toc}{subsection}{Greene Gynger}
\begin{abc}[name=latex_gresley8]
X:8
I:linebreak $
T:Greene Gynger
N:Matches Pile 46
T:for two
C:Richard Schweitzer
P:One dance: AABCCDDE
M:6/8
L:1/8
K:G major
P:A
"^Drone: G/D"d2c B2G | A2B A2G | d2c BG2 | A6 :| 
P:B
B3 c3 | A3 G3 | 
A2G FE2 | D6 |: 
P:C
G2G G2A | B6 :: 
P:D
d2c B2G | A2G F2D | 
G2A Bc2 | d6 :| 
P:E
B3 c3 | A3 G3 | d2c BA2 | G6 |] 


\end{abc}
\index{Ly Bens Distonys}
\addcontentsline{toc}{subsection}{Ly Bens Distonys}
\begin{abc}[name=latex_gresley9]
X:9
I:linebreak $
T:Ly Bens Distonys
N:Arr. Kathy Van Stone. Matches Pile 46
C:Gresley Manuscript, c. 1500
T:for two
P:ABBC or ABC
M:6/8
L:1/8
K:G major
P:A
"G"B3 "D"A2B | "C"c2B "D"A2G | "G"B3 "D"A2B | "C"c2A "G"G3 || 
P:B
"G"B2c "D"d3 | "C"e3 "D"d3 | 
"G"g3 "D"d3 | "C"e3 "D"d3 | "G"B2c "D"d3 | "C"e3 "D"d3 | "^Repeat B for 'long' version""C"c2B GA2 | "G"G6 || 
P:C
"G"B2B "D"A2B | "C"c2B "D"A2G | "G"B2B "D"A2B | "C"c2A "G"G3 |]


\end{abc}
\index{Mowbray}
\addcontentsline{toc}{subsection}{Mowbray}
\begin{abc}[name=latex_gresley10]
X:10
T:Mowbray
T:for three
C:Richard Schweitzer
P:AA BCDE
M:6/8
L:1/8
K:Gmaj
P:A
"^Drone: G/D"G3 F3 | G3 E3 | A2G FE2 | D6 :| \
P:B
G3 A3 | B2B B2G | \
G2E FG2 | A6 || \
P:C
G2F G2A | B2B B2G |]


\end{abc}
\index{New Yer}
\addcontentsline{toc}{subsection}{New Yer}
\begin{abc}[name=latex_gresley11]
X:11
T:New Yer
T:for three
C:Richard Schweitzer
P:One dance: AA BBB CCC DDD E
M:6/8
L:1/8
K:G major
P:A
"^Drone G/D"d3 B2A | G2F G3 | E2G FA2 | G6 :: \
P:B (3x)
A2G FG2 | AB2 "^            (3)"G3 :: 
P:C (3x)
G2E2F2 | G2G2 "^      (3)"G2:: \
P:D (3x)
A2G FE2 | D6 "^(3)":| \
P:E
E2G FA2 | G6 |] 


\end{abc}
\index{Newcastell}
\addcontentsline{toc}{subsection}{Newcastell}
\begin{abc}[name=latex_gresley12]
X:12
T:Newcastell
T:for two
C:Gresley Manuscript, c. 1500
P:AA B C DD EE F
M:C
M:6/8
L:1/8
K:Gmaj
P:A
"^Drone: G/D"G3 D3 | E3 D3 | A3 B2A | GF2 E3 | G3 D3 | E3 D3 | A2G FE2 | D6 :| 
P:B
F3 A3 | B3 F3 | F2F A2A | B2B B3 || \
P:C
B3 A3 | dcB AG2 |: \
P:D
B3 F3 | cBA GF2 :: 
P:E
G2A B2c | B2A G3 :| \
P:F
F2F F3 | F2F F3 | G2A B2c | B2A G3 |] 


\end{abc}
\index{Northumberland}
\addcontentsline{toc}{subsection}{Northumberland}
\begin{abc}[name=latex_gresley13]
X:13
T:Northumberland
C:Gresley Manuscript, c. 1500
P:AAA B CCC D E FF G
M:6/8
L:1/8
K:Gmaj
P:A (3x)
"^Drone: G/D"G2G A2A | B2B B2G | F3 E3 | F3 E3 | G2F G2A | B2B B2B | E3 A2G | \
M:9/8
AB2 A6 "^(3)":| 
M:2/4
P:B
E2 D2 | E2 D2 |: \
M:6/8
P:C (3x)
G3 A3 | B3 B2B | AGA4 "^(3)":| \
M:2/4
P:D
E2 D2 | E2 D2 | 
M:6/8
P:E
G3 A3 | B3 B2B | AGA4 |: \
M:2/4
P:F
G2 A2 | F2 G2 | A2 A2 | \
M:3/4
E2A4 :| \
M:6/8
P:G
d2c BA2 | G6 |] 


\end{abc}
\index{Oringe}
\addcontentsline{toc}{subsection}{Oringe}
\begin{abc}[name=latex_gresley14]
X:14
I:linebreak $
T:Oringe
T:for three
C:Richard Schweitzer
P:AA BBB CCC D E
M:C
M:6/8
L:1/8
K:Gmaj
P:A
"^Drone: G/D"d2e d2c | B3 G3 | A2G AB2 | A3 A3 | d2e d2c | BGA G2B | 
d2c AB2 | G6 :: 
P:B
G3 A3 | B3 B3 | G2G A2c | B2c B3 | 
G2G A2d | c2G A3 | G2B cAd | cBA G3 "^(3)":: 
P:C
d3 c3 | B3 G2A "^(3)":| 
P:D
d2e d2c | B3 G3 | d2e d2c | BGA G2B || 
P:E
d2c AB2 | G6 |] 


\end{abc}
\index{Petagay}
\addcontentsline{toc}{subsection}{Petagay}
\begin{abc}[name=latex_gresley15]
X:15
I:linebreak $
T:Petagay
T:for three
C:Gresley Manuscript, c. 1500
C:Music for La Duches
P:A BBB C D
M:C
L:1/8
K:A minor
P:A
"^Drone: A/E"e3f ed/c/BA | cdfe ed/c/BA | c3/d/ef g2e2 | fde2 d4 |: 
M:6/8
P:B (3x)
e3 f3 | a2g e2f | g3 e3 "^(3)":| 
M:C
P:C
eeef ed/c/BA | cdfe ed/c/BA | c3/d/ef g2e2 | fde2 d4 | 
P:D
e3f ed/c/BA | cdfe e4 |] 


\end{abc}
\index{Prenes a Gard}
\addcontentsline{toc}{subsection}{Prenes a Gard}
\begin{abc}[name=latex_gresley16]
X:16
T:Prenes a Gard
T:for three
C:Gresley Manuscript, c. 1500
M:6/8
L:1/8
K:Gmaj
P:A (3x)
"^Drone: G/D"G2G G2G | D2D D2D | G3 BA2 | G6 "^(3)":| \
P:B
d3 B3 | e3 d3 | B2A BA2 | G6 |: 
P:C (3x)
D3 D3 | G3 G3 "^(3)":| \
P:D
B3 ^c3 | d3 d3 | B2c BA2 | G6 || \
P:E
G2D G2D | G2A G3 || \
P:F
B3 c3 | d3 |] 


\end{abc}
\index{Prenes in Gre}
\addcontentsline{toc}{subsection}{Prenes in Gre}
\begin{abc}[name=latex_gresley17]
X:17
T:Prenes in Gre
T:for two
C:Gresley Manuscript, c. 1500
N:Drone: D/A
M:6/8
L:1/8
K:D dorian
P:A (3x)
"^Drone: D/A"D2D E2F | G2G G2G | c2c cB2 | A6 "^(3)":| \
P:B
A2A A2A | G2G G2G | A2G FE2 | D6 |:  \
P:C
F2G | A2G FE2 | D6 :| 
P:D
D2D E2F | G2G G2G | c2c cB2 | A6 || \
P:E
A2A A2A | G2G G2G | A2G FE2 | D6 |] 


\end{abc}
\index{Princitore}
\addcontentsline{toc}{subsection}{Princitore}
\begin{abc}[name=latex_gresley18]
X:18
T:Princitore
T:for two
C:Gresley Manuscript, c. 1500
C:Music for Bayons
P:AAA BBB CCC
M:C
M:6/8
L:1/8
K:Gmaj
P:A (3x)
"^Drone: G/D"G2F E2D | G2F E2D | G3 G3 | BA2 "^        (3)"G3 :: \
P:B (3x)
B3 A2B | c2d B2A |
B2B A2c | BA2 G2G | "^         (3)"G6 :: \
P:C (3x)
G2F E2D | G2F E2D | G3 G3 | BA2 "^        (3)"G3 :| 


\end{abc}
\index{Rawty}
\addcontentsline{toc}{subsection}{Rawty}
\begin{abc}[name=latex_gresley19]
X:19
T:Rawty
T:for two
C:Richard Schweitzer
P:AABC
M:C
M:6/8
L:1/8
K:Gmaj
P:A
"^Drone G/D"G3 D2D | E2F G3 | AAA AG2 | A2B c3 | B2c BA2 | G6 :| 
P:B
A2A A2G | F2G A3 | G2G ABc | B6 | A2A A2G | F2G A3 | d2B cBA | G6 || 
P:C
A2A A2B | c2B A3 | A2A A2B | c2B A3 | d2B cBA | G3 |] 


\end{abc}
\index{Roye}
\addcontentsline{toc}{subsection}{Roye}
\begin{abc}[name=latex_gresley20]
X:20
T:Roye
T:for three
C:Richard Schweitzer
P:AA BBB CCCC DDD
M:C
M:6/8
L:1/8
K:Gmaj
P:A
"^Drone: G/D"G2G A2B | d2d dcB | A2G GA2 | G6 :: \
P:B (3x)
G3 A3 | B3 B3 "^(3)":: 
P:C (4x)
d3 G2B | d2c B"^        (4)"G2 :: \
P:D (3x)
G3 A2B | d3 d2B | A2G G2A | G6 "^(3)":| 


\end{abc}
\index{Sofferance}
\addcontentsline{toc}{subsection}{Sofferance}
\begin{abc}[name=latex_gresley21]
X:21
I:linebreak $
T:Sofferance
T:for two
C:Richard Schweitzer
P:AAA BB CC DD EE
M:C
M:6/8
L:1/8
K:Gmaj
P:A (3x)
"^Drone: G/D"G2G A2B | d2d d2B | A2A ABA | "^             (3)"G6 :: 
P:B
c2c A2A | B2B G2G :: 
P:C
G3 A3 | B2c B3 :: 
P:D
G2G A2B | d2d cBA :: 
P:E
G2B AGF | G6 :| 


\end{abc}
\index{Talbott}
\addcontentsline{toc}{subsection}{Talbott}
\begin{abc}[name=latex_gresley22]
X:1
T:Talbott
C:Gresley Manuscript, c. 1500
M:6/8
L:1/8
K:D mixolydian
P:A (3x)
"^Drone: G/D"d2d B2B | A2B GF2 | E2E E3 | G2G E2E | A2G FE2 | D6 "^(3)":|\
M:3/4
P:B
G2G2G2 | G2G2G2 | 
G2G2G2 | G2G2G2 | A2A2A2 | B6 ||\
P:C
G2G2G2 | B2B2B2 |\
c2B2B2 | A6 | 
B2B2B2 | d2d2d2 | g2g2g2 | f2f4 | d6 | e2e2e2 |\
e2e2e2 |\
M:9/8
eee d2B dee |\
M:6/8
d6 |]


\end{abc}
\index{Tamrett}
\addcontentsline{toc}{subsection}{Tamrett}
\begin{abc}[name=latex_gresley23]
X:23
I:linebreak $
T:Tamrett
T:for two
C:Richard Schweitzer
M:C
M:6/8
L:1/8
K:Gmaj
P:A
"^Drone: G/D"G3 E3 | G3 E3 | D2D E2D | E2F G3 :| 
P:B
G3 E3 | G3 E3 | 
B2B A2G | A2B G3 || 
P:C
BB2 A2G | A2B G3 | dd2 c2B | A3 G3 |: 
P:D
B2B A2G | A2B c3 | d2d c2B | A3 G3 :| 


\end{abc}
\index{Temperans}
\addcontentsline{toc}{subsection}{Temperans}
\begin{abc}[name=latex_gresley24]
X:24
I:linebreak $
T:Temperans
T:for three
C:Gresley Manuscript, c. 1500
P:AAA BCDE
M:C
M:6/8
L:1/8
K:Dmix
P:A (3x)
"^Drone: D/A"D2D E2E | D3 D2D | FFF FE2 | A3 B3 | A2G EF2 | E6 "^(3)":| 
M:9/8
P:B
GGG GGG GGA | 
M:6/8
B2A GF2 | E6 || 
P:C
D2D E2E | D3 D2D | FFF FE2 | A3 B3 | 
A2G EF2 | E6 || 
P:D
G2B2A2 | G6 || 
P:E
D3 G3 | B3 A2A | 
G2G DE2 | D6 |] 


\end{abc}

\index{Whatsoever Ye Wyll}
\addcontentsline{toc}{subsection}{Whatsoever Ye Wyll}
\begin{abc}[name=latex_gresley25]
X:25
T:Whatsoever Ye Wyll
T:for two
C:Richard Schweitzer
M:C
M:6/8
L:1/8
K:Gmaj
P:A
"^Drone: G/D"G2G B2e | d2d d3 | c2B A2c | B3 G3 | G2G B2e | d2d d3 | c2B AB2 | G6 |: \
P:B
A3 c3 | B3 G3 | 
A2B cA2 | G6 :: \
P:C
A2G A2G | A2B c3 | B2c BA2 | G6 :| \
P:D
A3 c3 | B3 G3 | A2B cA2 | G6 |] 
\end{abc}


\chapter{Dances from the Inns of Court}

The dances in this section are from the Inns of Court: professional
associations for English barristers dating to the 15th century. There are
several known manuscripts dating from the mid-16th to mid-17th century
informally describing these dances, eight of which are believed to have been
performed in a fixed order at the beginning of revels at the Inns of Court. We
have preserved that order (for Quadran Pavane through Black Alman) to
facilitate dancing the entire suite, also known as ``The Old Measures''.

Tempos vary wildly, so check with the dancing master for their preference.
Reconstructions vary as well, so check for the desired roadmaps especially for
the more unusual ones such as Turkelone and Tinternell. We have included
suggested numbers of repeats when playing all 8 Old Measures as a suite, but
you may want to play more times through the dance if playing just one of the
dances.

\clearpage
\index{Quadran Pavane}
\addcontentsline{toc}{subsection}{Quadran Pavane}
\begin{abc}[name=latex_inns1]
X:1
T:Quadran Pavane
C:Melody from R.C.M. Ms. 1119
N:arr. Joseph Casazza; matches Pile 2018.
M:C|
L:1/4
K:G major
G/A/ | "G"B3/A/ G"C"e | "G"dc BA | B3/A/ "Em"G3/G/ | "D"A/B/"C"c "Gsus4 "c "G"B | "C"c2 Ac | "Gsus4"cB/A/ c"G"B | "C"c3/c/ c"G"B | "D"A/G/A/B/ "Am"c/B/"D"d/c/ ||
"G"B3/A/ Gd | "G"dB/c/ d/c/B/A/ | B3/A/ "Em"GA/B/ | "Am"cB AG | "D"F3/E/ D"Am"A | "Em"A"G"G "D"F"A"E | "D"F2 F2 | "Am"E"D"D "G"GA ||
"G"B3/A/ G"C"e | "G"dc BA | B3/A/ "Em"G3/G/ | "D"A/B/"C"c "Gsus4 "c"G"B | "C"c2 Ac | "Gsus4"cB/A/ c"G"B | "C"c3/c/ c"G"B | "D"A/G/A/B/ "Am"c/B/"D"d/c/ ||
"G"B"D"A/B/ "C"c"G"B | "D"AG F"C"E | "D"FD "G"G2 | "Dsus4"GF/E/ "D"F/G/E/F/ | "G"G2 d2 | "C"c3/B/ "D"A/B/c/d/ | "G"B2 BB | "G"B4 |]


\end{abc}
\index{Turkelone}
\addcontentsline{toc}{subsection}{Turkelone}
\begin{abc}[name=latex_inns2]
X:2
I:linebreak $
T:Turkelone
N:Arr. Monique Rio. Matches Pile 2018.
C:Willoughby Lute Book, c.1585
P:AA BB CC D x 4
M:6/4
L:1/4
K:G dorian
P:A
"D"^F3/F/FF3/G/F | "Gm"G2"D"F"Gm"G2"D"A | "Bb"B2B"F"A3/B/A | "Bb"B3/B/BB/B/"F"A2 | "Gm"G3/G/G"D"^F3/G/F | "G"G3/G/GGGG ::
P:B
"Gm"G3/G/G"D"^F3/G/F | "Gm"G3/G/G"D"^F2F ::
P:C
"Gm"G3/G/G"Dm"A3/B/A | "Bb"B3/B/BBB"F"A | "Gm"G3/G/G"D"^F3/E/F | "G"G3/G/GGGG :| 
P:D
"C"E3/E/EEEE | "C"E3/E/EE2E | "F"F3/E/F"C"E3/D/E | "D"^F3/E/F/G/FFF |]


\end{abc}
\index{Earl of Essex Measure, the}
\addcontentsline{toc}{subsection}{The Earl of Essex Measure}
\begin{abc}[name=latex_inns3]
X:3
T:The Earl of Essex Measure
P:AAB x 2
C:R.C.M. Ms. 1119
N:Arr. Dave Lankford. Matches Pile 2018
M:3/4
L:1/8
K:G major
P:A
G2 | "G"G3AB2 | "C"c6 | c3de2 | "D"d4c2 | "G"B6- | B4G2 | "G"G3AB2 | "C"c3d"G"B2 | "D"A4G2 | F3EF2 | "G"G6- | G4 :| 
P:B
D2 | "C"E3FG2 | "D"F3EF2 | "G"G4D2 | "C"E3FG2 | "D"F3EF2 | "G"G4D2 | "C"E3FG2 | "D"F3EF2 | "G"G2d2c2 | B3AG2 | "D"F3EF2 | "G"G4 |] 


\end{abc}
\index{Tinternell}
\addcontentsline{toc}{subsection}{Tinternell}
\begin{abc}[name=latex_inns4]
X:4
T:Tinternell
P:A BB C AAA BB C
C:Dallis Lute Book, c.1583
N:Transcribed by Lisa Koch; arr. Al Cofrin. Matches Pile 2018.
M:C
L:1/8
K:F major
P:A (3x second time only)
"Dm"F2 | "A"E3G "Dm"FEF2- | "C"FG2F E2"Dm"D2 | "A"^C2D2 "Asus4"DC/=B,/DC | "D"D4 D2 ::
P:B
"F"CD | "C"E2E2 "Dm"D4- | "A"D^C2D C2 :|\
P:C
"Dm"F2 | "C"E3D E2"Dm"F2 | "Gm"G3F E2D2 | "A"^C2"Dm"D2 "Em"E2"A7"C2 | "D"D4 D2 |]


\end{abc}
\index{Old Alman}
\addcontentsline{toc}{subsection}{Old Alman}
\begin{abc}[name=latex_inns5]
X:5
T:Old Alman
P:AAB x2
C:Anthony Holborne, the Cittharn Schoole, 1597
N:arr. Paul Butler; matches Pile 2018
M:C|
L:1/4
K:G dorian
P:A
"Gm"GA BG | "F"A3/G/ "Dm"F/E/F/D/ | "C"E/F/G G"D"^F | "Gm"G2 G2 :| \
P:B
"Gm"dc BA | "Bb"BA/G/ "F"FG/A/ | 
"Gm"Bc "Bb"d2 | "C"c2 "F"A2 | "Bb"B2 "Gm"G2 | "Dm"A2 "F"F2 | "C"EG G"D"^F | "Gm"G2 G2 |] 


\end{abc}
\index{Queen's Alman}
\addcontentsline{toc}{subsection}{Queen's Alman}
\begin{abc}[name=latex_inns6]
X:6
T:Queen's Alman
N:Arr. Robert Smith; matches Pile 2018
P:AABB x2
C:William Byrd, Fitzwilliam Virginal Book, c.1600
M:C|
L:1/4
K:G dorian
"Gm"GA BG | "D"A/G/^F/E/ F2 | "Cm"G2 "D"A2 | "Gm"D2 D2 ::"Bb"d3/c/ BA | "Bb"BA/G/ FF | 
"Gm"Bc dB | "D"c/B/A/G/ ^FG/A/ | "Gm"B/A/G/F/ "C"EF/G/ | "F"A/G/F/E/ "Bb"D/E/F/D/ | "C"E/F/"Dsus4"G2^F | "G5"G2 G2 :| 


\end{abc}
\index{Madam Sosilia's Alman}
\addcontentsline{toc}{subsection}{Madam Sosilia's Alman}
\begin{abc}[name=latex_inns7]
X:7
T:Madam Sosilia's Alman
P:AABB x2
C:Joseph Casazza
N:Matches Pile 2018
M:C|
L:1/4
K:G major
P:A
"G"Bd    "Am"c"G"B | "D"A2        A"G"G/A/ | Bd       "Am"c3/B/ | "D"A"G"B   "D"A"C"G-  | "G"G"D"F "G"G2 ::\
P:B
"G"BA    G"D"A/B/  | "Am"c"G"B    "D"AA    | "C"e"G"d "C"c"D"d  | "Asus4"d^c "D"d2     ||
"G"BA    G"D"A/B/  | "Am"c"G"B    "D"AA    | "C"e"G"d "C"c"D"d  | "Asus4"d^c "D"d2     || \
"G"d3/c/ B"D"d     | "Am"c/B/c/d/ "C"c"G"B | "D"A"C   G"G2"D"F  | "G"G2      "G"G2     :|


\end{abc}
\index{Black Alman}
\addcontentsline{toc}{subsection}{Black Alman}
\begin{abc}[name=latex_inns8]
X:8
T:Black Alman
C:R.C.M Ms. 1119
N:Arr. Dave Lankford; matches Pile 2018.
M:6/4
L:1/8
K:D minor
D2 |:\
P:A
"Dm"D3EF2 "Gm"G3AG2 | "F"F3EF2 "Gm"G4A2 | B3AG2 "Bb"B3cd2 |  [1 "D"A6- A4D2 :|]  [2 "D"A6- A4d2 ::
P:B
"F"c3BA2 "Gm"B3cd2 | "Dm"A2A2A2 A4d2 | "F"c3BA2 "Gm"B3cd2 | [1 "D"A6- A4d2 :|]  [2 "D"A12 ::
P:C
"Dm"d4e2 f3ed2 | "Am"c3Bc2 "Dm"d4A2 | "Dm"d2a2g2 "D"^f3ed2 | "A"^c3=Bc2 "D"d6 ::
P:D
"Dm"d4e2 f3ed2 | "F"c2d2B2 A4G2 | "Bb"B2A2G2 B2A2G2 |  [1 "C"c2A4 "G"G6 :|]  [2 "C"c2A4 "G"G4A2 :|
P:E
"Bb"B2A2G2 B2A2G2 | "C"c2A4 "Gm"G4A2 | "Bb"B2A2G2 B2A2G2 | "C"c2A4 "G"G4 |]


\end{abc}
\index{Lorayne Alman}
\addcontentsline{toc}{subsection}{Lorayne Alman}
\begin{abc}[name=latex_inns9]
X:9
T:Lorayne Alman
N:Ed. Aaron Elkiss; Matches Pile 2018
C:Pierre Phalese (1571)
M:C
L:1/8
K:G major
P:A
 |:"G"G3A BABc | "Dm"d4 d2cB | "F"A2"G"B2 "C"c2"Dm"d2 | "G"B4 "C"G4 ::\
P:B
"G"G3A B2AG | "D"A4 A4 | 
"G"G3A B2AG | "D"A4 A4 | "G"G3A BABc | "D"d4 d3c | "G"B2"C"AG "D"FEGF | "G"G4 G4 :| 
\end{abc}

\index{New Alman}
\addcontentsline{toc}{subsection}{New Alman}
\begin{abc}[name=latex_inns10]
X:11
T:New Alman
N:Arr. Robert Smith; Matches Pile 2018
P:ABB
C:Bernard Schmid (c. 1577)
M:C
L:1/8
K:C major
P:A
"C"c4 c2"G"B2 | "C"c4 z2"G"B2 | "Am"A2"C"G2 "G"G2"D"^F2 | "G"G4 z2G2 |\
 "C"cB"Dm"AG "Am"ABcA | "G"B2"Am"AB cB"D"AG |
"D"AG^FE FEF2 | "G"G4 z2"Dm"AB |:\
P:B
"C"c2A2 "G"G2EF | "G"G4 z2G2 | "Dm"FEFG A2"C"EF | "G"G4 z2"Am"AB | 
"Am"c2A2 "G"G2EF | "C"G4 z2"Dm"F2 | "C"EDC2 C2"G"B,2 |  [1 "C"C4 z2"Dm"AB :|]  [2 "C"C8 :| 
\end{abc}


\chapter{16th Century Italian Dances}

The major sources for 16th century Italian dances are the published books of
Fabritio Caroso (c. 1526-1605) and Cesare Negri (c. 1535-1605).

Many of the dances included in this collection are {\em cascarda}, a bouncy,
triple time kind of dance unique to Caroso. We have used a 3/4 time signature
for these but the dances should really be felt in 1, with a tempo of
approximately dotted half = 110-120.

The other dances (mostly {\em balletti}) in common time such as Bizzarria and
Lo Spagnoletto should work well with a tempo of half note = 100-110. Some of
these dances shift to 3/4 time partway through; let dotted half note in the 3/4
section = half note in the common time section. 

A few exceptions: Passo e Mezzo is written with doubled note values in cut
time, so use a tempo of whole note = 100-110. There are also a few dances we
have transcribed in 3/4 that are not cascarda such as Contrapasso and
Villanella. For Contrapasso, use a tempo of dotted half = 50-55. For
Villanella, always check with the dance master: it is sometimes danced (at the
same speed) to the music played slowly for 3 repeats and sometimes to the music
played twice as fast for 6 repeats. 

\clearpage
\index{Allegrezza d'Amore}
\addcontentsline{toc}{subsection}{Allegrezza d'Amore}
\begin{abc}[name=latex_16italian1]
X:1
T:Allegrezza d'Amore
C:Fabritio Caroso, il Ballarino, 1581
I:linebreak $
N:Transcribed & arranged by Monique Rio. Matches Pennsic Pile 46
M:3/4
L:1/8
K:C major
g2f2 |:
P:A
"C"e3de2 | c2d2e2 | "Bb"f4f2 | f2g2f2 | "C"e3de2 | "Am"c3dc2 | 
"G"d4d2 |  [1 "G"d2g2f2 :|]  [2 "G"d4"Am"c2 :| 
P:B
"G"d4d2 | d4"Am"c2 | "G"d4d2 | 
"G"d4"Am"c2 | "G"d4d2 | d2d2"C"c2 | "G"d3cB2 | "D"A3GA2 | "G"B4B2 | 
B2g2f2 | "C"e3dc2 | "G"d2c2d2 | "C"e4e2 | e2_B2c2 | "Bb"d4d2 | 
d4"F"c2 | "Bb"d4d2 | d2g2f2 | "C"e3dc2 | "G"d3cd2 | "C"e4e2 | 
e4 |] 


\end{abc}
\index{Alta Regina}
\addcontentsline{toc}{subsection}{Alta Regina}
\begin{abc}[name=latex_16italian2]
X:2
T:Alta Regina
C:Fabritio Caroso, il Ballarino, 1581
N:Transcribed & arranged by Aaron Elkiss. Matches Pennsic Pile 46
P:For Alta Regina:AB x 6 For Squilina:A x 21
M:3/4
L:1/8
K:F major
P:A
"C"G2F2G2 | G4"F"A2 | "Eb"B4B2 | B2G2A2 | \
B2A2B2 | "F"A2G2F2 | "C"G4G2 | G4G2 | \
G2F2G2 | G4"F"A2 | "Bb"B4B2 | B2c2B2 | 
"F"A2G2F2 | "C"G2F2G2 | "F"A4A2 | "F"F2G2A2 || \
P:B
"Bb"B6 | B2A2G2 | "F"A6 | A2G2F2 | \
"C"G4"Dm"F2 | "Bb"F4"C"G2 | "F"A4A2 | A6 :| 


\end{abc}
\index{Bassa Toscana}
\addcontentsline{toc}{subsection}{Bassa Toscana}
\begin{abc}[name=latex_16italian3]
X:3
I:linebreak $
C:Fabritio Caroso, il Ballarino, 1581
T:Bassa Toscana
N:Transcribed by Aaron Elkiss
M:C
L:1/8
M:C
K:F major
P:A (5x)
"Gm"G2G2 GBAG | "D"^F2F2 F2GA | "Gm"BGAB "F"cFGA | "Bb"B2B2 B4 | "F"AGAB cBAG | "F"AGFE FGAB | 
"F"AFGA "Eb"GBAG | "D"^F2F2 F4 | "Gm"G2GA GFEG | "C"GDEF ECDE | "F"FBAG F_EDC | "Bb"D4 D4 | 
"Bb"B2B2 B2"F"A2 | "Gm"GABG cBAG | "D"^F2G2 G2F2 | "G"G4 G4 "^(5)":| 
M:6/8
P:B
"Gm"G2G G2A/G/ | "D"^F2F F2G/A/ | "Bb"B2B "F"A2G/A/ | "Bb"B3 B3 | "F"AGA B2A | "F"A3 A2A | 
"Gm"B2A G2A/G/ | "D"^F2F F3 |:"Gm"G2G G3/E/F | "C"G3 G2D/E/ | "F"F2B/A/ G2F/_E/ | "Bb"D3 D3 | 
"Bb"B2B B2"F"A | "Gm"G2A/B/ B2A/G/ | "D"^F2G G2"D"F | "G"G3 G3 :| 


\end{abc}
\index{Bella Gioiosa}
\addcontentsline{toc}{subsection}{Bella Gioiosa}
\begin{abc}[name=latex_16italian4]
X:4
I:linebreak $
C:Fabritio Caroso, il Ballarino, 1581
T:Bella Gioiosa
N:Transcribed & arranged by Al Cofrin. Matches Pile 2018
P:AA BBB x 7 (or sometimes AA BBB AA x 6)
M:3/4
L:1/8
K:G major
P:A
d2 |:"G"d2c2d2 | G2A2B2 | "C"c2d2e2 | c2d2c2 | "G"B2c2B2 | G2A2G2 | 
"D"A6 | A4d2 | "G"d2c2A2 | G2A2B2 | "C"c2d2e2 | c2d2c2 | 
"G"B2A2G2 | "D"F2G2A2 | "G"G6 | G4 ::
dc | "G"B2A2G2 | "D"F2G2F2 | "G"G4G2 | G4 "^3":| 


\end{abc}
\index{Bianco Fiore, il}
\addcontentsline{toc}{subsection}{Il Bianco Fiore}
\begin{abc}[name=latex_16italian5]
X:5
T:Il Bianco Fiore
C:Cesare Negri, le Grazie d'Amore, 1602
N:Transcribed & arranged by Emma Badowski
M:6/4
L:1/8
K:F major
P:A
"F"F2E2F2 G2A2B2 | c6 A2B2c2 | "Bb"d4"F"c4"Gm"B4 | "F"A6 G2A4 | "Dm"F2E2F2 G2A2B2 | "F"c6 B2A2G2 | \
"Dm"F4"Csus4"F4"C"E4 | "F"F12 ::
P:B
"Dm"A4A4G2F2 | "C"E6 D2C4 | "F"c2B2A4G2F2 | "C"E6 D2C4 | \
"F"F2G2A2 G2F2E2 | "Bb"D6 C2D2E2 | "Dm"F4"Csus4"F4"C"E4 | "F"F12 ::
M:C
P:C
"C"c6B2 | "F"A4 "Dm"F4 | "Gm"G2A2 B2G2 | "F"A4 F4 | c6B2 | A4 G2F2 | \
"C"E4 F4 | "F"F8 :| 


\end{abc}
\index{Bizzarria d'Amore}
\addcontentsline{toc}{subsection}{Bizzarria d'Amore}
\begin{abc}[name=latex_16italian6]
X:6
T:Bizzarria d'Amore
C:Cesare Negri, le Grazie d'Amore, 1602
N:Transcribed & arranged by Monique Rio. Matches Pile 2018.
P:AA BB CC x 6
M:C
L:1/8
K:F major
P:A
c2 | "F"c2A2 B2c2 | "Bb"d3c B2d2 | "F"c2B2 "C"A2G2 | A4 z2A2 |\
"C"G2E2 "F"F2D2 | "C"C4 "F"c2BA | "Csus4"G2F2 "C"F2E2 | "F"F4 z2 ::
P:B
GA | "Gm"B4 "F"A4 | "C"G4 "Bb"d2cB | "F"A2"G"G2 "Dsus4"G2"D"^F2 | "G"G4 z2 ::\
P:C
GF | "C"E2C4GF | E2C4AB | "Am"c2"Bb"BA "C"G2G2 | "F"F4 z2 :|


\end{abc}
\index{Caccia d'Amore, la}
\addcontentsline{toc}{subsection}{La Caccia d'Amore}
\begin{abc}[name=latex_16italian7]
X:7
I:linebreak $
T:La Caccia d'Amore
N:Edited by Aaron Elkiss. Matches Pile 2018
C:Giovanni Giacomo Gastoldi, Balletti a cinque voci, 1591
M:C
L:1/8
K:D minor
"F"f2 | "Gm"d2"F"f2 "Bb"f2"C"e2 | "F"f4 f2f2 | "Gm"d2"F"f2 "Bb"f2"C"e2 | "F"f4 z2ff | "C"e2e2 "Dm"d2d2 | "A"^c2c2 z2"F"ff | 
"C"e2"Dm"d2 "Asus4"d2"A"^c2 | "D"d4 z2 ::"Dm"d2 | "C"e2"G"d2 "Am"e2"D"^f2 | "G"g4 g2"F"c2 | "Bb"d2"F"c2 "Gm"d2"C"e2 | "F"f4 z2ff | 
"C"e2ee "Dm"d2dd | "A"^c2c2 z2"F"ff | "C"e2"D"d2 "Gm"d2"A"^c2 | "D"d6 :| 


\end{abc}
\index{Candida Luna}
\addcontentsline{toc}{subsection}{Candida Luna}
\begin{abc}[name=latex_16italian8]
X:8
I:linebreak $
T:Candida Luna
C:Fabritio Caroso, il Ballarino, 1581
N:Transcribed & arranged by Aaron Elkiss. Matches Pile 2018
P:AA BB CC x 3
M:3/4
L:1/8
K:C major
P:A
"C"c2 | "G"B3AB2 | "Em"G3AB2 | "F"A3GA2 | A3B"C"c2 | "G"B4"F"A2 | A4"G"B2 | 
"C"c4c2 | c4 ::
P:B
"C"c2 | "G"B3AB2 | "C"G3AB2 | "F"A3GA2 | "Dm"F3GA2 | 
"C"G3FG2 | E3FG2 | "Dm"F3EF2 | "Bb"D3EF2 | "C"E3DE2 | E4"G"D2 | 
"Am"E3DE2 | E3DC2 | "G"D4"F"C2 | C4"G"D2 | "C"E4E2 | E4 ::
P:C
DE | 
"Bb"F3EF2 | D3EF2 | "Am"E3DE2 | E3DC2 | "G"D4"F"C2 | C4"G"D2 | 
"C"E4E2 | E4 :| 


\end{abc}
\index{Castellana, la}
\addcontentsline{toc}{subsection}{La Castellana}
\begin{abc}[name=latex_16italian9]
X:9
T:La Castellana
C:Fabritio Caroso, il Ballarino, 1581
N:Transcribed & arranged by Aaron Elkiss. Matches Pennsic Pile 46
P:AABBCC x 3
M:3/4
L:1/8
K:D minor
P:A
D2E2 |:"Dm"F3EF2 | "C"G3FG2 | "F"A4A2 | A3B"Dm"c2 | "Gm"B4"F"A2 | "C"G3FG2 | \
"F"A4A2 |  [1 "F"A2D2E2 :|]  [2 "F"A2E2F2 ::
P:B
"C"G4G2 | G2E2F2 | G4G2 | \
"C"G2A2G2 | "Dm"A4"C"G2 | "Dm"F4"G"D2 | "A"E4E2 | E4F2 ::
P:C
"C"G3FG2 | \
E3FG2 | "Dm"F3ED2 | "Em"E3^CD2 | "A"E4"Dm"D2 | "G"D4"A"^C2 | "D"D4D2 | \
 [1 "D"D4F2 :|]  [2 "D"D2 :| 


\end{abc}
\index{Chiara Stella}
\addcontentsline{toc}{subsection}{Chiara Stella}
\begin{abc}[name=latex_16italian10]
X:10
T:Chiara Stella
C:Fabritio Caroso, il Ballarino, 1581
N:Transcribed & arranged by Dennis Sherman. Matches Pile 2018
P:ABB x 4
M:3/4
L:1/8
K:D minor
P:A
"A"^C2D2E2 | E4E2 | E4E2 | "Dm"F4G2 | "F"A6 | "C"G2F2G2 | \
"F"A6 | A4G2 | "Dm"F3ED2 | "A"^C2D2E2 
| "Dm"D4D2 | D4D2 | \
"A"^C2D2E2 | E4E2 | E4E2 | "Dm"F4G2 | "F"A6 | "C"G2F2G2 | \
"F"A6 | A4G2 | "Dm"F3ED2 
| "A"^C2D2E2 | "Dm"D4F2 | F3ED2 | \
"A"E4^C2 | "G"=B,3^CD2 | "A"E4"Dm"F2 | F3ED2 | "A"E4^C2 | "G"=B,3^CD2 |\
"A"E4E2 | E6 |:
P:B
"F"A4A2 | A4"C"G2 | "F"A4A2 | A4"C"G2 | \
"Dm"F3ED2 | "A"^C2D2E2 | "Dm"D4D2 | D6 :| 


\end{abc}
\index{Chiaranzana}
\addcontentsline{toc}{subsection}{Chiaranzana}
\begin{abc}[name=latex_16italian11]
X:11
I:linebreak $
T:Chiaranzana
C:Fabritio Caroso, il Ballarino, 1581
N:Transcribed & arranged by Emma Badowski. Matches Pennsic Pile 46
M:6/4
L:1/8
K:A minor
"F"A4A2 B2c2d2 | "C"e2d2c2 g2f2e2 | "Dm"f2e2d2 f2e2d2 | "A"^c2B2A2 G2F2E2 | "F"F2G2A2 B2c2d2 | "C"e2g2f2 e2d2c2 | 
"G"B2f2e2 d2"A"d2^c2 | "D"d2e2f2 e2d2c2 | "G"B2GABc d2"A"d2^c2 | "D"d4d2 e2d2c2 | "G"B2GABc d2"A"d2^c2 | "D"d8d4 ::
M:3/4
"F"f4f2 | f3ed2 | "C"e3dc2 | g2f2e2 | "Dm"f3ed2 | f3ed2 | 
"A"^c3BA2 | G2F2E2 | "F"F3GA2 | "G"B3cd2 | "C"e3dc2 | e2d2c2 | 
"G"B3ed2- | "A"d2d2^c2 | "D"d3e^f2 | e2d2c2 | "G"B3ed2- | "A"d2d2^c2 | 
"D"d6 | e2d2c2 | "G"B3ed2- | "A"d2d2^c2 | "D"d4d2 | d6 :| 


\end{abc}
\index{Contentezza d'Amore}
\addcontentsline{toc}{subsection}{Contentezza d'Amore}
\begin{abc}[name=latex_16italian12]
X:12
T:Contentezza d'Amore
C:Cesare Negri, le Grazie d'Amore, 1602
N:Transcribed & arranged by Robert Smith. Matches Pennsic Pile 46
P:Ax5 B Cx3
M:C
L:1/8
K:F major
AB |:\
P:A
"F"c2c2 c2"Gm"B2 | "F"A2A2 A2GA | "Bb"B2B2 B2"F"A2 | "Gm"G2G2 G2AB | "F"c2c2 c2"Gm"B2 | "F"A3G A2GF |
"G"G2G2 G2"D"D2 | "G"G2G2 G2z2 | "F"A2AB A2"C"G2 | "F"A2A2 A2GA | "Gm"B2B2 B2"F"A2 | "G"G2G2 G2z2 |
"F"ABAG FEDC | "G"D4 G4 | "C"G2G2 G2"Dm"F2 | "C"G2GG cBAG | "F"ABAG FEDC | "G"G4 C2G2 |
"C"G2G2 G2"Dm"F2 | "C"G2G2 cBAG | "F"AF"C"GA "Bb"BAGF | "C"G2"F"F2 "Bb"F2"C"FE | "F"A2A2 A2"C"G2 |  [1-4 "F"A3A A2AB :|] \
 [5 "F"A3A A4 || 
M:3/4
P:B
"F"c4"Gm"B2 | "F"A2G2A2 | "Gm"B4"D"A2 | "Gm"G4G2 | G3AB2 | B3AG2 | \
G4"D"^F2 | "G"G4G2 | 
"F"c4"Gm"B2 | "F"A3GA2 | "Gm"B4"D"A2 | "Gm"G4G2 | \
G4F2 | "C"E3DC2 | "Gsus4"D6 | "C"E4E2 |:
P:C
"C"G4"Dm"F2 | "C"E3DC2 | \
"G"D2C2"G"D2 | "C"E4E2 | "F"c4"Gm"B2 | "F"A3GF2 | "C"G2F2G2 |  [1-2 "F"A4A2 :|] \
 [3 "F"A6 |] 


\end{abc}
\index{Conto Dell'Orco, il}
\addcontentsline{toc}{subsection}{Il Conto Dell'Orco}
\begin{abc}[name=latex_16italian13]
X:13
T:Il Conto Dell'Orco
C:Fabritio Caroso, il Ballarino, 1581
N:Transcribed & arranged by Dave Lankford. Matches Pennsic Pile 46
P:(AABB)x2 Cx2 or 3
M:C
L:1/8
K:C major
P:A
 EF | "C"G2"F"A2 "C"G2EF | "C"G2"F"A2 "C"G2EF | "C"GEFG "F"A3/G/FE | "G"DCCD "C"E2 ::
P:B
cd | "C"e2"F"f2 "C"e2cd | "C"e2"F"f2 "C"e2cd | \
"C"ecde "F"f3/e/dc | "G"BAAB "^Repeat AABB!""C"c2 ::
P:C
EF | "C"G2"F"A2 "C"G2EF | "C"G2"F"A2 "C"G2EF | "C"GEFG "F"A3/G/FE | "G"DCCD "C"E2 :| 


\end{abc}
\index{Contrapasso}
\addcontentsline{toc}{subsection}{Contrapasso}
\begin{abc}[name=latex_16italian14]
X:14
T:Contrapasso
C:Fabritio Caroso, Nobiltà di Dame, 1600
N:Transcribed & arranged by Monique Rio. Matches Pile 2018.
%%text for Contrapasso in Due & in Ruota:AAA BBB AA BBB
%%text for Contra Passo (Chigi):AA BBB AA BBB
%%text for Contrapasso Nuovo:AAA BBB AAA BBB
M:3/4
L:1/8
K:F major
P:A
DE | "F"F2F2"Csus4"G2 | "F"A2AGF_E | "Bb"D2F2"C"G2 | "F"A2A2DE | FGAF"C"GE | "F"FBAGF_E | "Bb"DE"Dm"F2"C"GE | "F"F2F2 ::
P:B
AB |"F"c2cFGA | "Bb"B2B2B2 | B2B3G | "F"A2A2AB | c2cFGA | "Bb"B2B2B2 | B2B3G | "F"A2A2z2 |
FGAFGA | "Bb"B2B2"F"A2 | "Eb"G2G2"Dm"F2 | "C"G2G2GB | "F"ABAGFE | "Bb"D2"C"EG"Dm"FA | "Bb"GF"Csus4"G2"C"E2 | "F"F2F2 :|

\end{abc}
\index{Dimostra, lo}
\addcontentsline{toc}{subsection}{Lo Dimostra}
\begin{abc}[name=latex_il_papa2]
X:2
I:linebreak $
T:Lo Dimostra
C:Nathan Kronenfeld
P:AAA BB C
M:6/4
L:1/8
K:C major
P:A
 |: "G"G3AB2 B3cd2 | "C"e3dc2 c3de2 | "D"d4A2 A3Bc2 | "D"d4A2 A3BA2 | "Em"G4E2 E6 | "Am"G3AB2 B3cB2 | 
"G"c3BA2 A3Bc2 | "C"B3AG2 G3AG2 | "C"G3FE2 "Dm"D4C2 | "G"C3DE2 "C"D3EF2 | "F"G3AB2 c4G2 | "F"A3GF2 "D"F3GA2 | 
"G"A3Bc2 d4A2 | "G"G3AB2 "F"d3cB2 | "C"B3AG2 A3BA2 | "G"G3FE2 "C"E3FG2 | "C"G3AB2 "^ (3)"c6 :: 
P:B
"F"G3FE2 E3FG2 | 
"G"A3GF2 F3GA2 | "G"B3AG2 "C"G3AB2 |  [1 "G"G3AB2 "C"c4A2 :|]  [2 "Am"G3AB2 c6 :| 
P:C
"G"c3Bc2 A3Bc2 | "F"B3AB2 G3AB2 | 
"G"A3GA2 "C"F3GA2 | G3AB2 c6 |] 


\end{abc}
\index{Fedelta}
\addcontentsline{toc}{subsection}{Fedelta}
\begin{abc}[name=latex_16italian15]
X:15
T:Fedelta
C:Fabritio Caroso, il Ballarino, 1581
N:Transcribed & arranged by Aaron Elkiss. Matches Pile 2018
P:AAB x 3
M:3/4
L:1/8
K:D major
P:A
"D"F2 |:F4F2 | F4F2 | "G"G4G2 | G4G2 | "D"F4F2 | "A"E4D2 | \
E4E2 | "A"E4"D"F2 
| "D"F4D2 | E4F2 | "G"G4G2 | "D"F4F2 | \
"A"E4"D"D2 | "A"C4E2 | "D"D4D2 | D4F2 :| 
P:B
"G"G4"D"F2 | "A"E4E2 | \
"D"F4F2 | F4F2 | "G"G4"D"F2 | "Asus4"E4"A"E2 | "D"D4D2 | D4 |] 


\end{abc}
\index{Fiamma d'Amore}
\addcontentsline{toc}{subsection}{Fiamma d'Amore}
\begin{abc}[name=latex_16italian16]
X:16
T:Fiamma d'Amore
C:Fabritio Caroso, il Ballarino, 1581
N:Transcribed & arranged by Al Cofrin. Matches Pennsic Pile 46
P:AA B x 4
M:3/4
L:1/8
K:D minor
P:A
 |:F2 | "Dm"F3EF2 | "C"G3FG2 | "F"A4A2 | A4"C"G2 | "Bb"F3ED2 | "A"E3DE2 | "D"D4D2 | D4 :|
P:B
F2 |"Dm"F3EF2 | "C"G3FG2 | "F"A4A2 | A4c2 | "Gm"B4A2 | "C"G3FG2 | "F"A4A2 | A4F2 |
"Dm"F3EF2 | "C"G3FG2 | "F"A4A2 | A4G2 | "Bb"F3ED2 | "A"E3DE2 | "D"D4D2 | D6 |
M:2/4
"Bb"F2 FG | "F"A4 | "Bb"F2 FG | "F"A4 |\
M:3/4
"F"A4"C"G2 | "Bb"F3ED2 | "A"E3DE2 | "D"D4D2 | D4 |]


\end{abc}
\index{Ballo del Fiore}
\index{Torche, Bransle de la}
\index{Bransle!Torche, de la}
\index{Fiore, Ballo del}
\addcontentsline{toc}{subsection}{Ballo del Fiore}
\begin{abc}[name=latex_16italian17]
X:17
T:Ballo del Fiore
T:Bransle de la Torche
C:Michael Praetorius, Terpsichore, 1612
P:Intro:A; one dance = (AB)x4
N:Edited by Aaron Elkiss. Matches Pile 2018
M:C
L:1/8
K:D dorian
P:A
"Dm"d3e f2f2 | "C"e4 e2e2 | "Dm"d3d d2d2 | "A"^c4 A4 | "Dm"d3e f2f2 | "C"e3f g2"Am"a2 | \
"Dm"f2ed "A"^cde2 | "D"d4 d4 || 
P:B
"F"a3g fgaf | "Em"g2ef geag | "Dm"f3e defg | "A"a4 a4 | \
"F"a3g fgaf | "C"g3f efge | "Dm"f2gf "A"ede2 | "D"d4 d4 :| 


\end{abc}
\index{Florido Giglio}
\addcontentsline{toc}{subsection}{Florido Giglio}
\begin{abc}[name=latex_16italian18]
X:18
T:Florido Giglio
C:Fabritio Caroso, il Ballarino, 1581
N:Transcribed & arranged by Aaron Elkiss. Matches Pennsic Pile 46
P:AABBCDD AABBCCDDx2 AABBCDD
M:3/4
L:1/8
K:G mixolydian
P:A
d2c2 | "G"B3AB2 | G2A2B2    | "F"c4c2  | c2d2c2 | "G"B2A2G2 | "D"A2G2A2 | "G"B6 | B2 ::
P:B
F2G2 | "F"A4A2  | A2A2B2    | c4c2     | c2d2c2 | "G"B2A2G2 | "D"A2G2A2 | "G"B6 | B2 ::
P:C (No repeat for 1st & 4th verses)
C2D2 | "C"E4G2  | "D"A2G2A2 | "G"B3AG2 | F2E2D2 | "C"E3^FG2 | "D"A2G2A2 | "G"B6 | B2 ::
P:D
F2G2 | "D"A6    | A2^F2G2   | "D"A6    | A3dc2  | "G"B2A2G2 | "D"A2G2A2 | "G"B6 | B2 :|


\end{abc}
\index{Fulgente Stella}
\addcontentsline{toc}{subsection}{Fulgente Stella}
\begin{abc}[name=latex_16italian19]
X:19
I:linebreak $
T:Fulgente Stella
C:Fabritio Caroso, il Ballarino, 1581
N:Transcribed & arranged by Aaron Elkiss. Matches Pennsic Pile 46
P:AABB x4
M:3/4
L:1/8
K:F major
P:A
 | "G5"G4A2 | "Gm"B3AG2 | "D"A4A2 | "A"A6 | "G5"G2A2B2 | "Gm"B3AG2 | 
"D"A4A2 | A6 ::
M:2/4
P:B
"Bb"B4 | B2 "F"c2 | "Bb"d2 d2 | d2 d2 | d2 "F"c2 | "Gm"B3/A/ G2 | 
"F"A4 | A2 "C"G2 | "F"A2 "Bb"B2 | "F"c2 A2 | "Gm"B2 A2 | "C"G2 "F"F2 | 
"C"E^F G2 | "Dsus4"A2 "D"^F2 | "G"G4 | G4 :| 


\end{abc}
\index{Furioso all'Italiana}
\addcontentsline{toc}{subsection}{Furioso all'Italiana}
\begin{abc}[name=latex_16italian20]
X:20
I:linebreak $
T:Furioso all'Italiana
C:Fabritio Caroso, Nobiltà di Dame, 1600
N:Transcribed & arranged by Al Cofrin. Matches Pennsic Pile 46
P:Ax10 Bx3 C Bx2 C B
M:C
L:1/8
K:G major
BA |:
P:A
"G"B2B2 c2A2 | B2B2 B2dc | BGAB GFEG | "D"A2A2 A2A2 | A2A2 G2F2 | "C"E2ED EFG2 | 
"Dsus4"A2G2 G2GF |  [1-9 "G"G2G2 G2BA :|] 
M:3/4
 [10 "G"G4G2 | "G"G4B2 ::
P:B
"G"B3AB2 | "Am"c3Bc2 | "G"B4B2 | B4d2 | 
"Am"c4B2 | "D"A4G2 | A4A2 | A4A2 | A3GA2 | "Em"G4F2 | 
"C"E4E2 | E3FG2 | "D"A4G2 | "C"G4"D"F2 | "G"G4G2 |  [1-2 G4B2 :|] 
 [3 "G"G6 :| 
M:3/2
P:C
"G"B4B4A2B2 | "C"c4c4c4 | c4c8 | "G"B4B4B4 | "Em"G4"F"A4"G"B4 | "C"c4c2B2A2G2 | 
F2E2"Dsus4"G4F4 | "G"G4G4z2"^To Bx2 C B"B2 |] 


\end{abc}
\index{Giunto m'ha Amore}
\addcontentsline{toc}{subsection}{Giunto m'ha Amore}
\begin{abc}[name=latex_16italian21]
X:21
T:Giunto m'ha Amore
N:Transcribed & arranged by Dave Lankford. Matches Pennsic Pile 46.
C:Fabritio Caroso, il Ballarino, 1581
P:AABBx5
M:3/4
L:1/8
K:D minor
P:A
"Dm"F2 |:F3EF2 | "C"G3FG2 | "F"A6 | A4G2 | "Dm"F3ED2 | "A"^C3DE2 | "Dm"D4D2 | D4F2 ::\
P:B
"Dm"F3EF2 | "C"G3FG2 | 
"F"A6 | A4c2 | c3Bc2 | "Gm"B3cB2 | "F"A6 | A4G2 |\
"Dm"F3ED2 | "Am"E3DE2 | "Dm"D4D2 |  [1 "Dm"D4F2 :|]  [2 "Dm"D4 :|


\end{abc}
\index{Gloria d'Amore}
\addcontentsline{toc}{subsection}{Gloria d'Amore}
\begin{abc}[name=latex_16italian22]
X:22
T:Gloria d'Amore
N:Transcribed & arranged by Dave Lankford. Matches Pennsic Pile 46.
C:Fabritio Caroso, il Ballarino, 1581
T:Cascarda
P:Play five times
M:3/4
L:1/8
K:D minor
"Gm"G4G2 | G3AG2 | "D"^F4F2 | ^F4GA | "Bb"B4B2 | "F"A4GA | "Bb"B4B2 | B6 | "F"A4GA | 
A3Bc2 | "Bb"B4B2 | B4A2 | "Gm"G3AG2 | "D"^F2G2F2 | "G"G4"Gm"B2 | "Gm"B3AG2 | "D"^F4D2 | 
D4E2 | ^F4F2 | ^F6 | "F"A3GA2 | A3Bc2 | "Bb"B4B2 | B4A2 | "Gm"G3AG2 | "D"^F2G2F2 | "G"G6 | G6 :|


\end{abc}
\index{Gracca Amorosa}
\addcontentsline{toc}{subsection}{Gracca Amorosa}
\begin{abc}[name=latex_16italian23]
X:23
T:Gracca Amorosa
C:Fabritio Caroso, il Ballarino, 1581
P:Play five times
M:6/8
L:1/8
K:F major
   "F"c3   c2B    | A3     A2c     | c2B       A2A       |     \
   "C"G3   G2C    | G3     G2F     | G3        G2F       |
   "Gm"G2A B2G    | "F"A3  A3      | c3        c2B       |     \
   A3      A2c    | c2B    A2A     | "C"G3     G2F       |
|:"Bb"F2F "C"G2G | "F"A2A "Eb"G2B | "F"A3/G/F "C"G3/F/G | [1  "F"A3 A2F :|]  [2  "F"A3 A3 :|]
\end{abc}

\index{Lucretia}
\addcontentsline{toc}{subsection}{Lucretia}
\begin{abc}[name=latex_il_papa_3]
X:3
T:Lucretia
P:AA B CC x5
M:6/4
L:1/8
C:Nathan Kronenfeld
K:C major
P:A
 |: "G"d4c2 B4AG | "D"A2B2A2 "G"G6 | "G"G4d2 d4cB | "C"c4e2 e4dc | "G"B4d2 d4cB | "D"A4G2 A6 :| 
P:B
"G"G6 "D"^F4A2 | "G"B2c2B2 d6 | "G"G4d2 d4cB | "C"c4e2 e4dc | "G"B4d2 d4cB | "D"A4G2 A6 |: 
P:C
"G"G6 "D"^F4A2 | "G"B2c2B2 d6 | "G"d4c2 B4AG | "D"A2B2A2 "G"G6 :| 


\end{abc}
\index{Maraviglia d'Amore}
\addcontentsline{toc}{subsection}{Maraviglia d'Amore}
\begin{abc}[name=latex_16italian24]
X:24
I:linebreak $
C:Fabritio Caroso, il Ballarino, 1581
T:Maraviglia d'Amore
N:Transcribed & arranged by Aaron Elkiss. Matches Pennsic Pile 46
P:ABBCC x 4
M:3/4
L:1/8
K:G major
P:A
"G"B2 | "D"A2B2"Em"G2 | "D"A4"G"B2 | "C"c4c2 | "C"c4"G"B2 | "F"A2B2A2 | "Em"G2F2"C"G2 | 
"D"A4A2 | A4"G"B2 | "D"A2B2"C"G2 | "F"A4"G"B2 | "C"c4c2 | c2d2"G"B2 | 
"D"A2B2"Em"G2 | "C"E2G2"D"F2 | "G"G4G2 | G2B2c2 |:
P:B
"G"d4d2 | d2e2"Am"c2 | 
"G"B4B2 | B2c2d2 | "Am"c4"G"B2 | "D"A2B2c2 | "G"B4B2 |  [1 "G"B2B2c2 :|] 
 [2 "G"B4B2 ::
P:C
"D"A4"C"G2 | "D"A4"G"B2 | "C"c4c2 | c2d2c2 | "G"B2A2G2 | 
"D"A2G2A2 | "G"B4B2 |  [1 "G"B4B2 :|]  [2 "G"B4 |] 


\end{abc}
\index{Ombrosa Valle}
\addcontentsline{toc}{subsection}{Ombrosa Valle}
\begin{abc}[name=latex_16italian25]
X:25
T:Ombrosa Valle
C:Fabritio Caroso, il Ballarino, 1581
N:Transcribed & arranged by Aaron Elkiss. Matches Pennsic Pile 46
P:AB x 7
M:C
L:1/8
K:C major
P:A
"C"g4 g2"Dm"f2 | "C"e4 cBcd | e2"G"d2 "Am"c2dc | "G"d4 d2d2 | d3c dcBA | B2G2 d2"F"c2 | 
"Bb"d2"C"e2 "Dm"f2"G"d2 | "C"e4 e4 | "C"g4 g2"Dm"f2 | "C"e4 cBcd | "C"e2"G"d2 "Am"c2dc | "G"d4 d2"F"c2 || 
P:B
"Bb"f2f2 "G"d2d2 | "C"e2e2 "Bb"d2"Dm"f2 | "C"e2e2 "G"d2d2 | "C"e4 e2"F"c2 | "F"c2c2 "G"d2df | "C"e2e2 "Bb"d2"Dm"f2 | \ 
"C"e2dc "G"B2AB |  [1-6 "C"c4 c2c2 :|]  [7 "C"c4 c4 :| 


\end{abc}
\index{Passo e Mezzo}
\addcontentsline{toc}{subsection}{Passo e Mezzo}
\begin{abc}[name=latex_16italian26]
X:26
T:Passo e Mezzo
C:Fabritio Caroso, il Ballarino, 1581
N:Transcribed & arranged by Dave Lankford. Matches Pennsic Pile 46.
%%text for Passo e Mezzo:11 times through
%%text for Dolce Amoroso Fuoco:5 times through
%%text for Ardente Sola:7 times through
M:C|
L:1/4
K:G dorian
"Gm"d2 d2 | dc BA | Bf ed | cB AG | "F"A2 A2 | AG AB | cd cB | AF GA |
"Gm"B2 d2 | dc BA | Bd cB | "D"AG ^FE | ^F2 F2 | ^F2 d2 | dc de | fe dc |
"Gm"d2 d2 | dc BA | Bf ed | cB AG | "F"A2 A2 | AG AB | "F"c2 "Gm"B2 | "F"AF "Dm"GA |
"Gm"Bd cB | "D"AG ^FE | ^F2 "G"G2 | "Am"A2 "D"A2 | "G"G2 G2 | G2 "D"^F2 | "G"G4- | G4 |]


\end{abc}
\index{Rose e Viole}
\addcontentsline{toc}{subsection}{Rose e Viole}
\begin{abc}[name=latex_16italian27]
X:27
I:linebreak $
T:Rose e Viole
N:Transcribed & arranged by Paul Butler. Matches Pennsic Pile 46
C:attrib. Antonio Casteliono, 1536
P:AABB
M:3/4
L:1/8
K:C major
P:A
de | "F"f4f2 | f4e2 | "Dm"d4d2 | d2"C"c2c2 | "G"B4B2 | "Am"c2A2c2 | 
"Em"B6- | B4de | "F"f4f2 | f4e2 | "Dm"d4d2 | d2"C"c2c2 | 
"G"B4c2 | "Am"BAc2B2 | "C"c4c2 | c4de | "F"f4f2 | f4e2 | 
"Dm"d4d2 | d2"C"c2c2 | "G"B4B2 | "Am"c2e2c2 | "Em"B6- | B4de | 
"F"f4f2 | f4e2 | "Dm"d4d2 | d2"C"c2c2 | "G"B4c2 | "Am"BAc2B2 | 
"C"c4c2 | c4 ::
P:B
de | "F"f4f2 | f4e2 | "Dm"d4d2 | d2"C"c2c2 | 
"G"B4B2 | "Am"c2A2c2 | "Em"B6- | B4de | "F"f4f2 | f4e2 | 
"Dm"d4d2 | d2"C"c2c2 | "G"B4c2 | "Am"BAc2B2 | "C"c4c2 | c6 | 
"F"AGABcA | "Dm"d6 | "G"GABcBA | "Em"e6 | "Am"A2c2d2 | "G"d4B2 | 
"Am"A4B2 | "C"e3fe2 | "F"c6 | "Dm"d3ed2 | "G"B2c2d2 | "Em"e6 | 
"Am"ABcABc | "Dm"d4c2 | "C"c3cB2 | "C"c4 :| 


\end{abc}
\index{Se pensando al partire}
\addcontentsline{toc}{subsection}{Se pensando al partire}
\begin{abc}[name=latex_16italian28]
X:28
T:Se pensando al partire
C:Fabritio Caroso, il Ballarino, 1581
T:Balletto
N:Transcribed & arranged by Emma Badowski
%P:AA BBB
M:C
L:1/8
K:F major
P:A
F2 |:"Bb"F2F2 "C"G2G2 | "F"A2A2 A2"C"G2 | "Bb"F2"Gm"ED "Asus4"E4 | "Dm"D6D2 | \
D2"Am"E2 "Bb"F2"C"G2 | "F"A2A2 A2"C"G2 | "Dm"F2ED "Asus4"E2E2 | "D"^F2F2 F4 ::
P:B
"F"A4 "C"G3A | "Gm"B2B2 "F"A2c2 | "Gm"B2"F"A2 "C"G4 | "F"A4 A4 | \
A4 "C"G3A | "Gm"B2B2 "F"A2c2 | "Gm"B2"F"A2 "C"G4 | "F"A8 |
A2A2 "Dm"A2"C"G2 | "Dm"A4 "Bb"F4 | "Bb"F3E "Gm"G2"Dm"F2 | "Asus4"E4 "Dm"D4 | \
D2A2 A2"C"G2 | "Dm"A4 "Bb"F4 | F3A "Gm"G2"Dm"F2 | "Asus4"E4 "D"^F4 |
"Bb"F2F2 "C"G2G2 | "F"A2A2 A2"C"G2 | "Bb"F2"Gm"ED "Asus4"E4 | "Dm"D6 D2 | \
D2"Am"E2 "Bb"F2"C"G2 | "F"A2A2 A2"C"G2 | "Dm"F2ED "Asus4"E2E2 | "D"^F2F2 F4 "^(3)":|


\end{abc}
\index{Spagnoletta}
\addcontentsline{toc}{subsection}{Spagnoletta}
\begin{abc}[name=latex_16italian29]
X:29
I:linebreak $
T:Spagnoletta
C:Fabritio Caroso, il Ballarino, 1581
N:Transcribed & arranged by David Yardley. Matches Pennsic Pile 46
P:5 times through (6 for Spagnoletta Nuova)
M:3/4
L:1/8
K:G dorian
GA | "Gm"d4d2 | "F"c3Bc2 | "Bb"d4d2 | c2d2e2 | f3ed2 | "F"c3Bc2 | 
"Bb"d4d2 | "D"^F2G2A2 | "Gm"d4d2 | "F"c3Bc2 | "Bb"d4d2 | c2d2e2 | 
f3ed2 | "F"c3Bc2 | "Bb"d4d2 | c2d2e2 | f3ed2 | "Gm"d3cB2 | 
"F"c4c2 | c4c2 | "Gm"B3AG2 | "D"A3GA2 | "Gm"d4d2 | "C"c2d2e2 | 
"Bb"f3ed2 | "Gm"d3cB2 | "F"c4c2 | c3Bc2 | "Gm"B3AG2 | "D"d4d2 | 
"Gm"z2zAB2 | B3AG2 | "D"A4"C"G2 | "D"A3GA2 | "Gm"z2zdd2 | d3cB2 | 
"D"A4"C"G2 | "D"A3GA2 | "Gm"z2zAB2 | B3AG2 | "D"A4"C"G2 | "D"A3GA2 | 
"Gm"z2zdd2 | d3cB2 | "D"A4"C"G2 | "D"A3GA2 | "G"d4 :| 


\end{abc}
\index{Spagnoletto, lo}
\addcontentsline{toc}{subsection}{Lo Spagnoletto}
\begin{abc}[name=latex_16italian30]
X:30
T:Lo Spagnoletto
C:Cesare Negri, le Grazie d'Amore, 1602
N:Transcribed & arranged by Dave Lankford. Matches Pennsic Pile 46
P:AABBCC x 7
M:C
L:1/8
K:F major
P:A
 B2 | "Gm"B2AB "F"c2Bc | "Bb"d2d2 B2d2 | "F"c2B2 c2c2 | "Bb"B4 B2 ::
P:B
d2 | "F"c2B2 A2"C"G2 | "D"^F4 D2dc | "Gm"B2AG "D"^F2A2 | "G"G6 ::
P:C
FG | "F"A4 F2AB | c4 A2dc | "Gm"B2AG "D"^F2A2 | "G"G6 :| 


\end{abc}
% \index{Torneo Amoroso}
% \addcontentsline{toc}{subsection}{Torneo Amoroso}
% \begin{abc}[name=latex_16italian31]
% X:31
% T:Torneo Amoroso
% C:Cesare Negri, Le Grazie d'Amore, 1602
% P:(AABBCC)x2 DDEEFFGGHHJJ
% M:C|
% L:1/4
% K:F major
% P:A
% "G"G2 "Am"cc | "G"=B2 =B2 | "F"cc Ac | "Gm"B2 B2 | "F"A/G/A/B/ c"Bb"B | "F"A/B/c/A/ B/A/G/F/ | "C"E"F"F "Csus4"FE | "F"F4 ::
% P:B
% "C"G3/A/ G"Dm"F | "C"E2 "Bb"D2 | "Bb"DD "C"E2 | "F"F4 | "C"GF GA | "C"GF E3/C/ | "Dm"D/E/F "Csus4"FE | "F"F4 ::
% P:C
% "F"AA/B/ cA | "Bb"B3/c/ dc/B/ | "F"AA/B/ cA | "Bb"B3/c/ dc/B/ | "F"A3/B/ cA | "Bb"B3/A/ G"F"F | "C"E"F"F "^Repeat AABBCC!""Csus4"FE | "F"F4 ::
% M:6/8
% P:D
% "^Saltarello""F"FF/ "C"EE/ | "F"F3//G//A/ "Eb"GG/ | "Bb"BA/ "Cm"G"Dm"F/ | D/"C"E "F"F3/ ::\
% P:E
% "Bb"BA/ "C"GF/ | "F"F3//G//A/ "Eb"GF/ | "Bb"BA/ "C"G"Dm"F/ | F/"C"E "F"F3/ ::
% P:F
% "F"F3//G//A/ "Bb"B3//c//B/ | "F"F3//G//A/ "Bb"B3/ | "Bb"d3//c//B/ "F"A3//G//F/ | "C"E3//F//G/ "F"F3/ ::\
% M:6/4
% P:G
% "^Galliarda""Bb"FDF "C"E3/D/C | "F"cAc "Gm"B3/c/B | "Eb"G/A/B/c/B "F"A3/C/D/E/ | "F"F"Csus4"FE "F"F3 ::
% M:6/8
% P:H
% "^Saltarello""Eb"GA/ "C"GF/ | "Dm"F3//G//A/ "C"GF/ | "Bb"BA/ "Eb"G"Bb"F/ | F/"Csus4"F/E/ "F"F3/ ::\
% P:J
% "F"F3//G//A/ "Bb"BB/ | "F"F3//G//A/ "Bb"BB/ | "Bb"d3//c//B/ "F"A3//G//F/ | "Bb"D/"C"G "F"F3/ :| "^Reverance"F3 |] 
% \end{abc}

\index{Villanella}
\addcontentsline{toc}{subsection}{Villanella}
\begin{abc}[name=latex_16italian32]
X:32
I:linebreak $
T:Villanella
C:Fabritio Caroso, Il Ballarino, 1581
N:Transcribed & arranged by Kathy Van Stone. Matches Pennsic Pile 46
P:AABB x 6 (fast) or AABB x 3 (slow)
M:3/4
L:1/8
K:G major
P:A
"G"d4c2 | B3AB2 | "Am"A3Bc2 | "G"B4B2 | "D"A6 | "C"G6 | "D"A6 | "G"B4B2 ::
P:B
"G"B4B2 | "F"A4A2 | A4"C"G2 | "F"A4A2 | A3Bc2 | "G"B3AG2 | "D"A3GA2 | "G"B4B2 :| 
\end{abc}

\index{Vita, la}
\addcontentsline{toc}{subsection}{La Vita}
\begin{abc}[name=latex_il_papa1]
X:1
I:linebreak $
T:La Vita
C:Nathan Kronenfeld
P:5x
M:6/4
L:1/8
K:C major
P:A
"Am"E6 A3Bc2 | "Dm"d2A2d2 "C"c3BAG | "Am"c4B2 "Dm"d3cBA | "E"A4^G2 "Am2."A6 || 
P:B
"Am"B3GB2 "G"A2G2E2 | "Am"B3GB2 "Am"A2G2E2 | 
EFGFED E6 |] 
\end{abc}


\chapter{Dances from Arbeau's {\em Orchésographie}}

Published in 1589 in Langres, France, Orchésographie includes music and
instructions for many different kinds of dances. Numerically speaking, the bulk
of the dances in Arbeau are {\em bransles}. Most of the bransles are in duple
time and should be played at about half note = 115.  The triple time bransles
are Bransle Gay and Bransle de Poictou; for these, a tempo of dotted half =
60-65 should work.

Arbeau also includes instructions for the pavane and galliard, music for which
also appears in the Improvised Dances section.

\clearpage
\index{Belle qui tiens ma vie}
\index{Pavane!Belle qui tiens ma vie}
\addcontentsline{toc}{subsection}{Belle qui tiens ma vie}
\begin{abc}[name=latex_arbeau1]
I:abc-charset utf-8
X:1
I:linebreak $
T:Belle qui tiens ma vie
C:Thoinot Arbeau, Orchesographie, 1589
N:Transcribed & edited by Aaron Elkiss; matches Pile 2018
M:C|
L:1/4
K:G dorian
"Gm"G2 G"D"^F | "Gm"G"F"A "Bb"B2 | "Bb"Bd "C"c"Bb"B | "Eb"B"F"A "Bb"B2 | "Gm"G2 G"D"^F | "Gm"G"F"A "Bb"B2 | 
"Bb"Bd "C"c"Bb"B | "Eb"B"F"A "Bb"B2 | "Bb"B2 "F"A"Dm"A | "Gm"G"Cm"G "D"^F2 | "Bb"D2 "C"E/F/"Gm"G | "Dsus4"G^F "G"G2 | 
"Bb"B2 "F"A"Dm"A | "Gm"G"Cm"G "D"^F2 | "Bb"D2 "C"E/F/"Gm"G | "Dsus4"G^F "G"G2 |] 


\end{abc}
\index{Jouyssance vous donneray}
\addcontentsline{toc}{subsection}{Jouyssance vous donneray}
\begin{abc}[name=latex_arbeau2]
X:2
T:Jouyssance vous donneray
C:Thoinot Arbeau, Orchesographie, 1589
P:AA BB CC; retour: BB C A
N:arr. Steve Hendricks; matches Pile 46
M:6/4
L:1/8
K:D minor
P:A
"Dm"D2D2D2 F2"Am"EDC2 | "Dm"F2"C"GEFG "F"A4A2 | "F"AGFGAB c2cBA2 | "Bb"BAGF"C"GE "F"F4F2 |
"F"A2G2A2 "Bb"D4D2 | "C"EFGFED C4C2 | "F"cBAGFE F2"Gm"BAGF | "A"ED^CDDC "D"D4D2 :: 
P:B
"Dm"d2d2d2 "Am"c=Bcde2 | "Dm"dc"E"=BA=B^G "A"A4A2 | "Am"AGA=Bcd e2"D"dc=BA | "E"=BA^GA=BG "A"A4A2 ::
P:C
"Dm"DCDEDE F2"Am"EDC2 | "Dm"FEDEFG "F"A4A2 | "F"AGFGAB cBcdcA | "Bb"BA"C"GFGE "F"F4F2 | 
"F"A2G2A2 "Bb"D4D2 | "C"EFGFED C4C2 | "F"cBAGFE F2"Gm"BAGF | "A"ED^CDDC "D"D4D2 :| 


\end{abc}
\index{Tourdion}
\addcontentsline{toc}{subsection}{Tourdion}
\begin{abc}[name=latex_arbeau3]
X:3
T:Tourdion
C:Thoinot Arbeau, Orchesographie, 1589
M:6/4
L:1/4
K:F
BAG c3/2 B/A | GFE F3/2 E/D | BAG c3/2 B/A/G/ | G2F G2z :: 
GGG F/G/A/B/c | B3/2 A/A/G/4A/4 B3 | \
BBB A/B/c/B/A/G/ | G2 F G2 z :|


\end{abc}
\index{Galliard!Traditore my fa morire, la}
\index{Traditore my fa morire, Galliard la}
\addcontentsline{toc}{subsection}{Galliard: La traditore my fa morire}
\begin{abc}[name=latex_arbeau4]
X:4
T:Galliard: La traditore my fa morire
R:Gaillarde
C:Thoinot Arbeau, Orchesographie, 1589
M:6/4
L:1/4
K:F
d d c B2 A | A G G F2 D | d d c B2 A | G G F G2 z :: \
A A A c2 c | c c c d2 d | d d c B2 A | G G F G2 z :|


\end{abc}
\index{Galliard!Anthoinette}
\index{Anthoinette, Galliard}
\addcontentsline{toc}{subsection}{Galliard: Anthoinette}
\begin{abc}[name=latex_arbeau5]
X:5
T:Galliard: Anthoinette
C:Thoinot Arbeau, Orchesographie, 1589
M:6/4
L:1/4
K:C
AdG A2 A |  ddc "^(♭)"B2A :: ccc A>AG |GFE E2z :: DCD E>FG | AFG E2D :|


\end{abc}
\index{Galliard!Baisons nous belle}
\index{Baisons nous belle, Galliard}
\addcontentsline{toc}{subsection}{Galliard: Baisons nous belle}
\begin{abc}[name=latex_arbeau6]
X:6
T:Galliard: Baisons nous belle
C:Thoinot Arbeau, Orchesographie, 1589
M:6/4
L:1/4
K:F
F F F D z D | G G G A z A | F G A B z G | F F E F z F |]


\end{abc}
\index{Galliard!Si j'ayme ou non}
\index{Si j'ayme ou non, Galliard}
\addcontentsline{toc}{subsection}{Galliard: Si j'ayme ou non}
\begin{abc}[name=latex_arbeau7]
X:7
T:Galliard: Si j'ayme ou non
C:Thoinot Arbeau, Orchesographie, 1589
M:6/4
L:1/4
K:F
A A A z A A | F G A z B B | A G A F z G | F F E F z F |
A G A F z F | D E F G z G | A G A F z G | F F E F z F |]


\end{abc}
\index{Galliard!J'aymerois mieulx dormir seulette}
\index{J'aymerois mieulx dormir seulette, Galliard}
\addcontentsline{toc}{subsection}{Galliard: J'aymerois mieulx dormir seulette}
\begin{abc}[name=latex_arbeau8]
X:8
T:Galliard: J'aymerois mieulx dormir seulette
C:Thoinot Arbeau, Orchesographie, 1589
M:6/4
L:1/4
K:F
FFFFFF | GGG AzA | BBB AAA | AGG ^Fz^F |
FGA BBA | AGG ^FzF | FGA BBA | GG^F GzG |]


\end{abc}
\index{Galliard!L'ennuy qui me tourmente}
\index{L'ennuy qui me tourmente, Galliard}
\addcontentsline{toc}{subsection}{Galliard: L'ennuy qui me tourmente}
\begin{abc}[name=latex_arbeau9]
X:9
T:Galliard: L'ennuy qui me tourmente
C:Thoinot Arbeau, Orchesographie, 1589
M:6/4
L:1/4
K:C
DDD FFF | FGG AzF |\
AAA AAG | FFE FzF |
K:C
FFA AAF | AAA GzE | GGG FFD | AGF EzE |
FFG AAF/F/ | A/A/AA GzE | GAG FFE/E/ | D/D/DC DzD |]


\end{abc}
\index{Volte}
\addcontentsline{toc}{subsection}{Volte}
\begin{abc}[name=latex_arbeau10]
X:10
T:Volte
C:Thoinot Arbeau, Orchesographie, 1589
L:1/4
M:6/4
K:DDor
DDD A2A | GGF E2D | GGF E2D | DDC D2D |]


\end{abc}
\index{Coranto}
\addcontentsline{toc}{subsection}{Coranto}
\begin{abc}[name=latex_arbeau11]
X:11
T:Coranto
C:Thoinot Arbeau, Orchesographie, 1589
K:DDor
L:1/4
M:C
ABcd | edce | dcBA | GFGG |]


\end{abc}
\index{Alman}
\addcontentsline{toc}{subsection}{Alman}
\begin{abc}[name=latex_arbeau12]
X:12
T:Alman
M:C
R:Alman
C:Thoinot Arbeau, Orchesographie, 1589
L:1/4
K:F
FF/F/ GG/G/ | A/F/G/A/ BB |\
Bd c>B | A/F/G/A/ FF | \
B/A/G/F/ B/A/G/F/ | 
Bd c>B | \
A/F/G/A/ FF ||\
FF GG | AA BB |\
Bd cB | AG FF |]


\end{abc}
\index{Bransle!Double}
\index{Double, Bransle}
\addcontentsline{toc}{subsection}{Bransle Double}
\begin{abc}[name=latex_arbeau13]
X:13
I:linebreak $
C:Thoinot Arbeau, Orchesographie, 1589
N:Ed. Aaron Elkiss. Matches Pile 2018.
T:Bransle Double
M:C|
L:1/4
K:G dorian
 "^Drone: G/D"GG BB | AG FF |  [1 Bc dB | cc BB :|]  [2 BB AG | G^F GG :| 


\end{abc}
\index{Bransle!Simple}
\index{Simple, Bransle}
\addcontentsline{toc}{subsection}{Bransle Simple}
\begin{abc}[name=latex_arbeau14]
X:14
I:linebreak $
T:Bransle Simple
T:Single Bransle
N:Ed. Aaron Elkiss. Matches Pile 2018.
C:Thoinot Arbeau, Orchesographie, 1589
M:C|
L:1/4
K:G dorian
"^Drone: G/D"GG BB | AG FB | BA BB | GG BB | AG FG | G^F GG :| 


\end{abc}
\index{Bransle!Gay}
\index{Gay, Bransle}
\addcontentsline{toc}{subsection}{Bransle Gay}
\begin{abc}[name=latex_arbeau15]
X:15
I:linebreak $
T:Bransle Gay
N:Ed. Aaron Elkiss. Matches Pile 2018.
C:Thoinot Arbeau, Orchesographie, 1589
M:6/4
L:1/8
K:G dorian
  "^Drone: G/D"G2G2d2 B4B2 | c4c2 d4d2 | c4B2 A4G2 | G4^F2 G6 :| 


\end{abc}
\index{Bransle!Burgoigne, de}
\index{Burgoigne, Bransle de}
\addcontentsline{toc}{subsection}{Bransle de Burgoigne}
\begin{abc}[name=latex_arbeau16]
X:16
I:linebreak $
T:Bransle de Burgoigne
T:Burgundian Bransle
N:Ed. Aaron Elkiss. Matches Pile 2018.
C:Thoinot Arbeau, Orchesographie, 1589
M:C|
L:1/4
K:F major
 "^Drone: G/D"BB GG | AA FD | dd BB | cA Gz :| 


\end{abc}
\index{Bransle!Hault Barrois}
\index{Hault Barrois, Bransle}
\addcontentsline{toc}{subsection}{Bransle Hault Barrois}
\begin{abc}[name=latex_arbeau17]
X:17
T:Bransle Hault Barrois
C:Thoinot Arbeau, Orchesographie, 1589
M:C
L:1/4
K:F
"^Drone: F/C"cdef | efcc | BBAA | BAGF :|


\end{abc}
\index{Bransle!Cassandre}
\index{Cassandre, Bransle}
\addcontentsline{toc}{subsection}{Bransle Cassandre}
\begin{abc}[name=latex_arbeau18]
X:18
I:linebreak $
T:Bransle Cassandre
N:Ed. Aaron Elkiss. Matches Pile 2018.
C:Thoinot Arbeau, Orchesographie, 1589
M:C|
L:1/4
K:D dorian
"^Drone: A/D"c2 cc | c2 c2 | d/e/f cd | A2 A2 :: f2 dd | e2 cc | 
dd cd | A3/B/ cc | F2 F2 | G/A/_B GA | D2 D2 :| 


\end{abc}
\index{Bransle!Pinagay}
\index{Pinagay, Bransle}
\addcontentsline{toc}{subsection}{Bransle Pinagay}
\begin{abc}[name=latex_arbeau19]
X:19
I:linebreak $
T:Bransle Pinagay
N:Ed. Aaron Elkiss. Matches Pile 2018.
C:Thoinot Arbeau, Orchesographie, 1589
M:C|
L:1/4
K:G major
"^Drone: G/D"GG GG | AA Bd/c/ | Bz GG | GG AA | Bd/c/ Bd/c/ | Bd/c/ Bz | 
BA GA | FG AG | BB AG | GF G2 |] 


\end{abc}
\index{Bransle!Charlotte}
\index{Charlotte, Bransle}
\addcontentsline{toc}{subsection}{Bransle Charlotte}
\begin{abc}[name=latex_arbeau20]
X:20
I:linebreak $
T:Bransle Charlotte
N:Ed. Aaron Elkiss. Matches Pile 2018.
C:Thoinot Arbeau, Orchesographie, 1589
M:C|
L:1/4
K:G dorian
"^Drone: G/D"GG BB | cc d2 | gz dz | cA Bc | d2 G2 :| Gd dd | 
cd B2 | dz cz | 
M:3/2
GABcAG | 
M:C|
dz cz | 
M:3/2
GABcAG | 
M:C|
dz ez | dc BB | AA G2 |] 


\end{abc}
\index{Bransle!Guerre, de la}
\index{Bransle!War}
\index{Guerre, Bransle de la}
\index{War Bransle}
\addcontentsline{toc}{subsection}{Bransle de la Guerre (War Bransle)}
\begin{abc}[name=latex_arbeau21]
X:21
I:linebreak $
T:Bransle de la Guerre
T:War Bransle
N:Ed. Aaron Elkiss. Matches Pile 2018.
C:Thoinot Arbeau, Orchesographie, 1589
M:C|
L:1/4
K:G major
"^Drone: G/D"d/c/d/e/ dd | B2 Gd | cB AG | F2 D2 | A2 Bc | d2 dd | 
ed d^c | d2 d2 :| dd dB | dd dB | c/A/B c/A/B | AG BA | 
B/G/A B/G/A | GG FG | B/G/A GG | FG G2 |] 


\end{abc}
\index{Bransle!Aridan}
\index{Aridan, Bransle}
\addcontentsline{toc}{subsection}{Bransle Aridan}
\begin{abc}[name=latex_arbeau22]
X:22
I:linebreak $
T:Bransle Aridan
N:Ed. Aaron Elkiss. Matches Pile 2018.
C:Thoinot Arbeau, Orchesographie, 1589
M:C|
L:1/4
K:G major
"^Drone: G/D"Bc dd | ee dc | 
M:3/2
BzAzGz :: 
M:C|
A2 B2 | Gd cB | A2 GG | AF GA | B2 G2 | Bc dd | 
ee d2 | ez dz | cc BB | AA G2 | AB G3/d/ | c/B/A/A/ G2 :| 


\end{abc}
\index{Bransle!Poictou, de}
\index{Poictou, Bransle de}
\addcontentsline{toc}{subsection}{Bransle de Poictou}
\begin{abc}[name=latex_arbeau23]
X:23
I:linebreak $
T:Bransle de Poictou
N:Ed. Aaron Elkiss. Matches Pile 2018.
C:Thoinot Arbeau, Orchesographie, 1589
M:3/4
L:1/8
K:G mix
"^Drone: G/D"B2c2d2 | d4d2 | c4A2  |B2A2G2 | G2A2^F2 | G4z2 :| 


\end{abc}
\index{Bransle!d'Ecosse}
\index{Bransle!Scottish}
\index{Ecosse, Bransles des}
\index{Scottish Bransles}
\addcontentsline{toc}{subsection}{Bransles d'Ecosse (Scottish Bransles)}
\begin{abc}[name=latex_arbeau24]
X:24
T:Bransles d'Ecosse
T:Scottish Bransles
C:Thoinot Arbeau, Orchesographie, 1589
N:Ed. Aaron Elkiss. Matches Pile 2018.
M:C|
L:1/4
K:G dorian
"^Drone: G/D"GA BG | AB c2 | cB AG | Bc d2 |  [1 df ed | cB AG :|] \
 [2 cB AG | AF G2 :: 
d2 de | fg fe | d2 d2 |  [1 cB A2 | d2 cB | A2 G2 :|] \
 [2 cB AG | AF G2 :| 


\end{abc}
\index{Trihory de Bretagne}
\addcontentsline{toc}{subsection}{Trihory de Bretagne}
\begin{abc}[name=latex_arbeau25]
X:25
I:linebreak $
T:Trihory de Bretagne 
T:Triory of Brittany
N:Ed. Aaron Elkiss. Matches Pile 2018.
C:Thoinot Arbeau, Orchesographie, 1589
M:2/4
L:1/4
K:G dorian
 "^Drone: G/D"Bc | d2 | d2 | c c | B B | c A | B2 :|


\end{abc}
\index{Bransle!Malte, de}
\index{Bransle!Maltese}
\index{Malte, Bransle de}
\index{Maltese Bransle}
\addcontentsline{toc}{subsection}{Bransle de Malte (Maltese Bransle)}
\begin{abc}[name=latex_arbeau26]
X:26
I:linebreak $
T:Bransle de Malte
T:Maltese Bransle
N:Ed. Aaron Elkiss. Matches Pile 2018.
C:Thoinot Arbeau, Orchesographie, 1589
M:C|
L:1/4
K:C major
 "^Drone: C/G"c/B/c/d/ ee | dc BA/G/ | AA G2 :| c/B/c/d/ ef | de f2 | d/e/f e2 | 
d/e/f ed/c/ | B/A/c/B/ c2 |] 


\end{abc}
\index{Bransle!Lavandieres, des}
\index{Lavandieres, Bransle des}
\index{Bransle!Washerwomens'}
\index{Washerwomens' Bransle}
\addcontentsline{toc}{subsection}{Bransle des Lavandieres (Washerwomens')}
\begin{abc}[name=latex_arbeau27]
X:27
T:Bransle des Lavandieres
T:Washerwomens' Bransle
C:Thoinot Arbeau, Orchesographie, 1589
N:Ed. Aaron Elkiss. Matches Pile 2018.
M:C
L:1/4
K:G dorian
"^Drone: G/D"GG GG | FF B2 | cB AG | GF G2 ::\
Gd Bd | cB AG ::
GG F2 | G2 A2 | AA AB | cB AG |\
GG F2 | G2 A2 | cB AG | GF G2 :|


\end{abc}
\index{Bransle!Lavandieres, des}
\index{Lavandieres, Bransle des}
\index{Bransle!Washerwomens'}
\index{Washerwomens' Bransle}
\begin{abc}[name=latex_arbeau28]
X:28
I:linebreak $
T:Bransle des Lavandieres
T:Washerwomen's Bransle
C:Jean d'Estrées, Premier livre de danseries, 1559
N:Edited by Aaron Elkiss; matches Pile 2018
M:C
L:1/8
K:G dorian
 "Gm"G2G2 G2G2 | "D"^F2F2 "Bb"B4 | "F"cd"Cm"_ed "F"cB"Gm"AG | "Dsus4"^FG2F "G"G4 :: "Gm"G2d2 B2"Dm"d2 | "F"c2"Gm"B2 "D"A2"G"G2 :| 
"Gm"G2G2 "D"^F4 | "Gm"G4 "F"A4 | "F"A4 A2"C"GA | "Gm"BcB2 "D"A2"Gm"G2 | "Gm"G2G2 "D"^F4 | "Gm"G4 "F"A4 | 
"F"ABcB A2"Gm"G2- | "Dsus4"G2^F2 "G"G4 |] 


\end{abc}
\index{Bransle!Pois, des}
\index{Pois, Bransle des}
\addcontentsline{toc}{subsection}{Bransle des Pois}
\index{Bransle!Pease}
\index{Pease Bransle}
\addcontentsline{toc}{subsection}{Bransle des Pois (Pease)}
\begin{abc}[name=latex_arbeau29]
X:29
I:linebreak $
T:Bransle des Pois
T:Pease Bransle
N:Ed. Aaron Elkiss. Matches Pile 2018.
C:Thoinot Arbeau, Orchesographie, 1589
M:C|
L:1/4
K:G dorian
"^Drone: G/D"B2 Bc | d2 dd | _ed cc | d2 G2 :| GA F2 | GA BG | 
GA Bc | BA G2 | GA F2 | GA BG | GA Bc | BA G2 |] 


\end{abc}
\index{Bransle!Pois, sont des}
\index{Pois, Bransle sont des}
\index{Bransle!Pease}
\index{Pease Bransle}
\begin{abc}[name=latex_arbeau30]
X:30
T:Bransle sont des Pois
T:Pease Bransle
N:arr. Steven Hendricks. Matches Pile 46
C:Adrian Le Roy, Breve et facile instruction, 1565
M:C
L:1/8
K:G major
"G"G2A2 B2c2 | d4 d2"C"e2 | "D"d2c2 B2A2 | "G"B4 B2G2 | \
G3A BABc | d4 d2"C"e2 | "D"d2c2 B2A2 | "G"B8 |
"G"G2"D"A2 F4 | "Am"ABc2 "G"B2G2 | "G"G2"D"A2 F2"Am"c2 | "G"B2"D"d2 "G"B4 | \
"G"G2"D"A2 F4 | "Am"ABc2 "G"B2G2 | "G"G2"D"A2 F2"Am"c2 | "G"B2"D"d2 "G"B2G2 |]


\end{abc}
\index{Bransle!Hermites, des}
\index{Hermites, Bransle des}
\addcontentsline{toc}{subsection}{Bransle des Hermites}
\begin{abc}[name=latex_arbeau31]
X:31
I:linebreak $
T:Bransle des Hermites
N:Ed. Aaron Elkiss. Matches Pile 2018.
C:Thoinot Arbeau, Orchesographie, 1589
M:C|
L:1/4
K:G dorian
"^Drone: G/D"BB BB | BB AB | cB AG | G"^(♯)"F G2 :| GG GG | GG F2 | 
GG A2 | F2 D2 | GG GG | GG F2 | GG A2 | F2 D2 |] 


\end{abc}
\index{Bransle!Sabots, des}
\index{Sabots, Bransle des}
\index{Bransle!Clog}
\index{Clog Bransle}
\addcontentsline{toc}{subsection}{Bransle des Sabots (Clog Bransle)}
\begin{abc}[name=latex_arbeau32]
X:32
I:linebreak $
T:Bransle des Sabots
T:Clog Bransle
C:Thoinot Arbeau, Orchesographie, 1589
N:Ed. Aaron Elkiss. Matches Pile 2018.
M:C|
L:1/4
K:C major
"^Drone: C/G"c2 c2 | d/c/B/A/ BG | AA Gc | cB c2 :: d/c/B/A/ BG | d/c/B/A/ BG | 
M:3/2
GzGzGz :| 


\end{abc}
\index{Bransle!Chevaulx, des}
\index{Chevaulx, Bransle des}
\index{Bransle!Horses'}
\index{Horses' Bransle}
\addcontentsline{toc}{subsection}{Bransle des Chevaulx (Horses's Bransle)}
\begin{abc}[name=latex_arbeau33]
X:33
T:Bransle des Chevaulx
C:Thoinot Arbeau, Orchesographie, 1589
N:arr. Kathy Van Stone; matches Pile 2018
T:Horses' Bransle
M:C
L:1/8
K:Gmaj
"G"G3A B2B2 | "C"c2B2 A2c2 | "G"B2A2 G2F2 | "C"E4 "D"D4 | \
"G"G3A B2B2 | "C"c2B2 A2c2 | "G"B2G2 "D"A2A2 | "G"G8 ||
"G"d2cB "F"A2AB | "C"c2BA "G"G2B2 | "F"A2G2 "D"F2G2 | "D"A4 A4 | \
"G"d2cB "F"A2AB | "C"c2BA "G"G2B2 | "F"A2"C"G2 "D"G2F2 | "G"G4 G4 ||
K:Gdor
"Gm"B2AG B2AG | "Bb"F2G2 "Dm"A4 | "Dm"D2E2 "Bb"F2G2 | "Dm"A2B2 "F"A2G2 | \
"Gm"B2AG B2AG | "Bb"F2G2 "Dm"A4 | "Dm"D2E2 F2G2 | "Dsus4"G2"D"^F2 "G"G4 |]


\end{abc}
\index{Bransle!Montarde, de la}
\index{Montarde, Bransle de la}
\addcontentsline{toc}{subsection}{Bransle de la Montarde}
\begin{abc}[name=latex_arbeau34]
X:34
T:Bransle de la Montarde
P:AA Bx(number of dancers per set)
C:Thoinot Arbeau, Orchesographie, 1589
N:Ed. Al Cofrin; Matches Pile 48
M:C|
L:1/4
K:D dorian
P:A
"D5"AB cA | dB cA | AA AA | GF ED | AB cA | 
dB cA | GA/G/ FE | "G5"DE/C/ D2 ::\
P:B
"^Repeat once per dancer in each set""G5"de/d/ cB | cA G2 :|


\end{abc}
\index{Bransle!Montarde, de la}
\index{Montarde, Bransle de la}
\begin{abc}[name=latex_arbeau35]
X:35
T:Bransle de la Montarde
C:Pierre Phalese, 1571
C:Arr. Emma Badowski
M:C|
L:1/4
K:F major
P:A
"Gm"d"C"e "F"f"Gm"d | "C"ge "Dm"f"Gm"d | "Gm"d/c/d B/c/"Bb"d | "F"c"Gm"B "D"AA | "Gm"d"C"e "F"f"Gm"d | 
"C"ge "Dm"f"Gm"d | \
"Gm"d/c/d B/c/"Bb"d | "F"c/B/A "Gm"BG |: \
P:B - repeat for each dancer in set
"Bb"dd "Eb"B/c/"Bb"d | "F"c/B/A "Gm"BG :| 


\end{abc}
\index{Bransle!Haye, de la}
\index{Haye, Bransle de la}
\addcontentsline{toc}{subsection}{Bransle de la Haye}
\begin{abc}[name=latex_arbeau36]
X:36
I:linebreak $
T:Bransle de la Haye
C:Thoinot Arbeau, Orchesographie, 1589
N:Arr. Steve Hendricks. Matches Pile 46
M:C|
L:1/4
K:F major
"Gm"G | "C"cc c"Gm"d | "Gm"GG G"D"A | "Gm"BB GB | "Dm"A2 zA | "C"cc c"Gm"d | "Gm"GG GB | 
"D"AG A^F | "Gm"G2 zG | "C"c=B cd | "C"e2 ef | "C"ed/c/ "G"dd | "C"c2 zc | 
"F"ff f"C"e | "Gm"d2 dd | "C"cc "Gm"BB | "D"A2 Ad | "Cm"cB c"D"A | "Gm"G2 "D"Ad | 
"Cm"cB c"D"A | "Gm"G3 |] 


\end{abc}
\index{Bransle!Official}
\index{Official, Bransle}
\addcontentsline{toc}{subsection}{Bransle Official}
\begin{abc}[name=latex_arbeau37]
X:37
I:linebreak $
T:Bransle Official
C:Thoinot Arbeau, Orchesographie, 1589
N:Arr. Russell Almond. Matches Pile 2018
M:C
L:1/8
K:C major
 |: "C"c2c2 "G"dcBA | "C"G6G2 | "F"ABc2 "Gsus4"c2B2 | "C"c4 c4 :: "C"g3f efge | "Bb"f3e defd | 
"Am"e3d cdec | "G"d3c BcdB | "F"c3B ABcA | "G"B3A G2G2 | "C"AB"Gsus4"c4B2 | "C"c4 c4 :| 


\end{abc}
\index{Gavotte}
\addcontentsline{toc}{subsection}{Gavotte}
\begin{abc}[name=latex_arbeau38]
X:38
T:Gavotte
M:C|
L:1/4
C:Thoinot Arbeau, Orchesographie, 1589
K:F
A>G F/G/A/B/ | c2 c2 | d/e/ f ed | c2 c2 ::BA GA | F>G AF | BA Gc | F2 F2 :| 


\end{abc}
\index{Morisques}
\addcontentsline{toc}{subsection}{Morisques}
\begin{abc}[name=latex_arbeau39]
X:39
T:Morisques
M:C|
L:1/4
C:Thoinot Arbeau, Orchesographie, 1589
K:CMix
cc cd | c4 | AF FG | E2 C2 :: \
AF FG | AF FG | AF FG | E2 C2 ::
L:1/8
cBcd cBcd | c2AB c4 | \
AGAG FAGF | E2DE C4 ::
AFFF AGAG | FEFG AGAG | \
FEFG E2DE | C4 :| 


\end{abc}
\index{Canaries}
\addcontentsline{toc}{subsection}{Canaries}
\begin{abc}[name=latex_arbeau40]
X:40
T:Canaries
C:Thoinot Arbeau, Orchesographie, 1589
M:C|
L:1/4
K:GMix
G2 AB | G2 AB | c2 dB | A2 BG ::\
G2 dc | B2 cd | c2 cB | A2 AG :|


\end{abc}
\index{Canaries}
\begin{abc}[name=latex_arbeau41]
X:41
I:linebreak $
T:Canaries
C:Michael Praetorius, Terpsichore, 1612
N:Ed. Aaron Elkiss, matches Pile 48
M:6/4
L:1/8
K:G major
"G"G3AB2 G3AB2 | "C"c4"G"B2 "D"A4"G"G2 :: "G"d6 B3cd2 | "C"c4"G"B2 "D"A4"G"G2 :: "G"B3AB2 "C"G3A"G"B2 | "Am"c3d"G"B2 "D"A4"G"G2 | 
"G"G3AB2 "Em"G3A"G"B2 | "C"c3d"G"B2 "D"A4"G"G2 | "D"d3cd2 "G"B3cd2 | "Am"c4"G"B2 "D"A4"G"G2 | "D"d3ed2 "Em"B3c"Bm"d2 | "C"c3d"G"B2 "D"A4"G"G2 :: 
"C"G3A"G"B2 "C"G3A"G"B2 | "C"G3A"G"B2 "D"A6 | "D"d6 "G"B6 | "C"G2A2"G"B2 "D"A6 :| 


\end{abc}
\index{Spanish Pavan}
\addcontentsline{toc}{subsection}{Spanish Pavan}
\begin{abc}[name=latex_arbeau42]
X:42
T:Spanish Pavan
M:C
L:1/8
C:Thoinot Arbeau, Orchesographie, 1589
K:DDor
d4  c3 A | B2 c2  d3 e | \
dBcd  e4 | e2 e2  e3 g |
f2 e2  d3 c | dfed  c2 d2 |\
B2 c2  d3 c | Bcde  d2 |]


\end{abc}
\index{Pavane de Spaigne (XXX)}
\index{Spaigne (XXX), Pavane de}
\addcontentsline{toc}{subsection}{Pavane de Spaigne (XXX)}
\begin{abc}[name=latex_arbeau43]
X:43
T:Pavane de Spaigne (XXX)
T:For Pavaniglia (Caroso) and Spanish Pavane (Arbeau)
N:Ed. Aaron Elkiss. Matches Pennsic Pile 46.
M:C
L:1/8
K:C major
"Dm"def2 e2d2 | "A"^c3B A2B2 | ^c2d2 e2c2 | "Dm"d3e f2e2 | \
d2B2 c2d2 | "C"e2c2 c2ga | g2f2 e2d2 | e2c2 c2fg |
"Am"a2g2 "F"f2e2 | "Dm"f2d2 d2a2 | "G"g2f2 e2d2 | "A"^c3B A2B2 | \
^c2d2 e2c2 | "D"d6"A"e2 | "Dm"f2g2 "A"a4 | "D"^f8 |]


\end{abc}
\index{Bouffons}
\addcontentsline{toc}{subsection}{Bouffons}
\begin{abc}[name=latex_arbeau44]
X:44
T:Bouffons
C:Thoinot Arbeau, Orchesographie, 1589
M:C
L:1/4
K:F
FG AA | B4 | AGAF | G4 | FG AA | B4 | AF GG | F4 |
c2 dc | B4 | AG AB | c4 | c2dc | B4 | AF GG | F4 ||


\end{abc}

\index{Bouffons}
\begin{abc}[name=latex_arbeau45]
X:45
T:Bouffons
C:Jean d'Estrées, Tiers Livre de Danseries, 1559
M:C
L:1/8
K:G major
P:A
 |: "G"G3A B2G2 | "F"c6c2 | "G"B3B "F"A2"G"G2 | "D"F4 F4 | "G"G3A B2G2 | "F"c6"G"B2 | \
"F"A2"Em"G4"D"F2 | "G"G4 G4 :: 
P:B
"G"d4 d2"C"e2 | "F"c4 c2"C"c2 | "G"B2B2 "F"A2"G"G2 | "D"F4 F4 | \
"G"d4 d2"C"e2 | "F"c4 c2"Dm"d2 | "G"B2"C"G4"D"F2 | "G"G4 G4 :| 
\end{abc}


\chapter{Improvised Dances}

Improvised dances such as the pavane and galliard were very popular in the 16th
century all over Europe. Music and instructions for these dances appear in
numerous sources. Reductions are provided from such sources as Tylman Susato's
{\em Danserye} of 1551 and Praetorius' {\em Terpsichore} of 1612. We have also
included the tunes used in the SCA for some early Italian improvised dances,
the Piva and the Saltarello.

The Canarie is transcribed in 6/4. For the Canarie, use a tempo of
approximately dotted half = 70.

Galliards can be transcribed in either 3/2 or 6/4. We have chosen to use 3/2
for clarity for some of the more rhythmically complex settings while halving
the original note values and using 6/4 for the more straightforward ones. The
tempo for galliards (for the 6/4 settings) can be anywhere from dotted half =
45 - 60, depending on the whims of the dancing master. For transcriptions in
3/2 use dotted whole = 45 - 60 instead (two measures of a 3/2 galliard equating
to one measure of 6/4 galliard). The Volta is really just a variation on the
galliard and can be played as such.

The pavanes are transcribed in cut time, and again, the tempo can range from
half note = 45 to 60.

Preferences vary, so always check with the dancing master for desired tempo.
Additionally, modern choreographies have been created for some of these tunes,
so be sure to confirm the roadmap with the dancing master if these are being
danced.

%%06imp

\chapter{The English Dancing Master, 1651}

This section includes all 105 dances in the first edition of John Playford's
{\em The English Dancing Master} of 1651.  The dances are generally transcribed
in either cut time or in 6/4. For cut time use a tempo of approximately half
note = 115 or for 6/4, dotted half = 115.  Some dances such as Chestnut are
often danced slower, so be sure to check with the dancing master just in case.

\clearpage
\index{Health To Betty, A}
\addcontentsline{toc}{subsection}{A Health To Betty}
\begin{abc}[name=latex_playford1]
X:36
T:A Health To Betty
C:John Playford, the English Dancing Master, 1651
N:arr. Emma Badowski
M:6/4
L:1/8
K:F major
G2 | "Gm"G3AG2 "D"^F4D2 | "Gm"B6 "D"A4d2 | "Bb"d3ed2 "F"c4A2 | "Dm"f2d4- d4e2 | "Bb"f2d4 B3cd2 | "F"c2A4 F4F2 | "Gm"G3AG2 "D"^F4D2 | "Gm"B6 "D"A4 :| 


\end{abc}
\index{A La Mode De France}
\addcontentsline{toc}{subsection}{A La Mode De France}
\begin{abc}[name=latex_playford2]
X:1
T:A La Mode De France
C:John Playford, the English Dancing Master, 1651
N:arr. Emma Badowski
M:C
L:1/8
K:D major
a2 | "D"a2f2 "G"g2"A"a2 | "D"f3e d2a2 | "D"a2f2 "G"g2"A"a2 | "D"f6 ::\
f2 | "A"e2A2 d2e2 | "D"f3e d2f2 | 
"A"e2A2 d2e2 | "D"f6f2 | "A"e2A2 d2e2 | "D"f3e d2a2 | "D"a2f2 "G"g2"A"a2 | "D"f6 :|


\end{abc}
\index{Adson's Saraband}
\addcontentsline{toc}{subsection}{Adson's Saraband}
\begin{abc}[name=latex_playford3]
X:1
T:Adson's Saraband
C:John Playford, the English Dancing Master, 1651
N:arr. Jay Ter Louw. Matches Pennsic Pile 46
P:Play 12 times through
M:6/4
L:1/8
K:C major
"C"e4e2 "G"d4c2 | "C"e2f2g2 "G"d4c2 | "G"B2c2d2 "C"e2A2B2 | "Am"c2B2A2 "C"G4F2 | "Am"E2F2G2 "Em"E2F2G2 | "F"A2B2c2 "G"B4A2 | "C"G2d2e2 "G"f2A2B2 | "C"c2d2e2 "G"d4c2 |
"C"e4e2 "G"d4c2 | "C"e2f2g2 "G"d4c2 | "G"B2c2d2 "C"e2A2B2 | "Am"c2B2A2 "C"G4F2 | "Am"E2F2G2 "Em"E2F2G2 | "F"A2B2c2 "G"B4A2 | "C"G2d2e2 "Dm"f2e2d2 | "Am"c2A2B2 "C"c6 |]


\end{abc}
\index{All in a Garden Green}
\addcontentsline{toc}{subsection}{All in a Garden Green}
\begin{abc}[name=latex_playford4]
X:1
T:All in a Garden Green
C:John Playford, the English Dancing Master, 1651
N:arr. Dave Lankford. Matches Pile 2018
P:AA BB x 3
M:C|
L:1/4
K:G major
P:A
"G"d2 dd | B3/c/ dd | "C"ed cB | "D"A3B | "C"c3/d/ e/f/g | "G"G3/A/ B/c/d | "C"cB "D"A/G/A/B/ | "G"G4 ::
P:B
"G"dd/d/ dd | "C"ef "Em"g2 | BB/B/ BG | "Am"AB "C"c2 | "G"B3/A/ G/A/B | "Am"c3/B/ A/B/c | "G"d3/c/ "Em"B/c/d | "C"e3/d/ c/d/B/c/ |\
"D"AG GF | "G"G4 :|


\end{abc}
\index{Old Man Is A Bed Full Of Bones, An}
\addcontentsline{toc}{subsection}{An Old Man Is A Bed Full Of Bones}
\begin{abc}[name=latex_playford5]
X:5
T:An Old Man Is A Bed Full Of Bones
C:John Playford, the English Dancing Master, 1651
N:arr. Emma Badowski
M:6/4
L:1/8
K:C major
"Am"e2e2e2 e4e2 | "Am"e2c2A2 c4c2 | "G"d4c2 d4c2 | "Dm"d4e2 f4d2 | "Am"e2e2e2 e4e2 | "Am"e2c2A2 c4e2 | "Dm"f4f2 "Am"e3dc2 | "G"d4e2 "Dm"f3ed2 :| 


\end{abc}
\index{Argeers}
\addcontentsline{toc}{subsection}{Argeers}
\begin{abc}[name=latex_playford6]
X:6
T:Argeers
C:John Playford, the English Dancing Master, 1651
N:arr. Jay Ter Louws. Matches Pile 46
P:AA BB x 3
M:C|
L:1/4
K:C major
P:A
"C"c/d/e/f/ dc | "G"BG G2 | "G"gd g3/f/ | "C"ec c2 | "C"c/d/e/f/ dc | "G"B3/c/ dG | "F"A/B/c "G"B/c/d/e/ | "C"c4 ::
P:B
"G"B/c/d B/c/d | "G"gd gd | "G"B/c/d B/c/d | "G"gd gd | "C"gf/e/ "F"fg/f/ | "C"ed/c/ "G"d3/G/ | "F"A/B/c/d/ "G"B/c/d/e/ | "C"c4 :| 


\end{abc}
\index{Aye Me}
\index{Symphony, the}
\addcontentsline{toc}{subsection}{Aye Me, or The Symphony}
\begin{abc}[name=latex_playford7]
X:7
T:Aye Me, or The Symphony
C:John Playford, the English Dancing Master, 1651
N:arr. Emma Badowski
M:C
L:1/8
K:C major
"C"g4 e4 | "G"d2BA G2AB | "C"c2c2 "G"d2d2 | "C"e6ee | "C"e2fg "F"a2cc | "Am"c2de "Dm"f2AA | "G"BGe2 "G"d3c | "C"c6z2 ::
"Am"A2AB c2Bc | "G"d2G2 G2ee | "Am"e2de "Dm"fged | "G"d6gg | "Am"a2ee "F"f2^ff | "G"g2dc "G"B2gg | "F"a2e2 "G"d3c | "C"c6z2 :| 


\end{abc}
\index{Bath, the}
\addcontentsline{toc}{subsection}{The Bath}
\begin{abc}[name=latex_playford8]
X:93
T:The Bath
C:John Playford, the English Dancing Master, 1651
N:arr. Emma Badowski
M:C
L:1/8
K:C major
"G"g3d g3d | "G"gfed g2B2 | "F"c2A2 "D"d2d2 | "Am"A4 A4 | "G"GGGG "F"ABcd | "G"ddde d2B2 | "C"cdec "D"d3c | "G"B8 :| 


\end{abc}
\index{Beggar Boy, the}
\addcontentsline{toc}{subsection}{The Beggar Boy}
\begin{abc}[name=latex_playford9]
X:1
T:The Beggar Boy
C:John Playford, the English Dancing Master, 1651
N:arr. Emma Badowski
P:AA BB x 3
M:6/4
L:1/8
K:A phrygian
P:A
"Dm"A2A2A2 f4f2 | "Am"e2c4 "Dm"d4c2 | "F"A2F4 "C"G4G2 | "F"A4"Gm"B2 "Am"c2A4 ::\
P:B
"Dm"A2A2A2 f4f2 | "Am"e2c4 "Dm"d4"Am"c2 | 
"F"A2c4 "C"e2d2e2 | "F"f2A4 "C"G6 | "F"A2c2c2 "C"e3dc2 | "Dm"d2f2gf "Am"e2"F"f2"Dm"d2 | "F"c2A2F2 "C"G4G2 | "F"A4"Gm"B2 "Am"c2A4 :|


\end{abc}
\index{Blue Cap}
\addcontentsline{toc}{subsection}{Blue Cap}
\begin{abc}[name=latex_playford10]
X:1
T:Blue Cap
C:John Playford, the English Dancing Master, 1651
N:arr. Emma Badowski
M:6/4
L:1/8
K:Bb major
F2 | "Bb"B2B2f2 b2b2d2 | "Cm"c4B2 G4d2 | "Bb"B2B2f2 d2d2g2 | "C"c3f=e2 "F"f4 ::]\
f2 | "Bb"d2d2B2 "Cm"c2c2G2 | "Bb"F4d2 "Cm"e4g2 | 
"Bb"f2f2d2 f2g2a2 | "Bb"b2f2d2 "F"c4f2 | "Bb"d2d2B2 "Cm"c2c2G2 |\
M:9/4
"Bb"F4d2 "Cm"e6 e4g2 |\
M:6/4
"Bb"f2f2d2 f2g2a2 | "Bb"b2f2d2 "F"c4 :|


\end{abc}
\index{Boatman}
\addcontentsline{toc}{subsection}{Boatman}
\begin{abc}[name=latex_playford11]
X:10
T:Boatman
C:John Playford, the English Dancing Master, 1651
P:AABB x 3
N:arr. Dave Lankford. Matches Pile 2018
M:6/8
L:1/16
K:C major
P:A
"C"G2E4 G4G2 | G2E4 G4G2 | c4c2 B4A2 | "G"d6 D6 | "C"G2E4 G4G2 | G2E4 G4G2 | "F"c4d2 "G"e2d4 | "C"c6 C6 ::
P:B
"C"c4d2 e4d2 | c4B2 A4G2 | "F"F4F2 E4D2 | "G"d6 D4D2 | "C"G2E4 G4G2 | G2E4 G4G2 | "F"c4d2 "G"e2d4 | "C"c6 C6 :| 


\end{abc}
\index{Bobbing Joe}
\addcontentsline{toc}{subsection}{Bobbing Joe}
\begin{abc}[name=latex_playford12]
X:11
T:Bobbing Joe
C:John Playford, the English Dancing Master, 1651
N:arr. Emma Badowski
M:6/4
L:1/8
K:C major
"Am"A4e2 e4d2 | e3fg2 "G"d3cB2 | "Am"A4B2 c3dB2 | A2e4 A6 ::\
"G"B2d4 G6 | B2d4 G6 | "Am"A4B2 c3BA2 | c3de2 A6 :| 


\end{abc}
\index{Broome:The bonny bonny Broome}
\addcontentsline{toc}{subsection}{Broome:The bonny bonny Broome}
\begin{abc}[name=latex_playford13]
X:12
T:Broome:The bonny bonny Broome
C:John Playford, the English Dancing Master, 1651
N:arr. Steve Hendricks. Matches Pile 46
P:Play 6 times
M:C|
L:1/4
K:D major
A | "A"A3/B/ A3/B/ | "D"A/G/F/E/ D3/A/ | "D"dd/e/ f/e/d/c/ | "G"B3"A"c | "D"d3/e/ fe/f/ | "D"dD/E/ FE/D/ | "Em"EE B3/G/ | "A"E4 || 
"A"A3/B/ A3/B/ | "D"A/G/F/E/ D3/A/ | "D"dd/e/ f/e/d/c/ | "G"B3"A"c | "D"d3/e/ fe/f/ | "D"dD/E/ FE/D/ | "Em"EE B3/G/ | "A"E4 |] 


\end{abc}
\index{Cast A Bell}
\addcontentsline{toc}{subsection}{Cast A Bell}
\begin{abc}[name=latex_playford14]
X:13
T:Cast A Bell
C:John Playford, the English Dancing Master, 1651
N:arr. Emma Badowski
M:C
L:1/8
K:G major
"D"FGA2 f2ed | "D"f2ed "A"e2E2 | "D"FGA2 f2ed | "D"dGFE F2D2 :| 


\end{abc}
\index{Cheerily and Merrily}
\addcontentsline{toc}{subsection}{Cheerily and Merrily}
\begin{abc}[name=latex_playford15]
X:14
T:Cheerily and Merrily
C:John Playford, the English Dancing Master, 1651
N:arr. Emma Badowski
M:3/4
L:1/8
K:F major
"F"cBA2F2 | "Gm"GAG2D2 | "F"F4F2 | a4g2 | "C"g2e2c2 | "Bb"ded2B2 | "C"c3ede | c6 ::
"F"c2def2 | "Dm"d2c2A2 | "Gm"G3ABc | "Gm"d4c2 | "F"c2A2F2 | "Gm"GAG2D2 | "F"F3AGA | F6 :| 


\end{abc}
\index{Chestnut}
\index{Dove's Figary}
\addcontentsline{toc}{subsection}{Chestnut, or Dove's Figary}
\begin{abc}[name=latex_playford16]
X:15
T:Chestnut, or Dove's Figary
C:John Playford, the English Dancing Master, 1651
N:arr. Dave Lankford. Matches Pile 2018
P:AA BB x 3
M:C|
L:1/4
K:A minor
P:A
"Am"Ae dc | "E"B3/A/ ^GE | "Am"AB cc | "Dm"dc/d/ "E"e2 ::\
P:B
"Am"ee/f/ gf/e/ | "G"dd/e/ fe/d/ | 
"Am"ee "C"ed/c/ | "G"d3/c/ "Am"c2 | "Am"ef/e/ f/e/d/c/ | "G"de/d/ e/d/c/B/ | "Am"cA Ad | "Em"B3/A/ "Am"A2 :| 


\end{abc}
\index{Chirping Of The Lark}
\addcontentsline{toc}{subsection}{Chirping Of The Lark}
\begin{abc}[name=latex_playford17]
X:16
T:Chirping Of The Lark
C:John Playford, the English Dancing Master, 1651
N:arr. Emma Badowski
M:C
L:1/8
K:F major
"F"f4 "C"e2de | "F"f2F2 F2"Bb"d2 | "F"c2"Bb"B2 "F"A2"C"G2 | "F"A2F2 F4 ::\
"F"f3f "C"e2e2 | "Bb"dfed "A"^c2A2 | "Gm"Bcd2 "A"^cde2 | "Dm"d4 d4 :| 


\end{abc}
\index{Chirping of the Nightingale}
\addcontentsline{toc}{subsection}{Chirping of the Nightingale}
\begin{abc}[name=latex_playford18]
X:17
T:Chirping of the Nightingale
C:John Playford, the English Dancing Master, 1651
N:arr. Steve Hendricks
M:6/4
L:1/8
K:C major
"C"c4c2 "F"c3BA2 | "G"B4"C"c2 "G"d3ed2 | "C"c3dc2 "F"c3BA2 | "C"G4"F"F2 "C"E2C4 ::\
"C"E3FG2 "F"A4"C"G2 | "C"c4G2 E2C4 | 
"C"E3FG2 "F"A4"C"G2 | "C"c4G2 E2C4 | "C"E3FG2 "F"A2"G"B2G2 | "C"c3B"F"A2 "C"G2"F"A2F2 | "C"E3DE2 "G"D4"C"C2 | "C"C12 :| 


\end{abc}
\index{Confess (his tune)}
\addcontentsline{toc}{subsection}{Confess (his tune)}
\begin{abc}[name=latex_playford19]
X:18
T:Confess (his tune)
C:John Playford, the English Dancing Master, 1651
N:arr. Steve Hendricks
P:ABB
M:6/4
L:1/8
K:D minor
P:A
"Dm"D4"A"E2 "Dm"F4"Gm"G2 | "Dm"A4A2 A4A2 | "Gm"B6 "Dm"A4A2 | "Gm"d6 "A"^c6 ::\
P:B
"Am"e4c2 "Dm"f4d2 | "Am"e2c4 A4"Gm"B2 | 
"F"c2A4 "C"G4"F"A2 | "F"A2"Csus4"G4 "F"A6 | "F"F4"C"G2 "F"A3BA2 | "Gm"B2"Dm"A4 "Gm"G6 | "F"A4"Dm"d2 "A"^c4"Dm"d2 | "A"e2^c4 "D"d6 :| 


\end{abc}
\index{Country Coll}
\addcontentsline{toc}{subsection}{Country Coll}
\begin{abc}[name=latex_playford20]
X:19
T:Country Coll
C:John Playford, the English Dancing Master, 1651
N:arr. Emma Badowski
M:6/4
L:1/8
K:G major
"G"g4d2 g2d2G2 | "G"B3cd2 "C"e4d2 | "C"e2f2g2 "D"f2g2a2 | "G"d2g2e2 "D"f4d2 ::\
"C"e6 "G"d4B2 | "C"c2d2e2 "G"d4B2 | "G"g2d4 B3cd2 | "D"A2a4 "G"b4g2 :| 


\end{abc}
\index{Cuckolds all a Row}
\addcontentsline{toc}{subsection}{Cuckolds all a Row}
\begin{abc}[name=latex_playford21]
X:20
T:Cuckolds all a Row
C:John Playford, the English Dancing Master, 1651
N:arr. David Yardley. Matches Pile 2018
P:ABB x 3
M:6/4
L:1/8
K:G major
P:A
"D"a4a2 a4"Em"g2 | "D"f4g2 a4"G"d2 | "Am"e4e2 "D"f3ef2 |  [1 "G"g6 g6 :|]  [2 "G"g6 g4B2 ::\
P:B
"C"c2d2c2 "G"B2A2B2 | "D"A4A2 f2e2f2 | 
"G"g4d2 "Am"e2"D"d2c2 | "G"B6 G3AB2 | "C"c3dc2 "G"B3AB2 | "D"A4A2 f3ef2 | "G"g4d2 "Am"e2"D"d4 | "G"B6 G6 :| 


\end{abc}
\index{Daphne}
\addcontentsline{toc}{subsection}{Daphne}
\begin{abc}[name=latex_playford22]
X:1
T:Daphne
C:John Playford, the English Dancing Master, 1651
C:arr. Jay Ter Louw. Matches Pile 2018
M:6/4
L:1/8
K:F major
D2 | "Dm"F4G2 A4d2 | "A"^c3de2 "Dm"d4AB | "F"c2A2F2 "C"G2E2C2 |  [1 "Dm"D2F2E2 D4 :|]  [2 "Dm"D2  F2E2D6 ::
"Dm"f4f2"C"e4e2 | "Bb"d3ed2"A"^c2 A4 | "F"=c3BA2"C"G4F2 | "F"F2 E3FF6 :|\
"F"c2c2d2 c2A2F2 | 
"F"c3def "C"g2e2c2 | "Dm"A3GF2 "C"E4D2 | "Bb"d2d2c2 "Dm"d2A2A2 | "F"c3BA2 "Gm"G2D2F2 | "C"E3DE2"Dm"D6 |]


\end{abc}
\index{Dargason}
\index{Sedany}
\addcontentsline{toc}{subsection}{Dargason, or Sedany}
\begin{abc}[name=latex_playford23]
X:22
T:Dargason, or Sedany
C:John Playford, the English Dancing Master, 1651
N:arr. Robert Smith. Matches Pile 2018
M:6/4
L:1/8
K:G major
"G"B4G2 G4G2 | B4c2 d2c2B2 | "Am"c4A2 A4A2 | c4d2 e2d2c2 | "G"B4G2 G4G2 | g4g2 f2e2d2 | "Am"c4A2 A4A2 | a4g2 f2e2d2 |] 


\end{abc}
\index{Dissembling Love}
\addcontentsline{toc}{subsection}{Dissembling Love}
\begin{abc}[name=latex_playford24]
X:1
T:Dissembling Love
C:John Playford, the English Dancing Master, 1651
N:arr. Emma Badowski
M:3/4
L:1/8
K:F major
D2 | "Dm"D4E2 | "Dm"F4G2 | "Dm"A6 | "Dm"A4d2 | "F"c4A2 | "Gm"B4G2 | "Dm"A6- | "Dm"A4 ::\
Bc | "Dm Gm"d4A2 | "F"A2B4 | "F"c6 | "Gm"F3GAB | "A"G2A2FG | "Dm"E3DEF | "Dm"D6- | D4 :|


\end{abc}
\index{Drive The Cold Winter Away}
\addcontentsline{toc}{subsection}{Drive The Cold Winter Away}
\begin{abc}[name=latex_playford25]
X:24
T:Drive The Cold Winter Away
C:John Playford, the English Dancing Master, 1651
N:arr. Emma Badowski
M:3/4
L:1/8
K:F major
D2 | "Dm"F3ED2 | "Dm"A4B2 | "F"c3BA2 | "F"f4F2 | "Gm"G4A2 | "Gm"B3cB2 | "Dm"A6- | "Dm"A4 ::
d2 | "F"c3BA2 | "F"c4c2 | "Gm"B3AG2 | "Gm"B4B2 | "Dm"A3GF2 | "A"E4D2 | "F"f6- | "F"f4d2 | 
"F"c3BA2 | "F"c4c2 | "Gm"B3AG2 | "Gm"B4B2 | "Dm"A3GF2 | "A"E3DE2 | "Dm"D6- | "Dm"D4 :| 


\end{abc}
\index{Dull Sir John}
\addcontentsline{toc}{subsection}{Dull Sir John}
\begin{abc}[name=latex_playford26]
X:1
T:Dull Sir John
C:John Playford, the English Dancing Master, 1651
N:arr. Dave Lankford. Matches Pile 2018
P:AABB x 3
M:6/8
L:1/8
K:D minor
P:A
"Gm"G2^F G2A | B3/A/B "F"cAF | "Bb"B2A B2c | d3 d2c | "Bb"d2c "Dm"d2e | "F"f3/g/f F2B | "Dm"A2B "F"c3/d/c | "Bb"B3 B3 ::
P:B
"Bb"d2c d2e | f3/g/f F2B | "Dm"A2B "C"c3/d/=B | "F"c3 c2f | "Gm"d3/c/d "Am"c2A | "Bb"Bb2 "F"a2d | "Dm"d2g "Gm"g3/a/^f | g6 :|


\end{abc}
\index{Faine I Would}
\addcontentsline{toc}{subsection}{Faine I Would}
\begin{abc}[name=latex_playford27]
X:26
T:Faine I Would
C:John Playford, the English Dancing Master, 1651
N:arr. Kathy Van Stone. Matches Pile 2018
P:AABB x 3
M:6/4
L:1/8
K:G dorian
P:A
"Gm"g6 "D"^f6 | "Gm"g6 b6 | "F"a4g2 "C"g3ag2 | "F"f6- f4de | "F"f3gf2 "C"e2"Dm"d4 | "F"c6 "Gm"d6 | "F"c4B2 B3cA2 | "Bb"B6- B4 ::
P:B
F2 | "F"F4G2 A4B2 | c6- c4A2 | "Dm"d4c2 "Bb"B3cde | "F"f6- f3gf2 | "C"e2"Dm"d4 "C"c4B2 | "Dm"A6 "Gm"b6 | "Dm"a4d2 "Gm"g3a"D"^f2 | "G"g6- g4 :| 


\end{abc}
\index{Fine Companion, the}
\addcontentsline{toc}{subsection}{The Fine Companion}
\begin{abc}[name=latex_playford28]
X:94
T:The Fine Companion
C:John Playford, the English Dancing Master, 1651
N:arr. Paul Butler. Matches Pile 2018
P:AABB x 3
M:6/4
L:1/8
K:D minor
P:A
A2 | "Dm"d4f2 d2d2a2 | "Am"c3de2 "Dm"f3ed2 | "F"a2a2g2 "Dm"a2a2g2 | "Am"a2e4 e4d2 | "C"e2e2e2 e4d2 | "Am"c3de2 A4e2 | "Dm"f4d2 "Am"c2d2e2 | "Dm"d6 d4 ::
P:B
d2 | "Am"c2A2A2 e4d2 | "Am"c2A2B2 A4e2 | "Dm"f2d2e2 f2d2e2 | "Dm"f6 "C"g6 | "Am"a2a2a2 c3de2 | "Dm"f2f2e2 f2e2a2 | "Am"c2c2d2 e2c2A2 | "Dm"d6 f4 :| 


\end{abc}
\index{Friar and the Nun}
\addcontentsline{toc}{subsection}{Friar and the Nun}
\begin{abc}[name=latex_playford29]
X:27
T:Friar and the Nun
C:John Playford, the English Dancing Master, 1651
N:arr. Emma Badowski
M:C
L:1/8
K:D major
"D"d2d2 d2d2 | "D"d6d2 | "A"e2e2 c2d2 | "A"e6g2 | "D"f2d2 d2d2 | "D"d2d4d2 | "A"c2A2 A2B2 | "A"=c6=c2 | 
"G"B2G2 G2A2 | "G"B3c d2d2 | "D"A2A2 F2F2 | "D"A6=c2 | "G"B2G2 G2A2 | "G"B3c d2d2 | "A"c2d2 e3d | "D"d8 :| 


\end{abc}
\index{Gathering Peascods}
\addcontentsline{toc}{subsection}{Gathering Peascods}
\begin{abc}[name=latex_playford30]
X:1
T:Gathering Peascods
C:John Playford, the English Dancing Master, 1651
N:arr. Dave Lankford. Matches Pile 2018
P:AA BB CC x 3
M:C|
L:1/4
K:G major
P:A
"G"d2 dd | B3/c/ dd | ed cB | "D"A3B | "C"AG "D"GF |  [1 "G"G4 :|]  [2 "G"G3G ::
P:B
"D"FD FG | A2 BA | G/A/B AG | "D"F3F | "G"ED "A"E3/D/ |  [1 "D"D3G :|]  [2 "D"D3d ::
P:C
"G"BG GA/B/ | "C"c3c | "G"BG GA/B/ | "C"c3d | "G"BG GA/B/ | "C"c3/d/ "Am"ed/c/ | "G"Bc/B/ "D"A3/G/ |  [1 "G"G3d :|]  [2 "G"G4 :|


\end{abc}
\index{Glory of the West}
\addcontentsline{toc}{subsection}{Glory of the West}
\begin{abc}[name=latex_playford31]
X:29
T:Glory of the West
C:John Playford, the English Dancing Master, 1651
N:arr. Jay Ter Louw. Matches Pile 46
P:AABB x 3 or AAB x 3
M:C|
L:1/4
K:D minor
P:A
"Dm"F/E/F/E/ DD | "C"E/D/E/D/ "Am"CC | "Dm"FE/F/ "C"GF/G/ | "Am"AE/F/ "Dm"DD ::
P:B
"F"A/G/A/G/ Fc/B/ | "F"A/B/A/G/ FF | "C"cc/d/ GG | "Am"cc/d/ "C"GG | 
"Bb"dd/e/ "C"cc/d/ | "Am"AG/A/ "Dm"FF | "Am"cc/d/ "Dm"AG/F/ | "C"G/F/E/D/ CC | "Dm"FE/F/ "C"GF/G/ | "Am"AE/F/ "Dm"DD :| 


\end{abc}
\index{Goddesses}
\addcontentsline{toc}{subsection}{Goddesses}
\begin{abc}[name=latex_playford32]
X:30
T:Goddesses
C:John Playford, the English Dancing Master, 1651
N:arr. Kathy Van Stone. Matches Pile 2018
P:AA BB x 11
M:C|
L:1/4
K:A minor
P:A
"Am"AA cB/A/ | "G"BB dA/B/ | "Am"AA cB/A/ | e"E"e e2 ::\
P:B
"C"ge c3/e/ | "G"dB G3/B/ | "Am"cA "Em"G3/B/ | "Am"AA A2 :| 


\end{abc}
\index{Gray's Inn Mask}
\addcontentsline{toc}{subsection}{Gray's Inn Mask}
\begin{abc}[name=latex_playford33]
X:31
T:Gray's Inn Mask
C:John Playford, the English Dancing Master, 1651
N:arr. Emma Badowski
M:C
L:1/8
K:C major
P:A
"Dm"d2de f2d2 | "A"a2A2 A4 | "A"A2AB ^c2A2 | "Dm"d2d4d2 | "F"F3F "C"G3G | "F"A_BcB A3A | "G"Bcde "Dm"f2"A"e2 | "Dm"d8 ::
M:2/4
P:B
"Dm"d/e/f/g/ aa | "C"gf e2 | "Dm"dd ^cB | "A"A4 ::\
P:C
"A"^cA e2 | "Dm"fd f2 | "C"e3/f/ g2 |"F" f4 ::
M:C
P:D
"Dm"f4 e2d2 | "A"^c8 | "Dm"a4 g2f2 | "A"e8 ::\
M:2/4
P:E
"D"^ff fd | ^fg aa | "G"bb ag |  [1 "D"^f4 :|]  [2 "D"(3:2:2^f4d2 ::
M:6/4
P:F
"D"d3e^f2 f3ga2 | "A"A6- A4A2 | "A"A3B^c2 c3de2 | "G"G6- G4G2 | \
"G"G3AB2 B3^cd2 | "D"^F6 d4e2 | "D"^f3gag f2e4 |  [1 "D"d6- d4d2 :|] \
 [2 "D"d12 |] 


\end{abc}
\index{Greenwood}
\addcontentsline{toc}{subsection}{Greenwood}
\begin{abc}[name=latex_playford34]
X:32
T:Greenwood
C:John Playford, the English Dancing Master, 1651
N:arr. Emma Badowski
M:6/4
L:1/8
K:F major
"F"f4f2 f4c2 | "F"d4c2 A6 | "Gm"B3cd2 d3ef2 | "Gm"g4f2 e4d2 | "F"f4f2 f4c2 | "F"d4c2 A4f2 | "C"e4d2 "F"c3BA2 | "Gm"B6 G6 :| 


\end{abc}
\index{Grimstock}
\addcontentsline{toc}{subsection}{Grimstock}
\begin{abc}[name=latex_playford35]
X:33
T:Grimstock
C:John Playford, the English Dancing Master, 1651
N:arr. Monica Cellio. Matches Pile 2018
P:AABB x 3
M:6/4
L:1/8
K:G major
P:A
"G"g4f2 g4d2 | "C"e3fg2 "D"f4d2 | "G"B3cd2 "C"e4d2 | "C"c3dB2 "D"A4G2 | "G"g4f2 g4d2 | "C"e3fg2 "D"f4d2 | 
"G"B3cd2 "C"e4d2 | "C"c3B"D"A2 "G"G6 ::\
P:B
"G"G2G2A2 B2G2A2 | B2G2A2 B2G2D2 | "G"G2G2A2 B2G2A2 | B2G2A2 G6 :| 


\end{abc}
\index{Gun, the}
\addcontentsline{toc}{subsection}{The Gun}
\begin{abc}[name=latex_playford36]
X:95
T:The Gun
C:John Playford, the English Dancing Master, 1651
N:arr. Emma Badowski
M:6/4
L:1/8
K:F major
"Dm"d4d2 "A"^c4A2 | "Dm"d6 "C"e6 | "F"f4f2 "C"e4c2 | "F"f6 "C"g6 | "F"a4a2 "Gm"b4a2 | "C"a2g4 "Dm1.m"a6 | "C"f4g2 "F"a3ba2 | "F"g4f2 f6 ::
"F"a4f2 "C"a4b2 | "C"a4g2 g6 | "Bb"e4c2 "F"d3dc2 | "F"B4A2 "Dm"A6 | "F"c4A2 d4e2 | "Dm"f4g2 a6 | "C"f4e2 "Bb"f4d2 | "F"g4a2 b6 | "A"f4g2 "Dm"a3gf2 | e4d2 d6 :| 


\end{abc}
\index{Half Hannikin}
\addcontentsline{toc}{subsection}{Half Hannikin}
\begin{abc}[name=latex_playford37]
X:34
T:Half Hannikin
C:John Playford, the English Dancing Master, 1651
N:arr. Steve Hendricks. Matches Pile 46
M:6/4
L:1/8
K:G major
"G"B3cd2 d4"C"c2 | "G"B3cB2 "D"A4"G"G2 | "G"B3cd2 d3cB2 | "D"A3GF2 "G"G6 ::"G"B6 "C"c6 | "G"B6 "D"A4"G"G2 | 
"G"B3AB2 "C"c4"G"B2 | "D"A3GF2 "G"G6 | "G"B3AB2 "C"c3Bc2 | "G"B3AB2 "D"A4"G"G2 | "G"B3AB2 "C"c4"G"B2 | "D"A3GF2 "G"G6 :| 


\end{abc}
\index{Have at Thy Coat Old Woman}
\addcontentsline{toc}{subsection}{Have at Thy Coat Old Woman}
\begin{abc}[name=latex_playford38]
X:1
T:Have at Thy Coat Old Woman
C:John Playford, the English Dancing Master, 1651
N:arr. Emma Badowski
M:C
L:1/8
K:G major
"G"Bc | d3e "Am"d3B | cA3 e2"D"Bc | d3e "G"d3A | BG3- G2 ::\
"G"ef | gfed cB"Am"AB | cA3 e2"D"Bc | d3e "G"d3A | BG3- G2 :|


\end{abc}
\index{Health, the}
\addcontentsline{toc}{subsection}{The Health}
\begin{abc}[name=latex_playford39]
X:1
T:The Health
C:John Playford, the English Dancing Master, 1651
N:arr. Dave Lankford. Matches Pile 46
P:(AA BB)x3, or AA BB AA BB AA BA, or Ax12
M:C
L:1/8
K:G major
P:A
 |:Bc | "G"d2d2 d3c | B2G4cd | "C"e2d2 "G"d3c | d6AB | "C"c2c2 B2A2 | "G"B4 A2G2 | "Am"A2B2 "D"A3G | "G"G6 ::
P:B
 GA |"G"B6c2 | B6z2 | "C"c2d2 e2f2 | "G"g6z2 | "C"c3d e3e | "G"B3c d3d | "Am"A2B2 "D"c2d2 | "G"d6 :|


\end{abc}
\index{Hearts Ease}
\addcontentsline{toc}{subsection}{Hearts Ease}
\begin{abc}[name=latex_playford40]
X:1
T:Hearts Ease
C:John Playford, the English Dancing Master, 1651
N:arr. Drea Leed. Matches Pile 2018
P:AABB x 3
M:6/8
L:1/8
K:E minor
E |:\
P:A
"Am"ABc B2A | "E"^G2A B2E | "Am"A2B c2d | "E"e3 e2e ::\
P:B
"G"Bcd ded | g2d d2B | 
"Am"c2B c2d | "E"e3 e2e | "G"d2c Bcd | "Am"c2B ABc | "E"B2A ^GFG | "Am"A3 A3 :|


\end{abc}
\index{Hemp Dresser}
\index{London Gentlewoman, the}
\addcontentsline{toc}{subsection}{Hemp Dresser, or The London Gentlewoman}
\begin{abc}[name=latex_playford41]
X:37
T:Hemp Dresser, or The London Gentlewoman
C:John Playford, the English Dancing Master, 1651
N:arr. Emma Badowski
M:6/4
L:1/8
K:C major
"G"B3cd2 d4c2 | "G"B3cd2 "D"D3E^F2 | "G"G4G2 G4A2 | "G"B6 "C"c6 ::\
"G"B3AB2 G3AB2 | "D"A3GA2 D3E^F2 | "G"G4G2 G4A2 | "G"B6 "C"c6 :| 


\end{abc}
\index{Hit or Miss}
\addcontentsline{toc}{subsection}{Hit or Miss}
\begin{abc}[name=latex_playford42]
X:38
T:Hit or Miss
C:John Playford, the English Dancing Master, 1651
N:arr. Emma Badowski
M:6/4
L:1/8
K:C major
"C"c4d2 "Am"e2c2A2 | "G"G3ABc d2B2G2 | "C"e4f2 g2e2c2 | "G"d2B2G2 "C"c2G2E2 ::\
"C"c2c2c2 c2def2 | "C"e6 e3dc2 | 
"G"B3cd2 d3ef2 | "C"g6 e4c2 ::\
"G"e2d4 B4G2 | "C"g2f4 e4c2 | "C"e4d2 "Dm"f4e2 | "C"g6 e4c2 :| 


\end{abc}
\index{Hockley in the Hole}
\addcontentsline{toc}{subsection}{Hockley in the Hole}
\begin{abc}[name=latex_playford43]
X:39
T:Hockley in the Hole
C:John Playford, the English Dancing Master, 1651
N:arr. Emma Badowski
M:6/4
L:1/8
K:C major
"G"d2B2G2 d2B2G2 | "C"g4e2 c3def | "G"d2B2G2 d2B2G2 |  [1 "G"d4B2 G6 :|]  [2 "G"d4B2 G4G2 ::
"G"G2A2G2 G2B4 | "Am"A2B2A2 c2e4 | "G"d2B2G2 G2A2B2 | \
 [1 "D"A4G2 "G"G4G2 :|]  [2 "D"A4G2 "G"G6 |] 


\end{abc}
\index{Hyde Park}
\addcontentsline{toc}{subsection}{Hyde Park}
\begin{abc}[name=latex_playford44]
X:40
T:Hyde Park
C:John Playford, the English Dancing Master, 1651
N:arr. Kathy Van Stone. Matches Pile 2018
P:AABB x 3
L:1/8
M:6/4
K:D major
P:A
"D"f2g2a2 a2f2d2 | "D"f2g2a2 "A"e4d2 | "D"f2e2f2 "G"g2a2fg | "A"a2e3d "D"d6 ::\
P:B
"A"e3fe2 e2c2A2 | "A"A2B2c2 c2d2e2 | 
"D"d2e2f2 f2g2a2 | "A"a2g2a2 "D"f3ed2 | "A"e2f2g2 "G"g3ag2 | "A"e2f2g2 e3dc2 | "D"d2d2e2 f2f2g2 | "D"a2e3d d6 :| 


\end{abc}
\index{If all the World were Paper}
\addcontentsline{toc}{subsection}{If all the World were Paper}
\begin{abc}[name=latex_playford45]
X:41
T:If all the World were Paper
C:John Playford, the English Dancing Master, 1651
N:arr. Monica Cellio. Matches Pile 2018
P:AA BB x 3
M:6/4
L:1/8
K:C major
P:A
G2 | "F"A4G2 A4B2 | "C"c6 C6 | "Am"E4D2 E4F2 | "G"G6- G4E2 | "F"F4E2 F4G2 | "Dm"A4F2 D4c2 | "G"d4G2 A4B2 | "C"c6- c4 ::
P:B
c2 | "G"B4A2 B4c2 | d6 G4G2 | "C"c4B2 c4d2 | e6- e4e2 | "F"f4e2 "Gsus2"d4c2 | "G"B4A2 G4d2 | "G"g4G2 A4B2 | "C"c6- c4 :| 


\end{abc}
\index{Irish Lady}
\index{Anniseed Water Robin}
\addcontentsline{toc}{subsection}{Irish Lady, or Anniseed Water Robin}
\begin{abc}[name=latex_playford46]
X:42
T:Irish Lady, or Anniseed Water Robin
C:John Playford, the English Dancing Master, 1651
N:arr. Emma Badowski
M:3/4
L:1/8
K:C major
"Dm"d2e2f2 | "C"e2f2g2 | "F"c4c2 | "F"c4d2 | "C"e3fed | "C"c2d2ed | "Am"c2A2A2 |  [1 "Am"A6 :|]  [2 "Am"A4G2 ::
"Dm"F2F2F2 | "F"F3ED2 | "Dm"f2f2f2 | "C"f3ed2 | "C"e4f2 | "Dm"g3fef | "Dm"d4A2 |  [1 "Dm"d4G2 :|]  [2 d6 |] 


\end{abc}
\index{Irish Trot}
\addcontentsline{toc}{subsection}{Irish Trot}
\begin{abc}[name=latex_playford47]
X:43
T:Irish Trot
C:John Playford, the English Dancing Master, 1651
N:arr. Emma Badowski
M:C
L:1/8
K:G major
"Em"efgf e2B2 | "D"defe d2cd | "Em"eedc "G"B2d2 | "G"G2A2 B4 | "G"Bcde dBG2 | "D"FGAB AFD2 | "Em"EFGA B2e2 | "D"defg "Em"e4 :| 


\end{abc}
\index{Jack a Lent}
\addcontentsline{toc}{subsection}{Jack a Lent}
\begin{abc}[name=latex_playford48]
X:1
T:Jack a Lent
C:John Playford, the English Dancing Master, 1651
N:arr. Steve Hendricks. Matches Pile 46
P:AABB x 6
M:C
L:1/8
K:G major
P:A
"G"d4 d2cB | "D"AGAB "Am"c2BA | "Em"G2G2 "G"G2"Em"g2 | "Bsus4"f4 "E"e4 ::\
P:B
"G"B3c d2cB | "D"AGAB "Am"ABcd | "Em"BcBA "G"G2"Em"ga | "Bsus4"fefg "E"e4 :|


\end{abc}
\index{Jack Pudding}
\addcontentsline{toc}{subsection}{Jack Pudding}
\begin{abc}[name=latex_playford49]
X:1
T:Jack Pudding
C:John Playford, the English Dancing Master, 1651
N:arr Paul Butler. Matches Pile 46
P:AABB x 3
M:6/4
L:1/8
K:A minor
P:A
E2 | "Am"A4A2 "Em"B4G2 | "Am"c2A4 "G"d4B2 | "C"e2c4 "Em"B4A2 |  [1 "Am"A6 A4E2 :|]  [2 "Am"A6 A4a2 ::\
P:B
"C"g2e4 "Dm"f4d2 | "C"e2c4 "G"d4B2 | 
"Am"c2A4 "Em"B2G4 | "C"c6 c4c2 | "G"d2B4 "Em"e4B2 | "E"^G2E4 "C"c4d2 | "Am"e2c4 "G"d2B4 |  [1 "Am"A6 A4a2 :|]  [2 "Am"A6 A4 |]


\end{abc}
\index{Jenny Pluck Pears}
\addcontentsline{toc}{subsection}{Jenny Pluck Pears}
\begin{abc}[name=latex_playford50]
X:1
T:Jenny Pluck Pears
C:John Playford, the English Dancing Master, 1651
N:arr. Dave Lankford. Matches Pile 2018
P:AAB x 6
M:6/8
L:1/8
K:A minor
P:A
"Am"A2A "G"B3/c/d | "Am"c3/B/A "Em"G2E | "Am"A2A "G"B3/c/d | "Am"c3/B/A "Em"G2E | "G"e2e "Am"d2B | "Em"c3/B/A "D"G2E | "Em"^F3/G/A "Am"G3/F/G |  [1 A3 "Am"A2E :|]
 [2 A3 "Am"A3 :|\
M:3/4
P:B
A2e2d2 | c4B2 | A2e2d2 | c4B2 | "^Intro"A2e2d2 | c3"Em"BA2 | G3"Am"^FG2 | A6 |]


\end{abc}
\index{Jog On}
\addcontentsline{toc}{subsection}{Jog On}
\begin{abc}[name=latex_playford51]
X:47
T:Jog On
C:John Playford, the English Dancing Master, 1651
N:arr. Emma Badowski
M:6/4
L:1/8
K:D major
"D"d4A2 d4e2 | "D"f3gf2 "A"e2c2A2 | "D"d2e2d2 a4g2 | \
M:9/4
"D"f6 "A"e6- e4f2 | "G"g3ag2 "D"f3gf2 "A"e3dc2 | \
M:6/4
"D"d2e2d2 "A"c2B2c2 | "G"B6 "D"A6 :| 


\end{abc}
\index{Kemp's Jig}
\addcontentsline{toc}{subsection}{Kemp's Jig}
\begin{abc}[name=latex_playford52]
X:48
T:Kemp's Jig
C:John Playford, the English Dancing Master, 1651
N:arr. Emma Badowski
M:6/4
L:1/8
K:F major
|:"Dm"d3ed2 d3ed2 | "C"c3de2 e3dc2 | "Dm"d3ef2 e4d2 | "Am"c6 A6 :|\
"Dm"F6 "C"G6 | "F"A12 | 
"Dm"F6 "C"G6 | "D"A12 | \
"Dm"F4E2 F4G2 | A4A2 A3GF2 | "Am"E4D2 E3FE2 | "Dm"D12 :| 


\end{abc}
\index{Kettledrum}
\addcontentsline{toc}{subsection}{Kettledrum}
\begin{abc}[name=latex_playford53]
X:49
T:Kettledrum
C:John Playford, the English Dancing Master, 1651
N:arr. Emma Badowski
M:C
L:1/8
K:C major
"C"efgf e2d2 | "Dm"A2f2 A2f2 | "C"efgf e2d2 | "Dm"A2f2 d4 ::\
"C"efga g3a | g3a g2e2 | 
"Dm"fga2 "E"a2^g2 | "A"a6fg | "Dm"agfe defd | "A"e4 A4 | "Dm"F2A2 A2f2 | d8 :| 


\end{abc}
\index{Lady Lie Near Me}
\addcontentsline{toc}{subsection}{Lady Lie Near Me}
\begin{abc}[name=latex_playford54]
X:50
T:Lady Lie Near Me
C:John Playford, the English Dancing Master, 1651
N:arr. Emma Badowski
M:3/4
L:1/8
K:F major
P:A
"F"ABc2A2 | G2A2F2 | "Bb"d2c2f2 | "C"d4c2 ::\
P:B
"F"f2f2c2 | f2f2g2 | a4g2 | a2a2g2 | \
"Dm"f3ga2 | "F"c2A4 | "C"G4"F"F2 :| 


\end{abc}
\index{Lady Spellor}
\addcontentsline{toc}{subsection}{Lady Spellor}
\begin{abc}[name=latex_playford55]
X:51
T:Lady Spellor
C:John Playford, the English Dancing Master, 1651
N:arr. Emma Badowski
M:6/4
L:1/8
K:C major
"G"d4d2 d4d2 | "G"d6 B6 | "C"c4B2 c4d2 | "C"e6 c6 | \
"D"A4A2 A4B2 | "C"c4d2 e4c2 | "G"d4c2 B4A2 | "G"B6 G6 :| 


\end{abc}
\index{Lavena/Picking of Sticks}
\addcontentsline{toc}{subsection}{Lavena/Picking of Sticks}
\begin{abc}[name=latex_playford56]
X:52
T:Lavena/Picking of Sticks
C:John Playford, the English Dancing Master, 1651
P:for Picking of Sticks:Ax7 Bx3 Ax7 for Lavena:repeat A
N:arr. Robert Smith with changes by Aaron Elkiss. Matches Pile 2018
M:6/4
L:1/8
K:D minor
P:A - Lavena
d2 | "Dm"d4d2 "A"^c3=Bc2 | "Dm"d4d2 A4d2 | "Dm"d6 "A"^c3=Bc2 | "Dm"d6 A4B2 | 
"Gm"B3dc2 B3cB2 | "Dm"A3BA2 "C"G3FE2 | "Dm"F3GA2 "Gm"G3AF2 | "A"E6 "D"D4 ::
K:C major
P:B - Picking of Sticks
G2 | "G"B3cd2 d3cd2 | "F"c4A2 A4A2 | "F"A3Bc2 c3dc2 | "G"B4G2 G4G2 | 
"G"B3cd2 d3cd2 | "F"c4A2 A4A2 | "F"A3Bc2 c3dc2 | "G"B4G2 G4 :| 


\end{abc}
\index{Lord of Carnarvan's Jegg}
\addcontentsline{toc}{subsection}{Lord of Carnarvan's Jegg}
\begin{abc}[name=latex_playford57]
X:1
T:Lord of Carnarvan's Jegg
C:John Playford, the English Dancing Master, 1651
N:arr. Dave Lankford. Matches Pile 2018
P:Play 8 times
M:C|
L:1/4
K:C major
"G"BG B/c/d | "F"cA A/B/c/d/ | "G"BG de/f/ | g2 d2 | BG B/c/d | "F"cA A/B/c/d/ | "G"BG de/f/ | g2 de/f/ ||
"G"g/f/e/d/ g3/B/ | "F"AA cd/c/ | "G"BA b3/a/ | "C"g2 "G"d2 | "G"g/f/e/d/ g3/B/ | "F"AA cd/c/ | "G"BA b3/a/ | "C"g2 "G"d2 |]


\end{abc}
\index{Lull Me Beyond Thee}
\addcontentsline{toc}{subsection}{Lull Me Beyond Thee}
\begin{abc}[name=latex_playford58]
X:54
T:Lull Me Beyond Thee
C:John Playford, the English Dancing Master, 1651
N:arr. Steve Hendricks. Matches Pile 46
P:A BB x 3
M:6/4
L:1/8
K:A minor
E2 | \
P:A
"Am"E4E2 c3Bc2 | "Dm"d3cd2 "E"e4E2 | "Am"E4E2 c3BA2 | "E"^G6 "A"A4 ::\
P:B
"Am"A2 | "Am"c4c2 "Dm"d3cd2 | "C"e3fe2 "G"d4G2 | 
"C"c4c2 "G"d3cd2 | "C"e6 "G"d4d2 | "C"e3fe2 "G"d3cB2 | "Am"c3BA2 "E"e4E2 | "Am"E4E2 c3BA2 | "E"^G6 "A"A4 :| 


\end{abc}
\index{Mage on a Cree}
\addcontentsline{toc}{subsection}{Mage on a Cree}
\begin{abc}[name=latex_playford59]
X:55
T:Mage on a Cree
C:John Playford, the English Dancing Master, 1651
N:arr. Steve Hendricks. Matches Pile 46
P:14 or 12 times through
M:6/4
L:1/8
K:D minor
"Gm"G2 |:"Gm"B3cde "F"f4c2 | "F"A4F2 c4A2 | "Gm"B3cd2 "Bb"d3cd2 | 
"Gm"B3AG2 "D"d4"Gm"G2 | "Gm"B3cde "F"f4c2 | "F"A4F2 c4A2 | "Gm"B3cd2 "C"e3fg2 | "D"^f3ef2 "G"g4 :| 


\end{abc}
\index{Maid Peeped Out at the Window}
\index{Friar in the Well, the}
\addcontentsline{toc}{subsection}{Maid Peeped Out at the Window}
\begin{abc}[name=latex_playford60]
X:56
T:Maid Peeped Out at the Window, or The Friar in the Well
C:John Playford, the English Dancing Master, 1651
N:arr. Emma Badowski
M:3/4
L:1/8
K:C major
D2 | "G"G4G2 | "G"G3AB2 | "Am"A4G2 | "Am"E6 | "G"B4B2 | "C"c3de2 | "G"d4B2 | "G"G4 ::\
G2 | "G"B4c2 | "G"d3ed2 | "F"c4B2 | "F"A4G2 | 
"F"F4E2 | "F"F4G2 | "F"A3GF2 | "Am"E6 | "G"D4D2 | "F"G4G2 | "Am"F3EF2 | "G"E4E2 | "F"D4G2 | "G"F3GA2 | "G"G6 | G4 :| 


\end{abc}
\index{Maiden Lane}
\addcontentsline{toc}{subsection}{Maiden Lane}
\begin{abc}[name=latex_playford61]
X:57
T:Maiden Lane
C:John Playford, the English Dancing Master, 1651
N:arr. Steve Hendricks. Matches Pile 46
P:AABBCC x 3
M:C
L:1/8
K:G major
P:A
"G"d3c BAG2 | "G"g2"D"f2 "G"g2d2 | "G"g2"D"f2 "G"g2dc | "G"B2"D"A2 "G"G4 ::\
P:B
"Am"ABcd e2A2 | "Am"c2"E"B2 "Am"A2e2 | 
"Am"e2e2 e2"Dm"dc | "E"BABc A4 ::\
P:C
"G"d2Bc d2Bc | "G"dcBA B2G2 | "G"d2Bc "D"dedc | "G"B2"D"A2 "G"G4 :| 


\end{abc}
\index{Merry Merry Milkmaids}
\addcontentsline{toc}{subsection}{Merry Merry Milkmaids}
\begin{abc}[name=latex_playford62]
X:58
T:Merry Merry Milkmaids
C:John Playford, the English Dancing Master, 1651
N:arr. Jay Ter Louw. Matches Pile 2018
P:AABB x 3
M:6/4
L:1/8
K:G major
P:A
"G"G2 | "G"G3AG2 D4d2 | "G"B6- B4G2 | "G"G3AG2 D4=F2 | "C"E6- E4C2 | "C"C3DEF G4A2 |
 "D"A3GFE D4d2 | "G"B3cd2 "D"A4G2 | "G"G6- G4 ::\
P:B
G2 | "G"B3cd2 d3cd2 | "G"B3cd2 d3cB2 |
"C"c4d2 e4f2 | "G"g6- g4d2 | "Em"g4d2 "G"B3cd2 | "C"e4c2 "D"A3Bc2 | "G"d4B2 G3AB2 | 
"Am"A4F2 "D"D3EF2 | "C"G4G2 "Am"E3FG2 | "D"A4G2 F3ED2 | "G"B3cd2 "D"A4G2 | "G"G6- "G"G4 :| 


\end{abc}
\index{Milkmaids Bob}
\addcontentsline{toc}{subsection}{Milkmaids Bob}
\begin{abc}[name=latex_playford63]
X:59
T:Milkmaids Bob
C:John Playford, the English Dancing Master, 1651
N:arr. Emma Badowski
M:6/4
L:1/8
K:G major
"G"d2d2e2 d2d2B2 | "F"c2A2c2 "G"B4G2 | "G"d2d2e2 "D"f3ga2 | "C"g2e2g2 "D"f4d2 | "G"d2g2d2 "C"e3cA2 | "Am"e2a2e2 "D"f3ed2 | "G"d2g2d2 "Em"g3ab2 | "C"b2"D"a2g2 "G"g6 :| 


\end{abc}
\index{Millfield}
\addcontentsline{toc}{subsection}{Millfield}
\begin{abc}[name=latex_playford64]
X:60
T:Millfield
C:John Playford, the English Dancing Master, 1651
N:arr. Emma Badowski
M:6/4
L:1/8
K:G major
G2 | "D"A4 B2c3B | A2 | "G"B3A G2d4 | c2 | "D"A4 F2D4 | G2 | "D"F3E F2"G"G4 ::\
G2 | "D"A4 F2D4 | d2 | "G"B4 G2G4 | B2 | "D"A4 F2D4 | G2 | "D"F3E F2"G"G4 :| 


\end{abc}
\index{Millison's Jig}
\addcontentsline{toc}{subsection}{Millison's Jig}
\begin{abc}[name=latex_playford65]
X:61
T:Millison's Jig
C:John Playford, the English Dancing Master, 1651
N:arr. Steve Hendricks
M:6/4
L:1/8
K:C major
"G"g2 | "C"g3fe2 "G"d4"C"c2 | "G"B4"C"c2 "G"d4g2 | "C"g3fe2 "G"d4"C"c2 | "G"B6 "C"c4 ::"C"G2 | "G"G2B2G2 G2B2G2 | "G"G2B2G2 B4B2 |
 "C"c2e2c2 c2e2c2 | "C"c2e2c2 e4e2 | "Bb"d2f2d2 d2f2d2 | "Bb"d2f2d2 f4"G"g2 | "C"g3fe2 "G"d4"C"c2 | "G"B6 "C"c4 :| 


\end{abc}
\index{Mundesse}
\addcontentsline{toc}{subsection}{Mundesse}
\begin{abc}[name=latex_playford66]
X:1
T:Mundesse
C:John Playford, the English Dancing Master, 1651
N:arr. Steve Hendricks
M:C
L:1/8
K:G
"G"G2 GG G2 A2|B3 A G2 d2|"Am"c3 B ABcd|"G"B3 A G2 G2|\
"D"A3 G A2 B2|"C"c3d c2 B2|"Am"A2 G2 "D"G2 F2|"G"G8:|
|:"Em"G3 A BABc| "D"d3 e d2 ef|"C"g2 f2 "A"edef | [1 "D"d8 :| [2 "D"d6 \
|:f2 | "Em"g2 f2 g2 e2|"Bm"d6 ef| "Em"g2 f2 g2 e2|"Bm"d6 e2| 
f2 g2 f2 g2| d6 c2| "G"B2 AA "D"GFGA| "G"G6 :|\
|:G2|"G"B3 c d2 B2|"C"e3 d c2 B2|"D"A2 G2 G2 F2| "G"G6 :|


\end{abc}
\index{My Lady Cullen}
\addcontentsline{toc}{subsection}{My Lady Cullen}
\begin{abc}[name=latex_playford67]
X:63
T:My Lady Cullen
C:John Playford, the English Dancing Master, 1651
N:arr. Steve Hendricks
P:AABB x 4 = one progression
M:C|
L:1/4
K:D minor
P:A
"Dm"AA f3/f/ | "Gm"g/f/e/d/ "A"^c3/c/ | "Dm"dd Af | "Asus4"e2 "D"d2 ::\
P:B
"F"AA "C"c3/c/ | "F"F3/G/ A3/c/ | "Gm"BG GG | "A"A=B/^c/ "D"d2 :| 


\end{abc}
\index{New Bo Peep, the}
\addcontentsline{toc}{subsection}{The New Bo Peep}
\begin{abc}[name=latex_playford68]
X:97
T:The New Bo Peep
C:John Playford, the English Dancing Master, 1651
N:arr. Monica Cellio. Matches Pile 2018
P:AABB x 3
M:6/4
L:1/8
K:G mixolydian
P:A
ef | "G"g6 "Am"e6 | "G"d2e2d2 B3cd2 | "Am"c2d2B2 "D"A3GAB | "G"G6- G4 ::\
P:B
B2 | "C"c3de2 e3dc2 | "G"d2B4 G4B2 | "C"c3de2 e3dc2 | "G"d2B4 G4e^f | 
"G"g6- g4^fg | "D"a6- a4e^f | "G"g6- g4^fg | "D"a6- a4ga | "Em"b4e2 "C"e4^f2 | "G"g4d2 B2c2d2 | "C"c4B2 "D"A6 | "G"G6- G4 :| 


\end{abc}
\index{New Exchange, the}
\addcontentsline{toc}{subsection}{The New Exchange}
\begin{abc}[name=latex_playford69]
X:98
T:The New Exchange
C:John Playford, the English Dancing Master, 1651
N:arr. Emma Badowski
M:6/4
L:1/8
K:F major
D2 | "Gm"G3ABc B2"D"A4 | "Gm"G6- G4G2 | "F"A3Bc2 c2"Gm"B4 | "D"A6- A4A2 | \
"F"A3Bc2 "Gm"d4G2 | "Dm"F2F2E2 F4D2 | "Gm"G3ABc B2"D"A4 | "Gm"G6- G4 :| 


\end{abc}
\index{New New Nothing}
\addcontentsline{toc}{subsection}{New New Nothing}
\begin{abc}[name=latex_playford70]
X:64
T:New New Nothing
C:John Playford, the English Dancing Master, 1651
N:arr. Emma Badowski
M:C
L:1/8
K:Bb major
"Bb"d4 d4 | "Cm"cd"Gm"Bc "D"A2"Gm"G2 | "Bb"B3c def2 | "Gm"g2d2 "F"c2"Bb"B2 | "F"f4 f4 | "F"f2 "Cm"ga "Bb"b2f2 | "Cm"gfed c2"F"f2 | "Cm"gfed c2"Bb"B2 :| 


\end{abc}
\index{Newcastle}
\addcontentsline{toc}{subsection}{Newcastle}
\begin{abc}[name=latex_playford71]
X:1
T:Newcastle
C:John Playford, the English Dancing Master, 1651
N:arr. Dave Lankford
P:AA BB x 3
M:C|
L:1/4
K:G major
P:A
D | "G"Bd GA | G3/A/ GD | Bd Gd | "C"eg2f/e/ | "G"dB AG | "C"Ee2d/c/ | "G"dB "D"A3/G/ | "G"G3 ::
P:B
"C"e/f/ | "G"g/f/e/d/ g3/B/ | "C"Ag2A | "G"G3/A/ BF | "C"Ee2e/f/ | "G"g/f/e/d/ g3/B/ | "Am"AA "C"c3/d/ | "Am"eB "D"A3/G/ | "G"G3 :|


\end{abc}
\index{Night Peace}
\addcontentsline{toc}{subsection}{Night Peace}
\begin{abc}[name=latex_playford72]
X:66
T:Night Peace
C:John Playford, the English Dancing Master, 1651
N:arr. Emma Badowski
M:6/4
L:1/8
K:D major
P:A
"D"f4d2 a4f2 | d4A2 d2f4 | "Em"e2g4 "D"f3ed2 | "G"B2d2ef "A"g4a2 ::\
P:B
"G"b2b2g2 "D"a2a2g2 | f3ed2 "A"e4A2 | "D"d2d2c2 d2d2A2 | "A"f2e4 "D"d6 :| 


\end{abc}
\index{Nonesuch}
\addcontentsline{toc}{subsection}{Nonesuch}
\begin{abc}[name=latex_playford73]
X:1
T:Nonesuch
C:John Playford, the English Dancing Master, 1651
N:arr. Paul Butler, with changes by Aaron Elkiss. Matches Pile 2018
P:AABB x 9 (or sometimes 11 or 15)
K:A minor
M:C|
L:1/4
P:A
|:e | "Am"ec de | cB/c/ Ae | ec de |  [1 c2 c :|]  [2 c2 A ::\
P:B
B | "Em"BG AB | "Am"cB/c/ AB | "Em"BG AB | "Am"c2 A :|


\end{abc}
\index{Old Mole}
\addcontentsline{toc}{subsection}{Old Mole}
\begin{abc}[name=latex_playford74]
X:68
T:Old Mole
C:John Playford, the English Dancing Master, 1651
N:arr. Jay Ter Louw. Matches Pile 2018
P:11 times through
M:6/4
L:1/8
K:C major
"C"c4c2 "Am"A3Bc2 | "Dm"d4B2 "G"G4G2 | "C"c4c2 "Am"A3Bc2 | "G"d6 g6 | "C"e4c2 "Am"A3Bc2 | "G"d4B2 G4G2 | "Am"A3Bc2 "G"B3cd2 | "C"c6- c4c2 | 
"C"c3dc2 "Am"A3Bc2 | "G"d2B4 G4c2 | "C"c3dc2 "F"A3Bc2 | "G"d6 g6 | "Em"e3de2 "Am"c3de2 | "G"d2B4 "C"G4G2 | "F"A3Bc2 "G"d2B4 | "C"c12 |] 


\end{abc}
\index{Once I Loved a Maiden Fair}
\addcontentsline{toc}{subsection}{Once I Loved a Maiden Fair}
\begin{abc}[name=latex_playford75]
X:69
T:Once I Loved a Maiden Fair
C:John Playford, the English Dancing Master, 1651
M:C
L:1/8
K:D major
"D"f2f2 f4 | "A"e2f2 "D"d2cd | "A"e2e2 e2f2 | "A"e4 "D"d4 ::\
"G"B2c2 d4 | "A"e2f2 "D"d2cd | "A"e2e2 e2f2 | "A"e4 "D"d4 :| 


\end{abc}
\index{Parson's Farewell}
\addcontentsline{toc}{subsection}{Parson's Farewell}
\begin{abc}[name=latex_playford76]
X:1
T:Parson's Farewell
C:John Playford, the English Dancing Master, 1651
N:arr. Drea Leed. Matches Pile 2018.
M:C
L:1/8
P:AABB x3
K:A minor
P:A
|:"Am"c2A2 A2Bc | "G"d2G2 G3G | "F"ABc2 B2A2 | [1 "Em"G2E2 E4 :| [2 "Em"G2E2 E2e2 |:\
P:B
"C"c4 c2e2 | c4 c2e2 | 
cde2 cde2 | "G"d2B2 B4 | Bcd2 Bcd2 | "Am"c3B ABcd | "E"e2dc BAB2 |  [1 "Am"A6e2 :|] [2 A8 :|


\end{abc}
\index{Paul's Steeple}
\addcontentsline{toc}{subsection}{Paul's Steeple}
\begin{abc}[name=latex_playford77]
X:71
T:Paul's Steeple
C:John Playford, the English Dancing Master, 1651
N:arr. Emma Badowski
M:C
L:1/8
K:F major
"Gm"G2G2 G2G2 | B6cB | "F"A2F2 F2F2 | F6F2 | "Gm"G2G2 G2A2 | B6c2 | "D"d2d2 d2d2 | d6c2 | 
"Bb"B2B2 B2B2 | B6B2 | "F"c2c2 c2c2 | c6c2 | "Gm"d4 c2B2 | "D"A2G2 A4 | "G"=B2G2 G2G2 | G8 :| 


\end{abc}
\index{Paul's Wharf}
\addcontentsline{toc}{subsection}{Paul's Wharf}
\begin{abc}[name=latex_playford78]
X:72
T:Paul's Wharf
C:John Playford, the English Dancing Master, 1651
N:arr. Emma Badowski
M:6/4
L:1/8
K:D major
"D"f2d4 "Em"e2B2c2 | "D"d3AF2 "A"E4e2 | "D"f2d2cd "Em"e2B2c2 | "D"d3F"A"E2 "D"D6 ::\
"D"F2A2F2 "Em"G2B4 | "D"A2d2B2 "A"c2e4 | "D"f2e2d2 "Em"e2B2c2 | "D"d3F"A"E2 "D"D6 :| 


\end{abc}
\index{Pepper's Black}
\addcontentsline{toc}{subsection}{Pepper's Black}
\begin{abc}[name=latex_playford79]
X:73
T:Pepper's Black
C:John Playford, the English Dancing Master, 1651
N:arr. Emma Badowski
M:6/4
L:1/8
K:D major
d2 | "A"c4 A2e4 | e2 | "A"c3B A2e4 | A2 | "D"d2e2 d2"A"e3f | ga | "D"f6d4 ::\
d2 | "A"e2f2 g2"G"g3a | g2 | "A"e2f2 g2"G"g3f | g2 | "D"f4 g2"A"a3b | ga | "D"f6d4 :| 


\end{abc}
\index{Petticoat Wag}
\addcontentsline{toc}{subsection}{Petticoat Wag}
\begin{abc}[name=latex_playford80]
X:74
T:Petticoat Wag
C:John Playford, the English Dancing Master, 1651
N:arr. Emma Badowski
M:6/4
L:1/8
K:F major
"Gm"d4c2 B3cA2 | G4A2 "D"^F6 | "Gm"G4"F"A2 "Bb"B6 | "Bb"B4"F"c2 "Bb"d6 | "Bb"d2f2e2 "F"f3ga2 | "Gm"g4g2 "D"^f4d2 | "Gm"d3ed2 "F"c4A2 | "Gm"B4"F"c2 "Bb"d6 ::
"F"f4g2 a3ga2 | "Bb"b4d2 f6 | "F"A4B2 c3Bc2 | "Bb"d4e2 f3ed2 | "C"g6- g4e2 | "Dm"a6- a4f2 | "Gm"b3ag2 g3a"D"^f2 | "Gm"g12 :| 


\end{abc}
\index{Prince Rupert's March}
\addcontentsline{toc}{subsection}{Prince Rupert's March}
\begin{abc}[name=latex_playford81]
X:1
T:Prince Rupert's March
C:John Playford, the English Dancing Master, 1651
N:arr. Emma Badowski
M:C
L:1/8
K:D dor
d2 | "Dm"d2A2 d2"C"e2 | "F"f3g f2f2 | "A"e2"Dm"d2 "Gm"d2"A"^c2 | "Dm"d6 ::\
d2 | "C"e3d efg2 | c3d c2c2 | 
"F"f2f2 "C"gaga | "F"f6f2 | "C"e3d efg2 | c3d c2f2 | "A"e2"Dm"d2 "Gm"d2"A"^c2 | "Dm"d6 :|


\end{abc}
\index{Punk's Delight}
\addcontentsline{toc}{subsection}{Punk's Delight}
\begin{abc}[name=latex_playford82]
X:76
T:Punk's Delight
C:John Playford, the English Dancing Master, 1651
N:arr. Emma Badowski
M:C
L:1/8
K:G major
P:A
"D"A3A A3A | A4 F4 | "G"G3B A3G | d4 B4 | "D"A3B c3c | f4 F4 | "G"G3B A3G | d4 B4 ::
P:B
"D"A3F D3B | A3F D3D | "G"G3B A3G | d4 B4 | "D"A3F D3B | A3F D3D | "G"G3B A3G | d4 B4 :| 


\end{abc}
\index{Rose is White and Rose is Red}
\addcontentsline{toc}{subsection}{Rose is White and Rose is Red}
\begin{abc}[name=latex_playford83]
X:77
T:Rose is White and Rose is Red
C:John Playford, the English Dancing Master, 1651
N:arr. Emma Badowski
M:6/4
L:1/8
K:C major
"G"B4B2 B3AG2 | "Am"A4A2 c3BA2 | "G"B4B2 B3AG2 | B6 d4ef | \
"C"g3^fe2 "G"d3cB2 | "Am"A4A2 e3dc2 | "G"B4B2 B3AG2 | B6 d6 :| 


\end{abc}
\index{Row Well Ye Mariners}
\addcontentsline{toc}{subsection}{Row Well Ye Mariners}
\begin{abc}[name=latex_playford84]
X:78
T:Row Well Ye Mariners
C:John Playford, the English Dancing Master, 1651
N:arr. Monica Cellio. Matches Pile 2018
P:AA BB x 3
M:6/4
L:1/8
K:G major
G2 | "D"F4E2 F4G2 | A6 A4B2 | "C"c2d2c2 "G"B2A2B2 | "D"A4G2 "G"G4G2 | "D"F4E2 F4G2 | A6 A4B2 | "C"c2d2c2 "G"B2A2B2 | "D"A4G2 "G"G4d2 || 
"G"d2e2d2 "D"A4B2 | "G"c2d2B2 "D"A4d2 | "G"d2e2d2 "D"A4B2 | "G"c2d2B2 "D"A4G2 || "G"G2A2G2 "D"F4G2 | "D"A2F2D2 "G"G4G2 
| "G"G2A2G2 "D"F4G2 | "D"A2F2D2 "G"G6 |:\
"G"d6 "D"A6 | "G"d6 "D"A6 | "D"d4e2 d4c2 | "G"B4A2 G6 :| 


\end{abc}
\index{Rufty Tufty}
\addcontentsline{toc}{subsection}{Rufty Tufty}
\begin{abc}[name=latex_playford85]
X:1
T:Rufty Tufty
C:John Playford, the English Dancing Master, 1651
N:arr. Drea Leed. Matches Pile 2018.
P:AA BB CC x 3 or AA B CC x 3
M:C
L:1/8
K:C major
P:A
"G"G4 G2A2 | B4 A2B2 | "C"c2c2 "D"B3A |  [1 "G"G8 :|]  [2 "G"G6AB ::\
P:B
"C"c2B2 "F"A2G2 | "C"G3F E3F |  G2G2 F2E2 | 
[1 "G"D4 "C"C2AB :|]  [2 "G"D4 "C"C4 ::\
P:C
"C"E3F G2G2 | "F"A2F2 "C"G4 | "C"E3F G2G2 | "F"A2F2 "C"G2EF | "C"G2G2 F2 E2 | "G"D4 "C"C4 :|


\end{abc}
\index{Saint Martins}
\addcontentsline{toc}{subsection}{Saint Martins}
\begin{abc}[name=latex_playford86]
X:80
T:Saint Martins
C:John Playford, the English Dancing Master, 1651
N:arr. Steve Hendricks. Matches Pile 46
P:AA BB x 3
M:C|
L:1/4
K:G major
E |:"Am"A/B/c "G"B/c/d | "C"c2 c"G"d | "C"e/f/g "D"f3/e/ | "E"e3"Em"e/f/ | "G"gf/e/ dd/e/ | "F"=fe/d/ cc/d/ 
| "Am"ed/c/ "E"B3/A/ |  [1 "A"A3"Am"E :|] \
 [2 "A"A4 |:"Em"Be B"Am"c | "Em"B3/A/ G/F/E | "C"c/d/e/=f/ "G"d3/e/ | 
"C"c3e/f/ | "G"gd/e/ "F"=fc/d/ | "Em"eB/c/ "Dm"dc/d/ | "Am"ed/c/ "E"B3/A/ |  [1 "A"A4 :|]  [2 "A"A3 |] 


\end{abc}
\index{Saraband}
\addcontentsline{toc}{subsection}{Saraband}
\begin{abc}[name=latex_playford87]
X:81
T:Saraband
C:John Playford, the English Dancing Master, 1651
N:arr. Emma Badowski
M:6/4
L:1/8
K:D major
"D"a2f2a2 "Em"g2e2g2 | "D"f2d2f2 "A"e4A2 | "A"c2d2e2 "D"f2g2a2 | "G"g2a2"D"f2 "G"g4"D"f2 | "D"a2f2a2 "Em"g2e2g2 | "D"f2d2f2 "A"e4A2 | "A"c2d2e2 "D"f2g2a2 | "G"g2a2"D"f2 "G"g4"D"f2 | 
"A"c2c2c2 c3BA2 | "D"a2g2f2 "A"e4A2 | "A"c2d2e2 "D"f2g2a2 | "G"g2a2f2 "A"e4"D"d2 | "A"c2c2c2 c3BA2 | "D"a2g2f2 "A"e4c2 | "A"c2d2e2 "D"f2g2a2 | "G"g2a2f2 "A"e4"D"d2 :| 


\end{abc}
\index{Saturday Night and Sunday Morn}
\addcontentsline{toc}{subsection}{Saturday Night and Sunday Morn}
\begin{abc}[name=latex_playford88]
X:1
T:Saturday Night and Sunday Morn
C:John Playford, the English Dancing Master, 1651
N:arr. Emma Badowski
M:6/4
L:1/8
K:D major
f2 | "D"f3ed2 a4A2 | "A"e2e2e2 e4g2 | "D"f3ed2 a4A2 | d2d2d2 d4 ::\
c2 | "Em"B3AG2 "D"F3ED2 | "A"e2e2e2 e4c2 | "Em"B3AG2 "D"F3ED2 | d2d2d2 d4 :|


\end{abc}
\index{Scotch Cap}
\addcontentsline{toc}{subsection}{Scotch Cap}
\begin{abc}[name=latex_playford89]
X:83
T:Scotch Cap
C:John Playford, the English Dancing Master, 1651
N:arr. Jay Ter Louw. Matches Pile 2018
P:AABB x 3 or AABB x 5 or AA(BB x 3)(AABB x 2) 
M:6/4
L:1/8
K:A minor
P:A
D2 | "Dm"D4D2 d4d2 | "Am"e4d2 c4A2 | "G"B4d2 B4A2 | B6 d4 ::\
P:B
de | "Dm"f4A2 A4de | f4A2 A4de | 
f4A2 A4F2 | A6 d6 | "G"B4B2 "Dm"A4F2 | "C"G4G2 "Dm"d4d2 | "Em"B4B2 "Dm"A2F4 | "Em"E6 "Dm"D6 :| 


\end{abc}
\index{Shepherd's Holiday}
\index{Labour in Vain}
\addcontentsline{toc}{subsection}{Shepherd's Holiday, or Labour in Vain}
\begin{abc}[name=latex_playford90]
X:84
T:Shepherd's Holiday, or Labour in Vain
C:John Playford, the English Dancing Master, 1651
N:arr. Emma Badowski
M:6/4
L:1/8
K:Bb major
d2 | "Gm"d3ed2 "Eb"g6 | "F"f6 e6 | "Gm"d3ed2 "Cm"c4B2 | "F"A2G2A2 "Bb"BABcde | "Bb"f3ed2 "F"c4B2 | "Bb"B6- B4 ::
d2 | "Bb"d3ef2 d2c2B2 | d4B2 d6 | "Gm"B3cd2 B2A2G2 | "D"d4^F2 d6 | \
"Eb"G3AB2 G2F2E2 | "Bb"B6 "F"c3def | "Bb"d3cBA "F"B2A3G | "Gm"G6- G4 :| 


\end{abc}
\index{Skellamefago}
\addcontentsline{toc}{subsection}{Skellamefago}
\begin{abc}[name=latex_playford91]
X:85
T:Skellamefago
C:John Playford, the English Dancing Master, 1651
N:arr. Emma Badowski
M:6/4
L:1/8
K:C major
"G"B2c2d2 d4d2 | B2c2d2 d4d2 | "C"e2f2g2 g4G2 | "G"B3cB2 "D"A4B2 | \
"C"c3dc2 "G"B3cd2 | "D"A3BG2 "Em"E4G2 | "G"D2D2G2 G2B2G2 | G2B2G2 G6 :| 


\end{abc}
\index{Slip}
\addcontentsline{toc}{subsection}{Slip}
\begin{abc}[name=latex_playford92]
X:86
T:Slip
C:John Playford, the English Dancing Master, 1651
N:arr. Emma Badowski
M:C
L:1/8
K:D major
"D"d2A2 d2e2 | f3g a2f2 | "G"b2a2 gaf2 | "A"e4 "D"d4 :: "A"e2cd e2A2 | e2cd e2A2 | 
e2f2 "Em"e2b2 | "B"f3e "Em"e4 | "D"f3ga2f2 | "G"g2a2 b2B2 | "D"A2d2 c2d2 | "A"e3d "D"d4 :| 


\end{abc}
\index{Soldier's Life}
\addcontentsline{toc}{subsection}{Soldier's Life}
\begin{abc}[name=latex_playford93]
X:87
T:Soldier's Life
C:John Playford, the English Dancing Master, 1651
N:arr. Emma Badowski
M:6/4
L:1/8
K:D major
d2 | "D"f3ga2 f3ed2 | "A"e4c2 A4d2 | "D"f3ga2 f3ed2 | "A"e6 e4e2 | \
"D"f3ga2 f3ed2 | "A"e4c2 "D"A3GF2 | "G"G3AB2 "A"A2B3c | "D"d6 d4 :| 


\end{abc}
\index{Spaniard}
\addcontentsline{toc}{subsection}{Spaniard}
\begin{abc}[name=latex_playford94]
X:88
T:Spaniard
C:John Playford, the English Dancing Master, 1651
N:arr. Emma Badowski
M:6/4
L:1/8
K:D major
"D"f2d2d2 d2cdA2 | d2d2cd "A"e4A2 | "D"f2d2f2 f2efd2 | f2g2fg "A"a6 | \
"A"e2e2e2 c2BcA2 | "D"f2efd2 "G"g4a2 | "G"b2a2g2 "D"a3gfe | "A"f2e2de "D"d6 :| 


\end{abc}
\index{Spanish Gipsy}
\addcontentsline{toc}{subsection}{Spanish Gipsy}
\begin{abc}[name=latex_playford95]
X:89
T:Spanish Gipsy
C:John Playford, the English Dancing Master, 1651
N:arr. Emma Badowski
M:6/4
L:1/8
K:G major
A2 | "D"d3ef2 d3ef2 |d6- d4A2 |d3ef2 d3ef2 | d6- d4e2 | f4e2 d4c2 | "G"B6 "A"A6 | "D"A4F2 A4G2 | F12 | 
F4G2 A4A2 | "G"B4A2 "D"d4D2 | "D"F4G2 A3GF2 | "A"E4D2 "D"D6 | "D"F4G2 A4A2 | "G"B4A2 "D"d4D2 | "D"F4G2 A3GF2 | "A"E4D2 "D"D4 :| 


\end{abc}
\index{Staines Morris}
\addcontentsline{toc}{subsection}{Staines Morris}
\begin{abc}[name=latex_playford96]
X:90
T:Staines Morris
C:John Playford, the English Dancing Master, 1651
N:arr. Emma Badowski
M:C
L:1/8
K:Gm
"Gm"d2g2 "C"=e2"D"^f2 | "Gm"g2fe d3e | "Bb"fgfe "Gm"d2cB | "D"AGAB "Gm"G4 :: "Gm"Bcde d2cB | 
"D"AGAB "Gm"G4 :: "Bb"B2B2 "F"F2F2 | "Gm"GABc "Bb"d3e | "F"fgfe "Gm"d2cB | "D"AGAB "Gm"G4 :| 


\end{abc}
\index{Step Stately}
\addcontentsline{toc}{subsection}{Step Stately}
\begin{abc}[name=latex_playford97]
X:1
T:Step Stately
C:John Playford, the English Dancing Master, 1651
N:arr. Steve Hendricks. Maches Pile 46.
M:6/4
L:1/8
K:F major
"F"f2"C"e2"Gm"d2 "F"c3BA2 | "Gm"B2G2G2 G3d"C"ef | "C"g4e2 "F"c2f2c2 | "F"c2A2F2 F6 ::
"C"g4"F"f2 "C"e4"Bb"d2 | "Bb"d2f4 f3g"F"fg | "C"e4"Bb"d2 "F"c2f2c2 | "F"c2A2F2 F6 ::


\end{abc}
\index{Stingo}
\addcontentsline{toc}{subsection}{Stingo}
\begin{abc}[name=latex_playford98]
X:1
T:Stingo
C:John Playford, the English Dancing Master, 1651
P:AA BB x 3
N:arr. Dave Lankford. Matches Pile 2018.
M:6/8
L:1/8
K:C major
P:A
"Am"A2A e2c | "G"dB2 G2G | "Am"A2A e2c |  A3 c3 ::\
P:B 
"C"c2c c2B/c/ | "G"d2d d2d | 
"Am"e2e a2g | e3 "G"g3 | "C"c2c c2B/c/ | "G"d2d d3/e/f | "Am"e3/d/c "G"dB2 | "Am"A3 c3 :|


\end{abc}
\index{Tom Tinker}
\addcontentsline{toc}{subsection}{Tom Tinker}
\begin{abc}[name=latex_playford99]
X:99
T:Tom Tinker
C:John Playford, the English Dancing Master, 1651
N:arr. Steve Hendricks
P:AABB x many
M:6/4
L:1/8
K:C major
P:A
"C"c2 | "C"c2e2c2 "G"d2"Dm"f2AB | "C"c2"G"B3A G4f2 | "C"e3def g2e2c2 | "G"d2B2G2 "C"c4 ::
P:B
"C"c2 | "G"d2"C"e4 "G"d2f4 | "C"c2"G"B3A G4f2 | "C"e3def g2e2c2 | "G"d2B2G2 "C"c4 ::


\end{abc}
\index{Up Tails All}
\addcontentsline{toc}{subsection}{Up Tails All}
\begin{abc}[name=latex_playford100]
X:100
T:Up Tails All
C:John Playford, the English Dancing Master, 1651
N:arr. Emma Badowski
M:C
L:1/8
K:C major
d2 | "G"d4 B2AB | "Am"c2c2 A2A2 | "G"d2d2 B3c | "D"d6e2 | "G"d4 B4 | "Am"c4 A4 | "D"d4 D4 | "G"G6 :| 


\end{abc}
\index{Upon a Summer's Day}
\addcontentsline{toc}{subsection}{Upon a Summer's Day}
\begin{abc}[name=latex_playford101]
X:101
T:Upon a Summer's Day
C:John Playford, the English Dancing Master, 1651
N:arr. Jay Ter Louw. Matches Pile 2018
P:AABBB x 3
M:6/4
L:1/8
K:D minor
P:A
d2 | "Gm"d4d2 B4G2 | "Dm"d6- d4ef | "Gm"g4f2 "Dsus4"g4a2 | "Bb"b6- b4B2 | B4B2 "Dm"A4G2 | "Bb"f6 "Gm"g6 | "Dm"a4B2 A4G2 | "Gm"G6- G4 ::
P:B
a2 | "Dm"a4a2 f4d2 | a6- a4f2 | "Gm"g4f2 g4a2 | "Bb"b6- b4B2 | B4B2 "Dm"A4G2 | "Bb"f6 "Dsus4"g6 | "Dm"a4B2 A4G2 | "Gm"G6- G4 "^(3)":| 


\end{abc}
\index{Whirlygig}
\addcontentsline{toc}{subsection}{Whirlygig}
\begin{abc}[name=latex_playford102]
X:102
T:Whirlygig
%%text for Whirlygig: Woodycock (ABB)x3 A; Whirlygig (ABBx3) A; Woodycock (ABBx3) A
C:John Playford, the English Dancing Master, 1651
N:arr. Aaron Elkiss. Matches Pile 2018
M:6/4
L:1/8
K:C major
P:A
"C"G4G2 c4d2 | "C"e3fg2 "G"d3ef2 | "C"e2g4 "G"G3AB2 | "C"c2e4 "G"d6 ::\
P:B
"Dm"d2f4 f2g2e2 | "G"d3ef2 B3ed2 | 
"C"c3BA2 G3gf2 | "C"e6 "G"d3gf2 | \
"G"d2f4 B2d4 | "F"c3BA2 "G"G3ed2 | "C"e3fg2 "F"a3gf2 | "G"g2d4 "C"c6 :| 


\end{abc}
\index{Wish}
\addcontentsline{toc}{subsection}{Wish}
\begin{abc}[name=latex_playford103]
X:103
T:Wish
C:John Playford, the English Dancing Master, 1651
N:arr. Emma Badowski
M:6/4
L:1/8
K:C major
d2 | "G"B3cd2 "D"A4G2 | "G"G4g2 g4g2 | "F"f3ge2 "G"d3ef2 |\
     "C"g4G2 "G"B3AG2 | "Am"A6 e6 | "G"d2B4 "D"A4G2 | "G"G6- G4 :| 
\end{abc}

\index{Woodycock}
\addcontentsline{toc}{subsection}{Woodycock}
\begin{abc}[name=latex_playford104]
X:104
T:Woodycock
C:John Playford, The English Dancing Master, 1651
N:arr. Jay Ter Louw. Matches Pile 2018
M:6/4
L:1/8
K:D minor
P:A
"Dm"d4d2 f3ed2 | "Am"c4A2 c4c2 | "Dm"d4d2 f3ed2 | [1 "Am"c2A2A2 A6 :| [2 "Am"c2A2A2 A4d2 |:\
P:B
"F"c4B2 c3BA2 | "Gm"B2G2G2 G4G2 | 
"Dm"A2A2A2 "Am"A3Bc2 | "Dm"d2D2D2 D4d2 | \
"F"c4B2 c3BA2 | "Gm"B2G2G2 G4G2 | "Dm"A2A2A2 A3Bc2 | "Dm"d2D2D2 D6 :| 
\end{abc}


\chapter{Other English Country Dances}

This section includes a variety of other English Country dances including
reconstructed music for dances from pre-Playford manuscripts as well as
selected dances from later editions of Playford or later English Country dance
authors that have been danced in the SCA.

\clearpage
\index{Beginning of the World, the}
\index{St. Ledgers}
\addcontentsline{toc}{subsection}{The Beginning of the World}
\begin{abc}[name=latex_playford_later1]
X:1
T:The Beginning of the World
T:or, St. Ledgers
C:adapted from Sellinger's Round by William Byrd and John Playford by Henry of Maldon
P:AA B CC x 3
L:1/8
K:G
P:A
G |: G>AB GAB | c2 c d>e=f | ged c2B | [1 (C3 C) g :| [2 (C3 C) d ] |\
P:B
e3 e>dc | d3 d2 d | 
Bd>c BAG | A3 A2 B \
P:C
|:c>dc B2 G | c>de d2B | A2 G F>EF | [1 G2 G G>AB :| [2 (G3 G2) |]


\end{abc}
\index{Black Nag}
\addcontentsline{toc}{subsection}{Black Nag}
\begin{abc}[name=latex_playford_later2]
X:1
C:John Playford, the Dancing Master, 1670
T:Black Nag
N:arr. Jay Ter Louw. Matches Pile 2018
P:AA BB x 3
M:6/4
L:1/8
K:A minor
E2 |: \
P:A
"Am"A3BA2 "Em"B3AB2 | "Am"c3Bc2 "G"B2c2d2 | "Am"e3dc2 "G"B3AB2 | "Am"A6- A4E2 :: \
P:B
"Em"B2G2E2 B2G2E2 | B2G2E2 B2G2E2 | 
"Am"e2c2A2 e2c2A2 | e2c2A2 e2c2A2 | "Em"B2G2E2 B2G2E2 | B2G2E2 B2c2d2 | \
"Am"e3dc2 "E"B3cB2 |  [1 "Am"A6- A4A2 :|]  [2 "Am"A12 |] 


\end{abc}
\index{Epping Forest}
\addcontentsline{toc}{subsection}{Epping Forest}
\begin{abc}[name=latex_playford_later3]
X:2
I:linebreak $
C:John Playford, the Dancing Master, 1670
T:Epping Forest
K:D minor
M:6/4
L:1/8
P:A
 |: "Gm"d4d2 "F"c3BA2 | "Gm"B3AG2 "D"^F4e2 | "Bb"f3ed2 "F"c4B2 | "Dm"A6- A4d2 | "Bb"f3ed2 "F"c4A2 | "Gm"B3AG2 "D"^F4D2 | 
"C"E3FG2 "D"G4^F2 |  [1 "Gm"G12 :|]  [2 "Gm"G6 G4 :: 
P:B
Bc | "Bb"d6 "F"c6 | "Bb"B6- B4de | "Dm"f6 "A"e6 | 
"Dm"d6- d4 :: 
P:C
d2 | "Gm"d3ed2 "F"c3BA2 | "Gm"B3AG2 "D"^F4D2 | "C"E3FG2 "D"G4^F2 |  [1 "Gm"G6- G4 :|]  [2 "Gm"G12 |] 


\end{abc}
\index{Female Sailor}
\addcontentsline{toc}{subsection}{Female Sailor}
\begin{abc}[name=latex_playford_later4]
X:3
T:Female Sailor
T:Marche pour les Matelots
C:Marain Marais, Alcyone, 1706
M:6/8
L:1/8
K:E minor
"Em"E2B "B"B2A | "Em"G3 "D"A3 | "Em"B2A G2F | G2F E2^D | E2B "B"B2A | "Em"G3 "D"A3 | \
"Em"B2A G2F | E6 :|
"Em"g2f e2^d | e3 "B"B3 | "Em"g2f e2^d | e3 "E"B2d | \
"Am"c2A "Em"B2F | "Em"G2E G2A | B3 "A"^c2^d | "Em"e6 |
 "Em"g2f e2^d | e3 "B"B3 | \
"Em"g2f e2^d | e3 "E"B2d | "Am"c2A "Em"B2F | "Em"G2E G2A | B3 "B"F2E | "Em"E6 |] 


\end{abc}
\index{Hole in the Wall}
\addcontentsline{toc}{subsection}{Hole in the Wall}
\begin{abc}[name=latex_playford_later5]
X:4
I:linebreak $
T:Hole in the Wall
C:John Playford, the Dancing Master, 1695
K:G major
M:3/2
L:1/4
"G"B3/c/ B/c/d"D"Ad | "Em"G3/A/ G/A/B"Bm"FB | "C"E3/F/ E/F/G"G"DB | "Dsus4"G3F"G"G2 | "G"B3/c/ B/c/d"D"Ad | \
"Em"G3/A/ G/A/B"Bm"FB | 
"C"E3/F/ E/F/G"G"DB | "C Dsus4"G3F"G"G2 ||
"Em"g3/f/ e/f/g"D"fe | "B"^d3/e/ d/e/fBf | "Em"g3/f/ e/f/g"D"fe | "Am    B"e3^d"Em"e2 | 
"C"E3/F/ E/F/G"D"F/G/A | "G"G3/A/ G/A/B"Am"A/B/c | "G"B3/c/ B/c/d"D"Dd | "G D"B3A/B/"G"G2 |] 


\end{abc}
\index{Jamaica}
\addcontentsline{toc}{subsection}{Jamaica}
\begin{abc}[name=latex_playford_later6]
X:5
I:linebreak $
T:Jamaica
C:John Playford, the Dancing Master, 1670
M:C|
L:1/4
K:F major
"F"FA AB/c/ | "Bb"d"F"c "Bb"d2 | "F"cA AG/F/ | "Csus4"G2 "F"F2 | "F"FA AB/c/ | "Bb"d"F"c "Bb"d2 | 
"F"cA AG/F/ | "Csus4"G2 "F"F2 |: 
"F"ff "C"ee | "Bb"dd "F"cA | "Bb"ff e/f/g | "Bb"d2 "F"c2 | 
"F"ff "C"ed/c/ | "Bb"dd "F"cA | "Bb"B/c/d "F"cB/A/ | "Csus4"G2 "F"F2 :| 


\end{abc}
\index{Jumbling of Harry, the}
\addcontentsline{toc}{subsection}{The Jumbling of Harry}
\begin{abc}[name=latex_playford_later7]
X:12
T:The Jumbling of Harry
C:Jumbled by Henry of Maldon
C:for the Lovelace Ms. dance "The Jumbling of Joan"
P:AA BB x 3
L:1/8
M:C|
K:G
P:A
d3 c B2A2 | | BABc d2cd | e2d2 cdcB | A6 B2 | c2 B2 AGAB | [1 G8 :| [2 G6 G2 
|:F4 D2 F2 | A4 G2 A2 | B3 A G2 A2 | | F2 D2 E2 F2 | G3 A Bc d2 | e3 d cdcB | A2 G2 G2 F2 | [1 G6 G2 :| [2 G8 |]


\end{abc}
\index{Kelsterne Gardens}
\addcontentsline{toc}{subsection}{Kelsterne Gardens}
\begin{abc}[name=latex_playford_later8]
X:6
T:Kelsterne Gardens
C:John Playford, the Dancing Master, the Third Volume 1726
M:C|
B:Barnes 2nd ed.
Z:2007 John Chambers <jc@trillian.mit.edu>
F:http://trillian.mit.edu/~jc/music/abc/England/tune/KelsterneGardens_Dm.abc	 2007-09-26 22:34:42 UT
L:1/8
K:D minor
"Dm"D2d2 dcBA | "Gm"B2G2 E2G2 | "C"C2c2 cBAG | "F"BAGF "C"AGFE | \
"Dm"D2d2 dcBA | "Gm"B2G2 E2G2 | "Dm"A2F2 "A7"A,2^C2 | "Dm"D8 :: 
"Dm"d2a2 a2ga | "Gm"bagf "C"e2c'2 | "Bb"d2b4d2 | "A"^c2a4ga | \
"Gm"bagf "A7"ed^ce | "A7"A2^c2 "Dm"d4 :| 


\end{abc}
\index{Lightly Love}
\addcontentsline{toc}{subsection}{Lightly Love}
\begin{abc}[name=latex_playford_later9]
X:7
I:linebreak $
T:Lightly Love
T:Light of Love, or Earl of Bedford
P:One round: AA BBB BBB; end with AA BB
C:16th. C English
K:C major
M:6/4
L:1/8
P:A
 |: "C"G3Gc2 B2G2E2 | "F"F3GE2 "G"D4D2 | "C"G3Gc2 B2G2E2 | "F"F3E"G"D2 "C"C4C2 :: 
P:B
"F"F3GF2 "C"E3DC2 | "F"F3GA2 "G"D4D2 | 
"C"G3Gc2 B2G2E2 | "F"F3E"G"D2 "C"C4C2 :| 


\end{abc}
\index{Oranges and Lemons}
\addcontentsline{toc}{subsection}{Oranges and Lemons}
\begin{abc}[name=latex_playford_later10]
X:8
T:Oranges and Lemons
B:3LF / Barnes
C:John Playford, the Dancing Master, 1670
M:C|
L:1/8
K:G
%
%
"D" d2 e2 afed | "G" e3 d "(A)" B3 A | "D" d3 f afed | [1 "A" e3 f "D" d4 :| [2 "A" e3 f "D" d2 ga |]
"G" b6 fg | "D" a6 fe | d3 f afed | "G" e3 d "(A)" B3 A | "D" d3 f afed | "A" e3 f "D" d2 ga |
"G" b6 fg | "D" a6 fe | d3 f afed | "G" e3 d "(A)" B3 A | "D" d3 f afed | [1 "A" e3 f "D" d2 ga :| [2 "A" e3 f "D" d4 |]


\end{abc}
\index{Prince William}
\addcontentsline{toc}{subsection}{Prince William}
\begin{abc}[name=latex_playford_later11]
X:9
T:Prince William
M:C|
C:Walsh c. 1731
B:3LF/Barnes
L:1/4
K:G
D | "G" G2 B A/G/| "D" A2 D "C" uc| "G" B2 "D" A2| "G" G/F/G/A/ G "D" A | "G" B G D B | "D" A3 "Em" G| "D" F d "A" A ^c | "D" d3 :|
|:c/d/|"G" d2 "Am" e d | "C" c>B "D" A d | "Am" c B "D" A "G" G | "D" F/G/A/F/ "G" D2 | "G" G F/G/ "D" A G/A/ | "G" B A/B/ "C" c d | "G" B A/G/ "D" D F | "G" G3 :|


\end{abc}
\index{Sellinger's Round}
\addcontentsline{toc}{subsection}{Sellinger's Round}
\begin{abc}[name=latex_playford_later12]
X:10
T:Sellinger's Round
N:arr. Robert Smith; matches Pile 2018
C:William Byrd, My Ladye Nevells Booke, 1591
P:AA BB x 4
M:6/4
L:1/8
K:C major
P:A
 |: "G"G6 "C"G3AB2 | "C"c6 c3de2 | "Dm"d4c2 "G"B3AB2 |  [1 "C"c12 :|]  [2 "C"c6- c4d2 :: \
P:B
"C"e6 e3dc2 | "G"d6- d4d2 | 
"G"B3cd2 d3cB2 | "D"A6 d4"G"B2 | "C"c3dc2 "G"B4G2 | "F"A3Bc2 "G"B4G2 | \
"F"A4"C"G2 "D"^F3EF2 |  [1 "G"G6- G4d2 :|]  [2 "G"G12 :| 


\end{abc}

\index{Trenchmore}
\index{Tomorrow the fox will come to towne}
\addcontentsline{toc}{subsection}{Trenchmore}
\begin{abc}[name=latex_playford_later13]
X:11
I:linebreak $
T:Trenchmore
T:Tomorrow the fox will come to towne
C:Thomas Ravenscroft, Deuteromelia, 1609
K:G major
M:6/4
L:1/8
"G"G2 | "D"F2F2"G"G2 "D"A4"G"B2 | "D"A4"G"G2 "D"A6 | "G"G6 B6 | "G"d4"C"c2 "G"B4G2 | "D"F2F2"G"G2 "D"A4"G"B2 | "F"c4"G"B2 "D"A4"G"Bc | 
"D"d4"Em"B2 "Am D"A4A2 | "G"G6- G4G2 | "D"F4"G"G2 "D"A4"G"B2 | "D"A4F2 A4"G"D2 | "G"G2G2G2 B4B2 | "G"d4"C"c2 "G"B4G2 | 
"D"F4"G"G2 "D"A4"G"B2 | "F"c4"G"B2 "D"A6 | "G"B6 d6 | "G"d4B2 d4G2 | "D"F4"G"G2 "D"A4"G"B2 | "F"c4"G"B2 "D"A4"G"Bc | 
"D"d4"Em"B2 "Am D"A4A2 | "G"G6- G4 |] 
\end{abc}


\chapter{Modern Folk Dances and SCA Choreography}

This section includes folk dances from later traditions that are popular within
the SCA as well as music for SCA choreographies in a variety of styles.

\clearpage
\index{Alta Battistina}
\addcontentsline{toc}{subsection}{Alta Battistina}
\begin{abc}[name=latex_other1]
X:1
I:linebreak $
T:Alta Battistina
C:James Blackcloak
P:(AABBC)x4 DDD
M:6/8
L:1/8
K:Dmin
B2A2G2 | ^FEF DEF | G^FG BAG | ^F2D2F2 | G2B2A2 | B3 AB2 | A3 GA2 | B2B4 |
A2G2^F2 | G^FG BAF | G2G2^F2 | G6 ::
P:B
d2c2B2 | AGA FGA | B2B2A2 | B6 |
A2G2^F2 | G3 cBA | G2G2^F2 | G6 :|
P:C
B2A2G2 | ^FEF DEF | G2G2^F2 | G3 cBA |
G2G2^F2 | G3 cBA | G2G2^F2 | G6 ::
M:3/8
P:D
B2B | B3/2A/2G | ^F3/2E/2F | DE^F | G3/2^F/2G | BAG | ^F3 | D2^F |
G3/2^F/2G | A3/2G/2A | BAG | BAG | AGA | FGA | B3/2B/2B | B3 |
A3/2G/2^F | A3/2G/2^F | G^FG | E^FG | DE^F | AG^F | G2G | G3 :|


\end{abc}
\index{Burley Mariners }
\addcontentsline{toc}{subsection}{Burley Mariners }
\begin{abc}[name=latex_other2]
X:1
T:Burley Mariners 
T:Rights Of Man
M:C
L:1/8
C:Traditional
S:O 141 From J&D Donaldson
R:Hornpipe
N:Second in set with "Twa Bonnie Maidens"
K:G
GA|:"Em"(3BcB AB GAFG|"Am"EFGA "Em"B2 ef |
"C"gfed "G"edBd|"Am"cBAG "D"A2 GA|!
"Em"(3BcB (3 ABA (3 GAG (3FGF|"Am"EFGA "Em"B2 ef|
"Am"gfed "D"Bgfg|1"Em"e2 E2 E2 GA:|2"Em"e2 E2 E2 ga|!
|:"Em"bagf efga|"Em"bged efga|
"D"(3fgf ef defg|"D"afdf a2 gf|!
"Em"(3ede ef (3gfg a2|"G"(3bc'b (3aba "C"(3gag (3fgf|
"Am"edBA "D"Ggfg|1"Em"e2 E2 E2 ga:|2"Em"e2 E2 E4|


\end{abc}
\index{Comitato d'Amore}
\addcontentsline{toc}{subsection}{Comitato d'Amore}
\begin{abc}[name=latex_other3]
X:1
T:Comitato d'Amore
C:The Committee of Love headed by James Blackcloak
P:AABB x 4
K:G dorian
M:3/4
L:1/8
P:A
"Gm"G4G2 | "F"A3GA2 | "Bb"B6 | B4d2 | "F"c4B2 | A3GA2 | "Bb"B6 | B6 |\
"Gm"G4G2 | "F"A3GA2 | "Bb"B6 | B4d2 | 
"F"c4B2 | "D"A3G^F2 | "Gm"G6 | G4D2 ::\
P:B
"C"E3^FG2 | "D"^F3EF2 | "Gm"G3AB2 | B2A2G2 | "C"E3^FG2 | "D"^F3EF2 | "Gm"G6 | G6 :|


\end{abc}
\index{Corwyn's Folly}
\addcontentsline{toc}{subsection}{Corwyn's Folly}
\begin{abc}[name=latex_other4]
X:2
I:linebreak $
T:Corwyn's Folly
T:Around the House and Mind the Dresser
C:Traditional
M:12/8
L:1/8
K:D major
g | "D"f2d A2=c "G"B2A G2g | "D"f2d "A"AB=c "D"d3 d2"G"g | "D"f2d A2=c "G"B2A G3 | "D"fag "A7"fB=c "D"d3 d2 :: 
a | "D"f2g agf "Em"e2f gfe | "D"f2g agf "A7"e3 g3 | "D"f2g agf "Em"e2f "G"g2a | "Em"ggf "A7"eAB "D"=cd3d2 :| 


\end{abc}
\index{Crossed Purposes}
\addcontentsline{toc}{subsection}{Crossed Purposes}
\begin{abc}[name=latex_other5]
% Output from ABC2Win  Version 2.1  on 2/11/98
X:3
I:linebreak $
T:Crossed Purposes
T:for six couples
C:Johann Sebastien Bach
C:for dance by Sean Andreas O Wynedd
M:C
M:6/8
L:1/8
K:Emin clef=G-8
ef |: gfe ^def | B^c^d e=d=c | BAG FGA | BAG/2F/2 Eef | gfe ^def | B^c^d e=d=c | 
BAG F2G |  [1 G4ef :|]  [2 G4BG :: dAc Bgd | eBd cBA | ^GAB cBA | A4dA | Bgd eBd | cae f^ce | d^cB ^A2B |
B4bf | ^gfe ae=g | fed gd=f | eae f^ce | ^d2B2eB | cdA BcG | ABF GFE | ^DEF GFE |
 [1 E4BG :|]  [2 E4 |]


\end{abc}
\index{Ginevra Weasley}
\addcontentsline{toc}{subsection}{Ginevra Weasley}
\begin{abc}[name=latex_other6]
X:1
T:Ginevra Weasley
T:Sylvi
M:3/4
L:1/8
K:A minor
EA Bc BA | ed ef ed | cd ed cA | Bc BA ^GE | EA Bc BA | ed ef ed | ecA ecA | BA ^GB A2 :|
fed fed | cde3A | ^GAB2d2 | c2B2c2 | fed fed | cde3c | Bc dB ^GF | E6 |
F^G AB cF | EF AB cA | e2d2B2 | ABc ABc | df ed   cf | e2c2A2 | B2c2B^G |  [1 A6 :|] [2 A2A2z2 :|%
%
X:1
T:Heralds in Love
C:Heather Rose Jones, 1990
P:AB x 3
C:arr. Paul Butler
K:C major
M:12/8
L:1/8
P:A
g2f | "C"e3 d3 c3 g2f | e2f d2e "Am"c2A B2c | "G"d3 G3 G3 B2c | d2cd2e "G7"f3 g2f | 
"C"e3 d3 c3 g2f | e2f d2e "Am"c2A B2c | "G"d3 G3 G3 A2B | "C"c6 c3 || 
P:B
G3 | "C"c3 G3 G3 A2B | c2B c2d "Em"e3 d2e | "F"f2e d2c "Em"e2c B2A | 
"F"c2B A3 "G"G3 A2B | "C"c2G E2G c3 d2e | "F"f2c A2c f3 g2f  | "C"e3 d2c "G"B2c d2B | "C"c6 c3 |]


\end{abc}
\index{John Tallis' (Tallow's) Canon}
\addcontentsline{toc}{subsection}{John Tallis' (Tallow's) Canon}
\begin{abc}[name=latex_other7]
X:4
T:John Tallis' (Tallow's) Canon
M:6/8
R:Jig
C:Pat Shaw, 1965
K:G
D |: "G"G2B d2B | "C"cde "D"d2"*"c | "G"B2d BAG | "C"ABG "D"FED | "G"G2B d2B | "C"cde "D"def | "G"g2d BAG | "C"ABG "D"FED :|
|:"G"Bdg dcB | "C"cdB "D"A2d | g"G"ab bag | "C"eag "D"fed | "G"Bdg gdB | "C"cdB "D"A2d | g2"G"d BAG | "C"ABG "D"FED :|


\end{abc}
\index{John Tallow's Canon}
\addcontentsline{toc}{subsection}{John Tallow's Canon}
\begin{abc}[name=latex_other8]
X:5
T:John Tallow's Canon
T:Chanconeta Tedescha
C:Anonymous, B.L Add. MS. 29987, 14th Century
N:Drone D/A
M:C
L:1/8
K:A minor
 |: D2 | D2FE D2A2 | B2AB AGF2 | E2FE D2D2 | E2FG FEFG | A2AG F2E2 | D2d2 cBAG | F2E2 FGFE | D6 :: 
D2 | D2AB dcB2 | AGF2 cdB2 | B2A2 A2BA | G2F2 E4 | D2dB e2de | f2cd edc2 | B2AG F2GF | E6 :| 


\end{abc}
\index{Karobushka}
\addcontentsline{toc}{subsection}{Karobushka}
\begin{abc}[name=latex_other9]
X:6
T:Karobushka
C:Ukranian Traditional
N:Arr. Paul Butler
M:C|
L:1/4
K:A minor
"E"B3/c/ dc/B/ | "A5"c3/d/ ed/c/ | "E"B3/c/ de | "A5"cA A2 | "Dm"f3/g/ ag/f/ | "A5"ec/d/ ed/c/ | "E"B3/c/ de | "A5"cA A2 |
"Dm"f2 ag/f/ | "A5"e/d/c/d/ e/e/d/c/ | "E"B3/c/ dc/B/ | "A5"cA Az ::
"E"^g/ee/ g/ee/ | "A5"a/ee/ a/ee/ | "E"^g/ee/ g/ee/ | "A5"a/ee/ a/e/f/f/ |
"Dm"a/f/a/f/ a/f/a/f/ | "A5"e/c/e/c/ e/c/e/c/ | "E"d/B/d/B/ d/B/d/B/ | "A5"A/c/e/c/ A/c/d/e/ | "Dm5"f2 ag/f/ | "A5"e/d/c/d/ e/e/d/c/ | "E"B3/c/ dc/B/ | "A5"AA Az :|


\end{abc}
\index{Luna Amorosa}
\addcontentsline{toc}{subsection}{Luna Amorosa}
\begin{abc}[name=latex_other10]
X:7
T:Luna Amorosa
C:Henry of Maldon
M:6/8
L:1/8
P:AA BB CC x3
K:G
P:A
d | d>ed f2e | d2B c2e | d2B c>BA | [1 B3 B2 :| [2 B3 B3 |: \
P:B
 e>de c>de | d>cd B>cd 
| c>Bc ABc | | d2B d2d :: \
P:C
c2 B d>cB | A2G B>AG | A2F G2A | [1 B3 B3 :| [2 B3 B2 |]


\end{abc}
\index{Mairi's Wedding}
\addcontentsline{toc}{subsection}{Mairi's Wedding}
\begin{abc}[name=latex_other11]
X:8
I:linebreak $
T:Mairi's Wedding
T:Lewis Bridal Song
C:John Roderick Bannerman, 1934
M:C
L:1/8
K:G major
 |: "G"D3E D2E2 | G2A2 B4 | "C"A2G2 E2G2 | "D7"B2A2 B2d2 | "G"D3E D2E2 | G2A2 B4 | 
"C"A2G2 E2C2 | "D7"D4 D4 :: 
"G"d3d d2e2 | d2c2 B4 | "C"A2G2 E2G2 | "D7"B2A2 B2d2 | 
"G"d3d d2e2 | d2c2 B4 | "C"A2G2 E2G2 | "D7"D4 D4 :| 


\end{abc}
\index{Maanschaft Pavane}
\addcontentsline{toc}{subsection}{Maanschaft Pavane}
\begin{abc}[name=latex_other12]
X:1
T:Maanschaft Pavane
T:from the movie Henry VIII
P:ABC DEF DEF
C:D. Munrow, based upon Turkelone
N:arr. Karen Kasper
K:C major
M:6/8
L:1/8
P:A
"A"e6 | "A"e3/d/e e2e | "Dm"f2e f2g | "F"a3 a3 | "C"g3 g3/f/g | "F"a2a "C"g2f | "Gm"e2d "A"^c3/B/c | "D"d3 d3 ::
P:B
"Dm"f3/e/d "A"^c3/B/c | "Dm"d3/e/d "A"^c3/d/e | "F"f3/g/a "C"e3/f/g | "Dm"f3/e/d "A"^c3 | "Dm"d3/e/f "C"e3/f/g | "F"a2a a3/g/f | "C"e2d "A"^c3/B/c | "D"d3 d3 ::
P:C
"G"G2^F G2A | "G"B3/A/B "Em"G3/A/B | "Am"c3/B/A "Dm"B3/A/"E"B | "A"A3 A3 ::\
P:D
"Dm"A2d d2^c | "Dm"d3/c/B A2B | "C"c2B "F"A2G/^F/ | "Gm"E/D/"A"E2 "Dm"D3 ::
P:E
"Dm"F/G/A2 "F"A2G/F/ | "C"E/D/E2 "Dm"D3 | "Dm"D2E F/G/A/F/G/F/ | "Dm"E/D/E"A"^C "D"D3 ::\
P:F
"F"F2F "C"C2C | "Dm"D2D "A"E3 | "A"E2F "C"E/F/"Gm"G/A/G/F/ | "A"E/D/E2 "D"D3 :|


\end{abc}
\index{Mi Amore Cadenza}
\addcontentsline{toc}{subsection}{Mi Amore Cadenza}
\begin{abc}[name=latex_other13]
X:1
T:Mi Amore Cadenza
P:AABBCC x 3 or AABBCC x 4
C:Gwommy Anpurpaidh & Felice Debbage, A.S. 48
N:arr. Monique Rio
M:3/4
L:1/8
K:Cmaj clef=G-8
P:A
"C"e3ee2 | "C"e2f2g2 | "Fm"_a3aa2 | "Fm"_a2_b2a2 | \
"C"g3ee2 | "C"e2f2e2 | "G"d6 |  [1 "G"d2g2f2 :|]  [2 "G"d4c2 ::
P:B
"G"B3BB2 | "G"B2A2B2 | "Cm"c3cd2 | "Cm"_e2d2c2 | \
"G"B3BB2 | "G"B2A2B2 | "C"c6 |  [1 "C"c2B2A2 :|]  [2 "C"c4d2 ::
P:C
"C"e4e2 | "F"f4f2 | "C"e3fe2 | "G"d4c2 | \
"C"e3cd2 | "Cm"_e3dc2 | "G"B3AB2 | "C"c6 :|


\end{abc}
\index{Mordred's Lullaby}
\addcontentsline{toc}{subsection}{Mordred's Lullaby}
\begin{abc}[name=latex_other14]
X:1
T:Mordred's Lullaby
C:Heather Dale
N:Modern Choreography by Rosina del Bosco Chiaro in the 15th C Italian Style
N:Permission granted for use in dancing
N:Please support Heather through her store, www.HeatherDale.com/store.
P:AA BCBC ACBC AA
K:D dorian
M:6/4
L:1/8
P:A
D6 A4zA | d2c2B2 A3G^F2 | A6 D6 | A2G2FE- E2D2C2 | \
D6 A4zA | d2c2B2 A3G^F2 | A6 D6 | A2G2FE- E2D2C2 ::
P:B
D6 D4D2 | E2E2EE- E2D2E2 | F4F2 F3ED2 | E2E2EE- E4EE | \
E2D2DD- D4zD | E2EE-E2 E2D2E2 | F4F2 FE-E2D2 | E2E2EE- E2D2C2 ||
P:C
D2D2D2 CDD2D2 | CDD2D2 E2E2E2 | D2D2D2 CDD2D2 | CDD2D2 E2E2E2 :|


\end{abc}
\index{On the Banks of the Helicon}
\addcontentsline{toc}{subsection}{On the Banks of the Helicon}
\begin{abc}[name=latex_other15]
X:9
T:On the Banks of the Helicon
C:Henry IV Plantagenet
S:M.G. Mudrey, Jr. <mgmudrey@wisc.edu> strathspey 2003-7-3 (Helicon.jpg)
M:C
L:1/4
F:http://trillian.mit.edu/~jc/music/abc/Scotland/misc/OnTheBanksOfTheHelicon_G.abc	 2007-09-26 22:37:35 UT
K:G
|:G | "G"Gd "D7"dc | "G"B>A Gd | "C"e/f/g fe | "D7"d>c "G"BG | "C"cB "D7"A/G/A | "G"G2 G :|
|:"G7"G | "C"c>B c"/B"d | "Am"ed ce/=f/ | "D"ed "A7"d^c | "D"d3 \
::"D7"d | "G"B>A "C"Gc | "G"B>A "C"Gc | BG "D7"GF | "G"G3 :|


\end{abc}
\index{Pennsic Dance}
\addcontentsline{toc}{subsection}{Pennsic Dance}
\begin{abc}[name=latex_other16]
X:10
I:linebreak $
T:Pennsic Dance
C:Emil Allzuwissender
M:6/8
L:1/8
K:C major
"Am"efe d2e | "G"c2d B3 | "Am"cBA cBA | "Am"c3 "G"d3 | "Am"efe d2e | "G"c2d B2G | 
"Am"A2c "G"c2B | "C"c3 c3 :: "Am"e3 g3 | "C"e2f "Em"e2c | "G"B2c dcB | "Dm"d3 e3 | 
"Am"e3 f3 | "G"d2e d2B | "Dm"A2B cBA | "C"c3 "G"d3 | "Am"efe d2e | "Em"c2d B2G | 
"Am"A2c "G"c2B | "Am"c3 "G"d3 :| "Am"efe d2e | "G"c2d B2G | "Am"A2c "G"c2B | "C"c3 c3 |] 


\end{abc}
\index{Pontyplas}
\addcontentsline{toc}{subsection}{Pontyplas}
\begin{abc}[name=latex_other17]
X:11
I:linebreak $
T:Pontyplas
T:Un jour Dieu se resolut
C:Michael Corrette, Nouveau Livre de noels, 1741
P:10 times
M:C
L:1/8
K:G major
 "G"B2c2 | "G"d2G2 G2"D"A2 | "G"B4 "G"B2c2 | "G"d2G2 G2"D"A2 | "G"B4 B2"Am"c2 | "G"d2G2 G2"D"A2 | "G"B4 B2"Am"c2 | 
"G"d2G2 G2"D"A2 | "G"B4 B2c2 | "G"d2e2 "D7"c2d2 | "G"d4 G2d2 | "G"d2d2 "C"e4 | "G"d4 "C"c2"D7"c2 | 
"G"B2c2 d2cB | "D"A4 c2c2 | "G"B2"Am"c2 "D"A4 | "G"G4 :| 


\end{abc}
\index{Posten's Jig}
\addcontentsline{toc}{subsection}{Posten's Jig}
\begin{abc}[name=latex_other18]
X:12
T:Posten's Jig
C:Traditional
M:6/8
L:1/8
K:G major
T:Old Maid at the Spinning Wheel
D | "G"GFG B2G | "G"BcA B2D | "G"GFG BAG | "D"F2G AFD |\
 "G"GFG B2G | "G"BcA B2g | "D"fed cAF | "G"GAG G2 :: 
c | "G"BAG "D"AFD | "D"DED AFD | "D"DF/E/D AFD | "C"EFG "D"ABc | \
"G"BAG "D"AFD | "D"DED AFD | "D"ded cAF | "G"GAG G2 :: 
D | "G"GBd gba | "G"gdB "C"ecA | "G"dBG "C"cAG | "D"F2G AFD | \
"G"GBd gba | "G"gdB "C"ecA | "D"fed cAF | "G"GAG G2 :| 
T:Ballykeale
c |: A=FA A2c | A=FA Adc | BGG DGG | BGB dcB | \
A=FA A2c | A=FA A2a | bag fdc |  [1 BGB dcB :|]  [2 BGF GBd :: 
g3 gag | fde fd^c | dgg gag | fd^c def |\
 g2a bag | fde fga | bag fdc |  [1 BGF GBd :|]  [2 BGB dcB |] 


\end{abc}
\index{Quen Quer Que}
\addcontentsline{toc}{subsection}{Quen Quer Que}
\begin{abc}[name=latex_other19]
X:13
I:linebreak $
T:Quen Quer Que
P:(AB) x 8 AA
C:13th Century Spanish for choreography by Sion Andreas o Wynedd
K:G mixolydian
M:C
L:1/8
P:A (Chorus)
d4 B4 | G4 A2B2 | c4 A2B2 | d4 d4 | e4 c4 | d4 e2f2 | 
g4 f2e2 | d2c2 B2c2 | A8 | A4 B2c2 | d4 d4 | c2B2 A2B2 | 
G4 F4 | A4 B2A2 | G4 A2B2 | d2c2 B2A2 | G2F2 G2A2 | G8 || 
P:B (Verse)
B4 c4 | d4 e2d2 | c2B2 B2A2 | G4 F4 | A4 B2A2 | G4 A2B2 | 
d2c2 B2A2 | B2A2 G2F2 | G8 | B4 c4 | d4 e2d2 | c2B2 B2A2 | 
G4 F4 | A4 B2A2 | G4 A2B2 | d2c2 B2A2 | B2A2 G2F2 | G8 | 
d4 B4 | G4 A2B2 | c4 A2B2 | d4 d4 | e4 c4 | d4 e2f2 | 
g4 f2e2 | d2c2 B2c2 | A8 | A4 B2c2 | d4 d4 | c2B2 A2B2 | 
G4 F4 | A4 B2A2 | G4 A2B2 | d2c2 B2A2 | G2F2 G2A2 | G8 :| 


\end{abc}
\index{Road to the Isles}
\addcontentsline{toc}{subsection}{Road to the Isles}
\begin{abc}[name=latex_other20]
X:14
I:linebreak $
T:Road to the Isles
T:Scotland the Brave
C:Traditional
K:D major
M:C
L:1/8
 |: "D"D4 D3E | "D"F2D2 F2A2 | "D"d4 d3c | "D"d2A2 F2D2 | "G"G4 B3G | "D"F2A2 F2D2 | 
"E7"E4 A3B | "A7"A3B AGFE | "D"D4 D3E | "D"F2D2 F2A2 | "D"d4 d3c | "D"d2A2 F2D2 | 
"G"G4 B3G | "D"F2A2 F2D2 | "A7"E4 D3E | "D"D2D2 F2A2 | 
"D"d4 d3c | "D"d2A2 F2D2 | 
"D"d4 d3c | "D"d2A2 F2D2 | "G"G4 B3G | "D"F2A2 F2D2 | "E7"E4 A3B | "A7"A3B AGFE | 
"D"D4 D3E | "D"F2D2 F2A2 | "D"d4 d3c | "D"d2A2 F2D2 | "G"G4 B3G | "D"F2A2 F2D2 | 
"A7"E4 D3E | "D"D8 :| 


\end{abc}
\index{Ronde IX}
\addcontentsline{toc}{subsection}{Ronde IX}
\begin{abc}[name=latex_other21]
X:15
I:linebreak $
T:Ronde IX
C:Tylman Susato
M:C|
L:1/4
K:G major
"D"d/ |: "Am"c/A/"D"A/c/ "G"B/G/"C"G/B/ | "F"A/F/"C"G/A/ "G"B"Em"G/d/ | "Am"c/A/"D"A/c/ "G"B"Am"A/G/ | "D"F/E/G/F/ "G"Gz/ :: G/ | "Am"c/B/c/d/ "Em"e3/e/ | "G"d/c/B/A/ "Em"BG/G/ | 
"Am"c/B/c/d/ "Em"e3/e/ |  [1 "G"d/c/B/A/ "E"Bz/ :|]  [2 "G"d/ | c/B/"E"A/B2 |] 


\end{abc}
\index{St. Barbara's}
\addcontentsline{toc}{subsection}{St. Barbara's}
\begin{abc}[name=latex_other22]
X:1
T:St. Barbara's
C:James Blackcloak
M:C
L:1/8
K:F major
 F2A2 | B4 B2cB | A2G2 "^*"F4 | G2A2 B2G2 | F4 "^*"F2F2 | E4 E2FE | F2E2 "^*"D4 | C4 C2E2 | F4 "^*"F2A2 |
c2c2 c4 | A2GA F4 | C2D2 E2C2 | F4 AGF2 | G2F2 E2G2 | A4 c2A2 | G2F2 E4 | F4 |]


\end{abc}
\index{St. Joan}
\addcontentsline{toc}{subsection}{St. Joan}
\begin{abc}[name=latex_other23]
X:16
T:St. Joan
T:Hunsdon House
M:6/8
L:1/8
K:C major
|:"C"C2G E2A | GE2 "G"D2C | "Am"c2e "G"dB2 | "D7"A3 "G"G3 :| \
"Em"g2f eB2 | "Am"c2B Ade | 
"Dm"f2e dB2 | "G7"A3 GAB | "Am"c2B AE2 | "Dm"F2E DAB | "Am"cde "Dm"f2e | "G7"d3 "C"c3 :| 


\end{abc}
\index{St. Paul's Cathedral}
\addcontentsline{toc}{subsection}{St. Paul's Cathedral}
\begin{abc}[name=latex_other24]
X:1
I:linebreak $
T:St. Paul's Cathedral
C:Master James Blackcloak
N:Free for non-commercial use within the SCA
M:3/4
L:1/8
K:D minor
"C"G3FE2 | "Dm"D4E2 | "F"F2c2B2 | "Dm"A6 | "Gm"d3cG2 | "F"A2B2F2 |  [1 "C"G2F2E2 | "Dm"D4EF :|]
 [2 "C"G4F2 | "Gm"G2A2D2 :: "F"c6 | "Am"A6 | "Gm"G2A2d2 | "F"c6 | "Am"A6 | "Gm"G2A2d2 |
"C"e6 | "F"f6 | "C"e3dc2 | "Dm"d2c2A2 |  [1 "C"G4F2 | "Gm"G2A2D2 :|]  [2 "Gm"G2A2"C"C2 | "Dm"D4E2 ::
"F"FGA2F2 | "Bb"B3AGF | "Gm"G4FG | "Am"ABc2A2 | "Dm"d3ef2 | "Gm"g6 | "Am"agfgfe | "F"f4"C"e2 |
"Dm"d3c"Am"A2 |  [1 "Bb"B2A2F2 | "C"EFGFEC | "Dm"D4E2 :|]  [2 "Bb"B2c2d2 | "C"efgfec | "Dm"d4e2 :| "F"f2g2a2 |
"Gm"g3fed | "A"^c2d2e2 | "Dm"d2e2f2 | "A"e3d^c2 | "Dm"dAd2c2 | "Gm"B2A2G2 | "F"FEFG"C"E2 | [1-4 "Dm"D4 EF :| [5 "Dm"D6 |]


\end{abc}
\index{St. Paul's Cathedral (Verse 4)}
\addcontentsline{toc}{subsection}{St. Paul's Cathedral (Verse 4)}
\begin{abc}[name=latex_other25]
X:1
I:linebreak $
T:St. Paul's Cathedral (Verse 4)
C:Margaret of Raynsford and Jadwiga Krzyzanowska
N:Free for non-commercial use within the SCA
M:3/4
L:1/8
K:D minor
"Dm"D3/E/ FFED | "A"^CDEECA, | "Dm"D3/E/ FFED | "C"E3FG2 | "F"F3/G/ AAGF | "C"E3/F/ GGFE | "Dm"D4"C"C2 | "Dm"D6 ::
"Gm"G3/A/ BBAG | "Dm"dcBABc | "Am"A3GA2 | "Dm"F3/G/ AAGF | "Bb"BAGFGA | "C"G3FGA | "Gm"BAGF"C"E2 | "Dm"AGFE"A"^C2 |
"Dm"A3/G/ FAGF | "C"G3/F/ EGFE | "Dm"D4"C"C2 | "Dm"D6 :: "F"F3/G/ AFAB | "C"cdcBAc | "Bb"BAGF"C"E2 | "Dm"D3/E/ FDFG |
"F"ABAGFA | "C"GFED"Am"C2 | "Gm"B,CDE"Dm"F2 | "Am"CDEF"C"G2 | "Dm"A3/G/ FAGF | "C"G3/F/ EGFE | "Dm"D4"C"C2 | "Dm"D6 :|
"Dm"D3/E/ FDEF | "A"ED^CDEC | "Dm"D3EF2 | "C"E3/F/ GEFG | "Bb"B,CDEFG | "C"E3FG2 | "Dm"A3/G/ FAGF | "Dm"D4"C"C2 |
"Dm"D6 |]


\end{abc}
\index{Turkish Bransle}
\addcontentsline{toc}{subsection}{Turkish Bransle}
\begin{abc}[name=latex_other26]
X:17
T:Turkish Bransle
T:Schiarazula Marazula
C:Giorgio Mainerio, il Primo Libro di Balli, 1578
M:C
L:1/8
K:G minor
 |: d2 | "Gm"d2c2 d2c2 | "Gm"B2B2 B2A2 | "Gm"G2F2 G2A2 | "Gm"G2G2 G2d2 | \
"Gm"d2c2 d2c2 | "Gm"B2B2 B2A2 | "Gm"G2F2 G2A2 | "Gm"G2G2 G2cB | 
"F"A2G2 A2B2 | "F"A2G2 A2dc | "Gm"B2G2 "Dsus4"G2^F2 | "Gm"G2G2 G2cB | \
"F"A2G2 A2B2 | "F"A2G2 A2dc | "Gm"B2G2 "Dsus4"G2^F2 | "G"G6 :| 


\end{abc}
\index{Two Fat Ladies}
\addcontentsline{toc}{subsection}{Two Fat Ladies}
\begin{abc}[name=latex_other27]
X:18
T:Two Fat Ladies
T:Barbarini's Tambourine
R:reel
Z:2007 John Chambers <jc@trillian.mit.edu>
B:Barnes 2nd ed.
M:2/4
L:1/16
K:Gmaj
Bc \
| "G"d2G2 G2G2 | G2A2 "D"A2B2 \
| "G"{c}B2AB "C"cBAG | "D"A2D2 D2Bc \
| "G"d2G2 G2G2 | G2A2 "D"A2B2 |
| "G"{c}B2AB "C"cBAG | "D"A6 :: FG \
| "D"A2D2 D2D2 | D6 GA \
| "Em"B2E2 E2E2 | E6 AB \
| "D"c2F2 F2Bc |
| "G"d2G2 G2cd \
| "C"e2c2 "D"f2d2 | "G"g2fe "C"d2c2 \
| "G"{c}B2AG "D"{B}A2GF | "G"{A}G2FE "C"D2C2 \
| "G"B,2G2 "D"A,2F2 | "G"[G6G,6] :|


\end{abc}
\index{Violet's Fancie}
\addcontentsline{toc}{subsection}{Violet's Fancie}
\begin{abc}[name=latex_other28]
X:19
T:Violet's Fancie
C:James Blackcloak
P:16 times through
L:1/8
M:6/8
K:D dorian
c |: "Am"A>Bc "Dm"BcA | "E"^GAB E2B | "C"c>de "Dm"dec | "G"Bcd "Am"e3
"C"e>fg "B(dim)"fed | "Am"c>de "G"dcB | "F"ABc "Dm"d2f | [1-15 "C" e3 "Dm"d2c :| [16 "C" e3 ("Dm" d3 | Hd6) |]


\end{abc}


\clearpage

\printindex

\end{document}
